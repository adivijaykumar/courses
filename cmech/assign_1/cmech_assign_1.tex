\documentclass{article}

\usepackage{fancyhdr}
\usepackage{extramarks}
\usepackage{amsmath}
\usepackage{amsthm}
\usepackage{amsfonts}
\usepackage{tikz}
\usepackage{physics}
\usepackage{amssymb}
\usepackage[plain]{algorithm}
\usepackage{algpseudocode}

\usetikzlibrary{automata,positioning}

%
% Basic Document Settings
%

\topmargin=-0.45in
\evensidemargin=0in
\oddsidemargin=0in
\textwidth=6.5in
\textheight=9.0in
\headsep=0.25in

\linespread{1.1}

\pagestyle{fancy}
\lhead{\hmwkAuthorName}
\chead{\hmwkClass\ (\hmwkClassInstructor\ \hmwkClassTime): \hmwkTitle}
\rhead{\firstxmark}
\lfoot{\lastxmark}
\cfoot{\thepage}

\renewcommand\headrulewidth{0.4pt}
\renewcommand\footrulewidth{0.4pt}

\setlength\parindent{0pt}

%
% Create Problem Sections
%

\newcommand{\enterProblemHeader}[1]{
    \nobreak\extramarks{}{Problem \arabic{#1} continued on next page\ldots}\nobreak{}
    \nobreak\extramarks{Problem \arabic{#1} (continued)}{Problem \arabic{#1} continued on next page\ldots}\nobreak{}
}

\newcommand{\exitProblemHeader}[1]{
    \nobreak\extramarks{Problem \arabic{#1} (continued)}{Problem \arabic{#1} continued on next page\ldots}\nobreak{}
    \stepcounter{#1}
    \nobreak\extramarks{Problem \arabic{#1}}{}\nobreak{}
}

\setcounter{secnumdepth}{0}
\newcounter{partCounter}
\newcounter{homeworkProblemCounter}
\setcounter{homeworkProblemCounter}{1}
\nobreak\extramarks{Problem \arabic{homeworkProblemCounter}}{}\nobreak{}

%
% Homework Problem Environment
%
% This environment takes an optional argument. When given, it will adjust the
% problem counter. This is useful for when the problems given for your
% assignment aren't sequential. See the last 3 problems of this template for an
% example.
%
\newenvironment{homeworkProblem}[1][-1]{
    \ifnum#1>0
        \setcounter{homeworkProblemCounter}{#1}
    \fi
    \section{Problem \arabic{homeworkProblemCounter}}
    \setcounter{partCounter}{1}
    \enterProblemHeader{homeworkProblemCounter}
}{
    \exitProblemHeader{homeworkProblemCounter}
}

%
% Homework Details
%   - Title
%   - Due date
%   - Class
%   - Section/Time
%   - Instructor
%   - Author
%

\newcommand{\hmwkTitle}{Assignment\ \#1}
\newcommand{\hmwkDueDate}{February 12, 2014}
\newcommand{\hmwkClass}{Classical Mechanics}
\newcommand{\hmwkClassTime}{}
\newcommand{\hmwkClassInstructor}{Prof.Manas Kulkarni}
\newcommand{\hmwkAuthorName}{\textbf{Aditya Vijaykumar}}

%
% Title Page
%

\title{
    %\vspace{2in}
    \textmd{\textbf{\hmwkClass:\ \hmwkTitle}}\\
    \normalsize\vspace{0.1in}\small{\hmwkDueDate\ }\\
%    \vspace{3in}
}

\author{\hmwkAuthorName}
\date{}

\renewcommand{\part}[1]{\textbf{\large Part \Alph{partCounter}}\stepcounter{partCounter}\\}

%
% Various Helper Commands
%

% Useful for algorithms
\newcommand{\alg}[1]{\textsc{\bfseries \footnotesize #1}}

% For derivatives
\newcommand{\deriv}[1]{\frac{\mathrm{d}}{\mathrm{d}x} (#1)}

% For partial derivatives
\newcommand{\pderiv}[2]{\frac{\partial}{\partial #1} (#2)}

% Integral dx
\newcommand{\dx}{\mathrm{d}x}

% Alias for the Solution section header
\newcommand{\solution}{\textbf{\large Solution}}

% Probability commands: Expectation, Variance, Covariance, Bias
\newcommand{\E}{\mathrm{E}}
\newcommand{\Var}{\mathrm{Var}}
\newcommand{\Cov}{\mathrm{Cov}}
\newcommand{\Bias}{\mathrm{Bias}}

\begin{document}

\maketitle

%\pagebreak

\begin{homeworkProblem}
\textbf{Part (a)}
\begin{flalign*}
V(x) = \alpha x^2/2 + \beta x^4/4\\
F(x) = -\pdv{V}{x} = -\alpha x - \beta x^3
\end{flalign*}
Including the damping term, we write the equation of motion as,
$$m\ddot{x} + \delta \dot{x} = -\alpha x - \beta x^3$$
$$m\ddot{x} + \delta \dot{x} + \alpha x + \beta x^3 = 0$$

The total energy of the system $E=T+V=m\dot{x}^2/2 + \alpha x^2/2 + \beta x^4/4$. Taking the time derivative, one gets
$$\dot{E} = m\ddot{x}\dot{x} + \alpha x \dot{x} + \beta x^3\dot{x}$$
Substituting from the equation of motion for $m\ddot{x}$,
$$\dot{E}=-(\delta \dot{x} + \alpha x + \beta x^3)\dot{x} + \alpha x \dot{x} + \beta x^3\dot{x}$$
$$\dot{E} = -\delta \dot{x}^2$$
Hence energy is dissipated from the system at a rate $\delta \dot{x}^2$.\\

\textbf{Part (b) and (c) }- \textit{code files attached}\\

\textbf{Part (d)}
$$E= m\dot{x}^2/2 + \alpha x^2/2 + \beta x^4/4$$

The turning points will satisfy $E = \alpha x^2/2 + \beta x^4/4$\\

\textbf{Part (f)}\\
Substituting $x = A \cos(\omega t + \phi)$ in $m\ddot{x} + \delta \dot{x} + \alpha x  = \gamma \cos(\omega t) $
\begin{flalign*}
	(-m\omega^2 + \alpha )\cos(\omega t + \phi) - \delta \omega \sin(\omega t + \phi) = \frac{\gamma}{A}\cos(\omega t)
\end{flalign*}	
Expanding the RHS and equating coefficients of $\cos(\omega t)$ and $\sin(\omega t)$, one gets
$$(-m\omega^2 + \alpha )\cos(\phi) - \delta \omega \sin(\phi) = \frac{\gamma}{A}$$
$$-(-m\omega^2 + \alpha )\sin(\phi) - \delta \omega \cos(\phi) = 0$$
Solving these, we get
$$A = \frac{\gamma}{(-m\omega^2 + \alpha )\cos(\phi) - \delta \omega \sin(\phi)} \text{ ; } \phi = \tan^{-1} \left(\frac{\delta \omega}{\alpha - m\omega^2}\right)$$

[PLOT THESE]\\

\textbf{Part (g)}\\
Substituting $x = \sum_{n=0}^{\infty} a_n x^n$ in $m\ddot{x} + \delta \dot{x} + \alpha x + \beta x^3  = \gamma \cos(\omega t)$

\end{homeworkProblem}



\end{document}
