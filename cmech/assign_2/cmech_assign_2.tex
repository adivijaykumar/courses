\documentclass{article}

\usepackage{fancyhdr}
\usepackage{extramarks}
\usepackage{amsmath}
\usepackage{amsthm}
\usepackage{amsfonts}
\usepackage{tikz}
\usepackage{physics}
\usepackage{amssymb}
\usepackage[plain]{algorithm}
\usepackage{algpseudocode}

\usetikzlibrary{automata,positioning}

% Basic Document Settings
%

\topmargin=-0.45in
\evensidemargin=0in
\oddsidemargin=0in
\textwidth=6.5in
\textheight=9.0in
\headsep=0.25in

\linespread{1.1}

\pagestyle{fancy}
\lhead{\hmwkAuthorName}
\chead{\hmwkClass\ : \hmwkTitle}
\rhead{\firstxmark}
\lfoot{\lastxmark}
\cfoot{\thepage}

\renewcommand\headrulewidth{0.4pt}
\renewcommand\footrulewidth{0.4pt}

\setlength\parindent{0pt}

%
% Create Problem Sections
%
\newcommand{\be}{\begin{equation}}
\newcommand{\ee}{\end{equation}}
\newcommand{\bes}{\begin{equation*}}
\newcommand{\ees}{\end{equation*}}
\newcommand{\bea}{\begin{flalign*}}
\newcommand{\eea}{\end{flalign*}}

\newcommand{\enterProblemHeader}[1]{
    \nobreak\extramarks{}{Problem \arabic{#1} continued on next page\ldots}\nobreak{}
    \nobreak\extramarks{Problem \arabic{#1} (continued)}{Problem \arabic{#1} continued on next page\ldots}\nobreak{}
}

\newcommand{\exitProblemHeader}[1]{
    \nobreak\extramarks{Problem \arabic{#1} (continued)}{Problem \arabic{#1} continued on next page\ldots}\nobreak{}
    \stepcounter{#1}
    \nobreak\extramarks{Problem \arabic{#1}}{}\nobreak{}
}

\setcounter{secnumdepth}{0}
\newcounter{partCounter}
\newcounter{homeworkProblemCounter}
\setcounter{homeworkProblemCounter}{1}
\nobreak\extramarks{Problem \arabic{homeworkProblemCounter}}{}\nobreak{}

%
% Homework Problem Environment
%
% This environment takes an optional argument. When given, it will adjust the
% problem counter. This is useful for when the problems given for your
% assignment aren't sequential. See the last 3 problems of this template for an
% example.
%
\newenvironment{homeworkProblem}[1][-1]{
    \ifnum#1>0
        \setcounter{homeworkProblemCounter}{#1}
    \fi
    \section{Problem \arabic{homeworkProblemCounter}}
    \setcounter{partCounter}{1}
    \enterProblemHeader{homeworkProblemCounter}
}{
    \exitProblemHeader{homeworkProblemCounter}
}

%
% Homework Details
%   - Title
%   - Due date
%   - Class
%   - Section/Time
%   - Instructor
%   - Author
%

\newcommand{\hmwkTitle}{Assignment\ \#2}
\newcommand{\hmwkDueDate}{Due 11th September 2018}
\newcommand{\hmwkClass}{Classical Mechanics}
\newcommand{\hmwkClassTime}{}
\newcommand{\hmwkClassInstructor}{Prof.Manas Kulkarni}
\newcommand{\hmwkAuthorName}{\textbf{Aditya Vijaykumar}}

%
% Title Page
%

\title{
    %\vspace{2in}
    \textmd{\textbf{\hmwkClass:\ \hmwkTitle}}\\
    \normalsize\vspace{0.1in}\small{\hmwkDueDate\ }\\
%    \vspace{3in}
}

\author{\hmwkAuthorName}
\date{}

\renewcommand{\part}[1]{\textbf{\large Part \Alph{partCounter}}\stepcounter{partCounter}\\}

%
% Various Helper Commands
%

% Useful for algorithms
\newcommand{\alg}[1]{\textsc{\bfseries \footnotesize #1}}

% For derivatives
\newcommand{\deriv}[1]{\frac{\mathrm{d}}{\mathrm{d}x} (#1)}

% For partial derivatives
\newcommand{\pderiv}[2]{\frac{\partial}{\partial #1} (#2)}

% Integral dx
\newcommand{\dx}{\mathrm{d}x}

% Alias for the Solution section header
\newcommand{\solution}{\textbf{\large Solution}}

% Probability commands: Expectation, Variance, Covariance, Bias
\newcommand{\E}{\mathrm{E}}
\newcommand{\Var}{\mathrm{Var}}
\newcommand{\Cov}{\mathrm{Cov}}
\newcommand{\Bias}{\mathrm{Bias}}

\begin{document}
\maketitle

\begin{homeworkProblem}
	\solution\\
	The Lagrangian for the given system can be written as,
	$$L = T + V = \frac{1}{2} m x^2 \omega ^2+\frac{1}{2} m \left(\dot{x}^2+\dot{y}^2\right)-m g y$$
	From the problem, we know that $y = k \left(\frac{x}{l}\right)^\alpha$, which means that $\dot{y} =k \alpha \frac{x^{\alpha -1 }}{l^\alpha}\dot{x}$. Substituting these into the form of the Lagrangian and simplifying, we get,
	$$L=\frac{1}{2} m \left(-2 g k \left(\frac{x}{l}\right)^{\alpha }+\dot{x}^2\left(\frac{\alpha ^2 k^2  x^{2 \alpha-2 }}{l^\alpha} + 1\right)+x^2 \omega ^2\right)$$
	The equation of motion can be written as,
	\bes
	\boxed{\alpha  g k x^2 \left(\frac{x}{l}\right)^{\alpha }+(\alpha -1) \alpha ^2 k^2 \dot{x}^2 \left(\frac{x}{l}\right)^{2 \alpha }+\alpha ^2 k^2 x \ddot{x} \left(\frac{x}{l}\right)^{2 \alpha }-x^4 \omega ^2+x^3 \ddot{x}=0}
	\ees
	The equlibrium points will satisfy $ \dot{x}=\ddot{x}=0 $. This means that the equilibrium point will be,
	\bes
	x_0 = \qty(\frac{\omega^2 l^\alpha}{gk\alpha})^\frac{1}{\alpha-2}
	\ees
	We substitute $ x(t) = x_0 + \epsilon y(t)$

	\bes
	\ddot{y}  + y\frac{(\alpha-2 )\omega^2}{{\alpha ^2 k^2    \left(\frac{x_0}{l}\right)^{2 \alpha }}+1}  = 0
	\ees
	
	For small oscillations, the coefficient of $ y $ in the above equation should be positive. Hence,
	
	\bes
	\frac{(\alpha - 2 )\omega^2}{{\alpha ^2 k^2    \left(\frac{x_0}{l}\right)^{2 \alpha }}+1}   > 0
	\ees
	\bes
	\therefore \alpha - 2 > 0 \implies \boxed{\alpha > 2}
	\ees
	The frequency of oscillations $ \omega_0 $ is simply the square root of the coefficient of $ y $,
	\bes
	\boxed{\omega_0 = \sqrt{\frac{\alpha - 2 }{{\alpha ^2 k^2    \left(\frac{x_0}{l}\right)^{2 \alpha }}+1}}\omega} \qq{where} x_0 = \qty(\frac{\omega^2 l^\alpha}{gk\alpha})^\frac{1}{\alpha-2}
	\ees
\end{homeworkProblem}

\begin{homeworkProblem}
	\textbf{Part (a)}\\
	We first write down the Lagrangian,
	\begin{align*}
	L &= T-V\\
	&= \underbrace{\dfrac{M (R_1 - R_2)^2 \dot{\theta_1^2}}{2}}_{\text{Centre of mass revolution}}+ \underbrace{\dfrac{M R_2^2 \dot{\theta_2}^2}{4}}_{\text{rotation}} + Mg (R_1-R_2)\cos \theta_1
	\end{align*}
	
	The equations of motion can now be written as,
	\begin{equation*}
	\boxed{\ddot{\theta_2} = 0 \qq{and} \ddot{\theta_1}= -\dfrac{g(R_1-R_2)}{R_2^2}\sin \theta_1}
	\end{equation*}
	
	\textbf{Part (b)}\\
	As there is rolling without slipping, the following constraint condition would hold,
	
	\begin{equation*}
	R_2 \dot{\theta_2} = (R_1 - R_2)\dot{\theta_1} \implies f = R_2 {\theta_2} - (R_1-R_2) {\theta_1} = 0
	\end{equation*}
	Note that differentiating this also gives,
	\begin{equation*}
	R_2 \ddot{\theta_2} = (R_1 - R_2)\ddot{\theta_1}
	\end{equation*}
	
	\textbf{Part (c)}\\
	Putting in constraint conditions, the equations of motion get modified as,
	\begin{equation*}
	\derivative{t} \qty(\pdv{L}{\dot{q}}) - \pdv{L}{{q}} = \lambda \pdv{f}{{q}}
	\end{equation*}
	which modifies the equations in this problem as,
	\begin{equation*}
	M R_2^2 \ddot{\theta_2} =  \lambda R_2 \qq{and} \dfrac{M (R_1-R_2)^2}{2} \ddot{\theta_1} = -{Mg(R_1-R_2)}\sin \theta_1 - \lambda (R_1-R_2)
	\end{equation*}
	\begin{equation*}
	M R_2 \ddot{\theta_2} =  \lambda  \qq{and} \dfrac{M (R_1-R_2)}{2} \ddot{\theta_1} = -{Mg}\sin \theta_1 - \lambda 
	\end{equation*}
	Taking the ratio of these two and substituting,
	\begin{equation*}
	\dfrac{2R_2}{R_1-R_2} \dfrac{R_1-R_2}{R_2} = \dfrac{\lambda}{-{Mg}\sin \theta_1 - \lambda}
	\end{equation*}
	\begin{equation*}
	\abs{\lambda} = \dfrac{Mg\sin \theta_1}{3} = \qq{Constraint Force}
	\end{equation*}
	
\end{homeworkProblem}

\begin{homeworkProblem}
	\textbf{Part (a)}\\
	A bicycle really has just two degrees of freedom in the simplest sense,
	\begin{itemize}
		\item The angle associated to the pedalling motion 
		\item The angle associated to the motion of the handle 
	\end{itemize}

	\textbf{Part (b)}\\
	The arrangement has $ M-1 $ links and hence $ M-2 $ angles in between. There is also one degree of freedom associated with the rotation of the chain if considered as a rigid body. Hence, there are a total of $ M-1 $ degrees of freedom.
	\\
	
	\textbf{Part (c)}\\
	Due to \textit{homogeneity and isotropy of space} Lagrangian of a free particle should be,
	\begin{itemize}
		\item \textit{Invariant under Rotation} - Hence, $L(v,x)$, where $ v $ and $ x $ are the absolute value of the velocity and position vectors.
		\item \textit{Invariant under Translation} - This means that L cannot depend on $ x $ at all. Hence $ L=L(v) $.
	\end{itemize}

	\textbf{Part (d)}\\
	\begin{equation*}
	\derivative{t} \qty(\pdv{L'}{v'}) = \derivative{t} (2v') = \derivative{t} v' = 0
	\end{equation*}
	which is the equation of motion for free particle in the original frame. Hence, $ L' = v'^2 $ is a possible choice.
	\\
	
	\textbf{Part (e)}\\
	\begin{equation*}
	L'(v+V_0) = L(v) +\dv{F(x,t)}{t} = L'(v) + \eval{V_0 \dv{L'}{v_c}}_{v_c=v} + \ldots
	\end{equation*}
	As $ L(v) = L'(v) $, we have $ \dv{F(x,t)}{t} =  \eval{V_0 \dv{L'}{v_c}}_{v_c=v} $. As $ F $ does not depend on v, it's first derivative can depend only linearly on $ v $. Hence,
	\begin{equation*}
	\dv{L'}{v} \sim v \implies L \sim v^2
	\end{equation*}
\end{homeworkProblem}

\begin{homeworkProblem}
	\textbf{Part (a)}\\
	The Schrodinger equation is given by 
	\bes
	-\frac{\hbar^2}{2m}\pdv[2]{\psi}{x} + V\psi = i \hbar \pdv{\psi}{t} \qq{and} -\frac{\hbar^2}{2m}\pdv[2]{\psi^*}{x} + V\psi^* = -i \hbar \pdv{\psi^*}{t}
	\ees
	We choose $ \psi $ and $ \psi^* $ as our generalized coordinates, and $ (t,x) $ as the dependent coordinates. One should be able to write the equations of motion in a compact form as follows,
	
	\bes
	\partial_\mu \qty(\pdv{L}{(\partial_\mu \psi)}) = \pdv{L}{\psi} \qq{and} \partial_\mu \qty(\pdv{L}{(\partial_\mu \psi^*)}) = \pdv{L}{\psi^*}
	\ees
	
	where the index $ \mu $ goes over $ (t,x) $. Let's analyze every term in the Schrodinger Equation and figure out what corresponding term in the Lagrangian will give rise to that term,
	\begin{itemize}
		\item The first term on the LHS is a double $ x $ derivative and will come from some single derivative term of the form $ L_1 = \psi' \psi'^*$
		\item The second term on the LHS has no derivatives and will come from a term of the form $ L_2 = \psi \psi^*$
		\item The RHS is a single $ t $ derivative and will come from some term of the form $ \dot{\psi} \psi^*$. To make it symmetric, let's consider $ L_3 = -\dot{\psi} \psi^* + \dot{\psi^*} \psi  $
	\end{itemize}
	So our final Lagrangian will be of the form $  L = a_1 L_1 + a_2 L_2 + a_3 L_3 $. Substituting our ansatz, we find our constants, and then the final Lagrangian can be written as,
	\begin{equation*}
	\boxed{L = -\frac{\hbar^2}{2m}\psi' \psi'^* + V \dot{\psi} \psi^* + i \hbar (-\dot{\psi} \psi^* + \dot{\psi^*} \psi)}
	\end{equation*}
	
	\textbf{Part (b)}\\
	Kinetic energy of the wire is zero. The Lagrangian can be written as,
	\bes
	L = - \int ds \ \rho g y = - \int \sqrt{dx^2 + dy^2} \ \rho g y = - \int dx y \sqrt{1 + y'^2} \ \rho g
	\ees
	
	Writing down the equation of motion for the Lagrangian density instead of the Lagrangian, one gets,
	\bes
	\deriv{\frac{yy'}{\sqrt{1+y'^2}}} - \sqrt{1+y'^2} = 0
	\ees
	\bes
	\frac{yy'' + y'^2}{\sqrt{1+y'^2}} - \frac{y y'^2 y''}{1+y'^2} - \sqrt{1+y'^2}=0
	\ees
	Expanding this out and simplifying a bit, one gets,
	\bes
	\frac{yy''}{(1+y'^2)^\frac{3}{2}} - \frac{1}{\sqrt{1+y'^2}}=0 \implies \deriv{\frac{y}{\sqrt{1+y'^2}}} = 0
	\ees
	\bes
	\therefore \frac{y}{\sqrt{1+y'^2}} = \alpha \implies \boxed{y = \alpha \cosh \qty(\frac{x}{\alpha} + \beta)}
	\ees	
	One can get the constants $ \alpha $ and $ \beta $ by imposing the end point conditions for the curve.
	\\
	
	\textbf{Part (c)}\\
	The distance metric on a sphere spherical polar coordinates is given by,
	\bes
	ds^2 = r^2 (d\theta^2 + \sin^2\theta d\phi^2) = d\theta^2 [ r^2 (1 + \sin^2\theta \phi'^2)]
	\ees
	\bes
	\therefore ds = d\theta \sqrt{r^2 (1 + \sin^2\theta \phi'^2)}
	\ees
	From the ansatz $  S = L d\tau $, we can identify that the Lagrangian $ L = \sqrt{r^2 (1 + \sin^2\theta \phi'^2)} $. For finding the \textit{equations of motion}, it is fine and also easier to work with $ L^2 $ rather than $ L $ in this problem. Writing down the equations of motion for $\phi(\theta)$,
	\bes
	\dv{\sin^2\theta \phi'}{\theta} = 0 \implies \phi' = \alpha \csc[2](\theta) \implies \boxed{\phi(\theta) = a \cot\theta + b}
	\ees
	where $ \alpha, a, b $ are constants. If the distance is to be found out between two points $ (\phi_1,\theta_1) $ and $ (\phi_2,\theta_2) $, then,
	
	\bes
	\phi_1 = a \cot\theta_1 + b \qq{and} \phi_2 = a \cot\theta_2 + b
	\ees
	which gives,
	\bes
	a = \frac{\phi_1-\phi_2}{\cot \theta_1 - \cot \theta_2} \qq{and} b= \frac{\phi_1 \tan \theta_1 - \phi_2 \tan \theta_2}{\tan \theta_1 - \tan \theta_2 }
	\ees
\end{homeworkProblem}
\newpage

\begin{homeworkProblem}
	The Lagrangian for a charged particle in a magnetic field is given by,
	\begin{equation*}
	L= \dfrac{1}{2}m (\dot{x}^2 + \dot{y}^2) + q(A_x \dot{x} + A_y \dot{y})
	\end{equation*}
	\textbf{Part (a)}\\
	The magnetic field $ \vb{B} $ is given by $ \vb{B} = \curl{\vb{A}}$. We can easily see that the addition of any term of the form $ \grad{\lambda} $, where $\lambda$ is a scalar, to the vector potential gives the same $ \vb{B} $.
	\\
	
	\textbf{Part (b)}\\
	Taking $ \vb{A} = \vb{A}(x) $, we write the Euler-Lagrange equations for the coordinate $ y $,
	\begin{equation*}
	\derivative{(m \dot{y} + qA_y)}{t} = 0  \implies m \dot{y} + qA_y = constant
	\end{equation*}
	
\end{homeworkProblem}
\end{document}
