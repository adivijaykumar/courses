\documentclass{article}

\usepackage{fancyhdr}
\usepackage{extramarks}
\usepackage{amsmath}
\usepackage{amsthm}
\usepackage{amsfonts}
\usepackage{tikz}
\usepackage{physics}
\usepackage{amssymb}
\usepackage[plain]{algorithm}
\usepackage{algpseudocode}

\usetikzlibrary{automata,positioning}

% Basic Document Settings
%

\topmargin=-0.45in
\evensidemargin=0in
\oddsidemargin=0in
\textwidth=6.5in
\textheight=9.0in
\headsep=0.25in

\linespread{1.1}

\pagestyle{fancy}
\lhead{\hmwkAuthorName}
\chead{\hmwkClass\ : \hmwkTitle}
\rhead{\firstxmark}
\lfoot{\lastxmark}
\cfoot{\thepage}

\renewcommand\headrulewidth{0.4pt}
\renewcommand\footrulewidth{0.4pt}

\setlength\parindent{0pt}

%
% Create Problem Sections
%
\newcommand{\be}{\begin{equation}}
\newcommand{\ee}{\end{equation}}
\newcommand{\bes}{\begin{equation*}}
\newcommand{\ees}{\end{equation*}}
\newcommand{\bea}{\begin{flalign*}}
\newcommand{\eea}{\end{flalign*}}

\newcommand{\enterProblemHeader}[1]{
    \nobreak\extramarks{}{Problem \arabic{#1} continued on next page\ldots}\nobreak{}
    \nobreak\extramarks{Problem \arabic{#1} (continued)}{Problem \arabic{#1} continued on next page\ldots}\nobreak{}
}

\newcommand{\exitProblemHeader}[1]{
    \nobreak\extramarks{Problem \arabic{#1} (continued)}{Problem \arabic{#1} continued on next page\ldots}\nobreak{}
    \stepcounter{#1}
    \nobreak\extramarks{Problem \arabic{#1}}{}\nobreak{}
}

\setcounter{secnumdepth}{0}
\newcounter{partCounter}
\newcounter{homeworkProblemCounter}
\setcounter{homeworkProblemCounter}{1}
\nobreak\extramarks{Problem \arabic{homeworkProblemCounter}}{}\nobreak{}

%
% Homework Problem Environment
%
% This environment takes an optional argument. When given, it will adjust the
% problem counter. This is useful for when the problems given for your
% assignment aren't sequential. See the last 3 problems of this template for an
% example.
%
\newenvironment{homeworkProblem}[1][-1]{
    \ifnum#1>0
        \setcounter{homeworkProblemCounter}{#1}
    \fi
    \section{Problem \arabic{homeworkProblemCounter}}
    \setcounter{partCounter}{1}
    \enterProblemHeader{homeworkProblemCounter}
}{
    \exitProblemHeader{homeworkProblemCounter}
}

%
% Homework Details
%   - Title
%   - Due date
%   - Class
%   - Section/Time
%   - Instructor
%   - Author
%

\newcommand{\hmwkTitle}{Assignment\ \#3}
\newcommand{\hmwkDueDate}{Due 21st September 2018}
\newcommand{\hmwkClass}{Classical Mechanics}
\newcommand{\hmwkClassTime}{}
\newcommand{\hmwkClassInstructor}{Prof.Manas Kulkarni}
\newcommand{\hmwkAuthorName}{\textbf{Aditya Vijaykumar}}

%
% Title Page
%

\title{
    %\vspace{2in}
    \textmd{\textbf{\hmwkClass:\ \hmwkTitle}}\\
    \normalsize\vspace{0.1in}\small{\hmwkDueDate\ }\\
%    \vspace{3in}
}

\author{\hmwkAuthorName}
\date{}

\renewcommand{\part}[1]{\textbf{\large Part \Alph{partCounter}}\stepcounter{partCounter}\\}

%
% Various Helper Commands
%

% Useful for algorithms
\newcommand{\alg}[1]{\textsc{\bfseries \footnotesize #1}}

% For derivatives
\newcommand{\deriv}[1]{\frac{\mathrm{d}}{\mathrm{d}x} (#1)}

% For partial derivatives
\newcommand{\pderiv}[2]{\frac{\partial}{\partial #1} (#2)}

% Integral dx
\newcommand{\dx}{\mathrm{d}x}

% Alias for the Solution section header
\newcommand{\solution}{\textbf{\large Solution}}

% Probability commands: Expectation, Variance, Covariance, Bias
\newcommand{\E}{\mathrm{E}}
\newcommand{\Var}{\mathrm{Var}}
\newcommand{\Cov}{\mathrm{Cov}}
\newcommand{\Bias}{\mathrm{Bias}}

\begin{document}
\maketitle
\begin{homeworkProblem}
	content...
\end{homeworkProblem}
\begin{homeworkProblem}
content...
\end{homeworkProblem}
\begin{homeworkProblem}
The Lagrangian for this system can be written as,
\begin{equation*}
L = \dfrac{1}{2}m_1 \abs{\dot{\vb{r_1}}}^2 + \dfrac{1}{2}m_2 \abs{\dot{\vb{r_2}}}^2 - V(\vb{r_1}-\vb{r_2})
\end{equation*}
We also know, from the question, that
\begin{equation*}
\vb{R} = \dfrac{m_1\vb{r_1} + m_2\vb{r_2}}{m_1 + m_2} \qq{and} \vb{r} = \vb{r_1} - \vb{r_2}
\end{equation*}
This leads us to,
\begin{equation*}
\vb{r_1} = \vb{R} + \dfrac{m_2 \vb{r}}{m_1 + m_2} \qq{and} \vb{r_2} = \vb{R} - \dfrac{m_1 \vb{r}}{m_1 + m_2}
\end{equation*}
\begin{equation*}
\abs{\dot{\vb{r_1}}}^2 = \abs{\dot{\vb{R}}}^2 + \dfrac{m_2^2 \abs{\dot{\vb{r}}}^2}{(m_1 + m_2)^2} + \dfrac{2 m_2}{m_1 + m_2}\dot{\vb{R}} \vdot \dot{\vb{r}} \qq{and} \abs{\dot{\vb{r_2}}}^2 = \abs{\dot{\vb{R}}}^2 + \dfrac{m_1^2 \abs{\dot{\vb{r}}}^2}{(m_1 + m_2)^2} - \dfrac{2 m_1}{m_1 + m_2}\dot{\vb{R}} \vdot \dot{\vb{r}}
\end{equation*}
Substituting into the expression for the Lagrangian, one gets,
\begin{equation*}
L = \dfrac{M}{2}\abs{\dot{\vb{R}}}^2  + \dfrac{\mu}{2}\abs{\dot{\vb{r}}}^2 - V(\vb{r}) \qq{where} M = m_1 + m_2 \qq{,} \mu = \dfrac{m_1 m_2}{M}
\end{equation*}
Each component of $ \dot{\vb{R}} $ will be conserved separately as all of them are cyclic coordinates. Using $ \vb{R} = X \vu{x} + Y \vu{y} + Z \vu{z} $ and $ \dot{\vb{{r}}} = \dot{r}\vu{r} + r \dot{\vu{r}} $,
\begin{equation*}
L = \dfrac{M}{2} (\dot{X}^2 + \dot{Y}^2 )  + \dfrac{\mu}{2} (\dot{r}^2 + r^2 \dot{\theta}^2)- V(r) 
\end{equation*}
\textbf{Part (a)}\\
We can see from the form of the above Lagrangian that,
\begin{equation*}
M\dot{X}= constant \qq{,} M\dot{Y}= constant \qq{,} \mu r^2 \dot{\theta}= constant  
\end{equation*}
Consider the infinitesimal area swept by the vector $\vb{r}$, 
\begin{equation*}
dA = \frac{r^2 d\theta}{2} \implies \dot{A} = r^2 \dfrac{\dot{\theta}}{2} = constant = l
\end{equation*}
Hence, the radius vector sweeps equal areas in equal intervals of time.
\\

\textbf{Part (b)}\\
If $  m_2 \gg m_1 $, $ \vb{R} \approx \vb{r_2}$

\textbf{Part (c)}\\
The Euler-Lagrange equation for the coordinate $ r $ is given by,
\begin{equation*}
\mu \ddot{r} = \mu r \dfrac{4 l^2}{r^4}- \dfrac{k}{r^2} \implies \ddot{r} - \dfrac{4l^2}{r^3} + \dfrac{k}{\mu r^2} = 0
\end{equation*}
Multiplying by $ \dot{r} $ and integrating with time, we get,
\begin{equation*}
\dot{r}^2 + \dfrac{2l^2}{r^2} - \dfrac{k}{\mu r} = constant = E
\end{equation*}
\end{homeworkProblem}
\end{document}
