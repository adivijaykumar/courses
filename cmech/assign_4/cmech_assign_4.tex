\documentclass{article}

\usepackage{fancyhdr}
\usepackage{extramarks}
\usepackage{amsmath}
\usepackage{amsthm}
\usepackage{amsfonts}
\usepackage{tikz}
\usepackage{physics}
\usepackage{amssymb}
\usepackage[plain]{algorithm}
\usepackage{algpseudocode}

\usetikzlibrary{automata,positioning}

% Basic Document Settings
%

\topmargin=-0.45in
\evensidemargin=0in
\oddsidemargin=0in
\textwidth=6.5in
\textheight=9.0in
\headsep=0.25in

\linespread{1.1}

\pagestyle{fancy}
\lhead{\hmwkAuthorName}
\chead{\hmwkClass\ : \hmwkTitle}
\rhead{\firstxmark}
\lfoot{\lastxmark}
\cfoot{\thepage}

\renewcommand\headrulewidth{0.4pt}
\renewcommand\footrulewidth{0.4pt}

\setlength\parindent{0pt}

%
% Create Problem Sections
%
\newcommand{\be}{\begin{equation}}
\newcommand{\ee}{\end{equation}}
\newcommand{\bes}{\begin{equation*}}
\newcommand{\ees}{\end{equation*}}
\newcommand{\bea}{\begin{flalign*}}
\newcommand{\eea}{\end{flalign*}}

\newcommand{\enterProblemHeader}[1]{
    \nobreak\extramarks{}{Problem \arabic{#1} continued on next page\ldots}\nobreak{}
    \nobreak\extramarks{Problem \arabic{#1} (continued)}{Problem \arabic{#1} continued on next page\ldots}\nobreak{}
}

\newcommand{\exitProblemHeader}[1]{
    \nobreak\extramarks{Problem \arabic{#1} (continued)}{Problem \arabic{#1} continued on next page\ldots}\nobreak{}
    \stepcounter{#1}
    \nobreak\extramarks{Problem \arabic{#1}}{}\nobreak{}
}

\setcounter{secnumdepth}{0}
\newcounter{partCounter}
\newcounter{homeworkProblemCounter}
\setcounter{homeworkProblemCounter}{1}
\nobreak\extramarks{Problem \arabic{homeworkProblemCounter}}{}\nobreak{}

%
% Homework Problem Environment
%
% This environment takes an optional argument. When given, it will adjust the
% problem counter. This is useful for when the problems given for your
% assignment aren't sequential. See the last 3 problems of this template for an
% example.
%
\newenvironment{homeworkProblem}[1][-1]{
    \ifnum#1>0
        \setcounter{homeworkProblemCounter}{#1}
    \fi
    \section{Problem \arabic{homeworkProblemCounter}}
    \setcounter{partCounter}{1}
    \enterProblemHeader{homeworkProblemCounter}
}{
    \exitProblemHeader{homeworkProblemCounter}
}

%
% Homework Details
%   - Title
%   - Due date
%   - Class
%   - Section/Time
%   - Instructor
%   - Author
%

\newcommand{\hmwkTitle}{Assignment\ \#3}
\newcommand{\hmwkDueDate}{Due 5th October 2018}
\newcommand{\hmwkClass}{Classical Mechanics}
\newcommand{\hmwkClassTime}{}
\newcommand{\hmwkClassInstructor}{Prof.Manas Kulkarni}
\newcommand{\hmwkAuthorName}{\textbf{Aditya Vijaykumar}}

%
% Title Page
%

\title{
    %\vspace{2in}
    \textmd{\textbf{\hmwkClass:\ \hmwkTitle}}\\
    \normalsize\vspace{0.1in}\small{\hmwkDueDate\ }\\
%    \vspace{3in}
}

\author{\hmwkAuthorName}
\date{}

\renewcommand{\part}[1]{\textbf{\large Part \Alph{partCounter}}\stepcounter{partCounter}\\}

%
% Various Helper Commands
%

% Useful for algorithms
\newcommand{\alg}[1]{\textsc{\bfseries \footnotesize #1}}

% For derivatives
\newcommand{\deriv}[1]{\frac{\mathrm{d}}{\mathrm{d}x} (#1)}

% For partial derivatives
\newcommand{\pderiv}[2]{\frac{\partial}{\partial #1} (#2)}

% Integral dx
\newcommand{\dx}{\mathrm{d}x}

% Alias for the Solution section header
\newcommand{\solution}{\textbf{\large Solution}}

% Probability commands: Expectation, Variance, Covariance, Bias
\newcommand{\E}{\mathrm{E}}
\newcommand{\Var}{\mathrm{Var}}
\newcommand{\Cov}{\mathrm{Cov}}
\newcommand{\Bias}{\mathrm{Bias}}

\begin{document}
\maketitle
\begin{homeworkProblem}
	\textbf{Part (a)}\\
	For $ m=constant $
	\begin{align*}
	T &= \dfrac{m \vb{v} \vdot \vb{v} }{2}\\
	\dv{T}{t} &= m \dot{\vb{v}}  \vdot \vb{v}
	= \vb{F} \vdot \vb{v}
	\end{align*}
	
	If $ m $ varies with time,
	\begin{align*}
	mT &= \dfrac{m^2 \vb{v} \vdot \vb{v} }{2}\\
	\dv{(mT)}{t} &= m^2 \dot{\vb{v}}  \vdot \vb{v} + m \dot{m} \vb{v} \vdot \vb{v}\\
	&= (m {\vb{v}}) \vdot (m \dot{\vb{v}} + \dot{m} {\vb{v}})\\
	\dv{(mT)}{t}&= \vb{p}\vdot \vb{F}
	\end{align*}
	
	\textbf{Part (b)}
	
\end{homeworkProblem}

\begin{homeworkProblem}
	Let $ R $ be the radius of the disc. The generalized coordinates for the motion are the horizontal coordinate $ x $ and angular coordinate $ \theta $. For rolling, we have,
	\begin{equation*}
	R \dot{\theta} = \dot{x} \implies R d \theta - dx = 0
	\end{equation*}
	
	It is straightforward to see that the above equation a specific instance of an equation of the form,
	\begin{equation*}
	\sum_{i=1}^{n} g(x_1,x_2,\ldots, x_n) d x_i = 0
	\end{equation*}
	with $ x_1 = \theta, x_2 = x, g_1 = R, g_2 = -1 $.
	\textcolor{red}{complete the problem}
\end{homeworkProblem}

\begin{homeworkProblem}
	\textbf{Part (a)}\\
	Let $ r, \theta, \phi $ be the generalized coordinates in their usual polar form, and $ l_0 $ be the equilibrium length of the spring. The Lagrangian of the problem $ L $ can be written as,
	\begin{equation*}
	L = \dfrac{m(\dot{r}^2 + r^2\dot{\theta}^2 + r^2 \sin^2 \theta \dot{\phi}^2)}{2} + mgr\cos \theta - \dfrac{k(r - l_0)^2}{2}
	\end{equation*}
	The equations of motion are,
	\begin{align*}
	m\ddot{r} &= mr\dot{\theta}^2 + mr \sin^2 \theta \dot{\phi}^2 + mg \cos \theta  - k(r - l_0)\\
	m r^2 \ddot{\theta} + 2 m r \dot{r} \dot{\theta} &= mr^2 \sin \theta \cos \theta \dot{\phi}^2 - mgr \sin \theta\\
	mr^2 \sin^2 \theta \ddot{\phi} + 2 m r \sin^2 \theta \dot{r} \dot{\phi} + 2 m r^2 \sin \theta \cos \theta \dot{\theta} \dot{\phi} &= 0
	\end{align*}
	Constraining the motion in a plane implies using $ \phi = constant \implies \dot{\phi} = \ddot{\phi} = 0$. \textcolor{red}{Is constraining possible?}. Our equations then reduce to,
	\begin{align*}
	m\ddot{r} &= mr\dot{\theta}^2 + mg \cos \theta  - k(r - l_0)\\
	m r^2 \ddot{\theta} + 2 m r \dot{r} \dot{\theta} &= - mgr \sin \theta
	\end{align*}
	The equilibrium positions can be found by substituting all time derivatives of $ r $ and $ \theta $ as zero. This gives the equilibrium $ r_0 = l_0 + \dfrac{mg}{k} $ and $ \theta_0 = 0r $.
	We need to solve the above for small stretching in $ r $ and small angular displacements $ \theta $. Let's substitute $ r = r_0 + \epsilon x $ and $ \theta = \epsilon \alpha $ in the equations. We get,
	\begin{align*}
	m\epsilon \ddot{x} &= m(r_0 + \epsilon x) \epsilon^2 \dot{\alpha}^2 + mg \left( 1 - \dfrac{\alpha^2}{2} \epsilon^2 + \ldots\right)   - k(\epsilon x + r_0 - l_0)\\
	m ( r_0 + \epsilon x)^2 \epsilon \ddot{\alpha} + 2 m \epsilon^2 ( r_0 + \epsilon x) \dot{x} \dot{\alpha} &= - mg( r_0 + \epsilon x ) \left(\epsilon \alpha + \ldots \right)
	\end{align*}
	Using only $ \order{\epsilon} $ terms,
	\begin{align*}
	\ddot{x} &= -\dfrac{k}{m} x\\
	\ddot{\alpha} &= - \dfrac{g}{r_0}\alpha 
	\end{align*}
	
	\textcolor{red}{Solve numerically}
	
	\textbf{Part (b)}\\
	The Lagrangian is given as,
	\begin{equation*}
	L = e^{\gamma t}\qty(\dfrac{m\dot{q}^2}{2} - \dfrac{k q^2}{2})
	\end{equation*}
	Writing down the equations of motion for hte generalized coordinate $ q $,
	\begin{align*}
	\derivative{t}\qty(e^{\gamma t}m \dot{q}) &= -e^{\gamma t} k {q}\\
	\implies e^{\gamma t} \qty(\gamma m \dot{q} + m \ddot{q}) &= -e^{\gamma t} k {q}\\
	\implies   \ddot{q} + \gamma  \dot{q}+ \dfrac{k}{m}q &= 0 
	\end{align*}
	This is the equation of motion for a damped harmonic oscillator. 
	
	Let's perform the transformation $ s = e^{\gamma t} q \implies \dot{s} = e^{\gamma t} (\gamma q + \dot{q}) = \gamma s + e^{\gamma t} \dot{q} $. Inverting these, we have the following,
	\begin{align*}
	q &= e^{-\gamma t} s\\
	\dot{q} &= e^{-\gamma t}(\dot{s} - \gamma s)
	\end{align*}
	Substituting this back into the expression for $ L $,
	\begin{equation*}
	L = e^{-\gamma t}\qty(\dfrac{m\dot{s}^2}{2} + \dfrac{(m\gamma^2 - k) s^2}{2} - m\gamma s \dot{s})
	\end{equation*}
	Writing the equations of motion for $ s $,
	\begin{align*}
	\derivative{t}\qty(e^{-\gamma t}(m \dot{s} - m\gamma s)) &= -e^{-\gamma t} ((k - m\gamma^2) {s} - m\gamma \dot{s})\\
	m \ddot{s} - m\gamma \dot{s} -\gamma (m \dot{s} - m \gamma s) &=  (k - m \gamma^2) {s} - m\gamma \dot{s}\\
	\ddot{s} -\gamma \dot{s} + \qty(2  \gamma^2-\dfrac{k}{m}) s &=  0 
	\end{align*}
\end{homeworkProblem}

\begin{homeworkProblem}
	\textbf{Part (a)}\\
	As given, we take $ y = at + bt^2 \implies \dot{y} = a + 2bt$. The Lagrangian $ L $ can be written as follows,
	\begin{align*}
	L &= \dfrac{m \dot{y}^2}{2} - mgy = \dfrac{m (a+2bt)^2}{2} - mg (at + bt^2)\\
	&= \dfrac{ma^2 }{2}+ (2mab - mga)t + (2mb^2 - mgb)t^2
	\end{align*}
	Let's evaluate $ \int L dt $,
	\begin{align*}
	\int_0^{t_0} L dt &= \int_0^{t_0} [\dfrac{ma^2}{2} + (2mab - mga)t + (2mb^2 - mgb)t^2] dt\\
	& = \dfrac{ma^2}{2} t_0 + \dfrac{2mab - mga}{2}t_0^2 + \dfrac{2mb^2 - mgb}{3}t_0^3\\
	& = \dfrac{ma^2}{2} \sqrt{\dfrac{2y_0}{g}} + \dfrac{2mab - mga}{2}\dfrac{2y_0}{g} + \dfrac{2mb^2 - mgb}{3}\dfrac{2y_0}{g}\sqrt{\dfrac{2y_0}{g}}\\
	&= 0 \impliedby \left(a=0 \qq{and} b = \dfrac{g}{2}\right)
	\end{align*}
	\textit{Hence Proved.}
	\\
	
	\textbf{Part (b)}\\
	Given, $ L = L(q_i, \dot{q_i}, \ddot{q_i},t) $, and we know that $ S = \int_{t_i}^{t_f} L(q_i, \dot{q_i}, \ddot{q_i},t) dt  $. Variation of the action can be written as,
	
	\begin{align*}
	\delta S &= \int_{t_i}^{t_f} \delta L dt = 0\\
	&= \int_{t_i}^{t_f} \sum_{i} \qty(\pdv{L}{q_i} \delta q_i+ \pdv{L}{\dot{q_i}} \delta \dot{q_i} + \pdv{L}{\ddot{q_i}} \delta \ddot{q_i})dt\\
	&= \int_{t_i}^{t_f} \sum_{i} \qty(\pdv{L}{q_i} \delta q_i+ \qty(\pdv{L}{\dot{q_i}} - \dv{t}\pdv{L}{\ddot{q_i}} ) \delta \dot{q_i} + \dv{t} \qty(\pdv{L}{\ddot{q_i}} \delta \dot{q_i}))dt\\
	&= \int_{t_i}^{t_f} \sum_{i} \qty[ \qty{\pdv{L}{q_i} - \dv{t}\qty(\pdv{L}{\dot{q_i}} - \dv{t}\pdv{L}{\ddot{q_i}} )} \delta q_i + \dv{t}\qty(\qty(\pdv{L}{\dot{q_i}} - \dv{t}\pdv{L}{\ddot{q_i}} ) \delta {q_i}) + \dv{t} \qty(\pdv{L}{\ddot{q_i}} \delta \dot{q_i})]dt\\
	\end{align*}
	As the variation of $ q_i $ and $ \dot{q_i} $ at the endpoints is zero, the total derivative terms vanish. Accounting for the fact that all $ q_i $'s are independent, one can write the equation of motion as,
	\begin{equation*}
	\boxed{\pdv{L}{q_i} - \dv{t}\pdv{L}{\dot{q_i}} + \dv[2]{t}\pdv{L}{\ddot{q_i}} = 0}
	\end{equation*}
	Taking $ L = -\dfrac{m}{2}q \ddot{q} - \dfrac{k}{2}q^2$, we can write,
	\begin{equation*}
	-kq + \dfrac{m}{2}\ddot{q} = 0 \implies \ddot{q} - \dfrac{2k}{m}q  = 0
	\end{equation*}
	\textcolor{red}{Where have I seen this?}
\end{homeworkProblem}

\begin{homeworkProblem}
	
\end{homeworkProblem}

\begin{homeworkProblem}
	The Lagrangian for the problem is,
	\begin{equation*}
	L = \dfrac{m(\dot{r}^2 + r^2 \dot{\theta}^2 )}{2} - V(r)
	\end{equation*}
	The equations of motion for this Lagrangian, with $ V(r) = -V_0 e^{-\lambda^2 r^2} $, are,
	\begin{align*}
	m r^2 \dot{\theta} &= constant = L_0 \qq{and}\\
	m \ddot{r} &= m r \dot{\theta}^2 + (2 \lambda^2 r) V(r)\\
	\implies m\ddot{r}&= \dfrac{L_0^2}{m r^3} + (2 \lambda^2 r) V(r)
	\end{align*}
	For stable circular orbit, $ \dot{r} = \ddot{r} = 0 $. Let $ r_0 $ be radius of stable circular orbit. We can see that $ r_0 $ will be given by the root of the equation,
	\begin{equation*}
	\dfrac{L_0^2}{m r_0^3} - 2 \lambda^2 r_0 V_0 e^{-\lambda^2 r_0^2} = 0\\
	\implies {L_0^2} = 2 \lambda^2 m r_0^4 V_0 e^{-\lambda^2 r_0^2}
	\end{equation*}
	
	As the factor $ e^{-\lambda^2 r_0^2} \le 1 $ for all choices of $ r $. For roots to exist, it should be the case that,
	\begin{align*}
	{L_0^2} &\le 2 \lambda^2 m r_0^4 V_0 \\
	\implies L_0 &\le \sqrt{2mV_0} \lambda r_0
	\end{align*}
	So, $ L_0 $ cannot exceed $ \sqrt{2mV_0} \lambda r_0 $.
	\textcolor{red}{Check this}
\end{homeworkProblem}

\begin{homeworkProblem}
	The radius of the circle $ r $ and the angle covered around the circle $ \theta $ are the generalized coordinates. The Lagrangian $ L $ can be written as,
	\begin{equation*}
	L =  \dfrac{m \dot{r}^2}{2} + \dfrac{m \dot{\theta}^2 r^2}{2} - mg r \cot \alpha
	\end{equation*}
	The equations of motion are,
	\begin{align*}
	r^2 \dot{\theta} &= constant = L_0\\
	\ddot{r} &= r\dot{\theta}^2 - g\cot \alpha \implies \ddot{r} = \dfrac{L_0^2}{r^3} - g\cot \alpha
	\end{align*}
	
	\textbf{Part (b)}\\
	If $ r = r_0 $, $ \ddot{r} =0 $ and,
	\begin{equation*}
	L_0^2 = r_0^4 \omega^2 =   g r_0^3 \cot \alpha \implies \boxed{\omega = \sqrt{\frac{g \cot \alpha}{r_0}}} \implies L_0 = r_0^3 g \cot \alpha
	\end{equation*}
	
	\textbf{Part (c)}\\
	
	We consider perturbations along the surface of the cone ie $ l = r_0 \csc \alpha + \epsilon x $. This in turn corresponds to a radial perturbation of the form $ r = r_0 + \epsilon x \sin \alpha $, $ \epsilon \ll 1 $. Substituting this into the equation of motion for $ r $,
	\begin{align*}
	\epsilon \ddot{x} \sin \alpha &= \dfrac{L_0^2}{(r_0 + \epsilon x \sin \alpha)^3} - g \cot \alpha\\
	&=\dfrac{L_0^2}{r_0^3}\qty(1 - \dfrac{3 \epsilon x \sin \alpha}{r_0} + \ldots) - g \cot \alpha\\
	&=\dfrac{L_0^2}{r_0^3}\qty(- \dfrac{3 \epsilon x \sin \alpha}{r_0} + \ldots) \\
	\end{align*}
	Choosing only the term first order in $ \epsilon $,
	\begin{equation*}
	\ddot{x} = -\dfrac{3 g \cot \alpha}{r_0} x \implies \boxed{\Omega = \sqrt{\dfrac{3 g \cot \alpha	}{r_0}}}
	\end{equation*}
	\textcolor{red}{Check if this is really correct}
\end{homeworkProblem}

\begin{homeworkProblem}
	Let $ \theta_1 $ and $ \theta_2 $ be the angles that the sticks make with the vertical. Each stick is of length $ 2r $. $ \theta_1 $ (lower stick) and $ \theta_2 $ (upper stick) are the generalized coordinates. The position coordinates of the lower and upper masses are,
	\begin{equation*}
	(x_1,y_1) = (r \sin \theta_1, r \cos \theta_1) \qq{and} (x_2,y_2) = (2r \sin \theta_1 + r \sin \theta_2, 2r \cos \theta_1 + r \cos \theta_2)
	\end{equation*}
	\begin{equation*}
	\implies v_1^2 = r^2 \dot{\theta}_1^2 \qq{and} v_2^2 = 4r^2 \dot{\theta}_1^2 + r^2 \dot{\theta}_2^2 + 4r^2 \cos (\theta_1 - \theta_2) \dot{\theta}_1 \dot{\theta}_2
	\end{equation*}
	
	The Lagrangian can be written as,
	\begin{equation*}
	L = \dfrac{ mr^2 }{2}\dot{\theta}_1^2 + \dfrac{ m(4r^2 \dot{\theta}_1^2 + r^2 \dot{\theta}_2^2 + 4r^2 \cos (\theta_1 - \theta_2) \dot{\theta}_1 \dot{\theta}_2) }{2} - mgr \cos \theta_1 - mg (2r\cos \theta_1 + r \cos \theta_2) 
	\end{equation*}
	The equations of motion are,
	\begin{align*}
	mr^2 \ddot{\theta}_1 + 4mr^2 \ddot{\theta}_1 + 2mr^2 \cos (\theta_1 - \theta_2) \ddot{\theta}_2 +  2mr^2 \cos (\theta_1 - \theta_2) \dot{\theta}_2 (\dot{\theta}_2 - \dot{\theta}_1)  =\\ - 2mr^2 \sin (\theta_1 - \theta_2)\dot{\theta}_1 \dot{\theta}_2 + mgr\sin \theta_1 + 2mgr \sin \theta_1
	\end{align*}
	\begin{equation*}
	mr^2 \ddot{\theta}_2 + 2mr^2 \cos (\theta_1 - \theta_2) \ddot{\theta}_1 +  2mr^2 \cos (\theta_1 - \theta_2) \dot{\theta}_1 (\dot{\theta}_2 - \dot{\theta}_1) = 2mr^2 \sin (\theta_1 - \theta_2)\dot{\theta}_1 \dot{\theta}_2 + mgr \sin \theta_2
	\end{equation*}
	In the above equations of motion, we substitute, $ \theta_1 = 0, \theta_2 = \epsilon \ll 1, \dot{\theta}_1 = \dot{\theta}_2 = 0 $,
	\begin{align*}
	5mr^2 \ddot{\theta}_1 + 2mr^2 \ddot{\theta}_2  &= 0 \qq{and}\\
	mr^2 \ddot{\theta}_2 + 2mr^2 \ddot{\theta}_1 &= mgr  \epsilon
	\end{align*}
	\begin{equation*}
	\boxed{\ddot{\theta}_1 = -\dfrac{2g \epsilon}{r} \qq{and} \ddot{\theta}_2 = \dfrac{5g \epsilon}{r}}
	\end{equation*}
	
	\textcolor{red}{check if correct}
\end{homeworkProblem}

\end{document}
