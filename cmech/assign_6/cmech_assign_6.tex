\documentclass{article}

\usepackage{fancyhdr}
\usepackage{extramarks}
\usepackage{amsmath}
\usepackage{amsthm}
\usepackage{amsfonts}
\usepackage{tikz}
\usepackage{physics}
\usepackage{amssymb}
\usepackage[plain]{algorithm}
\usepackage{algpseudocode}

\usetikzlibrary{automata,positioning}

% Basic Document Settings
%

\topmargin=-0.45in
\evensidemargin=0in
\oddsidemargin=0in
\textwidth=6.5in
\textheight=9.0in
\headsep=0.25in

\linespread{1.1}

\pagestyle{fancy}
\lhead{\hmwkAuthorName}
\chead{\hmwkClass\ : \hmwkTitle}
\rhead{\firstxmark}
\lfoot{\lastxmark}
\cfoot{\thepage}

\renewcommand\headrulewidth{0.4pt}
\renewcommand\footrulewidth{0.4pt}

\setlength\parindent{0pt}

%
% Create Problem Sections
%
\newcommand{\be}{\begin{equation}}
\newcommand{\ee}{\end{equation}}
\newcommand{\bes}{\begin{equation*}}
\newcommand{\ees}{\end{equation*}}
\newcommand{\bea}{\begin{flalign*}}
\newcommand{\eea}{\end{flalign*}}

\newcommand{\enterProblemHeader}[1]{
    \nobreak\extramarks{}{Problem \arabic{#1} continued on next page\ldots}\nobreak{}
    \nobreak\extramarks{Problem \arabic{#1} (continued)}{Problem \arabic{#1} continued on next page\ldots}\nobreak{}
}

\newcommand{\exitProblemHeader}[1]{
    \nobreak\extramarks{Problem \arabic{#1} (continued)}{Problem \arabic{#1} continued on next page\ldots}\nobreak{}
    \stepcounter{#1}
    \nobreak\extramarks{Problem \arabic{#1}}{}\nobreak{}
}

\setcounter{secnumdepth}{0}
\newcounter{partCounter}
\newcounter{homeworkProblemCounter}
\setcounter{homeworkProblemCounter}{1}
\nobreak\extramarks{Problem \arabic{homeworkProblemCounter}}{}\nobreak{}

%
% Homework Problem Environment
%
% This environment takes an optional argument. When given, it will adjust the
% problem counter. This is useful for when the problems given for your
% assignment aren't sequential. See the last 3 problems of this template for an
% example.
%
\newenvironment{homeworkProblem}[1][-1]{
    \ifnum#1>0
        \setcounter{homeworkProblemCounter}{#1}
    \fi
    \section{Problem \arabic{homeworkProblemCounter}}
    \setcounter{partCounter}{1}
    \enterProblemHeader{homeworkProblemCounter}
}{
    \exitProblemHeader{homeworkProblemCounter}
}

%
% Homework Details
%   - Title
%   - Due date
%   - Class
%   - Section/Time
%   - Instructor
%   - Author
%

\newcommand{\hmwkTitle}{Assignment\ \#6}
\newcommand{\hmwkDueDate}{Due 16th November 2018}
\newcommand{\hmwkClass}{Classical Mechanics}
\newcommand{\hmwkClassTime}{}
\newcommand{\hmwkClassInstructor}{Prof.Manas Kulkarni}
\newcommand{\hmwkAuthorName}{\textbf{Aditya Vijaykumar}}

%
% Title Page
%

\title{
    %\vspace{2in}
    \textmd{\textbf{\hmwkClass:\ \hmwkTitle}}\\
    \normalsize\vspace{0.1in}\small{\hmwkDueDate\ }\\
%    \vspace{3in}
}

\author{\hmwkAuthorName}
\date{}

\renewcommand{\part}[1]{\textbf{\large Part \Alph{partCounter}}\stepcounter{partCounter}\\}

%
% Various Helper Commands
%

% Useful for algorithms
\newcommand{\alg}[1]{\textsc{\bfseries \footnotesize #1}}

% For derivatives
\newcommand{\deriv}[1]{\frac{\mathrm{d}}{\mathrm{d}x} (#1)}

% For partial derivatives
\newcommand{\pderiv}[2]{\frac{\partial}{\partial #1} (#2)}

% Integral dx
\newcommand{\dx}{\mathrm{d}x}

% Alias for the Solution section header
\newcommand{\solution}{\textbf{\large Solution}}

% Probability commands: Expectation, Variance, Covariance, Bias
\newcommand{\E}{\mathrm{E}}
\newcommand{\Var}{\mathrm{Var}}
\newcommand{\Cov}{\mathrm{Cov}}
\newcommand{\Bias}{\mathrm{Bias}}

\begin{document}
\maketitle

\begin{homeworkProblem}[1]
	Liouville Theorem states that in a Hamiltonian system, the phase space density is constant in time.
	
	Let our system consist of $ N $ points $ (q_k, p_k) $ in a $ 2N $ dimensional phase space.
\end{homeworkProblem}






\begin{homeworkProblem}[2]
	Transformations of coordinates $ (q,p,t) \rightarrow (Q,P,t)$ which preserves the form of Hamilton's equations are called canonical transformations. So, by definition,
	\begin{equation*}
	\dot{p} = \pdv{H}{q} \qq{,} \dot{q} = - \pdv{H}{p} \qq{and} \dot{P} = \pdv{K}{Q} \qq{,} \dot{Q} = -\pdv{K}{P}
	\end{equation*}
	The definition also implies that,
	\begin{align*}
	\delta(p \dot{q} - H) = 0 &\qq{and} \delta(P \dot{Q} - K) = 0\\
	\lambda(p \dot{q} - H) &= P \dot{Q} - K + \dv{F}{t}
	\end{align*}
	We deal with the $ \lambda = 1 $ case. The $ \dv{F}{t} $ term comes from the fact that Lagrangians are not unique and we can always add a total time derivative term without changing the equations of motion. If the above condition is satisfied, the transformation $ (q,p,t) \rightarrow (Q,P,t)$ is guaranteed to be canonical, and the function $ F $ is called a generating function. We deal with four classes of generating functions case-by-case,
	\begin{itemize}
		\item $ F = F_1 (q,Q,t) $,
		\begin{equation*}
		p \dot{q} - H = P \dot{Q} - K + \dv{F_1}{t} = P \dot{Q} - K + \pdv{F_1}{q}\dot{q} + \pdv{F_1}{Q}\dot{Q} + \pdv{F_1}{t}
		\end{equation*}
		As $ q $ and $ Q $ are independent, the coefficients should vanishh independently, such that $ K = H + \pdv{F_1}{t} $. This implies,
		\begin{equation*}
		\pdv{F_1}{q} = p \qq{and} \pdv{F_1}{Q} = -P
		\end{equation*}
		
		\item $ F = F_2 (q,P,t) - QP$,
		\begin{equation*}
		p \dot{q} - H = P \dot{Q} - K + \dv{F_2}{t} - \dv{(QP)}{t} = P \dot{Q} - K + \pdv{F_2}{q}\dot{q} + \pdv{F_2}{P}\dot{P} + \pdv{F_2}{t} - P \dot{Q} - Q \dot{P}
		\end{equation*}
		\begin{equation*}
		\implies \pdv{F_2}{q} = p \qq{and} \pdv{F_2}{P} = Q
		\end{equation*}
		\item $ F = F_3 (p,Q,t) + qp$,
		\begin{equation*}
		p \dot{q} - H = P \dot{Q} - K + \dv{F_3}{t} + \dv{(qp)}{t} = P \dot{Q} - K + \pdv{F_3}{Q}\dot{Q} + \pdv{F_3}{p}\dot{p} + \pdv{F_3}{t} + p \dot{q} + q \dot{p}
		\end{equation*}
		\begin{equation*}
		\implies \pdv{F_3}{Q} = -P \qq{and} \pdv{F_2}{p} = -q
		\end{equation*}
		\item $ F = F_4 (p,P,t) + qp - QP$,
		\begin{equation*}
		p \dot{q} - H = P \dot{Q} - K + \dv{F_4}{t} + \dv{(qp - QP)}{t} = P \dot{Q} - K + \pdv{F_4}{P}\dot{P} + \pdv{F_4}{p}\dot{p} + \pdv{F_4}{t} + p \dot{q} + q \dot{p} - P \dot{Q} - Q \dot{P}
		\end{equation*}
		\begin{equation*}
		\implies \pdv{F_4}{P} = Q \qq{and} \pdv{F_4}{p} = -q
		\end{equation*}
	\end{itemize}
\end{homeworkProblem}

\begin{homeworkProblem}[3]
	We are given the Hamiltonian and generating function,
	\begin{equation*}
	H = \dfrac{p^2}{2} + \dfrac{\omega^2 x^2 }{2} + \alpha x^3 + \beta x p^2 \qq{and} \phi = xP + ax^2 P + bP^3
	\end{equation*}
	$ \phi = \phi (x,P) $. For $ \phi $ to be a canonical transformation,
	\begin{align*}
	\pdv{\phi}{x} = p &\qq{and} \pdv{\phi}{P} = Q\\
	\implies P + 2 a x P = p &\qq{and} x + ax^2 + 3b P^2 = Q \\
	\implies P + 2 a x P = p &\qq{and} \dfrac{-1 \pm \sqrt{1 + 4a(Q - 3bP^2)}}{2a}= x \\
	\end{align*}
	We know that,
	\begin{align*}
	p \dot{x} - H  &= P \dot{Q} - K + \dv{\phi}{t}\\
	p \dot{x} -  \dfrac{p^2}{2} + \dfrac{\omega^2 x^2 }{2} + \alpha x^3 + \beta x p^2 &= P \dot{Q} - K + (P + 2axP)\dot{x} + (x + ax^2 + 3bP^2)\dot{P}
	\end{align*}
\end{homeworkProblem}












\begin{homeworkProblem}[4]
	We first note that,
	\begin{equation*}
	y = x^2 \implies \dot{y} = 2 x \dot{x}
	\end{equation*}
	and write down the Lagrangian and Hamiltonian of the system,
	\begin{align*}
	L &= \dfrac{m \dot{x}^2}{2} + \dfrac{m \dot{y}^2}{2} - mgy \\
	L &= \dfrac{m \dot{x}^2}{2} + {2m  x^2 \dot{x}^2 } - mgx^2\\
	\implies p = m \dot{x} + 4 mx^2 \dot{x} &\implies  \dot{x} = \dfrac{p}{m(1 + 4x^2)}
	\end{align*}
	Thus, we can write the Hamiltonian as,
	\begin{align*}
	H(x,p) &= \dfrac{p^2}{m(1 + 4x^2)} - \dfrac{m}{2}(1 + 4x^2)\dfrac{p^2}{m^2 (1 + 4x^2)^2} + mg x^2\\
	H(x,p) &= \dfrac{p^2}{2m(1 + 4x^2)} + mgx^2
	\end{align*} 
	The Hamilton-Jacobi equation is given by,
	\begin{equation*}
	\dfrac{1}{2m(1 + 4x^2)}\qty(\pdv{S}{x})^2 + mgx^2 + \pdv{S}{t} = 0 
	\end{equation*}
	Substituting $ S = W(x) - Et $, we get,
	\begin{align*}
	\dfrac{1}{2m(1 + 4x^2)}\qty(\dv{W}{x})^2 + &mgx^2 - E = 0
	\implies \dv{W}{x} = \sqrt{2m(E - mgx^2)(1 + 4x^2)}\\
	\implies S &= \int dx \sqrt{2m(E - mgx^2)(1 + 4x^2)} - Et
	\end{align*}
	We know that $ \pdv{S}{E} = \alpha t + \beta $ for constants $ \alpha $ and $ \beta $. Hence the equation of motion is,
	\begin{equation*}
	\sqrt{\dfrac{m(1+4x^2)}{2(E - mgx^2)}} - E = \alpha t + \beta
	\end{equation*}
	\textbf{Part (b)}\\
	We first note that,
	\begin{equation*}
	z = \dfrac{\xi^2 - \eta^2}{2} \qq{,} \rho = \eta \xi \qq{,} \psi = \phi \implies \dot{z} = \xi \dot{\xi} - \eta \dot{\eta} \qq{,} \dot{\rho} = \eta \dot{\xi} + \xi \dot{\eta} \qq{,} \dot{\phi} = \dot{\psi}
	\end{equation*}
	We first write down the Lagrangian and canonical momenta,
	\begin{align*}
	L &= \dfrac{m (\dot{\rho}^2 + \rho^2 \dot{\phi}^2 + \dot{z}^2)}{2} - \dfrac{k}{\sqrt{\rho^2 + z^2} } + Fz\\
	&= \dfrac{m ( \eta^2 \dot{\xi}^2 + \xi^2 \dot{\eta}^2+ 2 \eta \xi \dot{\eta} \dot{\xi} + \eta^2 \xi^2 \dot{\psi}^2 + \xi^2 \dot{\xi}^2 - 2 \xi \dot{\xi} \eta \dot{\eta} + \eta^2 \dot{\eta}^2 )}{2} - \dfrac{k}{\sqrt{\qty(\dfrac{\xi^2 - \eta^2}{2})^2 + \eta^2 \xi^2}} + F\dfrac{\xi^2 - \eta^2}{2}\\
	L&= m\dfrac{(\eta^2 + \xi^2)( \dot{\xi}^2 + \dot{\eta}^2)+ \eta^2 \xi^2 \dot{\psi}^2}{2} - \dfrac{2k}{\eta^2 + \xi^2} + F\dfrac{\xi^2 - \eta^2}{2}\\
	\implies &p_\xi = m (\eta^2 + \xi^2) \dot{\xi} \qq{,} p_\eta = m (\eta^2 + \xi^2) \dot{\eta} \qq{,} p_\psi = m \eta^2 \xi^2 \dot{\psi}\\
	\implies H &= \dfrac{p_\xi^2 + p_\eta^2}{2 m(\eta^2 + \xi^2)} + \dfrac{p_\psi^2}{2m \eta^2 \xi^2} +\dfrac{2k}{\eta^2 + \xi^2}  - F\dfrac{\xi^2 - \eta^2}{2}
	\end{align*}
	
	Let's apply the transformations given in the problem
	We can now write down the Hamilton-Jacobi equation as,
	\begin{equation*}
	\pdv{S}{t} +  \dfrac{1}{2 m(\eta^2 + \xi^2)} \qty[\qty(\pdv{S}{\xi})^2 + \qty(\pdv{S}{\eta})^2 ]+ \dfrac{1}{2m \eta^2 \xi^2} \qty(\pdv{S}{\psi})^2 + \dfrac{2k}{\eta^2 + \xi^2}  - F\dfrac{\xi^2 - \eta^2}{2} = 0
	\end{equation*}
	Multiply the equation by $ 2m(\eta^2 + \xi^2) $,
	\begin{equation*}
	2m(\eta^2 + \xi^2)\pdv{S}{t} + \qty(\pdv{S}{\xi})^2 + \qty(\pdv{S}{\eta})^2 + \qty(\dfrac{1}{ \eta^2} + \dfrac{1}{\xi^2}) \qty(\pdv{S}{\psi})^2 + 4mk  - Fm ({\xi^4 - \eta^4} )= 0
	\end{equation*}
\end{homeworkProblem}


\end{document}
