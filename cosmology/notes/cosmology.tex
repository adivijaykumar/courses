\documentclass[a4paper,11pt]{article}

\usepackage{physics}
\usepackage{amsmath}
\usepackage{amssymb}
\usepackage{amsmath}
\usepackage{amsthm, mathtools}
%\usepackage{hyperref}
\usepackage{color}
\usepackage{jheppub}
\usepackage[T1]{fontenc} % if needed

% My Commands
\newcommand{\be}{\begin{equation}}
\newcommand{\ee}{\end{equation}}
\newcommand{\bes}{\begin{equation*}}
\newcommand{\ees}{\end{equation*}}
\newcommand{\bea}{\begin{flalign*}}
\newcommand{\eet}{\end{flalign*}}
%\linespread{1.0}
%\setlength{\parindent}{0em}
%\setlength{\parskip}{0.8em}

\title{\textbf{Notes on Cosmology}}
\author{Aditya Vijaykumar}
\affiliation{International Centre for Theoretical Sciences, Bengaluru, India.}
\emailAdd{aditya.vijaykumar@icts.res.in}
\abstract{One hopes one will finally be accustomed to cosmology with this exploration, largely following the book by Prof. Mukhanov and lecture notes by Prof. Baumann. One might also, if one feels like, try to survey modern day approaches to inflation inspired by string theory. }

\begin{document}
\maketitle

\section{Kinematics, Dynamics and Geometry}
For most of the 20th century, cosmologists agreed that the universe was homogenous and isotropic ie. there is no single \textit{preferred}  point or direction in the universe. This is called the \textit{Cosmological Principle} in popular terms. On the face of it, this is a very powerful assumption - it allows us to make predictions about the universe sitting at where we are on the earth. It remained an intelligent guess until the age of data took over cosmology.

Data now shows that the universe is indeed homogenous and isotropic, but only on large enough scales (greater than 100 Mpc). Hence, if we coarse grain the universe on a scale smaller than 100 Mpc, we will start seeing inhomogeneities like galaxies, galaxy clusters etc. This was not entirely unexpected. So we decided to live with it.

But then there is a twist. Theory suggests that the universe continues to be homogenous and isotropic for scales bigger than that of the observable universe (3000 Mpc and beyond), but inhomogeneities start to creep in at distance scales much larger than 3000 Mpc. Answering questions on these scales in indeed very tough, not only because one can't seem to pose the question in a mathematically precise fashion, but also because one can't imagine verifying these predictions empirically.

Nevertheless, as travellers in the conquest for truth, it is imperative for us to learn what our forefathers thought of the cosmos, and hopefully avoid making the same mistakes again.

\subsection{Hubble Law}

Hubble postulated that, in a homogenous, isotropic expanding universe, the relative velocities of observers obey the Hubble Law.$$\vec{v}_{B,A}=H(t)\vec{r}_{AB}$$ Is the Hubble law in agreement with the homogenous and isotropic assumption? Let's check,
$$\vec{v}_{B,A}=H(t)\vec{r}_{AB} \text{ ; } \vec{v}_{C,B}=H(t)\vec{r}_{BC}$$
$$\text{Hence, } \vec{v}_{C,A} = H(t)(\vec{r}_{AB} + \vec{r}_{BC}) = H(t)\vec{r}_{AC}$$ which is what we should have expected. In fact, one can show explicitly that Hubble Expansion Law is \textit{the only} law that is comaptible with homogeneity and isotropic expansion.

One could go ahead and write the Law as a differential equation as follows
$$\dot{\vec{r}}_{AB}=H(t)\vec{r}_{AB}$$ Integrating this,
$$ r_{AB} = r_0 \exp\int H(t)dt = a(t)  r_0 $$
where $\exp\int H(t)dt = a(t)$ is called the \textit{scale factor}. $r_0$ is the separation between A and B at some given instant of time (taken to be $t=0$ without loss of generality), and $r_{AB}$ denotes the distance between them after time $t$ has elapsed.

It is also straightforward to write the Hubble constant in terms of the scale factor and its time derivative,
$$H =\frac{\dot{a}}{a}$$
Obviously, the Hubble Law would only hold on cosmological length scales. The relative motion of the sun and the earth, for example, is not governed by this law, but by the inhomogeneities in the gravitational field.

\subsection{Dynamics of Newtonian Dust}
Consider a universe filled with \textit{dust} particles. What dust really means is matter which exerts negligible pressure compared to its energy density $\epsilon$. Lets also assume that gravity is a weak force and that the particles are not very far away from each other so as to avoid exceeding the speed of light. Now consider a sphere expanding about the origin, with its radius being given by $R(t) = a(t) \chi_{c}$. As the total mass is M, the energy density $\epsilon$ can be written as
$$\epsilon(t)= \frac{3M}{4 \pi R(t)^3} = \frac{3M}{4 \pi R_0^3} \left(\frac{a_0}{a(t)}\right)^3 = \epsilon_0 \left(\frac{a_0}{a(t)}\right)^3  $$
Taking the time derivative,
$$\dot{\epsilon}(t) = \epsilon_0 \left(\frac{a_0}{a(t)}\right)^3 \left(\frac{-3\dot{a}}{a}\right) = -3H \epsilon(t)$$

We can also write down Newton's second law for the sphere as follows,
$$\ddot{R} = -\frac{GM}{R^2} = -G \frac{4 \pi }{3} \epsilon(t) R $$
Dividing by $R_0$ we get,
$$\therefore  \ \ddot{a} = -  \frac{4 \pi G}{3} \epsilon a$$

We have essentially got the evolution equations for the energy density and the scale factor in a Newtonian dust setting. Substituting for $\epsilon$ in the equation for $a(t)$,
$$\ddot{a} = -  \frac{4 \pi G \epsilon_0}{3}  \left(\frac{a_0^3}{a^2}\right) $$
Multiplying by $\dot{a}$ and integrating we get,
$$\frac{\dot{a}^2}{2} - \frac{4 \pi G \epsilon_0}{3}  \left(\frac{a_0^3}{a}\right) = \text{constant} = E  $$

The form of this energy-like equation for the scale factor is similar to the energy equation for a rocket launched into orbit. If one provides the rocket too little kinetic energy, it will fall back to earth. Similarly, the fate of this Newtonian dust universe, ie. whether it keeps expanding or collapses unto itself depends on the sign of the energy $E$. Writing the equation in terms of the Hubble constant,
$$H^2 - \frac{2E}{a^2} = \frac{8 \pi G}{3}\epsilon$$
Setting $E=0$ in the above equation, one can derive the critical density,
$$\epsilon^{cr} = \frac{3H^2}{8\pi G}$$ Substituting this back in the other equation,
$$\epsilon^{cr}(1 - \Omega(t))= \frac{3E}{4 \pi G} \frac{\epsilon}{a^2}$$
where $\Omega(t) = \frac{\epsilon^{cr}}{\epsilon}$ is called the \textit{cosmological parameter}. $Omega(t)$ varies with time, but because the sign of $E$ is fixed, the quantity $(1-\Omega(t))$ does not change sign. Hence, we just need to measure the current value of the cosmological parameter to determine the sign of $E$ in the universe.

As is conveyed in popular talks, the sign of $E$ is linked to the spatial geometry of the Universe in general relativity. This spatial curvature has the sign opposite to that of $E$.

 Let's imagine we probe the value of $\Omega_0$. The possible scenarios are :-
\begin{itemize}
	\item $\epsilon_0 > \epsilon^{cr}$ ie $\Omega > 1$ and $E < 0$. The spatial curvature will be positive (closed universe). The scale factor reaches some maximum value and the universe recollapses.
	\item $\epsilon_0 < \epsilon^{cr}$ ie $\Omega < 1$ and $E > 0$. The spatial curvature will be negative (open universe). The universe exapnds hyperbolically.
	\item $\epsilon_0 = \epsilon^{cr}$ ie $\Omega = 1$ and $E = 0$. The spatial curvature will be zero (flat universe).
\end{itemize}

In all three cases, extrapolating the universe backwards, we face an initial singularity. Note that we are still working with dust-dominated universe (and also Newtonian theory, duh); the analyses we have done are only toy-model-like. Nonetheless, we hope we have gained some physical intuition from the same.

Lets just play with our toy-model for one last time. In a flat universe,
$$\frac{\dot{a}^2a}{2} = \frac{4}{9} \left(\derivative{a^{3/2}}{t} \right)^2 = \frac{4 \pi G \epsilon_0 a_0^3}{3} = \text{constant} $$
Hence, $a \propto t^{2/3}$, $H = \frac{2}{3t}$. We can then write the age of universe expression as,
$$t_0 = \frac{2}{3H_0} \text{ and also } \epsilon(t) = \frac{1}{6 \pi G t^2}$$

\subsection{Einstein it, duh}
Newtonian theory is \textit{obviously} wrong. So its better to quickly stop being stupid and start worshipping Einstein.

To start off, lets waste some time thinking what the geometry of a homogenous, isotropic space would look like. We can think of the universe as an evolving homogenous and istropic hypersurface.

Lets first consider surfaces of this kind in 2 dimensions. The well known surfaces are the plane and the 2-sphere. We write the embedding of a two sphere in 3D Euclidean space as follows,
$$x^2 + y^2 + z^2 = a^2$$
$a$ being the radius of the sphere. Taking the derivative,
$$2x dx + 2y dy +2z dz = 0$$
$$\therefore dz = -\frac{xdx + ydy}{z}$$
Substituting this in the distance metric,
$$dl^2 = dx^2 + dy^2 + \frac{(xdx + ydy)^2}{a^2 - x^2 - y^2}$$

Hence, we can represent distance between any two points on the hypersurface by just two independent coordinates $x$ and $y$. But lets, for the sake of easing up our calculations, substitute $x = r' \cos\phi$ and $y = r' \sin \phi$. Differentiting $x^2 + y^2 = r'^2$,
$$xdx + ydy = r' dr'$$
$$\therefore dl^2 = dr'^2 + r'^2d\phi^2 + \frac{r'^2 dr'^2}{a^2 - r'^2}$$
$$\therefore dl^2 = \frac{dr'^2 }{1 - \frac{r'^2}{a^2}} + r'^2d\phi^2 $$
We note that in the $a^2 \rightarrow \infty$ limit, this is just the flat space metric. On can also in principle have $a^2 < 0$; we would not be able to visualize it as an embedding in 3D, but that's not the fault of the subject of mathematics. This space with $a^2 < 0$ has negative curvature and is called the \textit{Lobachevski space}.

At this point, to simplify things further, lets rescale our coordinates as $r = r'/\sqrt{\abs{a^2}}$. Then,
$$dl^2 = \abs{a}^2\left\{\frac{dr^2}{1 - k{r^2}} + r^2d\phi^2 \right\}$$

where $k=1$ for sphere, $k=-1$ for pseudo-sphere, and $k=0$ for plane. The scale doesn't have any physical meaning as such, so we redefine our coordinates to absorb it. The generalization of the above analysis for three dimensions is as follows,
$$dl^2 = a^2\left\{\frac{dr^2}{1 - k{r^2}} + r^2(d\theta^2 + \sin^2\theta) \right\}$$
We can also work with a coordinate $\chi$ defined as,
$$d\chi^2 = \frac{dr^2}{1-kr^2}$$ One can explicitly verify by integrating that $\chi = \text{arcsinh } r$ for $k = -1$, $\chi = r$ for $k=0$ and $\chi = \arcsin r$ for $k=1$. One can now rewrite the metric as follows,
$$dl^2 = a^2(d\chi^2 + \sinh[2](\chi)d \Omega^2) \text{ ; } k = -1$$
$$dl^2 = a^2(d\chi^2 + \chi^2 d \Omega^2) \text{ ; } k = 0$$
$$dl^2 = a^2(d\chi^2 + \sin[2](\chi)d \Omega^2) \text{ ; } k = +1$$

\textbf{Consider the 2-sphere} in the positive curvature case at a given $\chi$. The surface area will be given by $S_{2d}(\chi) = 4 \pi a^2 \sin^2 \chi$. We note that as the radius $\chi$ increases, the surface area first increases and reaches a maximum, and thereafter starts decreasing, vanishing at $\chi=\pi$.

The differential volume element $dV$ is given by,
$$dV = S_{2d} \ dl_r = 4 \pi a^3 \sin[2](\chi) d \chi$$ Therefore the volume of a sphere with radius $\chi_0$ is,
$$V(\chi_0) = 2 \pi a^3 (\chi_0 - \frac{1}{2} \sin 2\chi_0)$$

At low $\chi_0$, $V = 4 \pi (a \chi_0)^3/3$ and $\chi_0=\pi$ gives $V = 2 \pi^2 a^3$

\textbf{Consider the 2-pseudosphere} in the negative curvature case at 	given $\chi$
$$dl^2 = a^2 \sinh[2](\chi)d \Omega^2 \text{ ; } S_{2d} =4 \pi a^2 \sinh^2 \chi $$

This increases exponentially for as $\chi$ increases.

In the most suitable coordinate system where the symmetries of our current construction are manifest \textcolor{red}{(Why?)}, we can write the metric of spacetime as 
$$ds^2 = dt^2 - a^2(t) \left(\frac{dr^2}{1 - k{r^2}} + r^2(d\theta^2 + \sin^2\theta)\right) = g_{\alpha \beta} dx^\alpha dx^\beta$$ On large scales, matter can be approximated to be a perfect fluid \textcolor{red}{(Why?)},
$$T^\alpha_\beta = (\epsilon+p)u^\alpha u_\beta - p \delta^\alpha_\beta$$
where $\epsilon$ is energy density and $p= p (\epsilon)$ depends on the properties of the matter. Another interesting case that one might encounter in cosmology is that of a classical scalar field. The energy momentum tensor for a scalar field is given as,
$$T^\alpha_\beta = \phi^{,\alpha} \phi_{,\beta} - \left( \frac{1}{2}\phi^{,\gamma} \phi_{,\gamma} - V(\phi) \right)\delta^\alpha_\beta$$
\textbf{\textcolor{red}{To be explored further}}.

\subsubsection{Friedmann Equations}

How are the Newtonian evolution equations modified when one accounts for GR? In pricniple, one has to the metric and the energy momentum tensor into the Einstein's equations and derive the expressions. Expressions thus derived are called \textit{Friedmann equations}. \textcolor{red}{(do the derivation on mathematica)}. Mukhanov wishes to enlighten us in a different way.

First law of thermodynamics tells us $dE = -pdV$. Since $V \propto a^3$, and $E = \epsilon V$, one can write,
$$d \epsilon = -3 (p+\epsilon) d\ln a \text{ ie. } \dot{\epsilon} = -3H(\epsilon + p)$$

The other Newtonian evolution equations get modified as follows, 
$$\ddot{a} = -\frac{4 \pi}{3}(\epsilon + 3p)a \text{ and } H^2 + \frac{k}{a^2} = \frac{8 \pi G}{3}\epsilon$$ 
The $k$ is indeed the curvature factor that was introduced a while ago.

In Newtonian cosmology, the Universe is always flat and the scale factor has no real interpretationin terms of geometry.. It also is able to describe matter which is presureless and is expanding with velocities much less as compared to the speed of lgiht. GR, in contrast, provides a self-consistent theory of the cosmos, and it can describe relativistic matter with any equation of state. The matter content indeed determines the geometry of the universe, and the scale factor has a geometrical interpretation as the radius of curvature.

\subsection{Conformal Time}
Consider defining the conformal time $\eta$ as follows,
$$\eta = \int \frac{dt}{a(t)} \text{ such that } a(t)d\eta = dt$$
The second of the above equations can be rewritten as,
$$ a'^2 + k a^2 = \frac{8\pi G}{3}\epsilon a^4$$  and the first can be written as,
$$a a'' -a'^2 = -\frac{4 \pi}{3}(\epsilon + 3p)a^4$$
Substituting, one gets,
$$a a'' + \frac{4 \pi}{3}(\epsilon + 3p)a^4 + k a^2 = \frac{8\pi G}{3}\epsilon a^4$$
$$a'' + k a = \frac{4 \pi}{3}(\epsilon - 3p)a^3 $$
The last equation is just the trace of Einstein equations \textbf{\textcolor{red}{Verify!}}. In case of radiation, $\epsilon = 3p$, and hence, $a'' + ka = 0$ which gives us the solutions,
$$a = \underbrace{a_m \sinh(\eta)}_{k=-1} \text{ , }\underbrace{a_m \eta}_{k=0} \text{ , } \underbrace{a_m \sin(\eta)}_{k=+1} $$ where the corresponding physical time $t$ is,
$$t = \underbrace{\cosh \eta - 1}_{k=-1} \text{ , }\underbrace{\frac{\eta^2}{2}}_{k=0} \text{ , } \underbrace{1 - \cos \eta}_{k=+1} $$

If we consider the flat radiation dominated universe specifically, we see that $\eta \propto \sqrt{t}$ and $a \propto \sqrt{t}$. Hence $H = \frac{1}{2t}$. Substituting this into the Friedmann equations, we see that $\epsilon \propto 1/t^2$ \textit{ie.} $\epsilon \propto 1/a^4$.

\textbf{\textcolor{red}{Do the conformal time exercises}}

\section{Milne Universe}
Consider open universe in the limit of vanishing energy density. The Friedmann equation boils down to $$ H^2  = \frac{1}{a^2} \implies \dot{a}^2 = 1 \implies a=t$$
The metric now becomes,
\bes
ds^2 = dt^2 -t^2 (d \chi^2 + \sinh^2 \chi d\Omega^2)
\ees
One can easily check that this is just the Minkowski spacetime written in expanding coordinates by making the substitutions $ \tau = t \cosh\chi $ and $  r = t \sinh\chi $
\begin{flalign*}
	ds^2 &= d\tau^2 - (dr^2 + r^2 d\Omega^2)\\
	&=  dt^2 -t^2 (d \chi^2 + \sinh^2 \chi d\Omega^2)
\end{flalign*}
The velocity $ v $ is just given by,
\be
v = \dfrac{r}{\tau} = \tanh \chi < 1
\ee

\end{document}