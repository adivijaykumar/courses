\documentclass{article}

\usepackage{fancyhdr}
\usepackage{extramarks}
\usepackage{amsmath}
\usepackage{amsthm}
\usepackage{amssymb}
\usepackage{amsfonts}
\usepackage{tikz}
\usepackage{physics}
\usepackage[plain]{algorithm}
\usepackage{algpseudocode}

\usetikzlibrary{automata,positioning}

\newcommand{\be}{\begin{equation}}
\newcommand{\ee}{\end{equation}}
\newcommand{\bes}{\begin{equation*}}
\newcommand{\ees}{\end{equation*}}
\newcommand{\bea}{\begin{flalign*}}
\newcommand{\eea}{\end{flalign*}}
%
% Basic Document Settings
%

\topmargin=-0.45in
\evensidemargin=0in
\oddsidemargin=0in
\textwidth=6.5in
\textheight=9.0in
\headsep=0.25in

\linespread{1.1}

\pagestyle{fancy}
\lhead{\hmwkAuthorName}
\chead{\hmwkClass\ : \hmwkTitle}
\rhead{\firstxmark}
\lfoot{\lastxmark}
\cfoot{\thepage}

\renewcommand\headrulewidth{0.4pt}
\renewcommand\footrulewidth{0.4pt}

\setlength\parindent{0pt}

%
% Create Problem Sections
%

\newcommand{\enterProblemHeader}[1]{
	\nobreak\extramarks{}{Problem \arabic{#1} continued on next page\ldots}\nobreak{}
	\nobreak\extramarks{Problem \arabic{#1} (continued)}{Problem \arabic{#1} continued on next page\ldots}\nobreak{}
}

\newcommand{\exitProblemHeader}[1]{
	\nobreak\extramarks{Problem \arabic{#1} (continued)}{Problem \arabic{#1} continued on next page\ldots}\nobreak{}
	\stepcounter{#1}
	\nobreak\extramarks{Problem \arabic{#1}}{}\nobreak{}
}

\setcounter{secnumdepth}{0}
\newcounter{partCounter}
\newcounter{homeworkProblemCounter}
\setcounter{homeworkProblemCounter}{1}
\nobreak\extramarks{Problem \arabic{homeworkProblemCounter}}{}\nobreak{}

%
% Homework Problem Environment
%
% This environment takes an optional argument. When given, it will adjust the
% problem counter. This is useful for when the problems given for your
% assignment aren't sequential. See the last 3 problems of this template for an
% example.
%
\newenvironment{homeworkProblem}[1][-1]{
	\ifnum#1>0
	\setcounter{homeworkProblemCounter}{#1}
	\fi
	\section{Problem \arabic{homeworkProblemCounter}}
	\setcounter{partCounter}{1}
	\enterProblemHeader{homeworkProblemCounter}
}{
	\exitProblemHeader{homeworkProblemCounter}
}

%
% Homework Details
%   - Title
%   - Due date
%   - Class
%   - Section/Time
%   - Instructor
%   - Author
%

\newcommand{\hmwkTitle}{Chapter \#1}
\newcommand{\hmwkDueDate}{\today}
\newcommand{\hmwkClass}{Mukhanov Cosmology}
\newcommand{\hmwkClassTime}{}
\newcommand{\hmwkClassInstructor}{}
\newcommand{\hmwkAuthorName}{\textbf{Aditya Vijaykumar}}

%
% Title Page
%

\title{
	%\vspace{2in}
	\textmd{\textbf{\hmwkClass:\ \hmwkTitle}}\\
	\normalsize\vspace{0.1in}\small{\hmwkDueDate\ }\\
	%    \vspace{3in}
}

\author{\hmwkAuthorName}
\date{}

\renewcommand{\part}[1]{\textbf{\large Part \Alph{partCounter}}\stepcounter{partCounter}\\}

%
% Various Helper Commands
%

% Useful for algorithms
\newcommand{\alg}[1]{\textsc{\bfseries \footnotesize #1}}

% For derivatives
\newcommand{\deriv}[1]{\frac{\mathrm{d}}{\mathrm{d}x} (#1)}

% For partial derivatives
\newcommand{\pderiv}[2]{\frac{\partial}{\partial #1} (#2)}

% Integral dx
\newcommand{\dx}{\mathrm{d}x}

% Alias for the Solution section header
\newcommand{\solution}{\textbf{\large Solution}}

% Probability commands: Expectation, Variance, Covariance, Bias
\newcommand{\E}{\mathrm{E}}
\newcommand{\Var}{\mathrm{Var}}
\newcommand{\Cov}{\mathrm{Cov}}
\newcommand{\Bias}{\mathrm{Bias}}

\begin{document}
	
	\maketitle
	
	%\pagebreak
	
	\begin{homeworkProblem}
		\textbf{Solution}
		\\
		
		The following condition should be satisfied for a law to be the same for all observers, and hence, physical.
		$$f(\va{r}_{CA} - \va{r}_{BA},t) = f(\va{r}_{CA},t) - f(\va{r}_{BA},t)$$
		One can recall that this is an example of a linear transformation/operation. A general transformation linear in $x$ can be written as $f(\va{r},t) = H(t)\va{r}$, which is exactly the form of the Hubble law.
	\end{homeworkProblem}

	\begin{homeworkProblem}
		Let $v_H$ be the Hubble velocity, and $v_P$ be the peculiar velocity of the galaxy. For peculiar velocity to be neglected, we assume that $v_H$ should be at least one order of magnitude larger than $v_P$ ie $v_H \approx$ 1000 km/s
		\begin{flalign*}
			v_H = Hr = 75r \text{ ; }
			r = 1000/75 \approx \boxed{13.33 \text{MPc}}
		\end{flalign*}
	\end{homeworkProblem}
	
	\begin{homeworkProblem}
		We know that, $$\frac{\dot{a}^2}{2} = \frac{4 \pi G \epsilon_0}{3}  \left(\frac{a_0^3}{a}\right) + \text{constant}$$
		Hence when $a \rightarrow \infty$, $\dot{a} \rightarrow \infty$ and $H = \frac{\dot{a}}{a} \rightarrow \infty$
		
		We also know that,$$\epsilon^{cr}(1 - \Omega(t))= \frac{3E}{4 \pi G} \frac{\epsilon}{a^2}$$
		As the RHS is a finite value, $E \rightarrow \infty$ when $a \rightarrow \infty$.
	\end{homeworkProblem}

	\begin{homeworkProblem}
		
		TO BE DONE
		\begin{flalign*}
			\epsilon^{cr}(1 - \Omega(t)) &= \frac{3E}{4 \pi G} \frac{\epsilon}{a^2}\\
			\Omega(t) &= 1 - \frac{3E}{4 \pi G} \frac{\epsilon}{\epsilon^{cr}}\frac{1}{a^2}
		\end{flalign*}
		$$PE = -\frac{GM^2}{R} \text{ ; } KE = \frac{M}{den}$$
	\end{homeworkProblem}

	\begin{homeworkProblem}
		Recall that Newton's second law applied to an expanding spherical ball of dust gives us
		\begin{flalign*}
			 \ \ddot{a} &= -  \frac{4 \pi G}{3} \epsilon a \\
			 \therefore \ q &= -\frac{\ddot{a}}{aH^2} =\frac{4 \pi G}{3} \frac{\epsilon}{H^2}
		\end{flalign*}
		We know that $\epsilon^{cr} = \frac{3H^2}{8 \pi G}$ and $\Omega(t) = \frac{\epsilon}{\epsilon^{cr}}$. Hence,
		$$\boxed{q = \frac{1}{2} \Omega(t)}$$
		
		For a spatially flat universe, $\Omega(t) =1$. Hence $\boxed{q=\frac{1}{2}}$.
	\end{homeworkProblem}

	\begin{homeworkProblem}
		Restoring units of $c$, the expression for energy density $\epsilon(t)$ is
		$$\epsilon(t) = \frac{c^2}{6 \pi G t^2} = \frac{7.162}{t^2} \times 10^{14} \text{ J/m$^3$}$$
		
		Substituting values, one gets $$\epsilon(t=10^{-43}s) = 7.162 \times 10^{100} \text{ J/m$^3$}$$
		$$\epsilon(t=1s) = 7.162 \times 10^{14} \text{ J/m$^3$}$$
		$$\epsilon(t=1 yr) = 0.72  \text{ J/m$^3$}$$
	\end{homeworkProblem}

	\begin{homeworkProblem}
		$$H^2 - \frac{2E}{a(t)^2} = \frac{8 \pi G a_0^3}{3a^3}\epsilon_0$$
		$$ \dot{a}^2 - 2E = \frac{8 \pi G a_0^3}{3a}\epsilon_0  $$
		For $ t\rightarrow \infty $, $ \dot{a} \sim 1 $ and hence $ a \sim t $.
	\end{homeworkProblem}
		
	\begin{homeworkProblem}
		
	\end{homeworkProblem}
\end{document}
