\documentclass{article}

\usepackage{fancyhdr}
\usepackage{extramarks}
\usepackage{amsmath}
\usepackage{amsthm}
\usepackage{amssymb}
\usepackage{amsfonts}
\usepackage{tikz}
\usepackage{physics}
\usepackage[plain]{algorithm}
\usepackage{algpseudocode}
\usepackage{graphicx,wrapfig,lipsum}
\usetikzlibrary{automata,positioning}

%
% Basic Document Settings
%

\topmargin=-0.45in
\evensidemargin=0in
\oddsidemargin=0in
\textwidth=6.5in
\textheight=9.0in
\headsep=0.25in

\linespread{1.1}

\pagestyle{fancy}
\lhead{\hmwkAuthorName}
\chead{\hmwkClass\ : \hmwkTitle}
\rhead{\firstxmark}
\lfoot{\lastxmark}
\cfoot{\thepage}

\renewcommand\headrulewidth{0.4pt}
\renewcommand\footrulewidth{0.4pt}

\setlength\parindent{0pt}

%
% Create Problem Sections
%
\newcommand{\be}{\begin{equation}}
\newcommand{\ee}{\end{equation}}
\newcommand{\bes}{\begin{equation*}}
\newcommand{\ees}{\end{equation*}}
\newcommand{\bea}{\begin{flalign*}}
\newcommand{\eea}{\end{flalign*}}


\newcommand{\enterProblemHeader}[1]{
    \nobreak\extramarks{}{Problem \arabic{#1} continued on next page\ldots}\nobreak{}
    \nobreak\extramarks{Problem \arabic{#1} (continued)}{Problem \arabic{#1} continued on next page\ldots}\nobreak{}
}

\newcommand{\exitProblemHeader}[1]{
    \nobreak\extramarks{Problem \arabic{#1} (continued)}{Problem \arabic{#1} continued on next page\ldots}\nobreak{}
    \stepcounter{#1}
    \nobreak\extramarks{Problem \arabic{#1}}{}\nobreak{}
}

\setcounter{secnumdepth}{0}
\newcounter{partCounter}
\newcounter{homeworkProblemCounter}
\setcounter{homeworkProblemCounter}{1}
\nobreak\extramarks{Problem \arabic{homeworkProblemCounter}}{}\nobreak{}

%
% Homework Problem Environment
%
% This environment takes an optional argument. When given, it will adjust the
% problem counter. This is useful for when the problems given for your
% assignment aren't sequential. See the last 3 problems of this template for an
% example.
%
\newenvironment{homeworkProblem}[1][-1]{
    \ifnum#1>0
        \setcounter{homeworkProblemCounter}{#1}
    \fi
    \section{Problem \arabic{homeworkProblemCounter}}
    \setcounter{partCounter}{1}
    \enterProblemHeader{homeworkProblemCounter}
}{
    \exitProblemHeader{homeworkProblemCounter}
}

%
% Homework Details
%   - Title
%   - Due date
%   - Class
%   - Section/Time
%   - Instructor
%   - Author
%

\newcommand{\hmwkTitle}{Assignment\ \#3}
\newcommand{\hmwkDueDate}{Due on 23rd October, 2018}
\newcommand{\hmwkClass}{Fluid Mechanics}
\newcommand{\hmwkClassTime}{}
\newcommand{\hmwkClassInstructor}{}
\newcommand{\hmwkAuthorName}{\textbf{Aditya Vijaykumar}}

%
% Title Page
%

\title{
    %\vspace{2in}
    \textmd{\textbf{\hmwkClass:\ \hmwkTitle}}\\
    \normalsize\vspace{0.1in}\small{\hmwkDueDate\ }\\
%    \vspace{3in}
}

\author{\hmwkAuthorName}
\date{}

\renewcommand{\part}[1]{\textbf{\large Part \Alph{partCounter}}\stepcounter{partCounter}\\}

%
% Various Helper Commands
%

% Useful for algorithms
\newcommand{\alg}[1]{\textsc{\bfseries \footnotesize #1}}

% For derivatives
\newcommand{\deriv}[1]{\frac{\mathrm{d}}{\mathrm{d}x} (#1)}

% For partial derivatives
\newcommand{\pderiv}[2]{\frac{\partial}{\partial #1} (#2)}

% Integral dx
\newcommand{\dx}{\mathrm{d}x}

% Alias for the Solution section header
\newcommand{\solution}{\textbf{\large Solution}}

% Probability commands: Expectation, Variance, Covariance, Bias
\newcommand{\E}{\mathrm{E}}
\newcommand{\Var}{\mathrm{Var}}
\newcommand{\Cov}{\mathrm{Cov}}
\newcommand{\Bias}{\mathrm{Bias}}

\begin{document}

\maketitle
\textbf{Acknowledgements} -


%\pagebreak
\begin{homeworkProblem}[1]
	\textbf{Part (a)}\\
	We first write down the unsteady state Bernoulli equation,
	\begin{align*}
	\pdv{\phi_1}{t} + \dfrac{P_{atm}}{\rho} + \dfrac{v_1^2}{2} + gz &= \pdv{\phi_2}{t} + \dfrac{P_{atm}}{\rho} + \dfrac{v_2^2}{2}\\
	\pdv{v_1}{t} + v_1 \pdv{v_1}{z} + g &= \pdv{v_2}{t} + v_2 \pdv{v_2}{z}
	\end{align*}
	where we have taken a partial derivative with respect to $ y $ in going from the first to the second step.
	The continuity equation tells us,
	\begin{equation*}
	A_1 v_1 = A v_2 \implies A_1 \pdv{v_1}{t} = A \pdv{v_2}{t}
	\end{equation*}
	If we assume, $ A_1 \gg A $, 
	From the geometry of the cone, we have,
	\begin{align*}
	r_1 &= r_2 +z \tan \alpha\\
	\pi r_1^2 &= \pi r_2^2 + \pi z^2 \tan^2 \alpha + 2\pi r_2 z \tan \alpha\\
	A_1 &= A + \pi z^2 \tan^2 \alpha + 2 \sqrt{A \pi } z \tan \alpha
	\end{align*}
	From this and the continuity equation, we get,
	\begin{align*}
	v_2 &= \qty(1 + \dfrac{\pi z^2 \tan^2 \alpha}{A} + 2 \sqrt{\dfrac{A}{\pi}} z \tan \alpha) v_1\\
	\pdv{v_2}{t} &= \qty(1 + \dfrac{\pi z^2 \tan^2 \alpha}{A} + 2 \sqrt{\dfrac{A}{\pi}} z \tan \alpha) \pdv{v_1}{t} + \qty(\dfrac{2 \pi z \tan^2 \alpha}{A} + 2 \sqrt{\dfrac{A}{\pi}} \tan \alpha) v_1^2\\
	&\approx \qty(\dfrac{2 \pi z \tan^2 \alpha}{A} + 2 \sqrt{\dfrac{A}{\pi}} \tan \alpha) v_1^2
	\end{align*}
	
	\textbf{Part (b)}\\
	From steady state Bernoulli equation,
	\begin{equation*}
	\dfrac{v_1^2}{2} + gz =  \dfrac{u_1^2}{2} \qq{and} \dfrac{v_2^2}{2} + gz =  \dfrac{u_2^2}{2}
	\end{equation*}
	From continuity equation,
	\begin{equation*}
	A_1 v_1 = A u_1 \qq{and} A_2 v_2 = A u_2
	\end{equation*}
	
\end{homeworkProblem}










\begin{homeworkProblem}[2]
	We first write the Bernoulli equation for between the point where water leaves the tap $ (z_1=0) $ and a point distance $ h $ below $ (z_2=-h) $,
	\begin{equation*}
	\dfrac{P_0 }{\rho} + \dfrac{v_1^2}{2} = \dfrac{P_0 }{\rho} + \dfrac{v_2^2}{2} - gh  \implies \dfrac{v_2^2}{v_1^2} = 1 + \dfrac{2gh}{v_1^2}
	\end{equation*}
	The continuity equation gives,
	\begin{equation*}
	\pi r_1^2 v_1 = \pi r_2^2 v_2 \implies \dfrac{v_2}{v_1} = \dfrac{r_1^2}{r_2^2}
	\end{equation*}
	Using the above two equations, we get,
	\begin{equation*}
	\dfrac{r_1^4}{r_2^4} = 1 + \dfrac{2gh}{v_1^2} \implies \boxed{\dfrac{R_0^4}{r^4} = 1 + \dfrac{2gH}{v_0^2}}
	\end{equation*}
	where $ r $ is the cross-sectional radius at height $ H $ below the tap, and $ R_0 $ and $ v_0 $ and the cross-sectional radius and velocity of the water the moment it leaves the tap.
	
\end{homeworkProblem}







\begin{homeworkProblem}[3]
	We work in cylindrical coordinates. The assumption of laminar flow $ \implies u_r = u_\phi = 0 $. The assumption of axisymmetry $ \implies u_z = u_z(r, z) $ The continuity condition $ \div{\va{u}} = 0$ gives,
	\begin{equation*}
	\pdv{u_z}{z} = 0 \implies u_z = u_z(r)
	\end{equation*}
	We now proceed and write the Navier-Stokes equation in cylindrical coordinates component-wise,
	\begin{align*}
	0 &= -\dfrac{1}{\rho}\pdv{P}{r}\\
	0 &= -\dfrac{1}{\rho r}\pdv{P}{\phi}\\
	0 &= -\dfrac{1}{\rho}\pdv{P}{z} + \nu \dfrac{1}{r} \pdv{r} \qty(r \pdv{u_z}{r})
	\end{align*}
	We can see from the first two equations that $ P = P(z) $. In the third equation, since the first term on the RHS depends only on $ z $ and the second term depends only on $ r $, we say that each of the terms should be constants. We get,
	\begin{align*}
	 \dfrac{1}{r} \dv{r} \qty(r \dv{u_z}{r}) &= \dfrac{1}{\mu}\dv{P}{z} = constant\\
	 \dv{r} \qty(r \dv{u_z}{r}) &= \dfrac{r}{\mu}\dv{P}{z}\\
	 \implies r \dv{u_z}{r} &= \dfrac{r^2}{2\mu}\dv{P}{z} + A\\
	 \implies \dv{u_z}{r} &= \dfrac{r}{2\mu}\dv{P}{z} + \dfrac{A}{r}\\
	 \implies {u_z} &= \dfrac{r^2}{4\mu}\dv{P}{z} + A \ln r + B\\
	\end{align*}
	We need the flow to be well-defined at $ r=0 $. As it stands, for non-zero $ A $, the flow will not be well-defined for $ r=0 $, which is undesirable. Hence, $ A=0 $.
	
	If $ R $ is the radius of the pipe, and the pipe is not moving, we get $ u_z(R) = 0 $, which means,
	\begin{equation*}
	0 = \dfrac{R^2}{4\mu}\dv{P}{z} + B \implies B = - \dfrac{R^2}{4\mu}\dv{P}{z}
	\end{equation*}
	So the final answer is,
	\begin{equation*}
	u_z = \dfrac{1}{4 \mu} \dv{P}{z} (r^2 - R^2) 
	\end{equation*}
\end{homeworkProblem}













\begin{homeworkProblem}[4]
	
\end{homeworkProblem}


\end{document}
