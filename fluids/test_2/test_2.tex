\documentclass{article}

\usepackage{fancyhdr}
\usepackage{extramarks}
\usepackage{amsmath}
\usepackage{amsthm}
\usepackage{amssymb}
\usepackage{amsfonts}
\usepackage{tikz}
\usepackage{physics}
\usepackage[plain]{algorithm}
\usepackage{algpseudocode}
\usepackage{graphicx,wrapfig,lipsum}
\usetikzlibrary{automata,positioning}

%
% Basic Document Settings
%

\topmargin=-0.45in
\evensidemargin=0in
\oddsidemargin=0in
\textwidth=6.5in
\textheight=9.0in
\headsep=0.25in

\linespread{1.1}

\pagestyle{fancy}
\lhead{\hmwkAuthorName}
\chead{\hmwkClass\ : \hmwkTitle}
\rhead{\firstxmark}
\lfoot{\lastxmark}
\cfoot{\thepage}

\renewcommand\headrulewidth{0.4pt}
\renewcommand\footrulewidth{0.4pt}

\setlength\parindent{0pt}

%
% Create Problem Sections
%
\newcommand{\be}{\begin{equation}}
\newcommand{\ee}{\end{equation}}
\newcommand{\bes}{\begin{equation*}}
\newcommand{\ees}{\end{equation*}}
\newcommand{\bea}{\begin{flalign*}}
\newcommand{\eea}{\end{flalign*}}


\newcommand{\enterProblemHeader}[1]{
    \nobreak\extramarks{}{Problem \arabic{#1} continued on next page\ldots}\nobreak{}
    \nobreak\extramarks{Problem \arabic{#1} (continued)}{Problem \arabic{#1} continued on next page\ldots}\nobreak{}
}

\newcommand{\exitProblemHeader}[1]{
    \nobreak\extramarks{Problem \arabic{#1} (continued)}{Problem \arabic{#1} continued on next page\ldots}\nobreak{}
    \stepcounter{#1}
    \nobreak\extramarks{Problem \arabic{#1}}{}\nobreak{}
}

\setcounter{secnumdepth}{0}
\newcounter{partCounter}
\newcounter{homeworkProblemCounter}
\setcounter{homeworkProblemCounter}{1}
\nobreak\extramarks{Problem \arabic{homeworkProblemCounter}}{}\nobreak{}

%
% Homework Problem Environment
%
% This environment takes an optional argument. When given, it will adjust the
% problem counter. This is useful for when the problems given for your
% assignment aren't sequential. See the last 3 problems of this template for an
% example.
%
\newenvironment{homeworkProblem}[1][-1]{
    \ifnum#1>0
        \setcounter{homeworkProblemCounter}{#1}
    \fi
    \section{Problem \arabic{homeworkProblemCounter}}
    \setcounter{partCounter}{1}
    \enterProblemHeader{homeworkProblemCounter}
}{
    \exitProblemHeader{homeworkProblemCounter}
}

%
% Homework Details
%   - Title
%   - Due date
%   - Class
%   - Section/Time
%   - Instructor
%   - Author
%

\newcommand{\hmwkTitle}{Test\ \#2}
\newcommand{\hmwkDueDate}{Due on 2nd November, 2018}
\newcommand{\hmwkClass}{Fluid Mechanics}
\newcommand{\hmwkClassTime}{}
\newcommand{\hmwkClassInstructor}{}
\newcommand{\hmwkAuthorName}{\textbf{Aditya Vijaykumar}}

%
% Title Page
%

\title{
    %\vspace{2in}
    \textmd{\textbf{\hmwkClass:\ \hmwkTitle}}\\
    \normalsize\vspace{0.1in}\small{\hmwkDueDate\ }\\
%    \vspace{3in}
}

\author{\hmwkAuthorName}
\date{}

\renewcommand{\part}[1]{\textbf{\large Part \Alph{partCounter}}\stepcounter{partCounter}\\}

%
% Various Helper Commands
%

% Useful for algorithms
\newcommand{\alg}[1]{\textsc{\bfseries \footnotesize #1}}

% For derivatives
\newcommand{\deriv}[1]{\frac{\mathrm{d}}{\mathrm{d}x} (#1)}

% For partial derivatives
\newcommand{\pderiv}[2]{\frac{\partial}{\partial #1} (#2)}

% Integral dx
\newcommand{\dx}{\mathrm{d}x}

% Alias for the Solution section header
\newcommand{\solution}{\textbf{\large Solution}}

% Probability commands: Expectation, Variance, Covariance, Bias
\newcommand{\E}{\mathrm{E}}
\newcommand{\Var}{\mathrm{Var}}
\newcommand{\Cov}{\mathrm{Cov}}
\newcommand{\Bias}{\mathrm{Bias}}

\begin{document}

\maketitle


\begin{homeworkProblem}[1]
	content...
\end{homeworkProblem}



















\begin{homeworkProblem}[2]
	We assume homogenous, isotropic turbulence. These assumptions tell us that the second-order structure function $ S_2 $ calculated between two spatial points should depend \textit{only} on the distance between the two points. More mathematically,
	\begin{equation*}
	\expval{\qty(\va{u}(\va{r} + \va{l}) - \va{u}(\va{r}))^2} = S_2 \qty(\abs{\va{l}}) = S_2(l)
	\end{equation*}
	
	Further, we define the \textit{energy spectrum} $ E(k) $ such that $ E(k)dk $ gives the mean kinetic energy contained within $ k $ and $ k + dk $. It follows from the definition that,
	\begin{equation}
	\int_{0}^{\infty} E(k) dk = \dfrac{1}{2}\expval{u^2}
	\label{ekkin}
	\end{equation}
	The \textit{Wiener-Khinchin} theorem tells us that energy spectrum is the Fourier Transform of the spatial autocorrelation function,
	\begin{equation}
	E(k) \sim \int_{0}^{\infty} e^{i k l} S_2(l) \implies S_2(l) \sim \int_{0}^{\infty} e^{i k l} E(k) dk
	\label{wkt}
	\end{equation}
	
	Now we try to guess the scaling form of $ E(k) $ as $E(k) \sim k^{-n}$. Substituting this ansatz into (\ref{ekkin}),
	\begin{align*}
	\int_{0}^{\infty} E(k) dk &\sim \int_{0}^{\infty} k^{-n} dk \\
	&\sim \eval{\dfrac{k^{1-n}}{1-n}}_{\infty}
	\end{align*}
	As the RHS of (\ref{ekkin}) is finite, it follows from above that $ n > 1 $. We now substitute our ansatz into (\ref{wkt}),
	
\end{homeworkProblem}



















\begin{homeworkProblem}[3]
	
\end{homeworkProblem}













\end{document}
