\documentclass[a4paper,11pt]{article}

\usepackage{physics}
\usepackage{amsmath}
\usepackage{amssymb}
\usepackage{amsmath}
\usepackage{amsthm, mathtools}
%\usepackage{hyperref}
\usepackage{color}
\usepackage{jheppub}
\usepackage[T1]{fontenc} % if needed

\newcommand{\be}{\begin{equation}}
\newcommand{\ee}{\end{equation}}
\newcommand{\bes}{\begin{equation*}}
\newcommand{\ees}{\end{equation*}}
\newcommand{\bea}{\begin{flalign*}}
\newcommand{\eea}{\end{flalign*}}

%\linespread{1.0}
%\setlength{\parindent}{0em}
%\setlength{\parskip}{0.8em}

\title{\textbf{Some Information about Gravitational Waves}}
\author{Aditya Vijaykumar}
\affiliation{International Centre for Theoretical Sciences, Bengaluru, India.}
\emailAdd{aditya.vijaykumar@icts.res.in}


\begin{document}
\maketitle

\section{Gravitational Waves as Geometry}

\textit{Though Maggiore, the main reference for this tome, prefers not to use geometric units, I shall unapologetically use them.}

The gravitational action is a sum of the Einstein-Hilbert action and the matter action; $S = S_E + S_M$ where,

$$S_E = \frac{1}{16 \pi} \int d^4 x \sqrt{-g}R$$

The \textit{energy-momentum tensor} $T_{\mu \nu}$ is defined from the variation of the matter action under a metric change $g_{\mu \nu} \rightarrow g_{\mu \nu} + \delta g_{\mu \nu}$ as,

$$\delta S_M = \frac{1}{2}  \int d^4 x \sqrt{-g} \ T^{\mu \nu} \delta g_{\mu \nu}$$

Variation of the total action with respect to $g_{\mu \nu}$ yields Einstein's equations,
$$\boxed{R_{\mu \nu} - \frac{1}{2}g_{\mu \nu}R = 8 \pi T_{\mu \nu}}$$

General relativity is invariant under a huge groups of transformations $x^\mu \rightarrow x'^\mu(x)$, as long as $x'$ is invertible, differentiable, and has a differentiable inverse. Such transformations are called \textit{diffeomorphisms}. Under diffeomorphisms, $g_{\mu \nu}$ transforms as,
$$g_{\mu \nu}(x') = \pdv{x^\rho}{x'^\mu} \pdv{x^\sigma}{x'^\nu} g_{\mu \nu}(x)$$

\subsection{Expansions around flat space}
Lets imagine that our system is a perturbation around the flat space metric,
$$g_{\mu \nu} = \eta_{\mu \nu} + h_{\mu \nu}$$
The magnitude of $h_{\mu \nu}$ is infinitesmal, and we only consider equations upto first order in $h_{\mu \nu}$.

Lets stare at the above equation for a few seconds. We know that the numerical values of the tensor depends on the choice of reference frame. So on what basis are we calling $h_{\mu \nu}$ infinitesmal? Actually, what we merely wish to say is that for some choice of reference frame, the above equation will be valid and we can do all our calculations. While this choice helps us simplify our calculations, it should be noted that choosing a specific reference frame breaks the diffeomorphism invariance.

Even after choosing the above transformation, we still have a residual gauge freedom left. Consider $$x'^\mu \rightarrow x^\mu + \xi^\mu (x)$$ where the derivatives of $\xi^\mu$ as as inifinitesmal as $h_{\mu \nu}$. By using the diffeomorphism property,
$$g'_{\mu \nu}(x') = \pdv{x^\rho}{x'^\mu} \pdv{x^\sigma}{x'^\nu} g_{\rho \sigma}(x)$$ and substituting for the partial derivatives, one gets,
\begin{flalign*}
	g'_{\mu \nu}(x') &= \left(\delta^\rho_\mu - \pdv{\xi^\rho}{x'^\mu}\right)  \left(\delta^\sigma_\nu - \pdv{\xi^\sigma}{x'^\nu}\right)  (\eta_{\rho \sigma} + h_{\rho \sigma}(x)) \\
	\eta_{\mu \nu} + h'_{\mu \nu}(x') &= \left(\delta^\rho_\mu \delta^\sigma_\nu - \pdv{\xi^\rho}{x'^\mu}\delta^\sigma_\nu - \pdv{\xi^\sigma}{x'^\nu} \delta^\rho_\mu \right)  (\eta_{\rho \sigma} + h_{\rho \sigma}(x))\\
	h'_{\mu \nu}(x') &= h_{\mu \nu}(x) - \pdv{\xi_\nu}{x'^\mu} - \pdv{\xi_\mu}{x'^\nu} 
\end{flalign*}

\end{document}\grid
