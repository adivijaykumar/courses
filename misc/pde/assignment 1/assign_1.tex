\documentclass{article}

\usepackage{fancyhdr}
\usepackage{extramarks}
\usepackage{amsmath}
\usepackage{amsthm}
\usepackage{amssymb}
\usepackage{amsfonts}
\usepackage{tikz}
\usepackage{physics}
\usepackage[plain]{algorithm}
\usepackage{hyperref}
\usepackage{algpseudocode}

\usetikzlibrary{automata,positioning}

%
% Basic Document Settings
%

\topmargin=-0.45in
\evensidemargin=0in
\oddsidemargin=0in
\textwidth=6.5in
\textheight=9.0in
\headsep=0.25in

\linespread{1.1}

\pagestyle{fancy}
\lhead{\hmwkAuthorName}
\chead{\hmwkClass\ : \hmwkTitle}
\rhead{\firstxmark}
\lfoot{\lastxmark}
\cfoot{\thepage}

\renewcommand\headrulewidth{0.4pt}
\renewcommand\footrulewidth{0.4pt}

\setlength\parindent{0pt}

%
% Create Problem Sections
%

\newcommand{\enterProblemHeader}[1]{
    \nobreak\extramarks{}{Problem \arabic{#1} continued on next page\ldots}\nobreak{}
    \nobreak\extramarks{Problem \arabic{#1} (continued)}{Problem \arabic{#1} continued on next page\ldots}\nobreak{}
}

\newcommand{\exitProblemHeader}[1]{
    \nobreak\extramarks{Problem \arabic{#1} (continued)}{Problem \arabic{#1} continued on next page\ldots}\nobreak{}
    \stepcounter{#1}
    \nobreak\extramarks{Problem \arabic{#1}}{}\nobreak{}
}

\setcounter{secnumdepth}{0}
\newcounter{partCounter}
\newcounter{homeworkProblemCounter}
\setcounter{homeworkProblemCounter}{1}
\nobreak\extramarks{Problem \arabic{homeworkProblemCounter}}{}\nobreak{}

%
% Homework Problem Environment
%
% This environment takes an optional argument. When given, it will adjust the
% problem counter. This is useful for when the problems given for your
% assignment aren't sequential. See the last 3 problems of this template for an
% example.
%
\newenvironment{homeworkProblem}[1][-1]{
    \ifnum#1>0
        \setcounter{homeworkProblemCounter}{#1}
    \fi
    \section{Problem \arabic{homeworkProblemCounter}}
    \setcounter{partCounter}{1}
    \enterProblemHeader{homeworkProblemCounter}
}{
    \exitProblemHeader{homeworkProblemCounter}
}

%
% Homework Details
%   - Title
%   - Due date
%   - Class
%   - Section/Time
%   - Instructor
%   - Author
%

\newcommand{\hmwkTitle}{Assignment\ \#1}
\newcommand{\hmwkDueDate}{Due on 7th September, 2021}
\newcommand{\hmwkClass}{Theory and Numerics of PDEs}
\newcommand{\hmwkClassTime}{}
\newcommand{\hmwkClassInstructor}{}
\newcommand{\hmwkAuthorName}{\textbf{Aditya Vijaykumar}}

%
% Title Page
%

\title{
    %\vspace{2in}
    \textmd{\textbf{\hmwkClass:\ \hmwkTitle}}\\
    \normalsize\vspace{0.1in}\small{\hmwkDueDate\ }\\
%    \vspace{3in}
}

\author{\hmwkAuthorName}
\date{}

\renewcommand{\part}[1]{\textbf{\large Part \Alph{partCounter}}\stepcounter{partCounter}\\}

%
% Various Helper Commands
%

% Useful for algorithms
\newcommand{\alg}[1]{\textsc{\bfseries \footnotesize #1}}

% For derivatives
\newcommand{\deriv}[1]{\frac{\mathrm{d}}{\mathrm{d}x} (#1)}

% For partial derivatives
\newcommand{\pderiv}[2]{\frac{\partial}{\partial #1} (#2)}

% Integral dx
\newcommand{\dx}{\mathrm{d}x}

% Alias for the Solution section header
\newcommand{\solution}{\textbf{\large Solution}}

% Probability commands: Expectation, Variance, Covariance, Bias
\newcommand{\E}{\mathrm{E}}
\newcommand{\Var}{\mathrm{Var}}
\newcommand{\Cov}{\mathrm{Cov}}
\newcommand{\Bias}{\mathrm{Bias}}

\begin{document}

\maketitle

%\pagebreak

\begin{homeworkProblem}
	\textbf{Question}: Show $ y(t)  = \sum_{n\ge0} \dfrac{a_0}{n!} t^n  $ converges.
	
	Let,
	\begin{equation}\label{key}
	x_n = \dfrac{a_0}{n!} t^n
	\end{equation}
	\begin{align}\label{key}
	\dfrac{x_{n+1}}{x_{n}} = \dfrac{t}{n+1}
	\end{align}
	which goes to 0 as $ n \rightarrow \infty $. So this passes the ratio test for convergence.

\end{homeworkProblem}

\begin{homeworkProblem}
	\textbf{Question}: Use $ y = \sum_{n\ge0} a_n t^n$ to solve $ y'' + y = 0 $.
	
	Substituting, we get,
	\begin{equation}\label{key}
	\sum_{n\ge0} a_n t^{n-2} + a_n t^n = 0 \implies \sum_{n\ge2} a_{n+2} t^{n-2} +  \sum_{n\ge0} a_n t^n = 0
	\end{equation}
	
	So we just need two constants, and the solution is fully specified. If $ a_0 = A $ and $ a_1 = B $,
	
	\begin{equation}\label{key}
	y = A \qty(1 - \dfrac{t^2}{2!} + \dfrac{t^4}{4!}+ \ldots) + B \qty(x - \dfrac{t^3}{3!} + \dfrac{t^5}{5!}) = A \cos(t) + B \sin(t) 
	\end{equation}
\end{homeworkProblem}

\begin{homeworkProblem}
\textbf{Question}: Consider
\begin{equation}\label{q1}
t^2 y'' - 4 t y' + 6 y = 0 \qq{;} y(1) = 1  \qq{;} y'(1) = 2 \qq{.}
\end{equation}

\begin{enumerate}
	\item Show that, $y(t) = t^2$ is a solution to the differential equation.
	\item Use Picard iteration to solve the differential equation.\\
\end{enumerate}

\textbf{Solution:}\\
For $ y = t^2 $, $ y' = 2 t $ and $  y'' =2 $. Hence, the LHS of (\ref{q1}) becomes,

\begin{equation}\label{key}
y'' - \dfrac{4}{t} y' + \dfrac{6}{t^2} y = 0
\end{equation}
We write two iterates, for $ y' $ and $ y $.

\begin{align}\label{key}
y_{1}' &=  2 + \int_{1}^{t} \dd{t } \qty[ \dfrac{8}{t} - \dfrac{6}{t^2} ] = 2 +  8\log t +  \dfrac{6}{t} - 6 \\
y_{1} &= 1 + \int_{2}^t (1) \dd{t} = 1 + 2t - 2 = 2t - 1
\end{align}

\end{homeworkProblem}


\begin{homeworkProblem}
	\textbf{Question}: Consider 
	\begin{equation}\label{q2}
	\dv{y}{t}  = y ( 1 -y) \qq{,} y(0) = 1/2	
	\end{equation}
	
	\begin{enumerate}
		\item Show that,
		\begin{equation}\label{key}
		y(t) = \dfrac{e^t}{e^t + 1} 
		\end{equation}
		solves the differential equation.
		\item Use Picard iterates to find the solution. Compare this solution to the true Taylor series.
		\item How many iterates do we need to obtain the true Taylor coefficient for $ t^k $.\\
				
	\textbf{Solution:}\\
	\begin{align}\label{key}
	\dv{y}{t} = \dfrac{e^t}{(e^t + 1)^2} &\qq{;} y(1-y) = \dfrac{e^t}{e^t + 1}  \dfrac{1}{e^t + 1} = \dfrac{e^t}{(e^t + 1)^2}\\
	\implies \dv{y}{t} &= y(1-y)
	\end{align}	
	
	Also, $ y(0) = 1/2  $. So the solution satisfies both the differential equation as well as the initial condition.
	
	Let $ \phi_{k} $ be the $ k^\mathrm{th} $ Picard iterate. We start off with $ \phi_0 = 1/2 $. Then,
	\begin{align}\label{key}
	\phi_1 &= \dfrac{1}{2} + \int_{0}^{t} \dfrac{1}{2}  \dfrac{1}{2} \dd{s} =\dfrac{1}{2} + \dfrac{t}{4}\\
	\phi_2 &= \dfrac{1}{2} + \int_{0}^{t} \qty(\dfrac{1}{2} + \dfrac{s}{4} )\qty(\dfrac{1}{2} - \dfrac{s}{4}) \dd{s} = \dfrac{1}{2} + \dfrac{s}{4} - \dfrac{t^3}{48}\\
	\phi_3 &= \dfrac{1}{2} + \int_{0}^{t} \dd{t}\qty(\dfrac{1}{2} + \dfrac{s}{4} - \dfrac{s^3}{48})\qty(\dfrac{1}{2} - \dfrac{s}{4} + \dfrac{s^3}{48}) = \dfrac{t}{4} - \dfrac{t^3}{48} + \dfrac{t^5}{480} - \dfrac{t^7}{16128}
	\end{align}
	
		
	\end{enumerate}
\end{homeworkProblem}
\end{document}
