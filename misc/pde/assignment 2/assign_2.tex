\documentclass{article}

\usepackage{fancyhdr}
\usepackage{extramarks}
\usepackage{amsmath}
\usepackage{amsthm}
\usepackage{amssymb}
\usepackage{amsfonts}
\usepackage{tikz}
\usepackage{physics}
\usepackage[plain]{algorithm}
\usepackage{hyperref}
\usepackage{algpseudocode}

\usetikzlibrary{automata,positioning}

%
% Basic Document Settings
%

\topmargin=-0.45in
\evensidemargin=0in
\oddsidemargin=0in
\textwidth=6.5in
\textheight=9.0in
\headsep=0.25in

\linespread{1.1}

\pagestyle{fancy}
\lhead{\hmwkAuthorName}
\chead{\hmwkClass\ : \hmwkTitle}
\rhead{\firstxmark}
\lfoot{\lastxmark}
\cfoot{\thepage}

\renewcommand\headrulewidth{0.4pt}
\renewcommand\footrulewidth{0.4pt}

\setlength\parindent{0pt}

%
% Create Problem Sections
%

\newcommand{\enterProblemHeader}[1]{
    \nobreak\extramarks{}{Problem \arabic{#1} continued on next page\ldots}\nobreak{}
    \nobreak\extramarks{Problem \arabic{#1} (continued)}{Problem \arabic{#1} continued on next page\ldots}\nobreak{}
}

\newcommand{\exitProblemHeader}[1]{
    \nobreak\extramarks{Problem \arabic{#1} (continued)}{Problem \arabic{#1} continued on next page\ldots}\nobreak{}
    \stepcounter{#1}
    \nobreak\extramarks{Problem \arabic{#1}}{}\nobreak{}
}

\setcounter{secnumdepth}{0}
\newcounter{partCounter}
\newcounter{homeworkProblemCounter}
\setcounter{homeworkProblemCounter}{1}
\nobreak\extramarks{Problem \arabic{homeworkProblemCounter}}{}\nobreak{}

%
% Homework Problem Environment
%
% This environment takes an optional argument. When given, it will adjust the
% problem counter. This is useful for when the problems given for your
% assignment aren't sequential. See the last 3 problems of this template for an
% example.
%
\newenvironment{homeworkProblem}[1][-1]{
    \ifnum#1>0
        \setcounter{homeworkProblemCounter}{#1}
    \fi
    \section{Problem \arabic{homeworkProblemCounter}}
    \setcounter{partCounter}{1}
    \enterProblemHeader{homeworkProblemCounter}
}{
    \exitProblemHeader{homeworkProblemCounter}
}

%
% Homework Details
%   - Title
%   - Due date
%   - Class
%   - Section/Time
%   - Instructor
%   - Author
%

\newcommand{\hmwkTitle}{Assignment\ \#2}
\newcommand{\hmwkDueDate}{Due on 15th September, 2021}
\newcommand{\hmwkClass}{Theory and Numerics of PDEs}
\newcommand{\hmwkClassTime}{}
\newcommand{\hmwkClassInstructor}{}
\newcommand{\hmwkAuthorName}{\textbf{Aditya Vijaykumar}}

%
% Title Page
%

\title{
    %\vspace{2in}
    \textmd{\textbf{\hmwkClass:\ \hmwkTitle}}\\
    \normalsize\vspace{0.1in}\small{\hmwkDueDate\ }\\
%    \vspace{3in}
}

\author{\hmwkAuthorName}
\date{}

\renewcommand{\part}[1]{\textbf{\large Part \Alph{partCounter}}\stepcounter{partCounter}\\}

%
% Various Helper Commands
%

% Useful for algorithms
\newcommand{\alg}[1]{\textsc{\bfseries \footnotesize #1}}

% For derivatives
\newcommand{\deriv}[1]{\frac{\mathrm{d}}{\mathrm{d}x} (#1)}

% For partial derivatives
\newcommand{\pderiv}[2]{\frac{\partial}{\partial #1} (#2)}

% Integral dx
\newcommand{\dx}{\mathrm{d}x}

% Alias for the Solution section header
\newcommand{\solution}{\textbf{\large Solution}}

% Probability commands: Expectation, Variance, Covariance, Bias
\newcommand{\E}{\mathrm{E}}
\newcommand{\Var}{\mathrm{Var}}
\newcommand{\Cov}{\mathrm{Cov}}
\newcommand{\Bias}{\mathrm{Bias}}

\begin{document}

\maketitle

%\pagebreak

\begin{homeworkProblem}
	Given:
	
	\begin{equation}\label{key}
	\lim\limits_{n \rightarrow \infty} a_n = L \qq{;} \lim\limits_{n \rightarrow \infty} b_n = 1 \qq{;} \lim\limits_{n \rightarrow \infty} \epsilon_n = 0 
	\end{equation}
	
	Using properties of limits for the  sum and products of functions, we can write:
	
	\begin{align}\label{key}
	\lim\limits_{n \rightarrow \infty} a_n b_n + \epsilon_n = \lim\limits_{n \rightarrow \infty} a_n \lim\limits_{n \rightarrow \infty} b_n + \lim\limits_{n \rightarrow \infty} \epsilon_n = L \times 1 + 0 = L
	\end{align}

\end{homeworkProblem}

\begin{homeworkProblem}
	Given:
	\begin{equation}\label{key}
	\lim\limits_{n \rightarrow \infty}  a_n = L \qq{;} M = \text{constant} > 0 \qq{;} S \in [L-M, L+M] \qq{;} f_n: \real \rightarrow \real \\
	\end{equation}
	\begin{equation}\label{key}
		\lim\limits_{n \rightarrow \infty} \qty(\sup\limits_{x \in S} |f_n(x) - x|) = 0
	\end{equation}
	
	Consider,
	\begin{align}\label{key}
	\abs{f_n (a_n)- L} &= \abs{f_n(a_n) - a_n + a_n - L} \\
	\abs{f_n(a_n) - a_n + a_n - L} &\le  \abs{f_n(a_n) - a_n} + \abs{a_n - L}\\
	\therefore \abs{f_n (a_n)- L} &\le  \abs{f_n(a_n) - a_n} + \abs{a_n - L}\\
	\therefore \abs{f_n (a_n)- L} &\le  \sup\limits_{x \in S} \abs{f_n(x) - x} + \abs{a_n - L}\\
	\therefore \lim\limits_{n \rightarrow \infty}\abs{f_n (a_n)- L} &\le  \lim\limits_{n \rightarrow \infty}\sup\limits_{x \in S} \abs{f_n(x) - x} +\lim\limits_{n \rightarrow \infty} \abs{a_n - L}\\
		\therefore \lim\limits_{n \rightarrow \infty}\abs{f_n (a_n)- L} &\le  0 + 0\\
		\therefore \lim\limits_{n \rightarrow \infty}\abs{f_n (a_n)- L} &= 0 \implies \lim\limits_{n \rightarrow \infty}f_n (a_n) = L
	\end{align}
	
	The $ \sup $ in the condition for $ f_n $ is indeed necessary, since without that we wouldn't have been able to manipulate the inequality to introduce the condition instead of $ \abs{f_n (a_n)- a_n} $.
\end{homeworkProblem}
\end{document}
