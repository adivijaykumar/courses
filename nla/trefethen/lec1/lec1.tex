\documentclass{article}

\usepackage{fancyhdr}
\usepackage{extramarks}
\usepackage{amsmath}
\usepackage{amsthm}
\usepackage{amsfonts}
\usepackage{tikz}
\usepackage{physics}
\usepackage[plain]{algorithm}
\usepackage{algpseudocode}

\usetikzlibrary{automata,positioning}

%
% Basic Document Settings
%

\topmargin=-0.45in
\evensidemargin=0in
\oddsidemargin=0in
\textwidth=6.5in
\textheight=9.0in
\headsep=0.25in

\linespread{1.1}

\pagestyle{fancy}
\lhead{\hmwkAuthorName}
\chead{\hmwkClass\ (\hmwkClassInstructor\ \hmwkClassTime): \hmwkTitle}
\rhead{\firstxmark}
\lfoot{\lastxmark}
\cfoot{\thepage}

\renewcommand\headrulewidth{0.4pt}
\renewcommand\footrulewidth{0.4pt}

\setlength\parindent{0pt}

%
% Create Problem Sections
%

\newcommand{\enterProblemHeader}[1]{
    \nobreak\extramarks{}{Problem \arabic{#1} continued on next page\ldots}\nobreak{}
    \nobreak\extramarks{Problem \arabic{#1} (continued)}{Problem \arabic{#1} continued on next page\ldots}\nobreak{}
}

\newcommand{\exitProblemHeader}[1]{
    \nobreak\extramarks{Problem \arabic{#1} (continued)}{Problem \arabic{#1} continued on next page\ldots}\nobreak{}
    \stepcounter{#1}
    \nobreak\extramarks{Problem \arabic{#1}}{}\nobreak{}
}

\setcounter{secnumdepth}{0}
\newcounter{partCounter}
\newcounter{homeworkProblemCounter}
\setcounter{homeworkProblemCounter}{1}
\nobreak\extramarks{Problem \arabic{homeworkProblemCounter}}{}\nobreak{}

%
% Homework Problem Environment
%
% This environment takes an optional argument. When given, it will adjust the
% problem counter. This is useful for when the problems given for your
% assignment aren't sequential. See the last 3 problems of this template for an
% example.
%
\newenvironment{homeworkProblem}[1][-1]{
    \ifnum#1>0
        \setcounter{homeworkProblemCounter}{#1}
    \fi
    \section{Problem \arabic{homeworkProblemCounter}}
    \setcounter{partCounter}{1}
    \enterProblemHeader{homeworkProblemCounter}
}{
    \exitProblemHeader{homeworkProblemCounter}
}

%
% Homework Details
%   - Title
%   - Due date
%   - Class
%   - Section/Time
%   - Instructor
%   - Author
%

\newcommand{\hmwkTitle}{Lecture \#1}
\newcommand{\hmwkDueDate}{\today}
\newcommand{\hmwkClass}{Trefethen and Bau}
\newcommand{\hmwkClassTime}{}
\newcommand{\hmwkClassInstructor}{Professor Isaac Newton}
\newcommand{\hmwkAuthorName}{\textbf{Aditya Vijaykumar}}

%
% Title Page
%

\title{
    %\vspace{2in}
    \textmd{\textbf{\hmwkClass:\ \hmwkTitle}}\\
    \normalsize\vspace{0.1in}\small{\hmwkDueDate\ }\\
%    \vspace{3in}
}

\author{\hmwkAuthorName}
\date{}

\renewcommand{\part}[1]{\textbf{\large Part \Alph{partCounter}}\stepcounter{partCounter}\\}

%
% Various Helper Commands
%

% Useful for algorithms
\newcommand{\alg}[1]{\textsc{\bfseries \footnotesize #1}}

% For derivatives
\newcommand{\deriv}[1]{\frac{\mathrm{d}}{\mathrm{d}x} (#1)}

% For partial derivatives
\newcommand{\pderiv}[2]{\frac{\partial}{\partial #1} (#2)}

% Integral dx
\newcommand{\dx}{\mathrm{d}x}

% Alias for the Solution section header
\newcommand{\solution}{\textbf{\large Solution}}

% Probability commands: Expectation, Variance, Covariance, Bias
\newcommand{\E}{\mathrm{E}}
\newcommand{\Var}{\mathrm{Var}}
\newcommand{\Cov}{\mathrm{Cov}}
\newcommand{\Bias}{\mathrm{Bias}}

\begin{document}

\maketitle

%\pagebreak

\begin{homeworkProblem}\small
$\left[\begin{array}{cccc}
1 & -1 & 0 & 0	\\
0 & 1 & 0 & 0	\\
0 & -1 & 1 & 0	\\
0 & -1 & 0 & 1	
\end{array}\right]
\left[\begin{array}{cccc}
1 & 0 & 1 & 0	\\
0 & 1 & 0 & 0	\\
0 & 0 & 1 & 0	\\
0 & 0 & 0 & 1	
\end{array}\right]
\left[\begin{array}{cccc}
1 & 0 & 0 & 0	\\
0 & 1 & 0 & 0	\\
0 & 0 & \frac{1}{2} & 1	\\
0 & 0 & 0 & 0	
\end{array}\right]$B
$\left[\begin{array}{cccc}
2 & 0 & 0 & 0	\\
0 & 1 & 0 & 0	\\
0 & 0 & 1 & 0	\\
0 & 0 & 0 & 1	
\end{array}\right]
\left[\begin{array}{cccc}
0 & 0 & 0 & 1	\\
0 & 1 & 0 & 0	\\
0 & 0 & 1 & 0	\\
1 & 0 & 0 & 0	
\end{array}\right]
\left[\begin{array}{cccc}
1 & 0 & 0 & 0	\\
0 & 1 & 0 & 0	\\
0 & 0 & 1 & 1	\\
0 & 0 & 0 & 0	
\end{array}\right]
\left[\begin{array}{cccc}
0 & 0 & 0	\\
1 & 0 & 0	\\
0 & 1 & 0	\\
0 & 0 & 1	
\end{array}\right]
$\\

A is just the product of all matrices to the left of B and C is just the product of all matrices to the right of B.

\end{homeworkProblem}

\begin{homeworkProblem}
\textbf{Part (a)}
\begin{flalign*}
	f_1&=k_{12} \left(-l_{12}-x_1+x_2\right)\\
	f_2&=k_{23} \left(-l_{32}-x_2+x_3\right)-k_{12} \left(-l_{12}-x_1+x_2\right)\\
	f_3&=k_{34} \left(-l_{34}-x_3+x_4\right)-k_{23} \left(-l_{32}-x_2+x_3\right)\\
	f_4&=-k_{34} \left(-l_{34}-x_3+x_4\right)
\end{flalign*}
$$\left[\begin{array}{cccc}
f_1 \\
f_2 \\
f_3\\
f_4 
\end{array}\right] = \left[\begin{array}{cccc}
-k_{12} & k_{12} & 0 & 0	\\
k_{12} & -(k_{12}+k_{23}) & k_{23} & 0	\\
0 & k_{23} & -(k_{23}+k_{34}) & k_{34}	\\
0 & 0 & k_{34} & -k_{34}	
\end{array}\right] \left[\begin{array}{cccc}
x_1 \\
x_2 \\
x_3\\
x_4 
\end{array}\right] + (constant \ matrix)$$

\textbf{Part (b)}
The dimensions of K will be that of $\frac{f}{x}$, ie. $\frac{kg}{s^2}$.\\

\textbf{Part (c)}
The dimensions of $\det(K)$ will be ${(\frac{kg}{s^2})}^4$.\\

\textbf{Part (d)}
$$1 \frac{kg}{s^2} = 1000\frac{g}{s^2}$$
$$1 (\frac{kg}{s^2})^4 = 10^{12}\frac{g}{s^2}$$
$$\det(K)=10^{12}\det(K')$$
\end{homeworkProblem}

\begin{homeworkProblem}
Let the identity $I=[e_1,e_2,\ldots,e_m]$, $R^{-1}=[a_1,a_2,\ldots,a_n]$. Let us write the expression $I = R^{-1}R$ as follows,
$$[e_1,e_2,\ldots,e_m]=[a_1,a_2,\ldots,a_n] \left[ \begin{array}{cccc}
r_{11} & r_{12} & \ldots & r_{1m}\\
r_{21} & r_{22} & \ldots & r_{2m}\\
\vdots & \vdots & \vdots & \vdots \\
r_{1m} & r_{2m} & \ldots & r_{mm}\\
\end{array}\right]$$
\end{homeworkProblem}
\end{document}
