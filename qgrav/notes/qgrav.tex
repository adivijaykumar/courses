\documentclass[a4paper,11pt]{article}

\usepackage{physics}
\usepackage{amsmath}
\usepackage{amssymb}
\usepackage{amsmath}
\usepackage{amsthm, mathtools}
%\usepackage{hyperref}
\usepackage{color}
\usepackage{jheppub}
\usepackage[T1]{fontenc} % if needed

%\linespread{1.0}
%\setlength{\parindent}{0em}
%\setlength{\parskip}{0.8em}

\title{\textbf{Notes on Gravity as a Quantum Theory}}
\author{Aditya Vijaykumar}
\affiliation{International Centre for Theoretical Sciences, Bengaluru, India.}
\emailAdd{aditya.vijaykumar@icts.res.in}

\begin{document}
\maketitle

\section{All is Classical, All is Quantum}

\subsection{The Classical Field}
$\phi(\va{x},t)$ gives the value of a classical field at every point in spacetime. The simplest classical field is the \textit{real scalar field}, which is characterized only by real numbers. The Klein-Gordon equation governs a free massive scalar field.

$$\pdv[2]{\phi}{t} - \sum_{x_j} \pdv[2]{\phi}{x_j} + m^2 \phi =0 $$

An interesting part about the free scalar field is that one can describe it as an infinite set of decoupled harmonic oscillators. Put this field into a box of length $L$ and volume $V=L^3$, and having periodic boundary conditions. One can Fourier decompose this as,
$$\phi(\va{x},t) = \frac{1}{\sqrt{V}} \sum_{\va{k}} \phi_{\vb{k}} (t) \exp(i \va{k}\vdot\va{x}) \text{ where } k_x =\frac
{2\pi n_x}{L} ,\ldots$$

Substituting this into the first equation, we find that the harmonic oscillators get nicely decoupled into an infinite set of ODEs of the form,
$$\ddot{\phi_{\vb{k}}} +(k^2+m^2)\phi_{\vb{k}} = 0$$
which is basically the harmonic oscillator equation with frequency $\omega_k = \sqrt{k^2 + m^2}$.The energy of oscillators in simply equal to the sum of individual energies of the oscillators,
$$E = \sum_{\vb{k}}\left[ \frac{1}{2} \dot{\phi_{\vb{k}}}^2 + \frac{1}{2}\omega_k^2 \phi_{\vb{k}}^2 \right]$$

Equivalently, when $V \rightarrow \infty $ and $k$ is a continuous variable, the summation is just replaced by an integral over all $k$,
$$\phi(\vb{x},t)= \int \frac{d^3\vb{k}}{(2\pi)^{3/2}}e^{i \vb{k} \vdot \vb{x}} \phi_{\vb{k}}(t)$$

\subsection{Quantizing Fields}
As mentioned earlier, a field can be thought of as a collection of decoupled harmonic oscillators. We quantize each field $\phi_{\vb{k}}$ as a separate harmonic oscillator. We identify the position and momentum as operators $\hat{\phi_{\vb{k}}}$ and $\hat{\pi_{\vb{k}}}$. The commutation relations for the harmonic oscillator as $V\rightarrow \infty$ can now be written as,
$$\comm{\hat{\phi_{\vb{k}}}(t)}{\hat{\pi_{\vb{k'}}}(t)} = i \delta(\vb{k}+\vb{k'})$$
The vacuum state is the state corresponding to the lowest energy configuration. One can clearly see that the commutation relations cannot be satisfied for the most intuitive low energy configuration \textit{ie.} $\phi(\vb{x},t) = 0$, implying that the vacuum state is really something non-trivial. But since, for a free field, all the $\phi_{\vb{k}}$ are decoupled, we can write the vacuum state wave functional as the product of all wavefunctions, each describing the ground state of the harmonic oscillator with the wavenumber $\vb{k}$. Again, for large volume, one can write,
$$\psi[\phi] \propto \exp(-\frac{1}{2}\int d^3 \vb{k} \abs{\phi_{\vb{k}}}^2 \omega_{\vb{k}})$$
The vacuum energy density is just the sum of all ground state energies,
$$\frac{E_o}{V}= \int\frac{d^3 \vb{k}}{(2\pi)^3} \frac{\omega_{k}}{2}$$
Okay, now this is a very interesting expression for the energy. We see that because $\omega_k = \sqrt{k^2+m^2}$, we can see that this integral diverges as $k^4$. If quantum gravity is assumed to be modelled as a scalar field, and we put a cutoff for our integration at let's say the Planckian scale, we see that the vacuum energy density is of the order unity in Planck units, which in turn corresponds to a mass density of $10^{94} g/cm^3$. The mass of the \textit{entire} observable universe is $10^{55}g$! One can try to resolve this problem by \textit{positing} that vacuum energy does not contribute to gravity, or by using some supersymmetric variants of such theories.

\subsection{Vacuum Fluctuations}
The fluctuation in the quantum field can be written as,
$$\delta \phi_{\vb{k}} = \sqrt{\expval{\abs{\phi_{\vb{k}}}^2} - \expval{\phi_{\vb{k}}}^2} = \sqrt{\expval{\abs{\phi_{\vb{k}}}^2}}$$
We know that $$\phi_{\vb{k}} = \frac{a_{\vb{k}} + a_{-\vb{k}}}{\sqrt{2\omega_k}}$$ which means that 
$$\abs{\phi_{\vb{k}}^2} = \frac{(a_{\vb{k}} + a_{-\vb{k}})(a_{\vb{k}} + a_{-\vb{k}})}{{2\omega_k}}$$
Taking the ground state expectation value of this expression, one obtains that $\delta \phi_{\vb{k}} \sim \omega_k^{-1/2}$. What if we measure the average value of a field over space? Lets consider a cubical box of side $L$  and define the average value $\phi_L$ as follows,
$$\phi_L = \frac{1}{L^3} \int_{-L/2}^{-L/2}dx \int_{-L/2}^{-L/2}dy \int_{-L/2}^{-L/2}dz \ \phi(\vb{x})$$
We again calculate fluctuations in this average value by the formula $\delta \phi_L = \sqrt{\expval{\phi_L^2}}$. Evaluating this quantity, we get $\delta \phi_L = \sqrt{(\delta \phi_{\vb{k}})^2/L^3}$. \textcolor{red}{(explictly verify)}
\end{document}