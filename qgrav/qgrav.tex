
\documentclass[a4paper,11pt]{article}

\usepackage{physics}
\usepackage{amsmath}
\usepackage{amssymb}
\usepackage{amsmath}
\usepackage{amsthm, mathtools}
%\usepackage{hyperref}
\usepackage{color}
\usepackage{jheppub}
\usepackage[T1]{fontenc} % if needed

%\linespread{1.0}
%\setlength{\parindent}{0em}
%\setlength{\parskip}{0.8em}

\title{\textbf{Notes on Gravity as a Quantum Theory}}
\author{Aditya Vijaykumar}
\affiliation{International Centre for Theoretical Sciences, Bengaluru, India.}
\emailAdd{aditya.vijaykumar@icts.res.in}

\begin{document}
\maketitle \tableofcontents

\section{Classical Fields}
$\phi(\va{x},t)$ gives the value of the classical field at every point in spacetime. The simplest classical field is the \textit{real scalar field}, which is characterized only by real numbers. The Klein-Gordon equation governs a free massive scalar field.

$$\pdv[2]{\phi}{t} - \sum_{x_j} \pdv[2]{\phi}{x_j} + m^2 \phi =0 $$

An interesting part about the free scalar field is that one can describe it as an infinite set of decoupled harmonic oscillators. Put this field into a box of length $L$ and volume $V=L^3$, and having periodic boundary conditions. One can Fourier decompose this as,
$$\phi(\va{x},t) = \frac{1}{\sqrt{V}} \sum_{\va{k}} \phi_{\vb{k}} (t) \exp(i \va{k}\vdot\va{x}) \text{ where } k_x =\frac
{2\pi n_x}{L} ,\ldots$$

Substituting this into the first equation, we find that the harmonic oscillators get nicely decoupled into an infinite set of ODEs of the form,
$$\ddot{\phi_{\vb{k}}} +(k^2+m^2)\phi_{\vb{k}} = 0$$
which is basically the harmonic oscillator equation with frequency $\omega_k = \sqrt{k^2 + m^2}$.The energy of oscillators in simply equal to the sum of individual energies of the oscillators,
$$E = \sum_{\vb{k}}\left[ \frac{1}{2} \dot{\phi_{\vb{k}}}^2 + \frac{1}{2}\omega_k^2 \phi_{\vb{k}}^2 \right]$$
\end{document}