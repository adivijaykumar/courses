
\documentclass[a4paper,11pt]{article}

\usepackage{physics}
\usepackage{amsmath}
\usepackage{amssymb}
\usepackage{amsmath}
\usepackage{amsthm, mathtools}
%\usepackage{hyperref}
\usepackage{color}
\usepackage{jheppub}
\usepackage[T1]{fontenc} % if needed

%\linespread{1.0}
%\setlength{\parindent}{0em}
%\setlength{\parskip}{0.8em}

\title{\textbf{Notes on Gravity as a Quantum Theory}}
\author{Aditya Vijaykumar}
\affiliation{International Centre for Theoretical Sciences, Bengaluru, India.}
\emailAdd{aditya.vijaykumar@icts.res.in}

\begin{document}
\maketitle \tableofcontents

\section{Classical Fields}
$\phi(\va{x},t)$ gives the value of the classical field at every point in spacetime. The simplest classical field is the \textit{real scalar field}, which is characterized only by real numbers. The Klein-Gordon equation governs a free massive scalar field.
$$\pdv[2]{\phi}{t} - \sum_{x_j} \pdv[2]{\phi}{x_j} + m^2 \phi =0 $$
\end{document}