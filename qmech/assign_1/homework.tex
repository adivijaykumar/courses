\documentclass{article}

\usepackage{fancyhdr}
\usepackage{extramarks}
\usepackage{amsmath}
\usepackage{amsthm}
\usepackage{amsfonts}
\usepackage{tikz}
\usepackage{physics}
\usepackage[plain]{algorithm}
\usepackage{algpseudocode}

\usetikzlibrary{automata,positioning}

%
% Basic Document Settings
%

\topmargin=-0.45in
\evensidemargin=0in
\oddsidemargin=0in
\textwidth=6.5in
\textheight=9.0in
\headsep=0.25in

\linespread{1.1}

\pagestyle{fancy}
\lhead{\hmwkAuthorName}
\chead{\hmwkClass\ (\hmwkClassInstructor\ \hmwkClassTime): \hmwkTitle}
\rhead{\firstxmark}
\lfoot{\lastxmark}
\cfoot{\thepage}

\renewcommand\headrulewidth{0.4pt}
\renewcommand\footrulewidth{0.4pt}

\setlength\parindent{0pt}

%
% Create Problem Sections
%

\newcommand{\enterProblemHeader}[1]{
    \nobreak\extramarks{}{Problem \arabic{#1} continued on next page\ldots}\nobreak{}
    \nobreak\extramarks{Problem \arabic{#1} (continued)}{Problem \arabic{#1} continued on next page\ldots}\nobreak{}
}

\newcommand{\exitProblemHeader}[1]{
    \nobreak\extramarks{Problem \arabic{#1} (continued)}{Problem \arabic{#1} continued on next page\ldots}\nobreak{}
    \stepcounter{#1}
    \nobreak\extramarks{Problem \arabic{#1}}{}\nobreak{}
}

\setcounter{secnumdepth}{0}
\newcounter{partCounter}
\newcounter{homeworkProblemCounter}
\setcounter{homeworkProblemCounter}{1}
\nobreak\extramarks{Problem \arabic{homeworkProblemCounter}}{}\nobreak{}

%
% Homework Problem Environment
%
% This environment takes an optional argument. When given, it will adjust the
% problem counter. This is useful for when the problems given for your
% assignment aren't sequential. See the last 3 problems of this template for an
% example.
%
\newenvironment{homeworkProblem}[1][-1]{
    \ifnum#1>0
        \setcounter{homeworkProblemCounter}{#1}
    \fi
    \section{Problem \arabic{homeworkProblemCounter}}
    \setcounter{partCounter}{1}
    \enterProblemHeader{homeworkProblemCounter}
}{
    \exitProblemHeader{homeworkProblemCounter}
}

%
% Homework Details
%   - Title
%   - Due date
%   - Class
%   - Section/Time
%   - Instructor
%   - Author
%

\newcommand{\hmwkTitle}{Homework\ \#2}
\newcommand{\hmwkDueDate}{February 12, 2014}
\newcommand{\hmwkClass}{Calculus}
\newcommand{\hmwkClassTime}{}
\newcommand{\hmwkClassInstructor}{Professor Isaac Newton}
\newcommand{\hmwkAuthorName}{\textbf{Aditya Vijaykumar}}

%
% Title Page
%

\title{
    %\vspace{2in}
    \textmd{\textbf{\hmwkClass:\ \hmwkTitle}}\\
    \normalsize\vspace{0.1in}\small{\hmwkDueDate\ }\\
%    \vspace{3in}
}

\author{\hmwkAuthorName}
\date{}

\renewcommand{\part}[1]{\textbf{\large Part \Alph{partCounter}}\stepcounter{partCounter}\\}

%
% Various Helper Commands
%

% Useful for algorithms
\newcommand{\alg}[1]{\textsc{\bfseries \footnotesize #1}}

% For derivatives
\newcommand{\deriv}[1]{\frac{\mathrm{d}}{\mathrm{d}x} (#1)}

% For partial derivatives
\newcommand{\pderiv}[2]{\frac{\partial}{\partial #1} (#2)}

% Integral dx
\newcommand{\dx}{\mathrm{d}x}

% Alias for the Solution section header
\newcommand{\solution}{\textbf{\large Solution}}

% Probability commands: Expectation, Variance, Covariance, Bias
\newcommand{\E}{\mathrm{E}}
\newcommand{\Var}{\mathrm{Var}}
\newcommand{\Cov}{\mathrm{Cov}}
\newcommand{\Bias}{\mathrm{Bias}}

\begin{document}

\maketitle

%\pagebreak

\begin{homeworkProblem}
Show that commutators in quantum mechanics and Poisson brackets in classical mechanics both obey the Jacobi identity.
$$[A, [B, C]] + [C, [A, B]] + [B, [C, A]] = 0.$$
$$\acomm{A}{\acomm{B}{C}_{PB}}_{PB} + \acomm{C}{\acomm{A}{B}_{PB}}_{PB} + \acomm{B}{\acomm{C}{A}_{PB}}_{PB} = 0$$

\textbf{Solution}\\
We solve each part separately.
\\

\textbf{Part One - Commutators}

We expand out each term as follows
$$\comm{A}{\comm{B}{C}} = ABC - ACB -BCA + CBA$$
$$\comm{C}{\comm{A}{B}} = CAB - CBA - ABC + BAC$$
$$\comm{B}{\comm{C}{A}} = BCA - BAC - CAB + ACB$$

Adding the three expressions above, we arrive at the expression
$$[A, [B, C]] + [C, [A, B]] + [B, [C, A]] = 0.$$
Hence Proved.

\textbf{Part One - TO BE DONE}

\begin{homeworkProblem}
Show that $$\comm{AB}{CD} = -AC\acomm{D}{B} + A\acomm{C}{B}D - C\acomm{D}{A}B + \acomm{C}{A}DB$$

\textbf{Solution}
\begin{flalign*}
\comm{AB}{CD} &= A\comm{B}{CD} + \comm{A}{CD}B\\
&= A\comm{B}{C}D + AC\comm{B}{D} + C\comm{A}{D}B + \comm{A}{C}DB\\
&= A(\acomm{B}{C}-2CB)D + AC(2BD - \acomm{B}{D}) + C(2AD - \acomm{A}{D}) + (\acomm{A}{C}-2CA)DB\\
&= A\acomm{B}{C}D-2ACBD + 2ACBD - AC\acomm{B}{D} + 2CADB - C\acomm{A}{D}B + \acomm{A}{C}DB-2CADB\\
&= -AC\acomm{B}{D} + A\acomm{B}{C}D - C\acomm{A}{D}B + \acomm{A}{C}DB\\
&= -AC\acomm{D}{B} + A\acomm{C}{B}D - C\acomm{D}{A}B + \acomm{C}{A}DB
\end{flalign*}
Hence Proved.
\end{homeworkProblem}

\end{homeworkProblem}

\begin{homeworkProblem}
Let $ \va{n} = n_x \vu{x}+ n_y\vu{y}  + n_z \vu{z}.$ $\va{\sigma} \dotproduct \va{n} = n_x \sigma_x + n_y \sigma_y + n_z \sigma_z$. Find its eigenvalues and eigenvectors.
\end{homeworkProblem}

\begin{homeworkProblem}
Let A be an observable whose spectral decomposition is$A = \sum_{i} \lambda_i P_i$ . What is the significance of $$\prod_{i\ne j}\frac{A-\lambda_i}{\lambda_j-\lambda_i}$$ where note that the product runs only over $i$ and $j$ is held fixed
\end{homeworkProblem}



\begin{homeworkProblem}
Show that commutators in quantum mechanics and Poisson brackets in classical mechanics both obey the Jacobi identity.
$$[A, [B, C]] + [C, [A, B]] + [B, [C, A]] = 0.$$
$$\acomm{A}{\acomm{B}{C}_{PB}}_{PB} + \acomm{C}{\acomm{A}{B}_{PB}}_{PB} + \acomm{B}{\acomm{C}{A}_{PB}}_{PB} = 0$$

    \begin{enumerate}
        \item \(f(n) = n^2 + n + 1\), \(g(n) = 2n^3\)
        \item \(f(n) = n\sqrt{n} + n^2\), \(g(n) = n^2\)
        \item \(f(n) = n^2 - n + 1\), \(g(n) = n^2 / 2\)
    \end{enumerate}

    \textbf{Solution}

    We solve each solution algebraically to determine a possible constant
    \(c\).
    \\

    \textbf{Part One}

    \[
        \begin{split}
            n^2 + n + 1 &=
            \\
            &\leq n^2 + n^2 + n^2
            \\
            &= 3n^2
            \\
            &\leq c \cdot 2n^3
        \end{split}
    \]

    Thus a valid \(c\) could be when \(c = 2\).
    \\

    \textbf{Part Two}

    \[
        \begin{split}
            n^2 + n\sqrt{n} &=
            \\
            &= n^2 + n^{3/2}
            \\
            &\leq n^2 + n^{4/2}
            \\
            &= n^2 + n^2
            \\
            &= 2n^2
            \\
            &\leq c \cdot n^2
        \end{split}
    \]

    Thus a valid \(c\) is \(c = 2\).
    \\

    \textbf{Part Three}

    \[
        \begin{split}
            n^2 - n + 1 &=
            \\
            &\leq n^2
            \\
            &\leq c \cdot n^2/2
        \end{split}
    \]

    Thus a valid \(c\) is \(c = 2\).

\end{homeworkProblem}

\end{document}
