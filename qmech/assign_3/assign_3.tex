\documentclass{article}

\usepackage{fancyhdr}
\usepackage{extramarks}
\usepackage{amsmath}
\usepackage{amsthm}
\usepackage{amssymb}
\usepackage{amsfonts}
\usepackage{tikz}
\usepackage{physics}
\usepackage[plain]{algorithm}
\usepackage{algpseudocode}

\usetikzlibrary{automata,positioning}

%
% Basic Document Settings
%

\topmargin=-0.45in
\evensidemargin=0in
\oddsidemargin=0in
\textwidth=6.5in
\textheight=9.0in
\headsep=0.25in

\linespread{1.1}

\pagestyle{fancy}
\lhead{\hmwkAuthorName}
\chead{\hmwkClass\ : \hmwkTitle}
\rhead{\firstxmark}
\lfoot{\lastxmark}
\cfoot{\thepage}

\renewcommand\headrulewidth{0.4pt}
\renewcommand\footrulewidth{0.4pt}

\setlength\parindent{0pt}

%
% Create Problem Sections
%
\newcommand{\be}{\begin{equation}}
\newcommand{\ee}{\end{equation}}
\newcommand{\bes}{\begin{equation*}}
\newcommand{\ees}{\end{equation*}}
\newcommand{\bea}{\begin{flalign*}}
\newcommand{\eea}{\end{flalign*}}


\newcommand{\enterProblemHeader}[1]{
    \nobreak\extramarks{}{Problem \arabic{#1} continued on next page\ldots}\nobreak{}
    \nobreak\extramarks{Problem \arabic{#1} (continued)}{Problem \arabic{#1} continued on next page\ldots}\nobreak{}
}

\newcommand{\exitProblemHeader}[1]{
    \nobreak\extramarks{Problem \arabic{#1} (continued)}{Problem \arabic{#1} continued on next page\ldots}\nobreak{}
    \stepcounter{#1}
    \nobreak\extramarks{Problem \arabic{#1}}{}\nobreak{}
}

\setcounter{secnumdepth}{0}
\newcounter{partCounter}
\newcounter{homeworkProblemCounter}
\setcounter{homeworkProblemCounter}{1}
\nobreak\extramarks{Problem \arabic{homeworkProblemCounter}}{}\nobreak{}

%
% Homework Problem Environment
%
% This environment takes an optional argument. When given, it will adjust the
% problem counter. This is useful for when the problems given for your
% assignment aren't sequential. See the last 3 problems of this template for an
% example.
%
\newenvironment{homeworkProblem}[1][-1]{
    \ifnum#1>0
        \setcounter{homeworkProblemCounter}{#1}
    \fi
    \section{Problem \arabic{homeworkProblemCounter}}
    \setcounter{partCounter}{1}
    \enterProblemHeader{homeworkProblemCounter}
}{
    \exitProblemHeader{homeworkProblemCounter}
}

%
% Homework Details
%   - Title
%   - Due date
%   - Class
%   - Section/Time
%   - Instructor
%   - Author
%

\newcommand{\hmwkTitle}{Assignment\ \#2}
\newcommand{\hmwkDueDate}{Due on 4th October, 2018}
\newcommand{\hmwkClass}{Advanced Quantum Mechanics}
\newcommand{\hmwkClassTime}{}
\newcommand{\hmwkClassInstructor}{}
\newcommand{\hmwkAuthorName}{\textbf{Aditya Vijaykumar}}

%
% Title Page
%

\title{
    %\vspace{2in}
    \textmd{\textbf{\hmwkClass:\ \hmwkTitle}}\\
    \normalsize\vspace{0.1in}\small{\hmwkDueDate\ }\\
%    \vspace{3in}
}

\author{\hmwkAuthorName}
\date{}

\renewcommand{\part}[1]{\textbf{\large Part \Alph{partCounter}}\stepcounter{partCounter}\\}

%
% Various Helper Commands
%

% Useful for algorithms
\newcommand{\alg}[1]{\textsc{\bfseries \footnotesize #1}}

% For derivatives
\newcommand{\deriv}[1]{\frac{\mathrm{d}}{\mathrm{d}x} (#1)}

% For partial derivatives
\newcommand{\pderiv}[2]{\frac{\partial}{\partial #1} (#2)}

% Integral dx
\newcommand{\dx}{\mathrm{d}x}

% Alias for the Solution section header
\newcommand{\solution}{\textbf{\large Solution}}

% Probability commands: Expectation, Variance, Covariance, Bias
\newcommand{\E}{\mathrm{E}}
\newcommand{\Var}{\mathrm{Var}}
\newcommand{\Cov}{\mathrm{Cov}}
\newcommand{\Bias}{\mathrm{Bias}}

\begin{document}

\maketitle
\begin{homeworkProblem}[1]
	Let's use the follwing convention $ (\ket{l,m}) $
	\begin{equation*}
	\ket{2,2} = \mqty(1\\0\\0\\0\\0) \qq{and} \ket{2,1} = \mqty(0\\1\\0\\0\\0) \qq{and} \ket{2,0} = \mqty(0\\0\\1\\0\\0) \qq{and} \ket{2,-1} = \mqty(0\\0\\0\\1\\0) \qq{and} \ket{2,-2} = \mqty(0\\0\\0\\0\\1)
	\end{equation*}
	We know that,
	\begin{equation*}
	J_3 \ket{l,m} = m\ket{l,m} \qq{and} J_{\pm} \ket{l,m} = \sqrt{(l \mp m)(l\pm m + 1)} \ket{l,m\pm 1}
	\end{equation*}
	\begin{equation*}
	 J_3 \ket{2,2} = 2\ket{2,2} \text{ , } J_3 \ket{2,1} = \ket{2,1} \text{ , } J_3 \ket{2,0} = 0\text{ , } J_3 \ket{2,-1} = -\ket{2,-1} \text{ , } J_3 \ket{2,-2} = -2\ket{2,-2}\\
	\end{equation*}
	Hence, we can see that,
	\begin{equation*}
	J_3 = \mqty(\dmat[0]{2,1,0,-1,-2})
	\end{equation*}
	We also note that,
	\begin{equation*}
	J_+ \ket{2,2} = 0 \text{ , } J_+ \ket{2,1} = 2 \ket{2,2} \text{ , } J_+ \ket{2,0} = \sqrt{6} \ket{2,1}\text{ , } J_+ \ket{2,-1} = \sqrt{6} \ket{2,0} \text{ , } J_+ \ket{2,-2} = 2 \ket{2,-1}\\
	\end{equation*}
	\begin{equation*}
	J_- \ket{2,2} = 2 \ket{2,1} \text{ , } J_- \ket{2,1} = \sqrt{6} \ket{2,0} \text{ , } J_- \ket{2,0} = \sqrt{6} \ket{2,-1} \text{ , } J_- \ket{2,-1} = 2 \ket{2,-2} \text{ , } J_- \ket{2,-2} = 0\\
	\end{equation*}
	So,
	\begin{equation*}
	J_+ = \mqty( 0 & 2 & 0 & 0 & 0 \\ 0 & 0 & \sqrt{6} & 0 & 0 \\ 0 & 0 & 0 & \sqrt{6} & 0 \\ 0 & 0 & 0 & 0 & 2 \\ 0 & 0 & 0 & 0 & 0) \qq{and} J_- = \mqty( 0 & 0 & 0 & 0 & 0 \\ 2 & 0 & 0 & 0 & 0 \\ 0 & \sqrt{6} & 0 & 0 & 0 \\ 0 & 0 & \sqrt{6} & 0 & 0 \\ 0 & 0 & 0 & 2 & 0)
	\end{equation*}
	Hence,
	\begin{equation*}
	J_1 = \dfrac{J_+ + J_-}{2} = \dfrac{1}{2} \mqty( 0 & 2 & 0 & 0 & 0 \\ 2 & 0 & \sqrt{6} & 0 & 0 \\ 0 & \sqrt{6} & 0 & \sqrt{6} & 0 \\ 0 & 0 & \sqrt{6} & 0 & 2 \\ 0 & 0 & 0 & 2 & 0) \qq{and} J_2 = \dfrac{J_+ - J_-}{2i} = \dfrac{i}{2} \mqty( 0 & -2 & 0 & 0 & 0 \\ 2 & 0 & -\sqrt{6} & 0 & 0 \\ 0 & \sqrt{6} & 0 & -\sqrt{6} & 0 \\ 0 & 0 & \sqrt{6} & 0 & -2 \\ 0 & 0 & 0 & 2 & 0)
	\end{equation*}
	
\end{homeworkProblem}

\begin{homeworkProblem}[2]
	Given,
	\begin{equation*}
	U = \exp(-i \dfrac{\sigma_3 \alpha}{2}) \exp(-i \dfrac{\sigma_2 \beta}{2}) \exp(-i \dfrac{\sigma_3 \gamma}{2})
	\end{equation*}
	
	Consider the trace of $ U  $,
	\begin{align*}
	\tr U &= \expval{U}{0} + \expval{U}{1}\\
	&= \expval{ \exp(-i \dfrac{\sigma_3 \alpha}{2}) \exp(-i \dfrac{\sigma_2 \beta}{2}) \exp(-i \dfrac{\sigma_3 \gamma}{2})}{0} + \expval{ \exp(-i \dfrac{\sigma_3 \alpha}{2}) \exp(-i \dfrac{\sigma_2 \beta}{2}) \exp(-i \dfrac{\sigma_3 \gamma}{2})}{1}\\
	\tr U&= e^{-i\qty(\frac{\alpha + \gamma}{2})} \expval{  \exp(-i \dfrac{\sigma_2 \beta}{2}) }{0} + e^{i\qty(\frac{\alpha + \gamma}{2})} \expval{ \exp(-i \dfrac{\sigma_2 \beta}{2}) }{1}\\
	\end{align*}
	We note that,
	\begin{align*}
	\exp(-i \dfrac{(\vu{n} \vdot \va{\sigma}) \beta}{2}) &= \sum_{n=0}^\infty  \dfrac{1}{2n!} \qty(-i \dfrac{ \beta}{2})^{2n} (\vu{n} \vdot \va{\sigma})^{2n} + \sum_{n=0}^\infty  \dfrac{1}{(2n+1)!} \qty(-i \dfrac{ \beta}{2})^{2n+1} (\vu{n} \vdot \va{\sigma})^{2n+1}\\
	&= \sum_{n=0}^\infty  \dfrac{(-1)^n}{2n!} \qty( \dfrac{ \beta}{2})^{2n} - \sum_{n=0}^\infty  \dfrac{(-1)^n}{(2n+1)!} \qty( \dfrac{ \beta}{2})^{2n+1} (\vu{n} \vdot \va{\sigma})\impliedby (\vu{n} \vdot \va{\sigma})^{2n} = 1\\
	\exp(-i \dfrac{(\vu{n} \vdot \va{\sigma}) \beta}{2}) &= \cos \dfrac{\beta}{2} - i \sin \dfrac{\beta}{2} \vu{n} \vdot \va{\sigma}\\
	\implies \expval{\exp(-i \dfrac{\sigma_2 \beta}{2})}{0} &= \expval{\exp(-i \dfrac{\sigma_2 \beta}{2})}{1} = \cos \dfrac{\beta}{2} \qq{and} \tr \qty[ \exp(-i \dfrac{(\vu{n} \vdot \va{\sigma}) \beta}{2})] = 2 \cos \dfrac{\beta}{2}
	\end{align*}
	Hence, we get,
	\begin{align*}
	\tr U &= \cos \dfrac{\beta}{2}(e^{i\qty(\frac{\alpha + \gamma}{2})} + e^{-i\qty(\frac{\alpha + \gamma}{2})} ) \\
	\cos \dfrac{\theta}{2} &= \cos \dfrac{\beta}{2}  \cos \dfrac{\alpha + \gamma}{2}
	\end{align*}
	$ \theta $ is given by the above equation.
\end{homeworkProblem}

\begin{homeworkProblem}[3]
	We know that,
	\begin{align*}
	J_1 = \dfrac{J_+ + J_-}{2} \qq{and} J_2 = \dfrac{J_+ - J_-}{2i} \qq{and} J_{\pm} \ket{l,m} = \sqrt{(l \mp m)(l\pm m + 1)} \ket{l,m\pm 1}
	\end{align*}
	As successive action of the form $ J_{\pm}^\alpha \ket{l,m} $ with integral $ \alpha  > 0 $ takes a state to one with higher/lower $ m $ , we can see that $ \ev{J_\pm^\alpha}{l,m} = 0  $.
	Lets consider $ \expval{J_1} $,
	\begin{align*}
	\expval{J_1} &= \ev{J_1}{l,m}\\
	&= \dfrac{1}{2}\qty(\ev{J_+}{l,m} + \ev{J_-}{l,m})\\
	&= 0
	\end{align*}
	
	Similarly for $ \ev{J_2} $,
	\begin{align*}
	\expval{J_2} &= \ev{J_2}{l,m}\\
	&= \dfrac{1}{2i}\qty(\ev{J_+}{l,m} - \ev{J_-}{l,m})\\
	&= 0
	\end{align*}
	
	Consider $ \ev{J_1^2} $ and $ \ev{J_2^2} $,
	\begin{align*}
	\ev{J_1^2} &= \dfrac{1}{4}(\ev{J_+^2} + \ev{J_-^2} + \acomm{J_+}{J_-}) = \dfrac{1}{4} \acomm{J_+}{J_-} \qq{and} \\
	\ev{J_2^2} &= \dfrac{1}{-4}(\ev{J_+^2} + \ev{J_-^2} - \acomm{J_+}{J_-}) = \dfrac{1}{4} \acomm{J_+}{J_-} \implies \ev{J_1^2} = \ev{J_2^2}
	\end{align*}
	We know,
	\begin{align*}
	\ev{J^2} &= l(l+1) \\
	\ev{J_1^2} + \ev{J_2^2} + \ev{J_3^2} &= l(l+1) \\
	2\ev{J_1^2} + m^2 &= l(l+1) \\
	\ev{J_1^2} = \ev{J_2^2} &= \dfrac{l(l+1) - m^2}{2}
	\end{align*} $ $
\end{homeworkProblem}

\begin{homeworkProblem}
	Let's use the follwing convention $ (\ket{l,m}) $
	\begin{equation*}
	\ket{1,1} = \mqty(1\\0\\0) \qq{and} \ket{1,0} = \mqty(0\\1\\0) \qq{and} \ket{1,-1} = \mqty(0\\0\\1)
	\end{equation*}
	We know that $ J_\pm\ket{1,m} = \sqrt{(1 \mp m)(1 \pm m + 1)} \ket{1,m\pm 1}  $, which means,
	\begin{align*}
	J_+ \ket{1,1} = 0 \qq{and} J_+ \ket{1,0} = \sqrt{2} \ket{1,1} \qq{and} J_+ \ket{1,-1} = \sqrt{2} \ket{1,0} \\
	J_- \ket{1,1} = \sqrt{2} \ket{1,0} \qq{and} J_- \ket{1,0} = \sqrt{2} \ket{1,-1} \qq{and} J_- \ket{1,-1} = 0\\
	\end{align*}
	Using the above relations, one can write $ J_+ $ and $ J_- $ as follows,
	\begin{align*}
		J_+ = \sqrt{2}\mqty(0 & 1 & 0 \\ 0 & 0 & 1 \\ 0 & 0 & 0) \qq{and} J_- = \sqrt{2} \mqty(0 & 0 & 0 \\ 1 & 0 & 0 \\ 0 & 1  & 0) \implies J_2 = \dfrac{i}{\sqrt{2}}\mqty(0 & -1 & 0 \\ 1 & 0 & -1 \\ 0 & 1  & 0)
	\end{align*}
	Thus, we have obtained $ J_2 $. Let's also note the following,
	\begin{equation*}
	J_2^2 = \dfrac{1}{2}\mqty(1 & 0 & -1\\ 0 & 2 & 0 \\ -1 & 0  & 1) \qq{and} J_2^3 = \dfrac{i}{\sqrt{2}}\mqty(0 & -1 & 0 \\ 1 & 0 & -1 \\ 0 & 1  & 0) = J_2
	\end{equation*}
	\begin{equation*}
	J_2^4 = J_2^3 J_2 = J_2^2 
	\end{equation*}
	We can see a pattern above, which can be written in a concise form as,
	\begin{equation*}
	J_2^{2n-1} = J_2 \qq{and} J_2^{2n} =  J_2^2
	\end{equation*}
	where $ n = 1,2,3,\ldots $. Consider $ e^{-iJ_2 \beta} $,
	\begin{align*}
	e^{-iJ_2 \beta} &= \sum_{n=0}^{\infty} \dfrac{(-i)^n \beta^n J_2^n}{n!}\\
	&= \sum_{n=0}^{\infty} \dfrac{(-i)^{2n} \beta^{2n} J_2^{2n}}{2n!} + \sum_{n=1}^{\infty} \dfrac{(-i)^{2n-1} \beta^{2n-1} J_2^{2n-1}}{(2n-1)!}\\
	&=1 +  \sum_{n=1}^{\infty} \dfrac{(-1)^{n} \beta^{2n} J_2^{2}}{2n!} + i \sum_{n=1}^{\infty} \dfrac{(-1)^{n} \beta^{2n-1} J_2}{(2n-1)!}\\
	&=1 +  (1-\cos \beta)J_2^2 - i J_2 \sin \beta
	\end{align*}
	\textit{Hence proved}.
\end{homeworkProblem}

\begin{homeworkProblem}
	\textit{Throughout the problem, we have assumed all relative phases to be $0$},
	
	We denote each representation by different subscripts, an example of which is shown below,
	\begin{equation*}
	\ket{2,2} = \ket{1,1}_1 \otimes \ket{1,1}_2
	\end{equation*}
	To get other states in $ j=2 $, we define the $ J_- $ operator as follows, and apply it successively,
	\begin{align*}
	J_- &= J_-^{(1)} \otimes I + I \otimes J_-^{(2)}\\
	\qq{where} J_- \ket{j,m} &= \sqrt{j(j+1) - m(m - 1)} \ket{j,m-1}
	\end{align*}
	Let's first consider $ \ket{2,1} $,
	\begin{align*}
	2\ket{2,1} &= J_- \ket{2,2}\\
	&= J_-^{(1)} \ket{1,1}_1 \otimes \ket{1,1}_2 + \ket{1,1}_1 \otimes J_-^{(2)}\ket{1,1}_2\\
	&= \sqrt{2} \ket{1,0}_1 \otimes \ket{1,1}_2 + \sqrt{2}  \ket{1,1}_1 \otimes \ket{1,0}_2 \\
	\ket{2,1} &= \dfrac{\ket{1,0}_1 \otimes \ket{1,1}_2 + \ket{1,1}_1 \otimes \ket{1,0}_2}{\sqrt{2}}
	\end{align*}
For $ \ket{2,0} $,
\begin{align*}
\sqrt{6}\ket{2,0} &= J_- \ket{2,1}\\
2\sqrt{3}\ket{2,0}&= J_-^{(1)} \ket{1,0}_1 \otimes \ket{1,1}_2 + J_-^{(1)} \ket{1,1}_1 \otimes \ket{1,0}_2 + \ket{1,0}_1 \otimes J_-^{(2)} \ket{1,1}_2 + \ket{1,1}_1 \otimes J_-^{(2)} \ket{1,0}_2\\
&= \sqrt{2}(\ket{1,-1}_1 \otimes \ket{1,1}_2 +  \ket{1,-1}_1 \otimes \ket{1,0}_2 + \ket{1,0}_1 \otimes \ket{1,0}_2 + \ket{1,1}_1 \otimes \ket{1,-1}_2) \\
\ket{2,0} &= \dfrac{\ket{1,-1}_1 \otimes \ket{1,1}_2 + 2 \ket{1,0}_1 \otimes \ket{1,0}_2 + \ket{1,1}_1 \otimes \ket{1,-1}_2}{\sqrt{6}}
\end{align*}
For $ \ket{2,-1} $,
\begin{align*}
\sqrt{6} \ket{2,-1} &= J_- \ket{2,0}\\
{6} \ket{2,-1}&= J_-^{(1)} \ket{1,-1}_1 \otimes \ket{1,1}_2 + 2 J_-^{(1)} \ket{1,0}_1 \otimes \ket{1,0}_2 + J_-^{(1)} \ket{1,1}_1 \otimes \ket{1,-1}_2 +\\
& \ket{1,-1}_1 \otimes J_-^{(2)}\ket{1,1}_2 + 2 \ket{1,0}_1 \otimes J_-^{(2)} \ket{1,0}_2 + \ket{1,1}_1 \otimes J_-^{(2)} \ket{1,-1}_2\\
&= 3\sqrt{2}(\ket{1,-1}_1 \otimes \ket{1,0}_2 + \ket{1,0}_1 \otimes \ket{1,-1}_2 )\\
\ket{2,-1} &= \dfrac{\ket{1,-1}_1 \otimes \ket{1,0}_2 + \ket{1,0}_1 \otimes \ket{1,-1}_2 }{\sqrt{2}}\\
\end{align*}
For $ \ket{2,-2} $,
\begin{align*}
2 \ket{2,-2} &= J_-\ket{2,-1}\\
2\sqrt{2}\ket{2,-2} &= J_-^{(1)} \ket{1,-1}_1 \otimes \ket{1,0}_2 + J_-^{(1)} \ket{1,0}_1 \otimes \ket{1,-1}_2  + \ket{1,-1}_1 \otimes J_-^{(2)} \ket{1,0}_2 + \ket{1,0}_1 \otimes J_-^{(2)} \ket{1,-1}_2\\
&= \sqrt{2}(\ket{1,-1}_1 \otimes \ket{1,-1}_2  + \ket{1,-1}_1 \otimes \ket{1,-1}_2)\\
\ket{2,-2} &= \ket{1,-1}_1 \otimes \ket{1,-1}_2 
\end{align*}

We now consider $ j=1 $. The highest weight state can be written as,
\begin{equation*}
\ket{1,1} = a \ket{1,0}_1 \otimes \ket{1,1}_2 + b \ket{1,1}_1 \otimes \ket{1,0}_2
\end{equation*}
Consider acting on this state with the operator $ J_+ =  J_+^{(1)} \otimes I + I \otimes J_+^{(2)} $,
\begin{align*}
a J_+^{(1)} \ket{1,0}_1 \otimes \ket{1,1}_2 + b \ket{1,1}_1 \otimes J_+^{(2)} \ket{1,0}_2 = 0\\
\implies a + b = 0
\end{align*}
Hence, we make a choice $ a = \dfrac{1}{\sqrt{2}}, b = - \dfrac{1}{\sqrt{2}} $,
\begin{equation*}
\therefore \ket{1,1} = \dfrac{1}{\sqrt{2}} \ket{1,0}_1 \otimes \ket{1,1}_2 - \dfrac{1}{\sqrt{2}} \ket{1,1}_1 \otimes \ket{1,0}_2
\end{equation*}
We apply a similar procedure as outlined previously for $ \ket{1,0} $
\begin{align*}
\sqrt{2} \ket{1,0} &= J_- \ket{1,1} \\
2 \ket{1,0} &= J_-^{(1)} \ket{1,0}_1 \otimes \ket{1,1}_2 -  J_-^{(1)} \ket{1,1}_1 \otimes \ket{1,0}_2 + \ket{1,0}_1 \otimes  J_-^{(2)} \ket{1,1}_2 - \ket{1,1}_1 \otimes J_-^{(2)} \ket{1,0}_2\\
&= \sqrt{2}( \ket{1,-1}_1 \otimes \ket{1,1}_2 -   \ket{1,0}_1 \otimes \ket{1,0}_2 + \ket{1,0}_1 \otimes   \ket{1,0}_2 - \ket{1,1}_1 \otimes \ket{1,-1}_2)\\
\ket{1,0} &= \dfrac{\ket{1,-1}_1 \otimes \ket{1,1}_2  - \ket{1,1}_1 \otimes \ket{1,-1}_2}{\sqrt{2}}
\end{align*}
For $ \ket{1,-1} $,
\begin{align*}
\sqrt{2} \ket{1,-1} &= J_- \ket{1,0}\\
2 \ket{1,-1} &=  J_-^{(1)}\ket{1,-1}_1 \otimes \ket{1,1}_2  - J_-^{(1)}\ket{1,1}_1 \otimes \ket{1,-1}_2 + \ket{1,-1}_1 \otimes  J_-^{(2)} \ket{1,1}_2  - \ket{1,1}_1 \otimes  J_-^{(2)} \ket{1,-1}_2\\
\ket{1,-1}&=  \dfrac{- \ket{1,0}_1 \otimes \ket{1,-1}_2 + \ket{1,-1}_1 \otimes   \ket{1,0}_2 }{\sqrt{2}}
\end{align*}
For $ j=0 $,
\begin{align*}
\ket{0,0} &= a \ket{1,1}\otimes \ket{1,-1} + b \ket{1,-1}\otimes \ket{1,1} + c \ket{1,0}\otimes \ket{1,0}\\
J_-\ket{0,0} &= a J_-^{(1)}\ket{1,1}\otimes \ket{1,-1} + b J_-^{(1)}\ket{1,-1}\otimes \ket{1,1} + c J_-^{(1)}\ket{1,0}\otimes \ket{1,0} \\&+ a \ket{1,1}\otimes J_-^{(1)} \ket{1,-1} + b \ket{1,-1}\otimes J_-^{(1)} \ket{1,1} + c \ket{1,0}\otimes J_-^{(1)}  \ket{1,0}\\
0 &=  a \ket{1,0}\otimes \ket{1,-1} + c \ket{1,-1}\otimes \ket{1,0} + b \ket{1,-1}\otimes \ket{1,0} + c \ket{1,0}\otimes \ket{1,-1}
\end{align*}
\begin{equation*}
\implies b+c =0 \qq{and} a+ c =0
\end{equation*}
We choose $ a=\dfrac{1}{\sqrt{3}},b=\dfrac{1}{\sqrt{3}},c=-\dfrac{1}{\sqrt{3}} $. Hence,
\begin{equation*}
\ket{0,0} = \dfrac{\ket{1,1}\otimes \ket{1,-1} +  \ket{1,-1}\otimes \ket{1,1} - \ket{1,0}\otimes \ket{1,0}}{\sqrt{3}}
\end{equation*}
\end{homeworkProblem}
%\pagebreak 	

\end{document}
