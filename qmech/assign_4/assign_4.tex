\documentclass{article}

\usepackage{fancyhdr}
\usepackage{extramarks}
\usepackage{amsmath}
\usepackage{amsthm}
\usepackage{amssymb}
\usepackage{amsfonts}
\usepackage{tikz}
\usepackage{physics}
\usepackage[plain]{algorithm}
\usepackage{algpseudocode}

\usetikzlibrary{automata,positioning}

%
% Basic Document Settings
%

\topmargin=-0.45in
\evensidemargin=0in
\oddsidemargin=0in
\textwidth=6.5in
\textheight=9.0in
\headsep=0.25in

\linespread{1.1}

\pagestyle{fancy}
\lhead{\hmwkAuthorName}
\chead{\hmwkClass\ : \hmwkTitle}
\rhead{\firstxmark}
\lfoot{\lastxmark}
\cfoot{\thepage}

\renewcommand\headrulewidth{0.4pt}
\renewcommand\footrulewidth{0.4pt}

\setlength\parindent{0pt}

%
% Create Problem Sections
%
\newcommand{\be}{\begin{equation}}
\newcommand{\ee}{\end{equation}}
\newcommand{\bes}{\begin{equation*}}
\newcommand{\ees}{\end{equation*}}
\newcommand{\bea}{\begin{flalign*}}
\newcommand{\eea}{\end{flalign*}}


\newcommand{\enterProblemHeader}[1]{
    \nobreak\extramarks{}{Problem \arabic{#1} continued on next page\ldots}\nobreak{}
    \nobreak\extramarks{Problem \arabic{#1} (continued)}{Problem \arabic{#1} continued on next page\ldots}\nobreak{}
}

\newcommand{\exitProblemHeader}[1]{
    \nobreak\extramarks{Problem \arabic{#1} (continued)}{Problem \arabic{#1} continued on next page\ldots}\nobreak{}
    \stepcounter{#1}
    \nobreak\extramarks{Problem \arabic{#1}}{}\nobreak{}
}

\setcounter{secnumdepth}{0}
\newcounter{partCounter}
\newcounter{homeworkProblemCounter}
\setcounter{homeworkProblemCounter}{1}
\nobreak\extramarks{Problem \arabic{homeworkProblemCounter}}{}\nobreak{}

%
% Homework Problem Environment
%
% This environment takes an optional argument. When given, it will adjust the
% problem counter. This is useful for when the problems given for your
% assignment aren't sequential. See the last 3 problems of this template for an
% example.
%
\newenvironment{homeworkProblem}[1][-1]{
    \ifnum#1>0
        \setcounter{homeworkProblemCounter}{#1}
    \fi
    \section{Problem \arabic{homeworkProblemCounter}}
    \setcounter{partCounter}{1}
    \enterProblemHeader{homeworkProblemCounter}
}{
    \exitProblemHeader{homeworkProblemCounter}
}

%
% Homework Details
%   - Title
%   - Due date
%   - Class
%   - Section/Time
%   - Instructor
%   - Author
%

\newcommand{\hmwkTitle}{Assignment\ \#4}
\newcommand{\hmwkDueDate}{Due on 8th November, 2018}
\newcommand{\hmwkClass}{Advanced Quantum Mechanics}
\newcommand{\hmwkClassTime}{}
\newcommand{\hmwkClassInstructor}{}
\newcommand{\hmwkAuthorName}{\textbf{Aditya Vijaykumar}}

%
% Title Page
%

\title{
    %\vspace{2in}
    \textmd{\textbf{\hmwkClass:\ \hmwkTitle}}\\
    \normalsize\vspace{0.1in}\small{\hmwkDueDate\ }\\
%    \vspace{3in}
}

\author{\hmwkAuthorName}
\date{}

\renewcommand{\part}[1]{\textbf{\large Part \Alph{partCounter}}\stepcounter{partCounter}\\}

%
% Various Helper Commands
%

% Useful for algorithms
\newcommand{\alg}[1]{\textsc{\bfseries \footnotesize #1}}

% For derivatives
\newcommand{\deriv}[1]{\frac{\mathrm{d}}{\mathrm{d}x} (#1)}

% For partial derivatives
\newcommand{\pderiv}[2]{\frac{\partial}{\partial #1} (#2)}

% Integral dx
\newcommand{\dx}{\mathrm{d}x}

% Alias for the Solution section header
\newcommand{\solution}{\textbf{\large Solution}}

% Probability commands: Expectation, Variance, Covariance, Bias
\newcommand{\E}{\mathrm{E}}
\newcommand{\Var}{\mathrm{Var}}
\newcommand{\Cov}{\mathrm{Cov}}
\newcommand{\Bias}{\mathrm{Bias}}

\begin{document}

\maketitle
	Let Hamiltonian $ H = H_0  + \lambda V$, $ H_0\ket{n^{0}} = E_n^{0}\ket{n^{0}}  $, $ H\ket{n} = E_n \ket{n} $. Let $ \Delta_n = E_n - E_n^{0} $ and $ \phi_n = 1 - \ket{n^{0}}\bra{n^{0}} $ be the projector onto the orthogonal space of $ \ket{n^{0}} $. $ \ket{n} $ and $ \Delta_n $ are given by,

	\begin{equation*}
	\ket{n} =\ket{n^{0}} + \dfrac{ \phi_n (\lambda V - \Delta_n)\ket{n}}{E_n^{0} - H_0} = \ket{n^{0}} +\sum_{k\ne n} \dfrac{ \lambda \mel{k^0}{V}{n} - \Delta_n \ip{k^{0}}{n}}{E_n^{0} - E_k^{0}}\ket{k^{0}} \qq{and} \Delta_n = \lambda \mel{n^{0}}{V}{n}
	\end{equation*}	
	We work with normalization $ \ip{n}{n^0} = 1 \implies \ip{n^j}{n^0} = 0 \qq{;} j\ne0$. 
	We assume the following,
	\begin{equation*}
	\Delta_n = \lambda \Delta^{1}_n + \lambda^2 \Delta_n^2 + \ldots \qq{and} \ket{n} = \ket{n^{0}} +  \lambda \ket{n^{1}} + \lambda^2 \ket{n^{2}} + \ldots
	\end{equation*}
	
	Putting these into the equations for $ \ket{n} $ and $ \Delta_{n} $ and equating order by order, we get,
	\begin{align}
	\Delta_n^{1} &= \ev{V}{n^0} \label{1}\\
	\ket{n^1} &= \sum_{k\ne n} \dfrac{ \mel{k^0}{V}{n^0} }{E_n^{0} - E_k^{0}}\ket{k^{0}}  \label{2}\\
	\Delta_n^2 &= \sum_{k\ne n} \dfrac{ \abs{\mel{k^0}{V}{n^0}}^2 }{E_n^{0} - E_k^{0}}\ket{k^{0}}  \label{3}\\
	\ket{n^2} &= \sum_{k\ne n} \dfrac{ \mel{k^0}{V}{n^1} - \Delta_n^1 \ip{k^{0}}{n^1}}{E_n^{0} - E_k^{0}}\ket{k^{0}} 
	\label{4}
	\end{align}
\begin{homeworkProblem}[1]
	We note that,
	\begin{align*}
	\ip{E_n}{E_n} &= \ip{E_n}{E_n^0} + \lambda\qty(\ip{E_n^1}{E_n^0} + \ip{E_n^0}{E_n^1})  + \lambda^2 \qty(\ip{E_n^2}{E_n^0} + \ip{E_n^0}{E_n^2} + \ip{E_n^1}{E_n^1}) \\
	&= 1 + \lambda^2 \qty(\ip{E_n^1}{E_n^1})
	\end{align*}
	One needs to find the following,
	\begin{align*}
	\dfrac{\ip{E_n^0}{E_n}}{\sqrt{\ip{E_n^1}{E_n^1}}} &= \dfrac{1}{\sqrt{1 + \lambda^2 \qty(\ip{E_n^1}{E_n^1})}}\\
	&= 1 - \dfrac{\lambda^2}{2}\sum_{k\ne n} \dfrac{\abs{\mel{E_k^0}{V}{E_n^0} }^2}{(E_n^{0} - E_k^{0})^2}
	\end{align*}
	where we have used (\ref{2}) in going to the last step. Hence, the required probability is $  1 - \dfrac{\lambda^2}{2}\sum_{k\ne n} \dfrac{\abs{\mel{E_k^0}{V}{E_n^0} }^2}{(E_n^{0} - E_k^{0})^2} $.
\end{homeworkProblem}








\begin{homeworkProblem}[2]
	\textbf{Part (a)}\\
	From the form of the Hamiltonian, we can see that the energy will have the form,
	\begin{equation*}
	E_{n_x,n_y} = (n_x + 0.5 + n_y + 0.5)\omega = (n_x + n_y + 1) \omega
	\end{equation*}
	The three lowest lying states are,
	\begin{align*}
	n_x = 0 \qq{,} n_y= 0 &\implies E_{00}^{(0)} = \omega\\
	n_x = 1 \qq{,} n_y= 0 &\implies E_{10}^{(0)} = 2 \omega\\
	n_x = 0 \qq{,} n_y= 1 &\implies E_{01}^{(0)} = 2 \omega
	\end{align*}
	\textbf{Part (b)}\\
	Let's denote states by $ \ket{n_x n_y} $. $ x $ and $ y $ can be written in terms of corresponding creation and annhilation operators as follows,
	\begin{equation*}
	x = \dfrac{1}{\sqrt{2m\omega}}(a_x + a_x^\dagger) \qq{and} y = \dfrac{1}{\sqrt{2m\omega}}(a_y + a_y^\dagger)
	\end{equation*}
	The perturbation is $ V = \lambda m \omega^2 x y $. Consider $ \mel{q_x q_y}{V}{n_x n_y} $	
	\begin{align*}
	\mel{q_x q_y}{V}{n_x n_y} &= \lambda m \omega^2\qty( \mel{q_x q_y}{a_x a_y}{n_x n_y} + \mel{q_x q_y}{a_x a_y^\dagger}{n_x n_y} + \mel{q_x q_y}{a_x^\dagger a_y ^\dagger}{n_x n_y} + \mel{q_x q_y}{a_x^\dagger a_y}{n_x n_y})\\
	&= \lambda m \omega^2( \sqrt{n_x n_y}\delta_{q_x,n_{x} -1} \delta_{q_y,n_{y} -1} + \sqrt{n_x (n_y+1)}\delta_{q_x,n_{x} -1} \delta_{q_y,n_{y} +1}\\
	& \qq{ }+ \sqrt{(n_x+1) (n_y+1)}\delta_{q_x,n_{x} +1} \delta_{q_y,n_{y} +1} + \sqrt{(n_x+1) (n_y)}\delta_{q_x,n_{x} +1} \delta_{q_y,n_{y}-1})\\
	\implies \ev{V}{n_xn_y} &= 0 \implies E_{n_x n_y}^{(1)} =0
	\end{align*}
	This means that there will be no energy shift at the first order in $ \lambda $ for any state under consideration. We now proceed to calculate $ \ket{n_x n_y^{(1)}} $,
	\begin{align*}
	 \ket{00^{(1)}} &= \sum_{(q_x,q_y)\ne(0,0)} \dfrac{\mel{q_x q_y}{V}{00}}{E_{00}^{(0)} - E_{q_x q_y}^{(0) }}\ket{q_xq_y}\\
	 &= \lambda m \omega^2 \sum_{(q_x,q_y)\ne(0,0)} \dfrac{ \delta_{q_x,1} \delta_{q_y,1}}{E_{00}^{(0)} - E_{q_x q_y}^{(0) }}\ket{q_xq_y}\\
	 \ket{00^{(1)}}&= -\dfrac{\lambda m \omega}{2}\ket{11}
	\end{align*}
	
	\begin{align*}
	\ket{10^{(1)}} &= \sum_{(q_x,q_y)\ne(1,0)} \dfrac{\mel{q_x q_y}{V}{10}}{E_{10}^{(0)} - E_{q_x q_y}^{(0) }}\ket{q_xq_y}\\
	&= \lambda m \omega^2 \sum_{(q_x,q_y)\ne(0,0)} \dfrac{ \delta_{q_x,1} \delta_{q_y,1}}{E_{00}^{(0)} - E_{q_x q_y}^{(0) }}\ket{q_xq_y}\\
	\ket{10^{(1)}}&= -\dfrac{\lambda m \omega}{2}\ket{11}
	\end{align*}
\end{homeworkProblem}









\begin{homeworkProblem}[3]
	We first note that $ x^2 - y^2 = r^2 \sin^2 \theta \cos 2 \phi$ when expressed in polar coordinates, and also the following eigenstates $ \psi_{n,l,m} $ of the hydrogen atom,
	\begin{equation*}
	\psi_{2,1,\pm 1}(r,\theta,\phi) = \dfrac{1}{8\sqrt{\pi}a_0^{5/2}}r e^{-\frac{2r}{a_0}} \sin\theta e^{\pm i\phi} \qq{and} \psi_{2,1,0}(r,\theta,\phi) = \dfrac{\sqrt{2}}{8\sqrt{\pi}a_0^{5/2}}r e^{-\frac{2r}{a_0}} \cos\theta
	\end{equation*}
	The perturbing Hamiltonian is $ V = \lambda (x^2 - y^2) = \lambda r^2 \sin^2 \theta \cos 2 \phi = \lambda V' $
	
	The first order correction for $ m=\pm 1 $ is given by,
	\begin{align*}
	\Delta_{\pm 1} &= -\int_{0}^{2\pi} \int_{0}^{\pi} \int_{0}^{\infty}\psi_{2,1,\pm 1}^* \psi_{2,1,\pm 1} V' r^2 d(\cos \theta) dr d\theta d\phi\\
	&= -\dfrac{1}{64\pi a_0^{5}}\int_{0}^{2\pi} \cos 2 \phi d\phi\int_{0}^{\pi} \sin^4 \theta d(cos\theta) \int_{0}^{\infty}r^6 e^{-\frac{4r}{a_0}}dr
	&= 0
	\end{align*}
\end{homeworkProblem}













\begin{homeworkProblem}[4]
	content...
\end{homeworkProblem}
















\begin{homeworkProblem}[5]
	Let $ L^2 = L_x^2 + L_y^2 + L_z^2 $. We work in the basis of states $ \ket{l,m} $ such that $ L^2 \ket{l,m} = l(l+1)\ket{l,m} $ and $ L_z \ket{l,m} = m \ket{l,m} $. The Hamiltonian then is,
	\begin{equation*}
	H = H_0 + \lambda V = AL^2 + BL_z + \lambda C L_y
	\end{equation*}
	The eigenstates of $ H_0 $ are,
	\begin{equation*}
	H_0 \ket{l,m} = \qty(Al(l+1) + Bm)\ket{l,m} = E_{lm}\ket{l,m}
	\end{equation*}
	
	For future use, let's evaluate $ \mel{l',m'}{V}{l,m} $, 
	\begin{align*}
	\mel{l',m'}{V}{l,m} &= C \mel{l',m'}{L_y}{l,m}\\
	&= \dfrac{C}{2i} \mel{l',m'}{L_+ - L_-}{l,m}\\
	&= \dfrac{C}{2i} \qty(\sqrt{l(l+1) - m(m+1)} \delta_{l',l} \delta_{m',m+1} - \sqrt{l(l+1) - m(m-1)} \delta_{l',l} \delta_{m',m-1})\\
	\abs{\mel{l',m'}{V}{l,m}}^2 &=  \dfrac{C^2}{4} \qty([{l(l+1) - m(m+1)} ]\delta_{l',l} \delta_{m',m+1} + [{l(l+1) - m(m-1)}] \delta_{l',l} \delta_{m',m-1})
	\end{align*}
	The first order energy shift is given by,
	\begin{align*}
	\Delta_{lm}^{(1)} &= \ev{V}{l,m}\\
	&=  \dfrac{C}{2i} \qty(\sqrt{l(l+1) - m(m+1)} \delta_{l,l} \delta_{m,m+1} - \sqrt{l(l+1) - m(m-1)} \delta_{l,l} \delta_{m,m-1})\\
	\Delta_{lm}^{(1)} & = 0
	\end{align*}
	Then one needs to find higher order energy shifts. Considering $ \Delta_{lm}^2$ and using (\ref{3}),
	\begin{align*}
	\Delta_{lm}^2 &= \sum_{l\ne l', m\ne m'}\dfrac{ \abs{\mel{l',m'}{V}{l,m}}^2}{E_{lm} - E_{l'm'}}\\
	 &= \dfrac{C^2}{4} \sum_{l\ne l', m\ne m'}\dfrac{\qty([{l(l+1) - m(m+1)} ]\delta_{l',l} \delta_{m',m+1} + [{l(l+1) - m(m-1)}] \delta_{l',l} \delta_{m',m-1})}{Al(l+1) + Bm - Al'(l'+1) - Bm'}\\
	 &= \dfrac{C^2}{4}\qty( \dfrac{-[{l(l+1) - m(m+1)}]}{B} + \dfrac{[{l(l+1) - m(m - 1)}]}{B} )\\
	 &= \dfrac{mc^2}{2B}
	\end{align*}
	
\end{homeworkProblem}














\begin{homeworkProblem}[6]
	The Hamiltonian to deal with is,
	\begin{equation*}
	H = \dfrac{p^2}{2m} + \dfrac{m\omega^2}{2}x^2
	\end{equation*}
	and the given trial wavefunction is,
	\begin{equation*}
	\psi_\beta(x) = N e^{-\beta\abs{x}}
	\end{equation*}
	where $ N $ is some normalization.
	Let's calculate $ \expval{H}{\psi_\beta} $,
	\begin{align*}
	\expval{H}{\psi_\beta} &= N^2 \int_{-\infty}^{\infty} e^{-2\beta\abs{x}}\qty(-\dfrac{1}{2m}\beta^2 + \dfrac{m\omega^2}{2}x^2)dx\\
	&=  \lim\limits_{\epsilon \rightarrow 0} N^2\qty[ 2\int_{\epsilon}^{\infty} e^{-2\beta{x}}\qty(-\dfrac{1}{2m}\beta^2 + \dfrac{m\omega^2}{2}x^2)dx + \int_{-\epsilon}^{\epsilon} e^{-2\beta\abs{x}}\qty(-\dfrac{1}{2m}\beta^2 + \dfrac{m\omega^2}{2}x^2)dx]\\
	&=  N^2\qty[ \qty(-\dfrac{\beta}{2m} + \dfrac{m \omega^2}{4\beta^3})+ \lim\limits_{\epsilon \rightarrow 0}\int_{-\epsilon}^{\epsilon} e^{-2\beta\abs{x}}\qty(-\dfrac{1}{2m}\beta^2 + \dfrac{m\omega^2}{2}x^2)dx]\\
	\end{align*}
	
	\begin{align*}
	N^2 \int_{-\infty}^{\infty}e^{-2\beta\abs{x}} = 1\\
	N^2 = \beta
	\end{align*}
	
	\begin{equation*}
	\expval{H}{\psi_\beta}  = -\dfrac{\beta^2}{2m} +  \dfrac{m\omega^2 }{4 \beta^2}
	\end{equation*}
\end{homeworkProblem}
\end{document}
