\documentclass{article}

\usepackage{fancyhdr}
\usepackage{extramarks}
\usepackage{amsmath}
\usepackage{amsthm}
\usepackage{amssymb}
\usepackage{amsfonts}
\usepackage{tikz}
\usepackage{physics}
\usepackage[plain]{algorithm}
\usepackage{algpseudocode}

\usetikzlibrary{automata,positioning}

%
% Basic Document Settings
%

\topmargin=-0.45in
\evensidemargin=0in
\oddsidemargin=0in
\textwidth=6.5in
\textheight=9.0in
\headsep=0.25in

\linespread{1.1}

\pagestyle{fancy}
\lhead{\hmwkAuthorName}
\chead{\hmwkClass\ : \hmwkTitle}
\rhead{\firstxmark}
\lfoot{\lastxmark}
\cfoot{\thepage}

\renewcommand\headrulewidth{0.4pt}
\renewcommand\footrulewidth{0.4pt}

\setlength\parindent{0pt}

%
% Create Problem Sections
%
\newcommand{\be}{\begin{equation}}
\newcommand{\ee}{\end{equation}}
\newcommand{\bes}{\begin{equation*}}
\newcommand{\ees}{\end{equation*}}
\newcommand{\bea}{\begin{flalign*}}
\newcommand{\eea}{\end{flalign*}}


\newcommand{\enterProblemHeader}[1]{
    \nobreak\extramarks{}{Problem \arabic{#1} continued on next page\ldots}\nobreak{}
    \nobreak\extramarks{Problem \arabic{#1} (continued)}{Problem \arabic{#1} continued on next page\ldots}\nobreak{}
}

\newcommand{\exitProblemHeader}[1]{
    \nobreak\extramarks{Problem \arabic{#1} (continued)}{Problem \arabic{#1} continued on next page\ldots}\nobreak{}
    \stepcounter{#1}
    \nobreak\extramarks{Problem \arabic{#1}}{}\nobreak{}
}

\setcounter{secnumdepth}{0}
\newcounter{partCounter}
\newcounter{homeworkProblemCounter}
\setcounter{homeworkProblemCounter}{1}
\nobreak\extramarks{Problem \arabic{homeworkProblemCounter}}{}\nobreak{}

%
% Homework Problem Environment
%
% This environment takes an optional argument. When given, it will adjust the
% problem counter. This is useful for when the problems given for your
% assignment aren't sequential. See the last 3 problems of this template for an
% example.
%
\newenvironment{homeworkProblem}[1][-1]{
    \ifnum#1>0
        \setcounter{homeworkProblemCounter}{#1}
    \fi
    \section{Problem \arabic{homeworkProblemCounter}}
    \setcounter{partCounter}{1}
    \enterProblemHeader{homeworkProblemCounter}
}{
    \exitProblemHeader{homeworkProblemCounter}
}

%
% Homework Details
%   - Title
%   - Due date
%   - Class
%   - Section/Time
%   - Instructor
%   - Author
%

\newcommand{\hmwkTitle}{Assignment\ \#4}
\newcommand{\hmwkDueDate}{Due on 8th November, 2018}
\newcommand{\hmwkClass}{Advanced Quantum Mechanics}
\newcommand{\hmwkClassTime}{}
\newcommand{\hmwkClassInstructor}{}
\newcommand{\hmwkAuthorName}{\textbf{Aditya Vijaykumar}}

%
% Title Page
%

\title{
    %\vspace{2in}
    \textmd{\textbf{\hmwkClass:\ \hmwkTitle}}\\
    \normalsize\vspace{0.1in}\small{\hmwkDueDate\ }\\
%    \vspace{3in}
}

\author{\hmwkAuthorName}
\date{}

\renewcommand{\part}[1]{\textbf{\large Part \Alph{partCounter}}\stepcounter{partCounter}\\}

%
% Various Helper Commands
%

% Useful for algorithms
\newcommand{\alg}[1]{\textsc{\bfseries \footnotesize #1}}

% For derivatives
\newcommand{\deriv}[1]{\frac{\mathrm{d}}{\mathrm{d}x} (#1)}

% For partial derivatives
\newcommand{\pderiv}[2]{\frac{\partial}{\partial #1} (#2)}

% Integral dx
\newcommand{\dx}{\mathrm{d}x}

% Alias for the Solution section header
\newcommand{\solution}{\textbf{\large Solution}}

% Probability commands: Expectation, Variance, Covariance, Bias
\newcommand{\E}{\mathrm{E}}
\newcommand{\Var}{\mathrm{Var}}
\newcommand{\Cov}{\mathrm{Cov}}
\newcommand{\Bias}{\mathrm{Bias}}

\begin{document}

\maketitle
(\textbf{Acknowledgements} - I would like to thank Chandramouli Chowdhury, Biprarshi Chakraborty and Junaid Majeed for discussions.)
\\

	Let Hamiltonian $ H = H_0  + \lambda V$, $ H_0\ket{n^{0}} = E_n^{0}\ket{n^{0}}  $, $ H\ket{n} = E_n \ket{n} $. Let $ \Delta_n = E_n - E_n^{0} $ and $ \phi_n = 1 - \ket{n^{0}}\bra{n^{0}} $ be the projector onto the orthogonal space of $ \ket{n^{0}} $. $ \ket{n} $ and $ \Delta_n $ are given by,

	\begin{equation*}
	\ket{n} =\ket{n^{0}} + \dfrac{ \phi_n (\lambda V - \Delta_n)\ket{n}}{E_n^{0} - H_0} = \ket{n^{0}} +\sum_{k\ne n} \dfrac{ \lambda \mel{k^0}{V}{n} - \Delta_n \ip{k^{0}}{n}}{E_n^{0} - E_k^{0}}\ket{k^{0}} \qq{and} \Delta_n = \lambda \mel{n^{0}}{V}{n} 
	\end{equation*}	
	We work with normalization $ \ip{n}{n^0} = 1 \implies \ip{n^j}{n^0} = 0 \qq{;} j\ne0$. 
	We assume the following,
	\begin{equation*}
	\Delta_n = \lambda \Delta^{1}_n + \lambda^2 \Delta_n^2 + \ldots \qq{and} \ket{n} = \ket{n^{0}} +  \lambda \ket{n^{1}} + \lambda^2 \ket{n^{2}} + \ldots
	\end{equation*}
	
	Putting these into the equations for $ \ket{n} $ and $ \Delta_{n} $ and equating order by order, we get,
	\begin{align}
	\Delta_n^{1} &= \ev{V}{n^0} \label{1}\\
	\ket{n^1} &= \sum_{k\ne n} \dfrac{ \mel{k^0}{V}{n^0} }{E_n^{0} - E_k^{0}}\ket{k^{0}}  \label{2}\\
	\Delta_n^2 &= \sum_{k\ne n} \dfrac{ \abs{\mel{k^0}{V}{n^0}}^2 }{E_n^{0} - E_k^{0}}  \label{3}\\
	\ket{n^2} &= \sum_{k\ne n} \dfrac{ \mel{k^0}{V}{n^1} - \Delta_n^1 \ip{k^{0}}{n^1}}{E_n^{0} - E_k^{0}}\ket{k^{0}} 
	\label{4}
	\end{align}
\begin{homeworkProblem}[1]
	We note that,
	\begin{align*}
	\ip{E_n}{E_n} &= \ip{E_n}{E_n^0} + \lambda\qty(\ip{E_n^1}{E_n^0} + \ip{E_n^0}{E_n^1})  + \lambda^2 \qty(\ip{E_n^2}{E_n^0} + \ip{E_n^0}{E_n^2} + \ip{E_n^1}{E_n^1}) \\
	&= 1 + \lambda^2 \qty(\ip{E_n^1}{E_n^1})
	\end{align*}
	One needs to find the following,
	\begin{align*}
	\dfrac{\ip{E_n^0}{E_n}}{\sqrt{\ip{E_n}{E_n}\ip{E_n^0}{E_n^0}}} &= \dfrac{1}{\sqrt{1 + \lambda^2 \qty(\ip{E_n^1}{E_n^1})}}\\
	&= 1 - \dfrac{\lambda^2}{2}\sum_{k\ne n} \dfrac{\abs{\mel{E_k^0}{V}{E_n^0} }^2}{(E_n^{0} - E_k^{0})^2}
	\end{align*}
	where we have used (\ref{2}) in going to the last step. Hence, the required probability is $  1 - \dfrac{\lambda^2}{2}\sum_{k\ne n} \dfrac{\abs{\mel{E_k^0}{V}{E_n^0} }^2}{(E_n^{0} - E_k^{0})^2} $.
\end{homeworkProblem}








\begin{homeworkProblem}[2]
	\textbf{Part (a)}\\
	From the form of the Hamiltonian, we can see that the energy will have the form,
	\begin{equation*}
	E_{n_x,n_y} = (n_x + 0.5 + n_y + 0.5)\omega = (n_x + n_y + 1) \omega
	\end{equation*}
	The three lowest lying states are,
	\begin{align*}
	n_x = 0 \qq{,} n_y= 0 &\implies E_{00}^{(0)} = \omega\\
	n_x = 1 \qq{,} n_y= 0 &\implies E_{10}^{(0)} = 2 \omega\\
	n_x = 0 \qq{,} n_y= 1 &\implies E_{01}^{(0)} = 2 \omega
	\end{align*}
	We see that there is a double-degeneracy with energy $ 2\omega $.
	\\
	
	\textbf{Part (b)}\\
	Let's denote states by $ \ket{n_x n_y} $. $ x $ and $ y $ can be written in terms of corresponding creation and annhilation operators as follows,
	\begin{equation*}
	x = \dfrac{1}{\sqrt{2m\omega}}(a_x + a_x^\dagger) \qq{and} y = \dfrac{1}{\sqrt{2m\omega}}(a_y + a_y^\dagger)
	\end{equation*}
	The perturbation is $ V = \lambda m \omega^2 x y $. Let's denote the $m$-th order energy shift by $ \Delta^m_{n_x n_y} $. Let's first consider $ \ket{00} $. The zeroth order energy eigenstate is given by,
	\begin{equation*}
	\psi_{00}^0 = \psi_0 (x) \psi_0 (y) = \sqrt{\dfrac{m\omega}{\pi}} e^{-\frac{m\omega}{2}(x^2 + y^2)}
	\end{equation*}
	Consider $\Delta^1_{00} $,
	\begin{align*}
	\Delta^1_{00} &= \ev{V}{00}\\
	&= \dfrac{m\omega^2}{2m\omega} \qty( \ev{a_x a_y}{00} + \ev{a_x a_y^\dagger}{00} + \ev{a_x^\dagger a_y ^\dagger}{00} + \ev{a_x^\dagger a_y}{00})\\
	\Delta^1_{00} &= 0
	\end{align*}
	The states $ \ket{10} $ and $ \ket{01} $ are degenerate, and hence we need to apply degenerate perturbation formalism. For this, we construct the matrix elements of $ V $ between the degenerate states. As we have seen in the $ \ket{00} $ case, the operator $ xy $ changes a given state to one which has at least one of the quantum numbers different (by 1). Hence $\ev{V}{01} = \ev{V}{10} = 0$. We proceed to calculate $ \mel{01}{V}{10} $,
	\begin{align*}
	\mel{01}{V}{10} &= \dfrac{\lambda m\omega^2}{2m \omega} \qty(\mel{01}{a_x a_y}{10} + \mel{01}{a_x a_y^\dagger}{10} + \mel{01}{a_x^\dagger a_y ^\dagger}{10} + \mel{01}{a_x^\dagger a_y}{10})\\
	&= \dfrac{\lambda \omega }{2}\qty(0+ 1+ 0 + 0)\\
	\mel{01}{V}{10} &= \dfrac{\lambda \omega }{2}\\
	\implies \mel{10}{V}{01} &= \dfrac{\lambda \omega }{2}\\
	\therefore V &= \dfrac{\lambda \omega }{2} \mqty(\admat[0]{1,1})
	\end{align*}

	The eigenvalues of the above matrix are $ \pm \dfrac{\lambda \omega }{2} \implies E = 2\omega \pm \lambda \dfrac{\omega}{2}$ with eigenstates $ \dfrac{1}{\sqrt{2}}(\ket{10} \pm \ket{01}) $
	\\
	\textbf{Part (c)}\\
	The aim is to now solve for the Hamiltonian exactly.
	\begin{align*}
	H &= \dfrac{p_x^2}{2m} + \dfrac{p_y^2}{2m} + \dfrac{m\omega^2}{2}(x^2 + y^2 + 2\lambda xy)\\
	&= \dfrac{p_x^2}{2m} + \dfrac{p_y^2}{2m} + \dfrac{m\omega^2}{4}((x+ y )^2 + (x-y)^2+ \lambda[(x+y)^2 - (x-y)^2])\\
	&= \dfrac{p_x^2}{2m} + \dfrac{p_y^2}{2m} + \dfrac{m\omega^2}{2}\qty[\qty(\dfrac{x+ y }{\sqrt{2}})^2(1+\lambda) + \qty(\dfrac{x-y}{\sqrt{2}})^2(1-\lambda)]
	\end{align*}
	Let $ \alpha = \dfrac{x+ y }{\sqrt{2}} $ and $ \beta = \dfrac{x - y }{\sqrt{2}} $. We can see that $ \dot{\alpha}^2 + \dot{\beta}^2 = \dot{x}^2 + \dot{y}^2 \implies p_\alpha^2 + p_\beta^2 = p_x^2 + p_y^2$. In the new coordinates, the Hamiltonian is,
	\begin{equation*}
	H = \dfrac{p_\alpha^2}{2m} + \dfrac{m(\omega\sqrt{1+\lambda})^2}{2}\alpha^2 +  \dfrac{p_\beta^2}{2m} + \dfrac{m(\omega\sqrt{1- \lambda})^2}{2}\beta^2
	\end{equation*}
	We have essentially decoupled the Hamiltonian into two harmonic oscillators of frequencies $ \omega_1 = \omega\sqrt{1+\lambda} $ and $ \omega_2 = \omega\sqrt{1 - \lambda} $. The three lowest energies are,
	\begin{align*}
	\dfrac{\omega(\sqrt{1+\lambda} + \sqrt{1 - \lambda})}{2} &= \omega + \order{\lambda^2}\\
	\dfrac{\omega(3\sqrt{1+\lambda} + \sqrt{1 - \lambda})}{2} &= 2\omega +  \lambda \dfrac{\omega}{2} + \order{\lambda^2}\\
	\dfrac{\omega(\sqrt{1+\lambda} + 3\sqrt{1 - \lambda})}{2} &= 2\omega - \lambda \dfrac{\omega}{2} + \order{\lambda^2}
	\end{align*}
	We see that the above values match with those calculated from perturbation theory.
	\end{homeworkProblem}









\begin{homeworkProblem}[3]
	We first note that $ x^2 - y^2 = r^2 \sin^2 \theta \cos 2 \phi = r^2 \sin^2 \theta \dfrac{e^{2i\phi} + e^{-2i\phi}}{2}$ when expressed in polar coordinates. As we are dealing with states that differ only in their $ m $ values, we label the states as $ \ket{m} $. We note the following eigenstates $ \psi_{n,l,m} $ of the hydrogen atom,
	\begin{equation*}
	\psi_{2,1,\pm 1}(r,\theta,\phi) = \ket{\pm 1} = \dfrac{1}{8\sqrt{\pi}a_0^{5/2}}r e^{-\frac{2r}{a_0}} \sin\theta e^{\pm i\phi} \qq{and} \psi_{2,1,0}(r,\theta,\phi) = \ket{0} = \dfrac{\sqrt{2}}{8\sqrt{\pi}a_0^{5/2}}r e^{-\frac{2r}{a_0}} \cos\theta
	\end{equation*}
	The above eigenstates are degenerate. The perturbing Hamiltonian is $ V = \lambda (x^2 - y^2) = \lambda r^2 \sin^2 \theta \cos 2 \phi = \lambda V' $. As we are dealing with states that differ only in their $ m $ values, we label the states as $ \ket{m} $. As in the previous problem, we proceed to construct the matrix elements of $ V' $. In each element $ \mel{p}{V}{q} $, the $ \phi $ integral will be $ \sim \int_{0}^{2\pi} e^{i(q - p)\phi} (e^{2i\phi} + e^{-2i\phi}) d\phi$. We can see that this integral will be zero unless $ q - p = \pm 2 $. Hence, only terms where $ p -q = \pm 2 $ will contribute, ie $ p=1, q = -1 $ and $ p=-1 , q=1 $. Let's evaluate $ \mel{-1}{V'}{1} $,
	\begin{align*}
	\mel{-1}{V'}{1} &= -\dfrac{1}{64\pi a_0^{5}} \int_{0}^{2\pi} \dfrac{e^{4i\phi} + 1}{2} d\phi\int_{0}^{\pi} \sin^4 \theta d(\cos\theta) \int_{0}^{\infty}r^6 e^{-\frac{4r}{a_0}}dr\\
	&= -\dfrac{1}{64\pi a_0^{5}} (\pi) \int_{0}^{\pi}  (1 - \cos^2\theta)^2  d(\cos\theta) \int_{-\infty}^0 \qty(\dfrac{a_0}{4})^7t^6 e^{t}dr \qq{ } \qty(\impliedby t = - \dfrac{4r}{a_0})\\
	&= -\dfrac{1}{64\pi a_0^{5}} (\pi) \int_{0}^{\pi} (\cos^4 \theta - 2 \cos^2 \theta + 1)  d(\cos\theta) \qty[\qty(\dfrac{a_0}{4})^7 6!] \qq{ } \\
	&= -\dfrac{ a_0^{2}}{64} \qty(\dfrac{-2}{5} + \dfrac{4}{3} - 2 ) \qty[\qty(\dfrac{45}{2^{10}})] = \dfrac{ a_0^{2}}{64} \qty(\dfrac{16}{15}) \qty[\qty(\dfrac{45}{2^{10}})] = \dfrac{3}{2^{12}}a_0^2 = \alpha
	\end{align*}
	\begin{equation*}
	\therefore \mel{-1}{V'}{1} = \mel{-1}{V'}{1} = \alpha  = \dfrac{3}{2^{12}}a_0^2
	\end{equation*}
	\begin{equation*}
	V' = \mqty(0 & 0 & \alpha\\0 & 0 & 0 \\\alpha & 0 & 0)
	\end{equation*}
	The eigenvalues of $ V' $ are $ 0 $, $ \pm \alpha $ and the corresponding eigenstates are $ \ket{0} $, $ \dfrac{\ket{1} \pm \ket{-1}}{\sqrt{2}} $ respectively. So the first order energy shifts are $ 0 $, $ \pm \alpha $.
\end{homeworkProblem}











\begin{homeworkProblem}[4]
	\begin{equation*}
	H = \mqty(E_1 & 0 & \lambda a\\ 0 & E_1 & \lambda b \\ \lambda a^* & \lambda b^* & E_2) = \underbrace{\mqty(\dmat[0]{E_1, E_1, E_2})}_{H_0} + \lambda \underbrace{\mqty(0 & 0 & a\\ 0 & 0 &  b \\  a^* &  b^* & 0)}_V
	\end{equation*}
	$ H $ has the following three eigenvalues.
	\begin{align*}
	\dfrac{{E_1}+{E_2}}{2} \pm \sqrt{\lambda^2 \left(a^2+b^2\right)+\qty(\dfrac{{E_1}-{E_2}}{2})^2} \qq{and} E_1\\
	\implies \approx E_1 + \lambda^2 \dfrac{\abs{a}^2 + \abs{b}^2}{E_1 - E_2} \qq{and} \approx E_2 - \lambda^2 \dfrac{\abs{a}^2 + \abs{b}^2}{E_1 - E_2} \qq{and} E_1
	\end{align*}
	We also notice that $ H_0 $ has a two-fold degeneracy with energy $ E_1 $, corresponding to the eigenstates $ \ket{0} = \mqty[1 & 0 & 0]^T $ and $ \ket{1} = \mqty[0 & 1 & 0]^T $, as well as energy eigenstate $ \ket{0} = \mqty[0 & 0 & 1]^T  $ with energy $ E_2 $. We also note that the block in $ V $ corresponding to these eigenstates in made up of zeros, hence there is no hope of breaking the degeneracy at first order in perturbation theory.
	
	Now we know that provided we choose the correct basis, $ \Delta_n^2 = \mel{n^1}{V}{n^0} $. For this, let's calculate the first order perturbed eigenstate. We have $ (H_0 + \lambda V) \ket{l_i} = E_{d_i} \ket{l_i}$. Substitute $ \ket{l_i} = \ket{l_i^0} + \lambda \ket{l_i^1} + \ldots $ (where $ \ket{l_i} = c_0\ket{0} + c_1 \ket{1} $) and $  E_{d_i} = E_1 +  \lambda E_{d_i}^1 + \ldots$. At first order, the equation then becomes,
	\begin{equation*}
	H_0 \ket{l_i^1} + V \ket{l_i^0} = E_1 \ket{l_i^1} + E_{d_i}^1 \ket{l_i^0}
	\end{equation*}
	But as degeneracy is not removed as first order, $ E_{d_i}^1 = 0 $. Taking the inner product with $ l_i^{0} $
	\begin{equation*}
	V \ket{l_i^0} = (E_1 - H_0) \ket{l_i^1} 
	\end{equation*}
	Taking the inner product with $ \bra{l_i^0} $, we get that $ \ev{V}{{l_i^0}}  = 0 \implies$ that $ V \ket{l_i^0} $ has no component along $ \ket{l_i^0} $. Then we can write,
	\begin{equation*}
	\ket{l_i^1} =  \dfrac{1}{E_1 - H_0} V\ket{l_i^0} \implies \mel{l_i^1}{V}{l_i^0} = \ev{V\dfrac{1}{E_1 - H_0} V}{l_i^0}
	\end{equation*}
	Now all that is left to do is diagonalize $G = V\dfrac{1}{E_1 - H_0} V$. We note,
	\begin{equation*}
	V \ket{0} = a^* \ket{2} \qq{and} V \ket{1} = b^* \ket{2}
	\end{equation*}
	We can use these to construct the matrix elements,
	\begin{align*}
	\mel{0}{G}{0} &= \abs{a}^2 \ev{\dfrac{1}{E_1 - H_0}}{2} =\dfrac{ \abs{a}^2 }{E_1 - E_2}\\
	\mel{0}{G}{1} = \dfrac{ab^*}{E_1 - E_2} &\qq{ } \mel{1}{G}{1} = \dfrac{\abs{b}^2}{E_1 - E_2} \qq{ } \mel{1}{G}{0} = \dfrac{a^* b}{E_1 - E_2}\\
	\therefore G &= \dfrac{1}{E_1-E_2}\mqty(\abs{a}^2 & ab^*\\ a^*b & \abs{b}^2)
	\end{align*}
	The eigenvalues of $ G $ are $ 0 $ and $ \dfrac{\abs{a}^2 + \abs{b}^2}{E_1 - E_2}  $ and the corresponding eigenstates are $ \dfrac{-b \ket{0} + a \ket{1}}{\sqrt{\abs{a}^2 + \abs{b}^2}} $ and $ \dfrac{a \ket{0} + b\ket{1}}{\sqrt{\abs{a}^2 + \abs{b}^2}} $. 
	
	We now look at the state with energy $ E_2 $. The second order perturbed energy for this case is straighforward,
	\begin{equation*}
	\Delta^2  =\dfrac{ \abs{\mel{0}{V}{2}}^2  + \abs{\mel{1}{V}{2}}^2}{E_2 - E_1} = - \dfrac{\abs{a}^2 + \abs{b}^2}{E_1 - E_2}
	\end{equation*}
	
	So then, the final energies upto second order in $ \lambda $ are,
	\begin{equation*}
	 E_1 \qq{,} E_1 + \lambda^2 \dfrac{\abs{a}^2 + \abs{b}^2}{E_1 - E_2} \qq{and} E_2 - \lambda^2 \dfrac{\abs{a}^2 + \abs{b}^2}{E_1 - E_2}
	\end{equation*}
	which is commensurate with the actual eigenvalues upto second order.
	
	
\end{homeworkProblem}
















\begin{homeworkProblem}[5]
	Let $ L^2 = L_x^2 + L_y^2 + L_z^2 $. We work in the basis of states $ \ket{l,m} $ such that $ L^2 \ket{l,m} = l(l+1)\ket{l,m} $ and $ L_z \ket{l,m} = m \ket{l,m} $. The Hamiltonian then is,
	\begin{equation*}
	H = H_0 + \lambda V = AL^2 + BL_z + \lambda C L_y
	\end{equation*}
	The eigenstates of $ H_0 $ are,
	\begin{equation*}
	H_0 \ket{l,m} = \qty(Al(l+1) + Bm)\ket{l,m} = E_{lm}\ket{l,m}
	\end{equation*}
	
	For future use, let's evaluate $ \mel{l',m'}{V}{l,m} $, 
	\begin{align*}
	\mel{l',m'}{V}{l,m} &= C \mel{l',m'}{L_y}{l,m}\\
	&= \dfrac{C}{2i} \mel{l',m'}{L_+ - L_-}{l,m}\\
	&= \dfrac{C}{2i} \qty(\sqrt{l(l+1) - m(m+1)} \delta_{l',l} \delta_{m',m+1} - \sqrt{l(l+1) - m(m-1)} \delta_{l',l} \delta_{m',m-1})\\
	\abs{\mel{l',m'}{V}{l,m}}^2 &=  \dfrac{C^2}{4} \qty([{l(l+1) - m(m+1)} ]\delta_{l',l} \delta_{m',m+1} + [{l(l+1) - m(m-1)}] \delta_{l',l} \delta_{m',m-1})
	\end{align*}
	The first order energy shift is given by,
	\begin{align*}
	\Delta_{lm}^{(1)} &= \ev{V}{l,m}\\
	&=  \dfrac{C}{2i} \qty(\sqrt{l(l+1) - m(m+1)} \delta_{l,l} \delta_{m,m+1} - \sqrt{l(l+1) - m(m-1)} \delta_{l,l} \delta_{m,m-1})\\
	\Delta_{lm}^{(1)} & = 0
	\end{align*}
	Then one needs to find higher order energy shifts. Considering $ \Delta_{lm}^2$ and using (\ref{3}),
	\begin{align*}
	\Delta_{lm}^2 &= \sum_{l\ne l', m\ne m'}\dfrac{ \abs{\mel{l',m'}{V}{l,m}}^2}{E_{lm} - E_{l'm'}}\\
	 &= \dfrac{C^2}{4} \sum_{l\ne l', m\ne m'}\dfrac{\qty([{l(l+1) - m(m+1)} ]\delta_{l',l} \delta_{m',m+1} + [{l(l+1) - m(m-1)}] \delta_{l',l} \delta_{m',m-1})}{Al(l+1) + Bm - Al'(l'+1) - Bm'}\\
	 &= \dfrac{C^2}{4}\qty( \dfrac{-[{l(l+1) - m(m+1)}]}{B} + \dfrac{[{l(l+1) - m(m - 1)}]}{B} )\\
	 &= \dfrac{C^2m}{2B}
	\end{align*}
	
\end{homeworkProblem}














\begin{homeworkProblem}[6]
	The Hamiltonian to deal with is,
	\begin{equation*}
	H = \dfrac{p^2}{2m} + \dfrac{m\omega^2}{2}x^2
	\end{equation*}
	and the given trial wavefunction is,
	\begin{equation*}
	\psi_\beta(x) = N e^{-\beta\abs{x}}
	\end{equation*}
	where $ N $ is some normalization. We first calculate $ \pdv[2]{\psi_\beta}{x}$,
	\begin{align*}
	\pdv[2]{\psi_\beta}{x} &=N \pdv{x}\qty(\pdv{e^{-\beta \abs{x}}}{x})\\
	&=-\beta N\pdv{x}\qty(\dfrac{x}{\abs{x}}e^{-\beta \abs{x}})\\
	&= N\qty(\beta^2e^{-\beta \abs{x}} -\dfrac{\beta}{\abs{x}}e^{-\beta \abs{x}} + \dfrac{\beta x}{x^2}e^{-\beta \abs{x}})
	\end{align*}
	Let's calculate $ \expval{H}{\psi_\beta} $,
	\begin{align*}
	\expval{H}{\psi_\beta} &= N^2 \int_{-\infty}^{\infty} e^{-\beta\abs{x}} \qty(-\dfrac{1}{2m}\pdv[2]{x}+ \dfrac{m\omega^2}{2}x^2)e^{-\beta\abs{x}}dx\\
	&=  \lim\limits_{\epsilon \rightarrow 0} N^2 \qty[ 2 \int_{\epsilon}^{\infty}  e^{-\beta x} \qty(-\dfrac{1}{2m}\pdv[2]{x}+ \dfrac{m\omega^2}{2}x^2) e^{-\beta x } dx + \int_{-\epsilon}^{\epsilon}  e^{-\beta\abs{x}} \qty(-\dfrac{1}{2m}\pdv[2]{x}+ \dfrac{m\omega^2}{2}x^2)e^{-\beta\abs{x}} dx]\\
	&=  \lim\limits_{\epsilon \rightarrow 0} N^2 \left[ 2 \int_{\epsilon}^{\infty}  e^{-2\beta x} \qty(-\dfrac{\beta^2}{2m} + \dfrac{m\omega^2}{2}x^2) dx - \int_{-\epsilon}^{\epsilon}  e^{-\beta\abs{x}} \qty(\dfrac{1}{2m}\pdv[2]{x} ) e^{-\beta\abs{x}} dx  + \int_{-\epsilon}^{\epsilon}  e^{-2\beta\abs{x}} \qty(\dfrac{m\omega^2}{2}x^2) dx \right]
	\end{align*}
	The last integral goes to zero as $ \epsilon \rightarrow 0 $. The second integral transforms into an integral only over the double derivative, since $  e^{-\beta\abs{x}} $ is almost constant from $ -\epsilon $ to $ \epsilon $. Then,
	\begin{align*}
	\expval{H}{\psi_\beta} &=  N^2 \left[ \qty(-\dfrac{\beta}{2m} + \dfrac{m\omega^2}{4\beta^3})-  \lim\limits_{\epsilon \rightarrow 0} \dfrac{e^{-\beta\abs{x}}}{2m} \eval{\qty(\pdv{x} e^{-\beta\abs{x}})}_{-\epsilon}^{\epsilon}  \right]\\
	&=  N^2 \left[ \qty(-\dfrac{\beta}{2m} + \dfrac{m\omega^2}{4\beta^3})+ \dfrac{\beta}{m} \right]\\
	&=  N^2 \qty[\dfrac{\beta}{2m} + \dfrac{m\omega^2}{4\beta^3}]
	\end{align*}
	We need to also fix the normalization,
	\begin{align*}
	N^2 \int_{-\infty}^{\infty}e^{-2\beta\abs{x}} = 1 \implies N^2 = \beta
	\end{align*}
	Hence,
	\begin{align*}
	\expval{H}{\psi_\beta} = &\ev{H} = \dfrac{\beta^2}{2m} +  \dfrac{m\omega^2 }{4 \beta^2}\\
	\eval{\dv{\ev{H}}{\beta}}_{\beta_0} = 0 \implies \dfrac{\beta_0}{m} - \dfrac{m\omega^2}{2\beta_0^3} = 0 \implies \beta_0 & = \pm \qty(\dfrac{m^2\omega^2}{2})^{1/4} \implies \ev{H}_{min} = \dfrac{\omega}{2\sqrt{2}} + \dfrac{\omega}{2\sqrt{2}} = \dfrac{\omega}{\sqrt{2}}
	\end{align*}
	Hence the ground state energy has an estimate of $ \dfrac{\omega}{\sqrt{2}} $ (which clearly is off from the actual value by a factor of $\dfrac{1}{\sqrt{2}}$ ).
\end{homeworkProblem}

\end{document}
