\documentclass{article}

\usepackage{fancyhdr}
\usepackage{extramarks}
\usepackage{amsmath}
\usepackage{amsthm}
\usepackage{amssymb}
\usepackage{amsfonts}
\usepackage{tikz}
\usepackage{physics}
\usepackage[plain]{algorithm}
\usepackage{algpseudocode}

\usetikzlibrary{automata,positioning}

%
% Basic Document Settings
%

\topmargin=-0.45in
\evensidemargin=0in
\oddsidemargin=0in
\textwidth=6.5in
\textheight=9.0in
\headsep=0.25in

\linespread{1.1}

\pagestyle{fancy}
\lhead{\hmwkAuthorName}
\chead{\hmwkClass\ : \hmwkTitle}
\rhead{\firstxmark}
\lfoot{\lastxmark}
\cfoot{\thepage}

\renewcommand\headrulewidth{0.4pt}
\renewcommand\footrulewidth{0.4pt}

\setlength\parindent{0pt}

%
% Create Problem Sections
%
\newcommand{\be}{\begin{equation}}
\newcommand{\ee}{\end{equation}}
\newcommand{\bes}{\begin{equation*}}
\newcommand{\ees}{\end{equation*}}
\newcommand{\bea}{\begin{flalign*}}
\newcommand{\eea}{\end{flalign*}}


\newcommand{\enterProblemHeader}[1]{
    \nobreak\extramarks{}{Problem \arabic{#1} continued on next page\ldots}\nobreak{}
    \nobreak\extramarks{Problem \arabic{#1} (continued)}{Problem \arabic{#1} continued on next page\ldots}\nobreak{}
}

\newcommand{\exitProblemHeader}[1]{
    \nobreak\extramarks{Problem \arabic{#1} (continued)}{Problem \arabic{#1} continued on next page\ldots}\nobreak{}
    \stepcounter{#1}
    \nobreak\extramarks{Problem \arabic{#1}}{}\nobreak{}
}

\setcounter{secnumdepth}{0}
\newcounter{partCounter}
\newcounter{homeworkProblemCounter}
\setcounter{homeworkProblemCounter}{1}
\nobreak\extramarks{Problem \arabic{homeworkProblemCounter}}{}\nobreak{}

%
% Homework Problem Environment
%
% This environment takes an optional argument. When given, it will adjust the
% problem counter. This is useful for when the problems given for your
% assignment aren't sequential. See the last 3 problems of this template for an
% example.
%
\newenvironment{homeworkProblem}[1][-1]{
    \ifnum#1>0
        \setcounter{homeworkProblemCounter}{#1}
    \fi
    \section{Problem \arabic{homeworkProblemCounter}}
    \setcounter{partCounter}{1}
    \enterProblemHeader{homeworkProblemCounter}
}{
    \exitProblemHeader{homeworkProblemCounter}
}

%
% Homework Details
%   - Title
%   - Due date
%   - Class
%   - Section/Time
%   - Instructor
%   - Author
%

\newcommand{\hmwkTitle}{Assignment\ \#5}
\newcommand{\hmwkDueDate}{Due on 20th November, 2018}
\newcommand{\hmwkClass}{Advanced Quantum Mechanics}
\newcommand{\hmwkClassTime}{}
\newcommand{\hmwkClassInstructor}{}
\newcommand{\hmwkAuthorName}{\textbf{Aditya Vijaykumar}}

%
% Title Page
%

\title{
    %\vspace{2in}
    \textmd{\textbf{\hmwkClass:\ \hmwkTitle}}\\
    \normalsize\vspace{0.1in}\small{\hmwkDueDate\ }\\
%    \vspace{3in}
}

\author{\hmwkAuthorName}
\date{}

\renewcommand{\part}[1]{\textbf{\large Part \Alph{partCounter}}\stepcounter{partCounter}\\}

%
% Various Helper Commands
%

% Useful for algorithms
\newcommand{\alg}[1]{\textsc{\bfseries \footnotesize #1}}

% For derivatives
\newcommand{\deriv}[1]{\frac{\mathrm{d}}{\mathrm{d}x} (#1)}

% For partial derivatives
\newcommand{\pderiv}[2]{\frac{\partial}{\partial #1} (#2)}

% Integral dx
\newcommand{\dx}{\mathrm{d}x}

% Alias for the Solution section header
\newcommand{\solution}{\textbf{\large Solution}}

% Probability commands: Expectation, Variance, Covariance, Bias
\newcommand{\E}{\mathrm{E}}
\newcommand{\Var}{\mathrm{Var}}
\newcommand{\Cov}{\mathrm{Cov}}
\newcommand{\Bias}{\mathrm{Bias}}

\begin{document}

\maketitle
(\textbf{Acknowledgements} - I would like to thank Chandramouli Chowdhury, Biprarshi Chakraborty and Junaid Majeed for discussions.)
\\

\begin{homeworkProblem}[1]
	\textbf{Part (a)}\\
	We first note that,
	\begin{align*}
	\lambda e^{-t/\tau} \mel{n}{x^2}{m} &= \lambda e^{-t/\tau}  \mel{n}{\dfrac{(a_+ + a_-)^2}{2m\omega}}{m} \\
	&=\lambda e^{-t/\tau}  \mel{n}{\dfrac{a_+^2 + a_-^2 + a_+ a_- + a_- a_+}{2m\omega}}{m}\\
	&= \lambda e^{-t/\tau} \dfrac{1}{2 m \omega} \qty[\sqrt{(m+1)(m+2)} \delta_{n,m+2} + \sqrt{m(m-1)} \delta_{n, m-2} + (2m-1) \delta_{n,m}]
	\end{align*}
	As is evident from above, a state $ \ket{m} $ can transition into  $ \ket{m}, \ket{m+2}, \ket{m-2} $ and no other states under a potential with spatial dependence that goes as $ x^2 $. In general, the $ k $-th order coefficient $ c^k_n(t) $ will have $ k $ terms of the form $ \mel{.}{x^2}{.} $. If we start out with ground state $ \ket{0} $, the final state will have contributions from the following states order by order
	\begin{align*}
	\order{\lambda} &\rightarrow \ket{0}, \ket{2} \\
	\order{\lambda^2} &\rightarrow \ket{0}, \ket{2}, \ket{4} \\
	\order{\lambda^3} &\rightarrow \ket{0}, \ket{2}, \ket{4}, \ket{6} \\ 
	\therefore \order{\lambda^k} &\rightarrow \ket{0}, \ket{2}, \ket{4}, \ldots \ket{2k}
	\end{align*}
	Hence, we see that the $ \ket{n} $ as mentioned in the question should be such that $ n $ is even, and the leading order contribution to the probability will $ \sim (\lambda^{n/2})^2 \sim \lambda^n $.\\
	
	\textbf{Part (b)}\\
	As described above, upto $ \order{\lambda^2} $ in probability (ie upto $ \order{\lambda} $ in the coefficients), $ \ket{2} $ is the only excited state that can be reached. From $ (5.7.17) $ of Sakurai, we have the relations, (with $ \ket{i,t_0;t} = \sum c_n(t) \ket{n} $)
	\begin{equation*}
	c_n^0 (t) = \delta_{ni} \qq{,} c_n^1 (t) = -i \int_{t_0}^{t} e^{i \omega_{ni} t'} V_{ni}(t') dt'
	\end{equation*}
	Let's calculate $ c_n^1 (t) $,
	\begin{align*}
	c_n^1 (t) &=  -i \lambda \int_{0}^{t} e^{i n \omega t'} \mel{n}{ x^2}{0} e^{-t'/\tau} dt' \\
	&= \dfrac{-i \lambda}{2 m \omega} (\sqrt{2} \delta_{n,2} + \delta_{n,0}) \int_{0}^{t} e^{i n \omega t'}  e^{-t'/\tau} dt'\\
	c_n^1 (t) &= \dfrac{-i \lambda}{2 m \omega} (\sqrt{2} \delta_{n,2} + \delta_{n,0} )\dfrac{e^{i n \omega t}  e^{-t/\tau} - 1}{i n \omega - 1/\tau} \\
	\implies c_2^1 (t) &= \dfrac{-i \lambda }{\sqrt{2} m \omega}  \dfrac{e^{2 i \omega t}  e^{-t/\tau} - 1}{2 i \omega - 1/\tau} \implies  \abs{c_2^1 (t)}^2 = \dfrac{\lambda^2 }{{2} m^2 \omega^2}  \dfrac{ e^{-2t/\tau} + 1 - 2 e^{-t / \tau } \cos 2 \omega t}{4 \omega^2 + 1/\tau^2} 
	\end{align*}
	$ \abs{c_2^1}^2 $ is the required probability.
\end{homeworkProblem}








\begin{homeworkProblem}[2]
	We don't need to apply any perturbation theory in this problem, and it can be solved exactly. The Hamiltonian is $ H = \lambda S_1 \cdot S_2 = \lambda( S^2 - S_1^2 - S_2^2 ) $. We consider the action of the Hamiltonian on the singlet state $ \ket{0 0} = \dfrac{\ket{+ -} - \ket{- +}}{\sqrt{2}} $ and $ \ket{1 0 }  = \dfrac{\ket{+ -} + \ket{- +}}{\sqrt{2}} $. We know $ S^2 \ket{00} = 0  $ and $ S^2 \ket{10} = \ket{10} $. Initially the system is in $ \ket{+-} = \dfrac{\ket{00} + \ket{10}}{\sqrt{2}} $. Then we know, by the usual rules of time-evolution,
	\begin{align*}
	\ket{\psi_f(t)} = e^{i H t} \ket{+-} &= \dfrac{e^{i\lambda t/4}}{\sqrt{2}} \ket{10} + \dfrac{e^{-i3\lambda t/4}}{\sqrt{2}} \ket{00}\\
	&= \qty(\dfrac{e^{i\lambda t/4}  +{e^{-i3\lambda t/4}}}{2}) \ket{+-} + \qty(\dfrac{e^{i\lambda t/4}  - {e^{-i3\lambda t/4}}}{2}) \ket{-+}\\
	\implies \abs{\ip{+-}{{\psi_f(t)}}}^2 &= \abs{\qty(\dfrac{e^{i\lambda t/4}  +{e^{-i3\lambda t/4}}}{2})}^2  = \dfrac{1 + \cos \lambda t}{2} =  P(\ket{+-})\\
	\implies \abs{\ip{-+}{{\psi_f(t)}}}^2 &= \abs{\qty(\dfrac{e^{i\lambda t/4}  - {e^{-i3\lambda t/4}}}{2})}^2  = \dfrac{1 - \cos \lambda t}{2} = P(\ket{-+})\\
	\implies \abs{\ip{++}{{\psi_f(t)}}}^2 &= 0 =  P(\ket{++})\\
	\implies \abs{\ip{--}{{\psi_f(t)}}}^2 &= 0 =  P(\ket{--})
	\end{align*}
	where $ P(\ket{}) $ denotes probability of initial state to be in state $ \ket{} $.
\end{homeworkProblem}










\begin{homeworkProblem}[3]
	From $ (5.7.17) $ of Sakurai, we have the relations, (with $ \ket{i,t_0;t} = \sum c_n(t) \ket{n} $)
	\begin{equation*}
	c_n^0 (t) = \delta_{ni} \qq{,} c_n^1 (t) = -i \int_{t_0}^{t} e^{i \omega_{ni} t'} V_{ni}(t') dt' 
	\end{equation*}
	For our problem, we have $ V = \lambda \delta (x - vt) $. We insert $ 1 = \int dx \ket{x} \bra{x} $ such that $ V_{ni}(t) = \int V(t) u_i^* (x) u_n (x) dx $. We have initial state $ u_i (x) $ and final state $ u_f(x) $. Hence, we can write the above coefficients as,
	\begin{align*}
	c_f^1 (t) &= -i \lambda \int_{-\infty}^{\infty} dx \int_{0}^{t} dt' e^{i (E_i - E_f) t'} \delta(x - vt') u_i^* (x) u_f (x)  \\
	 &= -i \lambda \int_{-\infty}^{\infty} dx  e^{i (E_i - E_f) x/v}  u_i^* (x) u_f (x) 
	\end{align*}
	Hence the probability is just $ \abs{c_f^1}^2 $
\end{homeworkProblem}














\begin{homeworkProblem}
	content...
\end{homeworkProblem}















\begin{homeworkProblem}[5]
	We first note for $ \ket{\psi_I(t)} = \sum_{n} {c_n(t) \ket{\alpha_n}} $
	\begin{align*}
	i \pdv{\ket{\psi_I}}{t} &= i \pdv{(e^{iH_0t} \ket{\psi_S})}{t} \\
	&= i \qty[e^{iH_0t}\pdv{\ket{\psi_S}}{t}  + iH_0 e^{iH_0t} \ket{\psi_S}]\\
	&= -e^{iH_0t} (H_0 + V) \ket{\psi_S} - H_0 e^{iH_0t} \ket{\psi_S}\\
	&=e^{iH_0t}  V \ket{\psi_S}\\
	i \pdv{\ket{\psi_I}}{t} &= V_I \ket{\psi_I}\\
	i \pdv{\ip{\alpha_n}{\psi_I}}{t} &=  \mel{\alpha_n}{V_I}{\psi_I}\\
	\dot{c_n} &= -i  \mel{\alpha_n}{V_I}{\psi_I}\\
	\dot{c_n} &= -i  \mel{\alpha_n}{V}{\alpha_m} e^{i{(E_n - E_m)t}} c_m
	\end{align*}
	So for the given problem, we have
	\begin{align*}
	\ket{\psi_I(t)} &=  {c_1(t) \ket{1} + c_2(t) e^{iEt} \ket{2}}	\\
	\dot{c_1} = -i V_{11} c_1 - i V_{12} e^{-iEt} c_2 = - i \gamma e^{i (\omega-E) t} c_2  &\qq{and} \dot{c_2} = -i V_{21}  e^{iEt}  c_1 - i V_{22}c_2 = -i \gamma e^{i(E-\omega)t} c_1
	\end{align*} 
	To solve the above equations, we make the substitution $  c_1 = b_1 e^{i\Delta t} $ and $c_2 = b_2 e^{-i \Delta t}  $, where $ 2 \Delta = \omega - E$. We then have the equations in terms of $ b $'s,
	\begin{equation*}
	i \dot{b_1} = \Delta b_1 + \gamma b_2 \qq{and} i \dot{b_2} = 
	\gamma b_1 - \Delta b_2
	\end{equation*}
	These are coupled equations, and we can solve these by making the substitution $ b_1 = A e^{i \Omega t} $ and $  b_2 = B e^{i \Omega t} $. We then have,
	\begin{align*}
	- A \Omega = \Delta A + \gamma B &\qq{and} - B \Omega = \gamma A - \Delta B \\
	\qq{For non-trivial solutions,} -\dfrac{\gamma}{\Delta + \Omega} &= \dfrac{\Delta - \Omega}{\gamma} \implies \Omega = \pm \sqrt{\gamma^2 + \Delta^2} = \pm \Omega_0\\
	\implies c_1 = A_1 e^{i(\Delta + \Omega_0)t} + A_2 e^{i(\Delta - \Omega_0)t} &\qq{and}  c_2 = B_1 e^{i(-\Delta + \Omega_0)t} + B_2 e^{i(-\Delta - \Omega_0)t}
	\end{align*}
	We are told that at $ t=0 $, the system is in state $ \ket{1} \implies c_1(0) = 0, c_2(0) = 1 \implies A_1 = - A_2, B_1 = 1 - B_2$. We also know that $ \dot{c}_2(0) = - i \gamma c_1 (0) $ and $ \dot{c}_1(0) = - i \gamma c_2 (0)  $ which means,
	\begin{align*}
	i(\Delta - \Omega_0) B_1 - i(\Delta + \Omega_0) (1 - B_1) = 0 &\implies B_1 = \dfrac{\Delta + \Omega_0}{2 \Delta} \qq{and}  B_2 = \dfrac{\Delta - \Omega_0}{2 \Delta} \\
	 A_1 i( \Delta + \Omega_0 - \Delta + \Omega_0)  = -i \gamma &\implies A_1 = \dfrac{ -i \gamma}{2 \Omega_0} \qq{and} A_2 = \dfrac{ i \gamma}{2 \Omega_0} \\
	 \implies c_1 = \dfrac{ -i \gamma}{2 \Omega_0}  e^{i(\Delta + \Omega_0)t} + \dfrac{ i \gamma}{2 \Omega_0} e^{i(\Delta - \Omega_0)t} &\qq{and}  c_2 = \dfrac{\Delta + \Omega_0}{2 \Delta}  e^{i(-\Delta + \Omega_0)t} + \dfrac{\Delta - \Omega_0}{2 \Delta} e^{i(-\Delta - \Omega_0)t}
	\end{align*}
\end{homeworkProblem}













\end{document}
