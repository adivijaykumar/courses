\documentclass{article}

\usepackage{fancyhdr}
\usepackage{extramarks}
\usepackage{amsmath}
\usepackage{amsthm}
\usepackage{amssymb}
\usepackage{amsfonts}
\usepackage{tikz}
\usepackage{physics}
\usepackage[plain]{algorithm}
\usepackage{algpseudocode}
\usepackage{hyperref}

\usetikzlibrary{automata,positioning}

%
% Basic Document Settings
%

\topmargin=-0.45in
\evensidemargin=0in
\oddsidemargin=0in
\textwidth=6.5in
\textheight=9.0in
\headsep=0.25in

\linespread{1.1}

\pagestyle{fancy}
\lhead{\hmwkAuthorName}
\chead{\hmwkClass\ : \hmwkTitle}
\rhead{\firstxmark}
\lfoot{\lastxmark}
\cfoot{\thepage}

\renewcommand\headrulewidth{0.4pt}
\renewcommand\footrulewidth{0.4pt}

\setlength\parindent{0pt}

%
% Create Problem Sections
%
\newcommand{\be}{\begin{equation}}
\newcommand{\ee}{\end{equation}}
\newcommand{\bes}{\begin{equation*}}
\newcommand{\ees}{\end{equation*}}
\newcommand{\bea}{\begin{flalign*}}
\newcommand{\eea}{\end{flalign*}}


\newcommand{\enterProblemHeader}[1]{
    \nobreak\extramarks{}{Problem \arabic{#1} continued on next page\ldots}\nobreak{}
    \nobreak\extramarks{Problem \arabic{#1} (continued)}{Problem \arabic{#1} continued on next page\ldots}\nobreak{}
}

\newcommand{\exitProblemHeader}[1]{
    \nobreak\extramarks{Problem \arabic{#1} (continued)}{Problem \arabic{#1} continued on next page\ldots}\nobreak{}
    \stepcounter{#1}
    \nobreak\extramarks{Problem \arabic{#1}}{}\nobreak{}
}

\setcounter{secnumdepth}{0}
\newcounter{partCounter}
\newcounter{homeworkProblemCounter}
\setcounter{homeworkProblemCounter}{1}
\nobreak\extramarks{Problem \arabic{homeworkProblemCounter}}{}\nobreak{}

%
% Homework Problem Environment
%
% This environment takes an optional argument. When given, it will adjust the
% problem counter. This is useful for when the problems given for your
% assignment aren't sequential. See the last 3 problems of this template for an
% example.
%
\newenvironment{homeworkProblem}[1][-1]{
    \ifnum#1>0
        \setcounter{homeworkProblemCounter}{#1}
    \fi
    \section{Problem \arabic{homeworkProblemCounter}}
    \setcounter{partCounter}{1}
    \enterProblemHeader{homeworkProblemCounter}
}{
    \exitProblemHeader{homeworkProblemCounter}
}

%
% Homework Details
%   - Title
%   - Due date
%   - Class
%   - Section/Time
%   - Instructor
%   - Author
%

\newcommand{\hmwkTitle}{Assignment\ \#6}
\newcommand{\hmwkDueDate}{Due on 28th November, 2018}
\newcommand{\hmwkClass}{Advanced Quantum Mechanics}
\newcommand{\hmwkClassTime}{}
\newcommand{\hmwkClassInstructor}{}
\newcommand{\hmwkAuthorName}{\textbf{Aditya Vijaykumar}}

%
% Title Page
%

\title{
    %\vspace{2in}
    \textmd{\textbf{\hmwkClass:\ \hmwkTitle}}\\
    \normalsize\vspace{0.1in}\small{\hmwkDueDate\ }\\
%    \vspace{3in}
}

\author{\hmwkAuthorName}
\date{}

\renewcommand{\part}[1]{\textbf{\large Part \Alph{partCounter}}\stepcounter{partCounter}\\}

%
% Various Helper Commands
%

% Useful for algorithms
\newcommand{\alg}[1]{\textsc{\bfseries \footnotesize #1}}

% For derivatives
\newcommand{\deriv}[1]{\frac{\mathrm{d}}{\mathrm{d}x} (#1)}

% For partial derivatives
\newcommand{\pderiv}[2]{\frac{\partial}{\partial #1} (#2)}

% Integral dx
\newcommand{\dx}{\mathrm{d}x}

% Alias for the Solution section header
\newcommand{\solution}{\textbf{\large Solution}}

% Probability commands: Expectation, Variance, Covariance, Bias
\newcommand{\E}{\mathrm{E}}
\newcommand{\Var}{\mathrm{Var}}
\newcommand{\Cov}{\mathrm{Cov}}
\newcommand{\Bias}{\mathrm{Bias}}

\begin{document}

\maketitle
(\textbf{Acknowledgements} - I would like to thank Chandramouli Chowdhury, Sarthak Duary and Junaid Majeed for discussions.)
\\











\begin{homeworkProblem}[1]
	The wave equation can be written as,
	\begin{align*}
	\pdv[2]{f}{t} &= c^2 \laplacian{f} \\
	\implies \pdv[2]{f}{t} &= c^2 \qty[\dfrac{1}{r}\pdv{r} \qty(r \pdv{f}{r}) + \dfrac{1}{r^2} \pdv[2]{f}{\theta} ]
	\end{align*}
	We write $ f(r, \theta, t)  = R(r) \Theta(\theta) T(t)$. We then have,
	\begin{equation*}
	\dfrac{1}{T}\dv[2]{T}{t} = c^2 \qty[\dfrac{1}{Rr} \dv{r} \qty(r \dv{R}{r}) + \dfrac{1}{r^2 \Theta} \dv[2]{\Theta}{\theta} ] = constant = - \lambda^2
	\end{equation*}
	where $ - \lambda^2 $ comes from the fact that LHS is only dependent on $ t $, and the RHS does not depend on $ t $. Let's look at the RHS. 
	\begin{align*}
	c^2 \qty[\dfrac{1}{Rr} \dv{r} \qty(r \dv{R}{r}) + \dfrac{1}{r^2 \Theta} \dv[2]{\Theta}{\theta} ] &= -\lambda^2 \\
	\implies \dfrac{r}{R} \dv{r} \qty(r \dv{R}{r}) + \dfrac{1}{\Theta} \dv[2]{\Theta}{\theta}  &= -\dfrac{\lambda^2 r^2}{c^2}\\
	\implies \dv[2]{\Theta}{\theta} = - \mu^2 \Theta \qq{and} \dfrac{r}{R} \dv{r} \qty(r \dv{R}{r}) &= - \dfrac{\lambda^2 r^2}{c^2} + \mu^2	\\
	\implies \Theta =A  e^{i \mu \theta} + B e^{-i \mu \theta} \qq{and} {r R''(r)+R'(r)} + &\qty(\frac{\lambda ^2 r^2}{c^2} - \mu^2) R(r) = 0 \\
	\implies \Theta =A  e^{i \mu \theta} + B e^{-i \mu \theta} \qq{and} R &= C J_\mu \qty(\frac{\lambda R}{c})
	\end{align*}
	So $ \mu, \lambda $ are the required quantum numbers.
	\\
	
	\textbf{Part (b)}\\
	We have to find $ G(\vb{x},\vb{x'}) $. We know that the Green's function in momentum space is $ \tilde{G}(\vb{p}) = \dfrac{1}{E - p^2/2m} $
	
	\begin{align*}
	\therefore  G(\vb{x},\vb{x'})  &= \mel{\vb{x}}{\tilde{G}(\vb{p})}{\vb{x'}} \\
	&= \int \dfrac{d^2 \vb{p'}}{(2\pi)^2} \dfrac{d^2 \vb{p}}{(2\pi)^2}  \ip{\vb{x}}{\vb{p}}  \mel{\vb{p}}{\tilde{G}(\vb{p})}{\vb{p'}} \ip{\vb{p'}}{\vb{x'}}\\
	&= \int \dfrac{d^2 \vb{p'}}{(2\pi)^2} \dfrac{d^2 \vb{p}}{(2\pi)^2}  \ip{\vb{x}}{\vb{p}}  \dfrac{1}{E - p^2/2m} \delta_{\vb{p},\vb{p'}} \ip{\vb{p'}}{\vb{x'}}
	\end{align*}
	
	\begin{align*}
	G(\vb{x},\vb{x'})   &= \int  \dfrac{d^2 \vb{p}}{(2\pi)^2}  \ip{\vb{x}}{\vb{p}}  \dfrac{1}{E - p^2/2m}  \ip{\vb{p}}{\vb{x'}}\\
	&= \int  \dfrac{d^2 \vb{p}}{(2\pi)^2}  e^{i \vb{p} \cdot (\vb{x} - \vb{x'})}  \dfrac{1}{E - p^2/2m}\\
	&= \dfrac{1}{4 \pi^2} \int_0^\infty  p dp  \int_{-1}^1 d(\cos \theta) e^{i p \abs{\vb{x} - \vb{x'}}}  \dfrac{1}{E - p^2/2m}\\
	&= \dfrac{2m}{4 \pi^2} \int_0^\infty  p dp  \int_{-1}^1 d(\cos \theta) e^{i p \abs{\vb{x} - \vb{x'}} \cos \theta}  \dfrac{1}{2mE - p^2}\\
	&=\dfrac{2m}{4 \pi^2} \int_0^\infty  p dp \dfrac{ e^{i p \abs{\vb{x} - \vb{x'}} } - e^{-i p \abs{\vb{x} - \vb{x'} }}}{i \abs{\vb{x} - \vb{x'} }}  \dfrac{1}{2mE - p^2}\\
	&= \dfrac{2m i}{4 \pi^2  \abs{\vb{x} - \vb{x'} }} \int_{-\infty}^\infty  p dp  e^{i p \abs{\vb{x} - \vb{x'}} } \dfrac{1}{ p^2  - 2 m E - i \epsilon} \\
	&= \dfrac{2m i}{4 \pi^2  \abs{\vb{x} - \vb{x'} }} ( \pi i) e^{i k \abs{\vb{x} - \vb{x'}}} \\
	G(\vb{x},\vb{x'})  &= - \dfrac{m  e^{i k \abs{\vb{x} - \vb{x'}}} }{2 \pi  \abs{\vb{x} - \vb{x'} }}
	\end{align*}
	The integration is exactly the same as the contour integral which was described in class, and so is the Green's function.
\end{homeworkProblem}














\begin{homeworkProblem}[2]
	We first lay out our notation. From the Lippmann-Schwinger equation, we know,
	\begin{equation*}
	\psi(\vb{r}) = e^{i \vb{k} \cdot \vb{r}} +  \dfrac{e^{i k r}}{r} \underbrace{\qty[\qty(\dfrac{-m}{2 \pi}) \int d^3 \vb{r'} e^{-i k \vu{r} \cdot \vb{r'}} V(\vb{r'}) \psi (\vb{r'})]}_{f(\vb{k},\vb{k'})} 
	\end{equation*}
	To solve this order by order, we use the ansatz $ \psi(\vb{r}) = e^{i \vb{k} \cdot \vb{r}}  + \sum_{n = 1}^{\infty} \phi_n(\vb{r})$. Substituting in the above equation, we get the recurrence relation,
	\begin{align*}
	\phi_{n+1}(\vb{r}) &=   \int d^3 \vb{r'} G_0^+ (k,\vb{r} - \vb{r'}) V(\vb{r'})\phi_n(\vb{r})\\
	\qq{in particular} \phi_{1}(\vb{r}) &= \qty(\dfrac{-m}{2 \pi}) \dfrac{e^{i k r}}{r} \int d^3 \vb{r'} e^{-i k \vu{r} \cdot \vb{r'}} V(\vb{r'}) e^{i \vb{k} \cdot \vb{r'}} = \qty(\dfrac{-m}{2 \pi}) \dfrac{e^{i k r}}{r}  \int d^3 \vb{r'} e^{i (\vb{k} - \vb{k'}) \cdot \vb{r'}} V(\vb{r'}) 
	\end{align*}
	In the Born approximation, one sets $ \psi(\vb{r}) = e^{i \vb{k} \cdot \vb{r}}  + \phi_1 (\vb{r}) \implies f^{(1)}(\vb{k},\vb{k'}) = \qty(\dfrac{-m}{2 \pi}) \int d^3 \vb{r'} e^{i (\vb{k} - \vb{k'}) \cdot \vb{r'}} V(\vb{r'}) $
	
	\begin{align*}
	\dv{\sigma}{\Omega} &= \abs{f^{(1)} (\vb{k},\vb{k'})}^2 \\
	&= \dfrac{m^2}{4 \pi^2} \int d^3 \vb{x}  d^3 \vb{x'} V(\vb{x}) V(\vb{x'}) e^{-i \vb{k'} \cdot \vb{x}} e^{i \vb{k} \cdot \vb{x}} e^{-i \vb{k} \cdot \vb{x'}} e^{i \vb{k'} \cdot \vb{x}}\\
	\dv{\sigma}{\Omega} &= \dfrac{m^2}{4 \pi^2} \int d^3 \vb{x}  d^3 \vb{x'} V(\vb{x}) V(\vb{x'}) e^{i (\vb{k} - \vb{k'}) \cdot (\vb{x} - \vb{x'})} 
	\end{align*}
	The total cross-section $ \sigma_{T} $ can be obtained by integrating over outgoing momenta and averaging over ingoing momenta.
	
	\begin{align*}
	\implies \sigma_T  &= \dfrac{m^2}{4 \pi^2} \int d^3 \vb{x}  d^3 \vb{x'} \dfrac{d \Omega_{\vb{k}}}{4 \pi} d \Omega_{\vb{k'}} V(\vb{x}) V(\vb{x'}) e^{i (\vb{k}) \cdot (\vb{x} - \vb{x'})}  e^{i (-\vb{k'}) \cdot (\vb{x} - \vb{x'})}   \\
	&= \dfrac{m^2}{4 \pi^2} \int d^3 \vb{x}  d^3 \vb{x'} V(\vb{x}) V(\vb{x'})\int_{0}^{2 \pi} \dfrac{d\phi_{\vb{k}}}{4 \pi} \int_{-1}^{1}  d (\cos \theta_{\vb{k}}) e^{i \abs{\vb{k}} \abs{\vb{x} - \vb{x'}} \cos \theta_{\vb{k}}}  \int_{0}^{2 \pi} d\phi_{\vb{k'}} \int_{-1}^{1}  d (\cos \theta_{\vb{k'}}) e^{-i \abs{\vb{k'}} \abs{\vb{x} - \vb{x'}} \cos \theta_{\vb{k'}}} 
	\end{align*}
	\begin{align*}
	\sigma_T &= \dfrac{m^2}{4 \pi} \int d^3 \vb{x}  d^3 \vb{x'} V(\vb{x}) V(\vb{x'}) \dfrac{e^{i k \abs{\vb{x} - \vb{x'}}} - e^{-i k \abs{\vb{x} - \vb{x'}} }}{i k \abs{\vb{x} - \vb{x'}}} \cross  \dfrac{e^{-i k \abs{\vb{x} - \vb{x'}}} - e^{i k \abs{\vb{x} - \vb{x'}} }}{-i k \abs{\vb{x} - \vb{x'}}} \impliedby (\abs{\vb{k}} = \abs{\vb{k'}} = k) \\
	&= \dfrac{m^2}{4 \pi} \int d^3 \vb{x}  d^3 \vb{x'} V(\vb{x}) V(\vb{x'})   \dfrac{2 \sin  k \abs{\vb{x} - \vb{x'}} }{k \abs{\vb{x} - \vb{x'}}} \cross \dfrac{2 \sin  k \abs{\vb{x} - \vb{x'}} }{k \abs{\vb{x} - \vb{x'}}} \\
	\sigma_{T} &= \dfrac{m^2}{ \pi} \int d^3 \vb{x}  d^3 \vb{x'} V(\vb{x}) V(\vb{x'})   \dfrac{ \sin^2  k \abs{\vb{x} - \vb{x'}} }{k^2 \abs{\vb{x} - \vb{x'}}^2}
	\end{align*}
	 The optical theorem tells us,
	 \begin{equation*}
	 \sigma_T = \dfrac{4 \pi}{k} 	\Im f(\vb{k}, \vb{k}) = - \dfrac{2 m L^3}{k}  \Im \mel{\vb{k}}{T}{\vb{k}} 
	 \end{equation*}
	 Upto first order the imaginary part is zero. Hence,
	 \begin{align*}
	 \sigma_T &=  - \dfrac{2 m L^3}{k}  \Im \mel{\vb{k}}{V \dfrac{1}{E -  H_0 + i \epsilon} V}{\vb{k}} \\
	 &=  - \dfrac{2 m L^3}{k}  \Im \int d^3 \vb{x} d^3 \vb{x'}  \mel{\vb{k}}{V}{\vb{x}} \mel{\vb{x}}{ \dfrac{1}{E -  H_0 + i \epsilon} }{\vb{x'}} \mel{\vb{x'}}{V}{\vb{k}} \\
	 &=  - \dfrac{2 m }{k}  \Im \int d^3 \vb{x} d^3 \vb{x'}  V(\vb{x}) V(\vb{x'})  e^{-i \vb{k} \cdot (\vb{x} - \vb{x'})} \qty(\dfrac{-m}{2 \pi})  \dfrac{e^{i k \abs{\vb{x} - \vb{x'} }}}{\abs{\vb{x} - \vb{x'} }}\\
	 &=  - \dfrac{2 m }{k}  \Im \int d^3 \vb{x} d^3 \vb{x'}  V(\vb{x}) V(\vb{x'})  e^{-i \vb{k} \cdot (\vb{x} - \vb{x'})} \qty(\dfrac{-m}{2 \pi})  \dfrac{e^{i k \abs{\vb{x} - \vb{x'} }}}{\abs{\vb{x} - \vb{x'} }}\\
	 &= \dfrac{m^2}{ \pi} \int d^3 \vb{x}  d^3 \vb{x'} V(\vb{x}) V(\vb{x'})   \dfrac{ \sin^2  k \abs{\vb{x} - \vb{x'}} }{k^2 \abs{\vb{x} - \vb{x'}}^2}
	 \end{align*}
\end{homeworkProblem}






























\begin{homeworkProblem}
	content...
\end{homeworkProblem}




















\begin{homeworkProblem}
	\textbf{Part (a)}\\
	If $ \rho $ is the density matrix of a pure state, we know that $ \rho^2 = \rho \implies \log \rho^2 = \log \rho \implies 2 \log \rho - \log \rho = 0 \implies \log \rho = 0 \implies \rho \log \rho  = 0 \implies S = - \Tr(\rho \log \rho) = 0 $  \\
	
	\textbf{Part (b)}\\
	We know that $ \Tr \rho = 1 $ and that $ \rho $ is positive semi-definite $ \implies \lambda_i \le 1 $, where $ \lambda_i $ are the eigenvalues. In the basis where the density matrix in diagonal, $ \lambda_i $'s are the diagonal elements.
	\begin{equation*}
	\Tr \rho = \sum_{i=1}^{d} \lambda_i = 1 \qq{and} -\Tr \rho \log \rho = \sum_{i=1}^{d} \lambda_i \log \dfrac{1}{\lambda_i} = \sum_{i=1}^{d}  \log \dfrac{1}{\lambda_i^{\lambda_i}} = \log \prod_{i=1}^d\dfrac{1}{  \lambda_i^{\lambda_i}}
	\end{equation*}
	The  Arithmetic Mean (AM) - Geometric Mean (GM) inequality with weights \footnote{https://www.jstor.org/stable/24340414} says that for non-negative numbers $ n_i $ and non-negative weights $ w_i $, the following holds
	\begin{equation*} 
	\dfrac{\sum_{i} w_i n_i}{\sum_{i} w_i} \ge 
	\qty(\prod_{i=1} n_i^{w_i})^{\frac{1}{\sum_{i} w_i}}
	\end{equation*}
	Using $	n_i = \dfrac{1}{\lambda_i}$ and $ w_i = \lambda_i $,
	\begin{align*}
	\dfrac{\sum_{i=1}^d \lambda_i \frac{1}{\lambda_i}}{\sum_{i}^d \lambda_i} &\ge 
	\qty(\prod_{i=1}  \frac{1}{\lambda_i^{\lambda_i}})^{\frac{1}{\sum_{i}^d \lambda_i}} \\
	\implies \prod_{i=1}  \frac{1}{\lambda_i^{\lambda_i}} &\le  d \\
	\implies \log \prod_{i=1}  \frac{1}{\lambda_i^{\lambda_i}} &\le  \log d \\
	\implies S \le \log d
	\end{align*}
	Hence the maximum value of $ S $ is $ \log d $.
\end{homeworkProblem}




























\begin{homeworkProblem}[5]
	Given,
	\begin{align*}
	\ket{\psi} &= \kappa \sum_E e^{- \frac{\beta E}{2}} \ket{E} \ket{\tilde{E}}\\
	\implies \ip{\psi}{\psi} &= \kappa^2 \sum_{E,E'} e^{- \frac{\beta (E + E')}{2}} \ip{E}{E'}  \ip{\tilde{E}}{\tilde{E'}}\\
	\implies 1 &= \kappa^2 \sum_{E,E'} e^{- \frac{\beta (E + E')}{2}} \delta_{E,E'}\\
	\implies \kappa &= \qty(\dfrac{1}{\sum_E e^{-{\beta E}}})^{1/2}
	\end{align*}
	
	The density matrix corresponding to $ \ket{\psi} $ is,
	\begin{align*}
	\rho &= \op{\psi}{\psi}\\
	&= \kappa^2 \sum_{E,E'} e^{- \frac{\beta (E + E')}{2}} \ket{E} \ket{\tilde{E}} \bra{E'} \bra{\tilde{E'}}
	\end{align*}
	
	We need to find $ \rho_1 = \Tr_2 \rho = \ev{\rho}{\tilde{E}} $,
	\begin{align*}
	\rho_1  &= \kappa^2 \sum_{E,E'} e^{- \frac{\beta (E + E')}{2}} \ket{E} \ip{\tilde{E}}{\tilde{E}} \bra{E'} \ip{\tilde{E'}}{\tilde{E}} \\
	&= \kappa^2 \sum_{E,E'} e^{- \frac{\beta (E + E')}{2}} \ket{E} \bra{E'} \delta_{E,E'}\\
	\rho_1 &= {\sum_{E} \kappa^2 e^{- {\beta E}} \ket{E} \bra{E}}
	\end{align*}
	We see that $ \rho_1 $ is diagonal in the energy basis. Hence, taking the trace of a function of $ \rho_1 $ is straightforward,
	\begin{align*}
	S = -\Tr \rho_1 \log \rho_1 = {\sum_{E} \kappa^2 e^{- {\beta E}}(-\beta E + \log \kappa^2)}
	\end{align*}
\end{homeworkProblem}









\end{document}
