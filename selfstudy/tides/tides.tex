
\documentclass[a4paper,11pt]{article}

\usepackage{physics}
\usepackage{amsmath}
\usepackage{amssymb}
\usepackage{amsmath}
\usepackage{amsthm, mathtools}
%\usepackage{hyperref}
\usepackage{color}
\usepackage{jheppub}
\usepackage[T1]{fontenc} % if needed

% My Documents
\newcommand{\be}{\begin{equation}}
\newcommand{\ee}{\end{equation}}
\newcommand{\bes}{\begin{equation*}}
\newcommand{\ees}{\end{equation*}}
\newcommand{\bea}{\begin{flalign*}}
\newcommand{\eea}{\end{flalign*}}


%\linespread{1.0}
%\setlength{\parindent}{0em}
%\setlength{\parskip}{0.8em}

\title{\textbf{Binaries and Tides}}
\author{Aditya Vijaykumar}
\affiliation{International Centre for Theoretical Sciences, Bengaluru, India.}
\emailAdd{aditya.vijaykumar@icts.res.in}
\abstract{}
\begin{document}
\maketitle

\section{Tanja's Lectures at Gravitational Wave School 2017}

Non-black hole objects differ from black holes in that  :-
\begin{itemize}
	\item They can deform due to their rotational motion
	\item They can have effects due to no presence of horizon
	\item They can get tidally deformed
\end{itemize}
Tanja's lectures focus on tidal effects, which are the most promising candidates (as of 2017) to detect parameters of neutron stars.

We start off with tides in Newtonian physics.

\subsection{Newtonian Physics}
Notation - $ \vb{a} $ is 3-vector and $ \va{a} $ is 4-vector.

The force between two bodies of masses $ m  $ and $ M $ is,
\begin{equation}\label{key}
\vb{F} = - \dfrac{GmM}{r^2} \vu{n} \qq{,} U = -\dfrac{GM}{r} \qq{,} \vb{a} = \grad{U}
\end{equation}
where $ U $ and $ \vb{a} $ are the gravitational potential due to mass $ M $ and the acceleration respectively.

To make things more specific, consider a body of mass $ m_A $ with position vector $ \vb{z}_A $. The potential at $\vb{x} $ due to $ m_A$ is,
\begin{equation}\label{key}
U_A(\vb{x}) = \dfrac{Gm_A}{\abs{\vb{x} - \vb{z}_A}}
\end{equation}
Alternatively, if we consider extended bodies with density $ \rho(x) $, the potential can be written as,
\begin{align}\label{key}
U_A(\vb{x}) &= G \int \dd^3{\vb{x}'}\dfrac{\rho(\vb{x}' )}{\abs{\vb{x} - \vb{x}'}}\\
\implies \laplacian{U_A} &= G \int \dd^3{\vb{x}'} \rho(\vb{x}' ) \laplacian{\dfrac{1}{\abs{\vb{x} - \vb{x}'}}} \\
&=  G \int \dd^3{\vb{x}'} \rho(\vb{x}' ) \qty(-4 \pi \delta(\vb{x} - \vb{x}'))  \\
\laplacian{U_A} &= -4 \pi G \rho(\vb{x} )
\end{align}
\end{document}