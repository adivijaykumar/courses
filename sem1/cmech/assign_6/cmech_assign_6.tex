\documentclass{article}

\usepackage{fancyhdr}
\usepackage{extramarks}
\usepackage{amsmath}
\usepackage{amsthm}
\usepackage{amsfonts}
\usepackage{tikz}
\usepackage{physics}
\usepackage{amssymb}
\usepackage[plain]{algorithm}
\usepackage{algpseudocode}

\usetikzlibrary{automata,positioning}

% Basic Document Settings
%

\topmargin=-0.45in
\evensidemargin=0in
\oddsidemargin=0in
\textwidth=6.5in
\textheight=9.0in
\headsep=0.25in

\linespread{1.1}

\pagestyle{fancy}
\lhead{\hmwkAuthorName}
\chead{\hmwkClass\ : \hmwkTitle}
\rhead{\firstxmark}
\lfoot{\lastxmark}
\cfoot{\thepage}

\renewcommand\headrulewidth{0.4pt}
\renewcommand\footrulewidth{0.4pt}

\setlength\parindent{0pt}

%
% Create Problem Sections
%
\newcommand{\be}{\begin{equation}}
\newcommand{\ee}{\end{equation}}
\newcommand{\bes}{\begin{equation*}}
\newcommand{\ees}{\end{equation*}}
\newcommand{\bea}{\begin{flalign*}}
\newcommand{\eea}{\end{flalign*}}

\newcommand{\enterProblemHeader}[1]{
    \nobreak\extramarks{}{Problem \arabic{#1} continued on next page\ldots}\nobreak{}
    \nobreak\extramarks{Problem \arabic{#1} (continued)}{Problem \arabic{#1} continued on next page\ldots}\nobreak{}
}

\newcommand{\exitProblemHeader}[1]{
    \nobreak\extramarks{Problem \arabic{#1} (continued)}{Problem \arabic{#1} continued on next page\ldots}\nobreak{}
    \stepcounter{#1}
    \nobreak\extramarks{Problem \arabic{#1}}{}\nobreak{}
}

\setcounter{secnumdepth}{0}
\newcounter{partCounter}
\newcounter{homeworkProblemCounter}
\setcounter{homeworkProblemCounter}{1}
\nobreak\extramarks{Problem \arabic{homeworkProblemCounter}}{}\nobreak{}

%
% Homework Problem Environment
%
% This environment takes an optional argument. When given, it will adjust the
% problem counter. This is useful for when the problems given for your
% assignment aren't sequential. See the last 3 problems of this template for an
% example.
%
\newenvironment{homeworkProblem}[1][-1]{
    \ifnum#1>0
        \setcounter{homeworkProblemCounter}{#1}
    \fi
    \section{Problem \arabic{homeworkProblemCounter}}
    \setcounter{partCounter}{1}
    \enterProblemHeader{homeworkProblemCounter}
}{
    \exitProblemHeader{homeworkProblemCounter}
}

%
% Homework Details
%   - Title
%   - Due date
%   - Class
%   - Section/Time
%   - Instructor
%   - Author
%

\newcommand{\hmwkTitle}{Assignment\ \#6}
\newcommand{\hmwkDueDate}{Due 16th November 2018}
\newcommand{\hmwkClass}{Classical Mechanics}
\newcommand{\hmwkClassTime}{}
\newcommand{\hmwkClassInstructor}{Prof.Manas Kulkarni}
\newcommand{\hmwkAuthorName}{\textbf{Aditya Vijaykumar}}

%
% Title Page
%

\title{
    %\vspace{2in}
    \textmd{\textbf{\hmwkClass:\ \hmwkTitle}}\\
    \normalsize\vspace{0.1in}\small{\hmwkDueDate\ }\\
%    \vspace{3in}
}

\author{\hmwkAuthorName}
\date{}

\renewcommand{\part}[1]{\textbf{\large Part \Alph{partCounter}}\stepcounter{partCounter}\\}

%
% Various Helper Commands
%

% Useful for algorithms
\newcommand{\alg}[1]{\textsc{\bfseries \footnotesize #1}}

% For derivatives
\newcommand{\deriv}[1]{\frac{\mathrm{d}}{\mathrm{d}x} (#1)}

% For partial derivatives
\newcommand{\pderiv}[2]{\frac{\partial}{\partial #1} (#2)}

% Integral dx
\newcommand{\dx}{\mathrm{d}x}

% Alias for the Solution section header
\newcommand{\solution}{\textbf{\large Solution}}

% Probability commands: Expectation, Variance, Covariance, Bias
\newcommand{\E}{\mathrm{E}}
\newcommand{\Var}{\mathrm{Var}}
\newcommand{\Cov}{\mathrm{Cov}}
\newcommand{\Bias}{\mathrm{Bias}}

\begin{document}
\maketitle

\begin{homeworkProblem}[1]
	Liouville Theorem states that in a Hamiltonian system, the total phase space volume is constant in time.
	
	Let our system consist of $ N $ points $ (q_k, p_k) $ in a $ 2N $ dimensional phase space. The volume of the phase space is,
	\begin{equation*}
	V = \prod_{i} d q_i \qq{and} \tilde{V} = \prod_{i} d \tilde{q}_i 
	\end{equation*}
	where the tilded coordinates represents the volume at a later time. We know from Hamilton's equations,
	\begin{equation*}
	\tilde{q}_i = q_i + \pdv{H}{p_i} dt \qq{and} \tilde{p}_i = p_i - \pdv{H}{q_i} dt
	\end{equation*}
	We know for a fact that the volume transformation is related as follows,
	\begin{align*}
	\tilde{V} &= \det(J) V \\
	J &= \mqty[\pdv{\tilde{q}_i}{q_j} & \pdv{\tilde{q}_i}{p_j} \\ \pdv{\tilde{p}_i}{q_j} & \pdv{\tilde{p}_i}{q_j}]\\
	J &= \mqty[1 + \pdv[2]{H}{q_j}{p_i} dt & \pdv[2]{H}{p_i} dt \\ -\pdv[2]{H}{q_j} dt & 1 - \pdv[2]{H}{p_j}{q_j} dt] \\
	\det(J) &= 1 + \order{dt^2}
	\end{align*}
	Hence upto first order, $ \tilde{V} = V$. Hence proved. (files for the next part are attached)
\end{homeworkProblem}






\begin{homeworkProblem}[2]
	Transformations of coordinates $ (q,p,t) \rightarrow (Q,P,t)$ which preserves the form of Hamilton's equations are called canonical transformations. So, by definition,
	\begin{equation*}
	\dot{p} = \pdv{H}{q} \qq{,} \dot{q} = - \pdv{H}{p} \qq{and} \dot{P} = \pdv{K}{Q} \qq{,} \dot{Q} = -\pdv{K}{P}
	\end{equation*}
	The definition also implies that,
	\begin{align*}
	\delta(p \dot{q} - H) = 0 &\qq{and} \delta(P \dot{Q} - K) = 0\\
	\lambda(p \dot{q} - H) &= P \dot{Q} - K + \dv{F}{t}
	\end{align*}
	We deal with the $ \lambda = 1 $ case. The $ \dv{F}{t} $ term comes from the fact that Lagrangians are not unique and we can always add a total time derivative term without changing the equations of motion. If the above condition is satisfied, the transformation $ (q,p,t) \rightarrow (Q,P,t)$ is guaranteed to be canonical, and the function $ F $ is called a generating function. We deal with four classes of generating functions case-by-case,
	\begin{itemize}
		\item $ F = F_1 (q,Q,t) $,
		\begin{equation*}
		p \dot{q} - H = P \dot{Q} - K + \dv{F_1}{t} = P \dot{Q} - K + \pdv{F_1}{q}\dot{q} + \pdv{F_1}{Q}\dot{Q} + \pdv{F_1}{t}
		\end{equation*}
		As $ q $ and $ Q $ are independent, the coefficients should vanishh independently, such that $ K = H + \pdv{F_1}{t} $. This implies,
		\begin{equation*}
		\pdv{F_1}{q} = p \qq{and} \pdv{F_1}{Q} = -P
		\end{equation*}
		
		\item $ F = F_2 (q,P,t) - QP$,
		\begin{equation*}
		p \dot{q} - H = P \dot{Q} - K + \dv{F_2}{t} - \dv{(QP)}{t} = P \dot{Q} - K + \pdv{F_2}{q}\dot{q} + \pdv{F_2}{P}\dot{P} + \pdv{F_2}{t} - P \dot{Q} - Q \dot{P}
		\end{equation*}
		\begin{equation*}
		\implies \pdv{F_2}{q} = p \qq{and} \pdv{F_2}{P} = Q
		\end{equation*}
		\item $ F = F_3 (p,Q,t) + qp$,
		\begin{equation*}
		p \dot{q} - H = P \dot{Q} - K + \dv{F_3}{t} + \dv{(qp)}{t} = P \dot{Q} - K + \pdv{F_3}{Q}\dot{Q} + \pdv{F_3}{p}\dot{p} + \pdv{F_3}{t} + p \dot{q} + q \dot{p}
		\end{equation*}
		\begin{equation*}
		\implies \pdv{F_3}{Q} = -P \qq{and} \pdv{F_2}{p} = -q
		\end{equation*}
		\item $ F = F_4 (p,P,t) + qp - QP$,
		\begin{equation*}
		p \dot{q} - H = P \dot{Q} - K + \dv{F_4}{t} + \dv{(qp - QP)}{t} = P \dot{Q} - K + \pdv{F_4}{P}\dot{P} + \pdv{F_4}{p}\dot{p} + \pdv{F_4}{t} + p \dot{q} + q \dot{p} - P \dot{Q} - Q \dot{P}
		\end{equation*}
		\begin{equation*}
		\implies \pdv{F_4}{P} = Q \qq{and} \pdv{F_4}{p} = -q
		\end{equation*}
	\end{itemize}

	\textbf{Part (b)}\\
	We first use the Poisson Bracket invariance approach. We are given,
	\begin{equation*}
	Q_1 = q_1 \qq{,} Q_2 = p_2 \qq{,} P_1 = p_1 - 2 p_2 \qq{,} P_2 = -2 q_1 - q_2
	\end{equation*}
	Consider $ \pb{Q_1}{Q_2} $,
	\begin{align*}
	\pb{Q_1}{Q_2} &= \sum_{i =1}^{2} \pdv{Q_1}{q_i} \pdv{Q_2}{p_i} - \pdv{Q_1}{p_i} \pdv{Q_2}{q_i} = 0 \\
	\pb{P_1}{P_2} &= \sum_{i =1}^{2} \pdv{P_1}{q_i} \pdv{P_2}{p_i} - \pdv{P_1}{p_i} \pdv{P_2}{q_i} =  - \pdv{P_1}{p_1} \pdv{P_2}{q_1} - \pdv{P_1}{p_2} \pdv{P_2}{q_2} = 2-2 = 0   \\
	\pb{Q_1}{P_2} &= \sum_{i =1}^{2} \pdv{Q_1}{q_i} \pdv{P_2}{p_i} - \pdv{Q_1}{p_i} \pdv{P_2}{q_i} =   \pdv{Q_1}{q_1} \pdv{P_2}{p_1} = 0 \\
	\pb{Q_2}{P_1} &= \sum_{i =1}^{2} \pdv{Q_2}{q_i} \pdv{P_1}{p_i} - \pdv{Q_2}{p_i} \pdv{P_1}{q_i} =  - \pdv{Q_2}{p_2} \pdv{P_1}{q_2} = 0\\
	\pb{Q_1}{P_1} &= \sum_{i =1}^{2} \pdv{Q_1}{q_i} \pdv{P_1}{p_i} - \pdv{Q_1}{p_i} \pdv{P_1}{q_i} =  \pdv{Q_1}{q_1} \pdv{P_1}{p_1}  = 1\\
	\pb{Q_2}{P_2} &= \sum_{i =1}^{2} \pdv{Q_2}{q_i} \pdv{P_2}{p_i} - \pdv{Q_2}{p_i} \pdv{P_2}{q_i} = - \pdv{Q_2}{p_2} \pdv{P_2}{q_2}  = 1
	\end{align*}
	Hence, as $ \pb{Q_i}{P_j} = \delta_{ij}, \pb{Q_i}{Q_j} = 0,  \pb{P_i}{P_j}  = 0$, the transformation is canonical. We now use the symplectic approach. If we denote $ X = \mqty[Q_1 & Q_2  & P_1 & P_2]^T, x = \mqty[q_1 & q_2  & p_1 & p_2]^T  $, then $ X = Mx $ where $ M $ is the transformation matrix. From the definitions of the $ X $, we can see that,
	\begin{equation*}
	M = \left(
	\begin{array}{cccc}
	1 & 0 & 0 & 0 \\
	0 & 0 &0 & 1 \\
	0 & 0 & 1 & -2 \\
	-2 & -1 & 0 & 0 \\
	\end{array}
	\right)
	\end{equation*}
	For the transformation to be a canonical transformation, $ M^T J M = J $, where,
	\begin{align*}
	J &= \left(
	\begin{array}{cccc}
	0 & 0 & 1 & 0 \\
	0 & 0 &0 & 1 \\
	-1 & 0 & 0 & 0 \\
	0 & -1 & 0 & 0 \\
	\end{array}
	\right)
	\end{align*}
	\begin{align*}
	M^T J M &= \left(
	\begin{array}{cccc}
	1 & 0 & 0 & 0 \\
	0 & 0 &0 & 1 \\
	0 & 0 & 1 & -2 \\
	-2 & -1 & 0 & 0 \\
	\end{array}
	\right) \left(
	\begin{array}{cccc}
	0 & 0 & 1 & 0 \\
	0 & 0 &0 & 1 \\
	-1 & 0 & 0 & 0 \\
	0 & -1 & 0 & 0 \\
	\end{array}
	\right) \left(
	\begin{array}{cccc}
	1 & 0 & 0 & 0 \\
	0 & 0 &0 & 1 \\
	0 & 0 & 1 & -2 \\
	-2 & -1 & 0 & 0 \\
	\end{array}
	\right) \\
	&= \left(
	\begin{array}{cccc}
	0 & 0 & 1 & 0 \\
	0 & -1 &0 & 0 \\
	-1 & 2 & 0 & 1 \\
	0 & 0 & -2 & 0 \\
	\end{array}
	\right)
	\left(
	\begin{array}{cccc}
	1 & 0 & 0 & 0 \\
	0 & 0 &0 & 1 \\
	0 & 0 & 1 & -2 \\
	-2 & -1 & 0 & 0 \\
	\end{array}
	\right) \\
	M^T J M &= \left(
	\begin{array}{cccc}
	0 & 0 & 1 & 0 \\
	0 & 0 &0 & 1 \\
	-1 & 0 & 0 & 0 \\
	0 & -1 & 0 & 0 \\
	\end{array}
	\right) = J
	\end{align*}
	Hence, it is a canonical transformation.\\
	\textbf{Part (c)}
	\begin{align*}
	l_i &= \epsilon_{i j k} x_j p_k \qq{using the Einstein summation convention}
	\end{align*}
	We now note the following,
	\begin{align*}
	\pb{l_i}{l_j} &= \epsilon_{i a b} \epsilon_{j m n} \pb{x_a p_b}{x_m p_n} \\
	&= \epsilon_{i a b} \epsilon_{j m n} \pb{x_a p_b}{x_m p_n}\\
	&= \epsilon_{i a b} \epsilon_{j m n} (\pb{x_a }{ p_n} x_m p_b + \pb{p_b}{x_m} x_a p_n)\\
	&= \epsilon_{i a b} \epsilon_{j m n} ( \delta_{an} x_m p_b - \delta_{bm} x_a p_n) \\
	&= \epsilon_{i n b} \epsilon_{j m n} x_m p_b - \epsilon_{i a m} \epsilon_{j m n}  x_a p_n \\
	&= - \epsilon_{i b n} \epsilon_{j m n} x_m p_b + \epsilon_{i m a} \epsilon_{j m n}  x_a p_n \\
	&= - (\delta_{ij} \delta_{bm} - \delta_{im} \delta_{j b}) x_m p_b + (\delta_{ij} \delta_{an} - \delta_{aj} \delta_{in}) x_a p_n \\
	&= - \delta_{ij}  x_b p_b  + x_i p_j + \delta_{ij}  x_a p_a  -  x_j p_i \\
	&= + x_i p_j -  x_j p_i \\
	\pb{l_i}{l_j} &= \epsilon_{i j k} l_k
	\end{align*}
	\begin{align*}
	\pb{x_i}{l_j} &= \epsilon_{j m n} \pb{x_i}{ x_m p_n}\\
	&= \epsilon_{j m n} x_m  \pb{x_i}{ p_n}\\
	&= \epsilon_{j m n} x_m  \delta_{in}\\
	\pb{x_i}{l_j} &= \epsilon_{i j m} x_m 
	\end{align*}
	\begin{align*}
	\pb{p_i}{l_j} &= \epsilon_{j m n} \pb{p_i}{ x_m p_n}\\
	&= \epsilon_{j m n} p_n  \pb{p_i}{ x_m}\\
	&= -\epsilon_{j m n} p_n  \delta_{im}\\
	\pb{p_i}{l_j}&= \epsilon_{i j n} p_n
	\end{align*}
\end{homeworkProblem}












\begin{homeworkProblem}[3]
	We are given the Hamiltonian and generating function,
	\begin{equation*}
	H = \dfrac{p^2}{2} + \dfrac{\omega^2 x^2 }{2} + \alpha x^3 + \beta x p^2 \qq{and} \phi = xP + ax^2 P + bP^3
	\end{equation*}
	$ \phi = \phi (x,P) $. For $ \phi $ to be a canonical transformation,
	\begin{align*}
	\pdv{\phi}{x} = p &\qq{and} \pdv{\phi}{P} = Q\\
	\implies P + 2 a x P = p &\qq{and} x + ax^2 + 3b P^2 = Q \\
	\implies P \sqrt{-12 a b P^2+4 a Q+1} = p &\qq{and}\frac{\sqrt{-12 a b P^2+4 a Q+1}-1}{2 a} = x
	\end{align*}
	where we have only considered the $ x $ root with positive sign before the discriminant. Then,
	\begin{align*}
	K(Q, P) &= \frac{\alpha  \left(\sqrt{-12 a b P^2+4 a Q+1}-1\right)^3}{8 a^3}+\frac{\omega ^2 \left(\sqrt{-12 a b P^2+4 a Q+1}-1\right)^2}{8 a^2} \\
	&+\frac{\beta  P^2 \left(-12 a b P^2+4 a Q+1\right) \left(\sqrt{-12 a b P^2+4 a Q+1}-1\right)}{2 a}+ \frac{1}{2} P^2 \left(-12 a b P^2+4 a Q+1\right)
	\end{align*}
	Expanding the above upto third order, we have,
	\begin{align*}
	K(Q, P) &= Q^3 \left(P^2 \left(-30 a^2 b \omega ^2-2 a^2 \beta +36 \alpha  a b\right)-a \omega ^2+\alpha \right)+Q^2 \left(P^2 \left(9 a b \omega ^2+3 a \beta -9 \alpha  b\right)+\frac{\omega ^2}{2}\right)\\&+P^2 Q \left(2 a-3 b \omega ^2+\beta \right)+\frac{P^2}{2}
	\end{align*}
	As anharmonic terms of third order should not be present, we can see from above that,
	\begin{equation*}
	- a \omega^2 + \alpha = 0 \qq{and} 2 a-3 b \omega ^2+\beta = 0 \implies a = \dfrac{\alpha}{\omega^2} \qq{and} b = \dfrac{1}{3 \omega^2} \qty(\dfrac{2 \alpha}{\omega^2} + \beta)
	\end{equation*}
	Now we need to find $ \dot{x} $. From Hamilton's equation of motion we have,
	\begin{align*}
	\dot{x} = \pdv{H}{p} = p (1 + 2 \beta x) \implies p = \dfrac{\dot{x}}{1 + 2 \beta x} \\
	\dot{p} = -\dfrac{2 \beta \dot{x}^2 - \ddot{x} (1 + 2 \beta x) }{(1 + 2 \beta x)^2} = -\pdv{H}{x} = -\omega^2 x - 3 \alpha x^2 - \beta p^2\\
	\implies -{2 \beta \dot{x}^2 - \ddot{x} (1 + 2 \beta x) } = -(\omega^2 x - 3 \alpha x^2) (1 + 2 \beta x)^2 - \beta \dot{x}^2 \\
	\implies {\ddot{x} (1 + 2 \beta x) -\beta \dot{x}^2 } + (\omega^2 x - 3 \alpha x^2) (1 + 2 \beta x)^2 = 0
	\end{align*}
	$ x(t) $ will be given by the solution of this equation.
	
	
	\textbf{Part (b)}\\
	\begin{itemize}
		\item $ \phi(\va{r}, \va{P}) = (\va{r} \cdot \va{P}) + (\delta \va{a} \cdot \va{P}) $\\
		This looks like $ F_2 (q,P) $. From the results of Problem 2, we can then write,
		\begin{align*}
		\pdv{\Phi}{r} = p_r = P_r \qq{,} \pdv{\Phi}{\theta} = p_\theta =0 \qq{,} \pdv{\Phi}{\phi} = p_\phi = 0 \\
		\pdv{\Phi}{P_r} = Q_r =  r + \delta a_x\qq{,} \pdv{\Phi}{P_\theta} = Q_\theta = \delta a_\theta \qq{,} \pdv{\Phi}{P_\phi} = Q_\phi = \delta a_\phi
		\end{align*}
		as $  r + \delta a = Q $, it is evident that the transformation is a translation by constant, as the momentum remains the same but the coordinates get shifted by a constant amount.
		
		\item $ \Phi(\va{r}, \va{P}) = (\va{r} \cdot \va{P}) + (\va{\delta \psi} \cdot \va{r} \cross \va{P}) $ \\
		This looks like $ F_2 $ too. We have,
		\begin{equation*}
		p \cdot \delta \psi = \qty(\dv{\Phi}{r} = P + \pdv{(\va{\delta \psi} \cdot \va{r} \cross \va{P})}{r})\cdot \delta \psi =  P\cdot \delta \psi + \pdv{(\va{r} \cdot \va{P} \cross \va{\delta \psi})}{r} \cdot \delta \psi= P \delta \psi 
		\end{equation*}
		\begin{equation*}
		Q = \pdv{\Phi}{p} = r +  r \delta \psi 
		\end{equation*}
		The above transformations look like rotations in the phase plane.
		
		\item $ \Phi = qP + \delta \tau H(q,p,t) $\\
		This looks like $ F_2 $ again. We write,
		\begin{align*}
		\pdv{\Phi}{q} &= P + \delta \tau (-\dot{p}) = p \qq{and}\\
		 \pdv{\Phi}{P} &= q + \delta \tau \pdv{H}{P}\\
		  &=  q + \delta \tau \pdv{H}{P} \\
		  &=  q + \delta \tau \qty( \pdv{H}{p} \pdv{p}{P} + \pdv{H}{q} \pdv{q}{P} ) \\
		  &=  q + \delta \tau \qty( \dot{q} (1 - \delta \tau \dot{p}) )\\
		  Q & \approx q + \delta \tau \dot{q}
		\end{align*}
		\begin{equation*}
		\therefore Q \approx q + \delta \tau \dot{q} \qq{and} P \approx p +  \delta \tau \dot{p}
		\end{equation*}
		So the canonical transformation just corresponds to time translation by parameter $ d\tau $.
		
		\item  $  \Phi = \va{r} \cdot \va{P} + (r^2 + P^2)\delta a  $
		\begin{align*}
		\pdv{\Phi}{r} = P_r + 2 r \delta a \implies P_r = p_r - 2 r \delta a \\
		\pdv{\Phi}{P} = r + 2 P \delta a \implies Q = r + 2 P \delta a 
		\end{align*}
		This is equivalent to rotation in the phase space by amount $ 2 \delta a $
	\end{itemize}
\end{homeworkProblem}












\begin{homeworkProblem}[4]
	We first note that,
	\begin{equation*}
	y = x^2 \implies \dot{y} = 2 x \dot{x}
	\end{equation*}
	and write down the Lagrangian and Hamiltonian of the system,
	\begin{align*}
	L &= \dfrac{m \dot{x}^2}{2} + \dfrac{m \dot{y}^2}{2} - mgy \\
	L &= \dfrac{m \dot{x}^2}{2} + {2m  x^2 \dot{x}^2 } - mgx^2\\
	\implies p = m \dot{x} + 4 mx^2 \dot{x} &\implies  \dot{x} = \dfrac{p}{m(1 + 4x^2)}
	\end{align*}
	Thus, we can write the Hamiltonian as,
	\begin{align*}
	H(x,p) &= \dfrac{p^2}{m(1 + 4x^2)} - \dfrac{m}{2}(1 + 4x^2)\dfrac{p^2}{m^2 (1 + 4x^2)^2} + mg x^2\\
	H(x,p) &= \dfrac{p^2}{2m(1 + 4x^2)} + mgx^2
	\end{align*} 
	The Hamilton-Jacobi equation is given by,
	\begin{equation*}
	\dfrac{1}{2m(1 + 4x^2)}\qty(\pdv{S}{x})^2 + mgx^2 + \pdv{S}{t} = 0 
	\end{equation*}
	Substituting $ S = W(x) - Et $, we get,
	\begin{align*}
	\dfrac{1}{2m(1 + 4x^2)}\qty(\dv{W}{x})^2 + &mgx^2 - E = 0
	\implies \dv{W}{x} = \sqrt{2m(E - mgx^2)(1 + 4x^2)}\\
	\implies S &= \int dx \sqrt{2m(E - mgx^2)(1 + 4x^2)} - Et
	\end{align*}
	We know that $ \pdv{S}{E} = \alpha t + \beta $ for constants $ \alpha $ and $ \beta $. Hence the equation of motion is,
	\begin{equation*}
	\sqrt{\dfrac{m(1+4x^2)}{2(E - mgx^2)}} - E = \alpha t + \beta
	\end{equation*}
	\textbf{Part (b)}\\
	We first note that,
	\begin{equation*}
	z = \dfrac{\xi^2 - \eta^2}{2} \qq{,} \rho = \eta \xi \qq{,} \psi = \phi \implies \dot{z} = \xi \dot{\xi} - \eta \dot{\eta} \qq{,} \dot{\rho} = \eta \dot{\xi} + \xi \dot{\eta} \qq{,} \dot{\phi} = \dot{\psi}
	\end{equation*}
	We first write down the Lagrangian and canonical momenta,
	\begin{align*}
	L &= \dfrac{m (\dot{\rho}^2 + \rho^2 \dot{\phi}^2 + \dot{z}^2)}{2} - \dfrac{k}{\sqrt{\rho^2 + z^2} } + Fz\\
	&= \dfrac{m ( \eta^2 \dot{\xi}^2 + \xi^2 \dot{\eta}^2+ 2 \eta \xi \dot{\eta} \dot{\xi} + \eta^2 \xi^2 \dot{\psi}^2 + \xi^2 \dot{\xi}^2 - 2 \xi \dot{\xi} \eta \dot{\eta} + \eta^2 \dot{\eta}^2 )}{2} - \dfrac{k}{\sqrt{\qty(\dfrac{\xi^2 - \eta^2}{2})^2 + \eta^2 \xi^2}} + F\dfrac{\xi^2 - \eta^2}{2}\\
	L&= m\dfrac{(\eta^2 + \xi^2)( \dot{\xi}^2 + \dot{\eta}^2)+ \eta^2 \xi^2 \dot{\psi}^2}{2} - \dfrac{2k}{\eta^2 + \xi^2} + F\dfrac{\xi^2 - \eta^2}{2}\\
	\implies &p_\xi = m (\eta^2 + \xi^2) \dot{\xi} \qq{,} p_\eta = m (\eta^2 + \xi^2) \dot{\eta} \qq{,} p_\psi = m \eta^2 \xi^2 \dot{\psi}\\
	\implies H &= \dfrac{p_\xi^2 + p_\eta^2}{2 m(\eta^2 + \xi^2)} + \dfrac{p_\psi^2}{2m \eta^2 \xi^2} +\dfrac{2k}{\eta^2 + \xi^2}  - F\dfrac{\xi^2 - \eta^2}{2}
	\end{align*}
	
	Let's apply the transformations given in the problem
	We can now write down the Hamilton-Jacobi equation as,
	\begin{equation*}
	\pdv{S}{t} +  \dfrac{1}{2 m(\eta^2 + \xi^2)} \qty[\qty(\pdv{S}{\xi})^2 + \qty(\pdv{S}{\eta})^2 ]+ \dfrac{1}{2m \eta^2 \xi^2} \qty(\pdv{S}{\psi})^2 + \dfrac{2k}{\eta^2 + \xi^2}  - F\dfrac{\xi^2 - \eta^2}{2} = 0
	\end{equation*}
	We now make the substitution $  S = S_1(\xi) + S_2(\eta) + S_3(\psi) - Et $. We then have,
	\begin{equation*}
	  \dfrac{1}{2 m(\eta^2 + \xi^2)} \qty[\qty(\dv{S_1}{\xi})^2 + \qty(\dv{S_2}{\eta})^2 ]+ \dfrac{1}{2m \eta^2 \xi^2} \qty(\dv{S_3}{\psi})^2 + \dfrac{2k}{\eta^2 + \xi^2}  - F\dfrac{\xi^2 - \eta^2}{2} = E
	\end{equation*}
	Out of the four terms above, we see that only the third terms depends on $ \psi $. As the RHS is a constant, the dependence on $ \psi $ also should vanish. This means,
	\begin{equation*}
	\qty(\dv{S_3}{\psi})^2 = \beta_1^2
	\end{equation*}
	Making the above substitution, multiplying the equation by $ 2m(\eta^2 + \xi^2) $, and collecting terms, we get,
	\begin{equation*}
	\qty[-2m \eta^2 E + \qty(\dv{S_2}{\eta})^2 + \dfrac{\beta_1^2}{ \eta^2} + Fm \eta^4] + \qty[-2m\xi^2 E + \qty(\pdv{S}{\xi})^2 + \dfrac{\beta_1^2}{ \xi^2} - Fm \xi^4] + 4mk = 0
	\end{equation*}
	We can see from the above form that the equation has become completely separable in the new coordinates.
\end{homeworkProblem}


\end{document}
