\documentclass{article}

\usepackage{fancyhdr}
\usepackage{extramarks}
\usepackage{amsmath}
\usepackage{amsthm}
\usepackage{amssymb}
\usepackage{amsfonts}
\usepackage{tikz}
\usepackage{physics}
\usepackage[plain]{algorithm}
\usepackage{algpseudocode}
\usepackage{graphicx,wrapfig,lipsum}
\usetikzlibrary{automata,positioning}

%
% Basic Document Settings
%

\topmargin=-0.45in
\evensidemargin=0in
\oddsidemargin=0in
\textwidth=6.5in
\textheight=9.0in
\headsep=0.25in

\linespread{1.1}

\pagestyle{fancy}
\lhead{\hmwkAuthorName}
\chead{\hmwkClass\ : \hmwkTitle}
\rhead{\firstxmark}
\lfoot{\lastxmark}
\cfoot{\thepage}

\renewcommand\headrulewidth{0.4pt}
\renewcommand\footrulewidth{0.4pt}

\setlength\parindent{0pt}

%
% Create Problem Sections
%
\newcommand{\be}{\begin{equation}}
\newcommand{\ee}{\end{equation}}
\newcommand{\bes}{\begin{equation*}}
\newcommand{\ees}{\end{equation*}}
\newcommand{\bea}{\begin{flalign*}}
\newcommand{\eea}{\end{flalign*}}


\newcommand{\enterProblemHeader}[1]{
    \nobreak\extramarks{}{Problem \arabic{#1} continued on next page\ldots}\nobreak{}
    \nobreak\extramarks{Problem \arabic{#1} (continued)}{Problem \arabic{#1} continued on next page\ldots}\nobreak{}
}

\newcommand{\exitProblemHeader}[1]{
    \nobreak\extramarks{Problem \arabic{#1} (continued)}{Problem \arabic{#1} continued on next page\ldots}\nobreak{}
    \stepcounter{#1}
    \nobreak\extramarks{Problem \arabic{#1}}{}\nobreak{}
}

\setcounter{secnumdepth}{0}
\newcounter{partCounter}
\newcounter{homeworkProblemCounter}
\setcounter{homeworkProblemCounter}{1}
\nobreak\extramarks{Problem \arabic{homeworkProblemCounter}}{}\nobreak{}

%
% Homework Problem Environment
%
% This environment takes an optional argument. When given, it will adjust the
% problem counter. This is useful for when the problems given for your
% assignment aren't sequential. See the last 3 problems of this template for an
% example.
%
\newenvironment{homeworkProblem}[1][-1]{
    \ifnum#1>0
        \setcounter{homeworkProblemCounter}{#1}
    \fi
    \section{Problem \arabic{homeworkProblemCounter}}
    \setcounter{partCounter}{1}
    \enterProblemHeader{homeworkProblemCounter}
}{
    \exitProblemHeader{homeworkProblemCounter}
}

%
% Homework Details
%   - Title
%   - Due date
%   - Class
%   - Section/Time
%   - Instructor
%   - Author
%

\newcommand{\hmwkTitle}{Assignment\ \#3}
\newcommand{\hmwkDueDate}{Due on 23rd October, 2018}
\newcommand{\hmwkClass}{Fluid Mechanics}
\newcommand{\hmwkClassTime}{}
\newcommand{\hmwkClassInstructor}{}
\newcommand{\hmwkAuthorName}{\textbf{Aditya Vijaykumar}}

%
% Title Page
%

\title{
    %\vspace{2in}
    \textmd{\textbf{\hmwkClass:\ \hmwkTitle}}\\
    \normalsize\vspace{0.1in}\small{\hmwkDueDate\ }\\
%    \vspace{3in}
}

\author{\hmwkAuthorName}
\date{}

\renewcommand{\part}[1]{\textbf{\large Part \Alph{partCounter}}\stepcounter{partCounter}\\}

%
% Various Helper Commands
%

% Useful for algorithms
\newcommand{\alg}[1]{\textsc{\bfseries \footnotesize #1}}

% For derivatives
\newcommand{\deriv}[1]{\frac{\mathrm{d}}{\mathrm{d}x} (#1)}

% For partial derivatives
\newcommand{\pderiv}[2]{\frac{\partial}{\partial #1} (#2)}

% Integral dx
\newcommand{\dx}{\mathrm{d}x}

% Alias for the Solution section header
\newcommand{\solution}{\textbf{\large Solution}}

% Probability commands: Expectation, Variance, Covariance, Bias
\newcommand{\E}{\mathrm{E}}
\newcommand{\Var}{\mathrm{Var}}
\newcommand{\Cov}{\mathrm{Cov}}
\newcommand{\Bias}{\mathrm{Bias}}

\begin{document}

\maketitle
\textbf{Acknowledgements} - I thank Saumav Kapoor for discussions.


%\pagebreak
\begin{homeworkProblem}[1]
	\textbf{Part (a)}\\
	The unsteady state Bernoulli equation tells us that,
	\begin{align*}
	\pdv{\phi}{t} + \dfrac{P_{atm}}{\rho} + \dfrac{v^2}{2} + gz &= constant\\
	\pdv{v}{t} + v\pdv{v}{y} + g &= 0
	\end{align*}
	where we have differentiated with $ y $ going from the first to the second line.
	
	From the geometry of the cone, we have,
	\begin{align*}
	r_1 &= r_0 +y \tan \alpha\\
	\pi r_1^2 &= \pi r_0^2 + \pi y^2 \tan^2 \alpha + 2\pi r_0 y \tan \alpha\\
	A_1 &= \pi r_0^2 + \pi y^2 \tan^2 \alpha + 2 \sqrt{A \pi } y \tan \alpha	\end{align*}
	From this and the continuity equation, we get,
\begin{align*}
v = \dfrac{r_0^2}{\beta^2}v_0 \qq{, } \beta^2 = r_0^2 +  y^2 \tan^2 \alpha + 2 r_0 y \tan \alpha
\end{align*}
Substituting back into earlier equation,
\begin{equation*}
\dfrac{r_0^2}{\beta^2}\pdv{v_0}{t} - \dfrac{r_0^2}{\beta^2} v_0^2 \dfrac{r_0^2}{\beta^4} (2y \tan \alpha + 2 r_0 \tan \alpha) + g = 0
\end{equation*}
The above equation holds for all $ y $ and specifically $ y=0 $. Let's put $ y=0$ and $ \beta^2 = r_0^2 $,
\begin{equation*}
	\pdv{v_0}{t} - \dfrac{2}{r_0} v_0^2 \tan \alpha + g = 0
\end{equation*}
Solving this gives,
\begin{equation*}
v_0 = \sqrt{\dfrac{gr_0}{2 \tan \alpha}} \coth\qty( \sqrt{\dfrac{2g\tan \alpha}{r_0}}t + C)
\end{equation*}
All that is left is to evaluate the constant $ C $.

We go back to the continuity equation, which says,
\begin{equation*}
\pi (y + y_0)^2 v(y,t) \tan^2 \alpha = K(t) \implies v(y,t) = \dfrac{K(t)}{\pi (y+y_0)^2  \tan^2 \alpha } \implies \phi(y,t) = -\dfrac{K(t)}{\pi (y+y_0) \tan^2 \alpha} 
\end{equation*}
Writing Bernoulli between points $ y=h $ and $ y=r_0 \tan \alpha $,
\begin{align*}
	-\dfrac{K'}{\pi (h+y_0) \tan^2 \alpha} + \dfrac{K^2}{2\pi^2 \tan^4 \alpha h^4} + gh &= -\dfrac{K'}{\pi (y_0) \tan^2 \alpha} + \dfrac{K^2}{2\pi^2 \tan^4 \alpha r_0^4}
\end{align*}
Substituting $ K(t) = v_0 \pi r_0^2 \tan^2 \alpha $, we can get an expression for the constant $ C $ in terms of height $ h $.



\textbf{Part (b)}\\
We know that,
\begin{equation*}
\dv{V}{t} = Q \implies t = \int \dfrac{dV}{Q}
\end{equation*}
\begin{align*}
\therefore t_1  =  \int \dfrac{dV_1}{Q_1} &\qq{ } t_2  =  \int \dfrac{dV_2}{Q_2}\\
t_1  =  \int \dfrac{\pi h^2 dh}{A_1 \sqrt{2gh}} &\qq{ } t_2  =  \int \dfrac{\pi h^2 dh}{A_2 \sqrt{2gh}}\\
t_1 - t_2  = \qty(\dfrac{1}{A_1} - \dfrac{1}{A_2})&\int \dfrac{\pi h^2 dh}{\sqrt{2gh}} \implies \qq{tank with larger base area will drain faster}
\end{align*}
\end{homeworkProblem}










\begin{homeworkProblem}[2]
	We first write the Bernoulli equation for between the point where water leaves the tap $ (z_1=0) $ and a point distance $ h $ below $ (z_2=-h) $,
	\begin{equation*}
	\dfrac{P_0 }{\rho} + \dfrac{v_1^2}{2} = \dfrac{P_0 }{\rho} + \dfrac{v_2^2}{2} - gh  \implies \dfrac{v_2^2}{v_1^2} = 1 + \dfrac{2gh}{v_1^2}
	\end{equation*}
	The continuity equation gives,
	\begin{equation*}
	\pi r_1^2 v_1 = \pi r_2^2 v_2 \implies \dfrac{v_2}{v_1} = \dfrac{r_1^2}{r_2^2}
	\end{equation*}
	Using the above two equations, we get,
	\begin{equation*}
	\dfrac{r_1^4}{r_2^4} = 1 + \dfrac{2gh}{v_1^2} \implies \boxed{\dfrac{R_0^4}{r^4} = 1 + \dfrac{2gH}{v_0^2}}
	\end{equation*}
	where $ r $ is the cross-sectional radius at height $ H $ below the tap, and $ R_0 $ and $ v_0 $ and the cross-sectional radius and velocity of the water the moment it leaves the tap.
	
\end{homeworkProblem}







\begin{homeworkProblem}[3]
	We work in cylindrical coordinates. The assumption of laminar flow $ \implies u_r = u_\phi = 0 $. The assumption of axisymmetry $ \implies u_z = u_z(r, z) $ The continuity condition $ \div{\va{u}} = 0$ gives,
	\begin{equation*}
	\pdv{u_z}{z} = 0 \implies u_z = u_z(r)
	\end{equation*}
	We now proceed and write the Navier-Stokes equation in cylindrical coordinates component-wise,
	\begin{align*}
	0 &= -\dfrac{1}{\rho}\pdv{P}{r}\\
	0 &= -\dfrac{1}{\rho r}\pdv{P}{\phi}\\
	0 &= -\dfrac{1}{\rho}\pdv{P}{z} + \nu \dfrac{1}{r} \pdv{r} \qty(r \pdv{u_z}{r})
	\end{align*}
	We can see from the first two equations that $ P = P(z) $. In the third equation, since the first term on the RHS depends only on $ z $ and the second term depends only on $ r $, we say that each of the terms should be constants. We get,
	\begin{align*}
	 \dfrac{1}{r} \dv{r} \qty(r \dv{u_z}{r}) &= \dfrac{1}{\mu}\dv{P}{z} = constant\\
	 \dv{r} \qty(r \dv{u_z}{r}) &= \dfrac{r}{\mu}\dv{P}{z}\\
	 \implies r \dv{u_z}{r} &= \dfrac{r^2}{2\mu}\dv{P}{z} + A\\
	 \implies \dv{u_z}{r} &= \dfrac{r}{2\mu}\dv{P}{z} + \dfrac{A}{r}\\
	 \implies {u_z} &= \dfrac{r^2}{4\mu}\dv{P}{z} + A \ln r + B\\
	\end{align*}
	We need the flow to be well-defined at $ r=0 $. As it stands, for non-zero $ A $, the flow will not be well-defined for $ r=0 $, which is undesirable. Hence, $ A=0 $.
	
	If $ R $ is the radius of the pipe, and the pipe is not moving, we get $ u_z(R) = 0 $, which means,
	\begin{equation*}
	0 = \dfrac{R^2}{4\mu}\dv{P}{z} + B \implies B = - \dfrac{R^2}{4\mu}\dv{P}{z}
	\end{equation*}
	So the final answer is,
	\begin{equation*}
	u_z = \dfrac{1}{4 \mu} \dv{P}{z} (r^2 - R^2) 
	\end{equation*}
\end{homeworkProblem}













\begin{homeworkProblem}[4]
	We solve the problem for a two-dimensional jet. The orthogonal directions are taken to be $ x $ and $ y $. We assume that steady state.
	
	The Continuity equation gives us,
	\begin{equation*}
	\pdv{u}{x} + \pdv{v}{y} = 0 \implies u\pdv{u}{x} + u\pdv{v}{y} = 0 
	\end{equation*}
	The $ x $-component of the Navier-Stokes gives us,
	\begin{equation}
	u \pdv{u}{x} + v \pdv{u}{y} = \nu \pdv[2]{u}{y}
	\label{xns}
	\end{equation}
	
	Adding up the two equations, one has,
	\begin{align*}
		2u \pdv{u}{x} + v \pdv{u}{y} + u \pdv{v}{y}  &= \nu \pdv[2]{u}{y}\\
		\pdv{(u^2)}{x} + \pdv{(uv)}{y} &= \nu \pdv[2]{u}{y}
	\end{align*}
	Integrating both sides with respect to $ y $,
	\begin{equation*}
	\pdv{x} \int_{-\infty}^{\infty} u^2 dy + \eval{uv}_{-\infty}^{\infty} = \nu \eval{\pdv{u}{y}}_{-\infty}^\infty
	\end{equation*}
	We now would like to impose boundary conditions. The velocity is purely along the $ x $-axis at $ y = 0 $. As $ y \rightarrow \pm \infty $, both $ u \rightarrow 0  $ and $ v $ $ \rightarrow 0 $, and so do their derivatives.
	
	We then have,
	\begin{equation}
	\pdv{x} \int_{-\infty}^{\infty} u^2 dy = 0 \implies \int_{-\infty}^{\infty} u^2 dy = constant = M
	\label{conserv}
	\end{equation}
	We now try to guess the form of the similarity solution for this problem. Let's assume,
	\begin{equation}
	x \rightarrow \lambda^a x' \qq{ } y \rightarrow \lambda^b y' \qq{ } \psi \rightarrow \lambda^c \psi'
	\label{ansatz}
	\end{equation}
	Using the fact that $ u = \psi_y $ and $ v = -\psi_x $, one can write (\ref{xns}) as,
	\begin{equation}
	\psi_y \psi_{xy} - \psi_{x}\psi_{yy} = \nu \psi_{yyy}
	\label{psixns}
	\end{equation}
	and (\ref{conserv}) as,
	\begin{equation}
	\int_{-\infty}^{\infty} \psi_y^2 dy = M
	\label{psiconserv}
	\end{equation}
	Substituting (\ref{ansatz}) into (\ref{psixns}) and (\ref{psiconserv}), we get,
	\begin{align*}
	2c - 2b - a &= c - 3b \implies a = b + c \qq{and} \\
	2(c-b) + b &= 0 \implies b = 2c
	\end{align*}
	Solving which we get,
	\begin{equation*}
	b = \dfrac{2a}{3} \qq{ } c = \dfrac{a}{3}
	\end{equation*}
	and the final form being,
	\begin{equation*}
		x \rightarrow \lambda^a x' \qq{ } y \rightarrow \lambda^{2a/3} y' \qq{ } \psi \rightarrow \lambda^{a/3} \psi'
	\end{equation*}
	This suggests that,
	\begin{equation*}
	\dfrac{\psi }{ x^{1/3} }\sim f\qty(\dfrac{y}{x^{2/3}}) \implies \psi = A x^{1/3}f(\eta)
	\end{equation*}
	where $ \eta = \dfrac{y}{x^{2/3}}$. We note the following,
	\begin{align*}
	\psi_y &= A x^{1/3}f'(\eta) \dv{\eta}{y} \\
	&= A x^{-1/3}f'\\
	\psi_x &= \dfrac{A}{3}x^{-2/3}f(\eta) + A x^{1/3}f'(\eta)\dv{\eta}{x}\\
	&=\dfrac{A}{3}x^{-2/3}f(\eta) - \dfrac{2A}{3} x^{-2/3}f'(\eta)\eta\\
	&= \dfrac{Ax^{-2/3}}{3}(-2 \eta f'+f)\\
	\psi_{xy} &= \dfrac{Ax^{-4/3}}{3}(-2 \eta f'' - f')\\
	\psi_{yy} &=  A x^{-1}f''\\
	\psi_{yyy} &=  A x^{-5/3}f'''
	\end{align*}
	Putting all this into (\ref{psixns}), we get,
	\begin{equation*}
	-f'(-2 \eta f''+f') - (-2 \eta f'+f)f'' = \dfrac{3\nu}{A} f''' \implies \dfrac{3\nu}{A} f''' + f'^2 + f''f = 0
	\end{equation*}
	If we set $ A =  \nu $ (we can always do that since it is an arbitrary constant), we get our final answer,
	\begin{equation*}
	3 f''' + f'^2 + f''f = 0
	\end{equation*}
\end{homeworkProblem}



\end{document}
