\documentclass{article}

\usepackage{fancyhdr}
\usepackage{extramarks}
\usepackage{amsmath}
\usepackage{amsthm}
\usepackage{amssymb}
\usepackage{amsfonts}
\usepackage{tikz}
\usepackage{physics}
\usepackage[plain]{algorithm}
\usepackage{algpseudocode}

\usetikzlibrary{automata,positioning}

%
% Basic Document Settings
%

\topmargin=-0.45in
\evensidemargin=0in
\oddsidemargin=0in
\textwidth=6.5in
\textheight=9.0in
\headsep=0.25in

\linespread{1.1}

\pagestyle{fancy}
\lhead{\hmwkAuthorName}
\chead{\hmwkClass\ : \hmwkTitle}
\rhead{\firstxmark}
\lfoot{\lastxmark}
\cfoot{\thepage}

\renewcommand\headrulewidth{0.4pt}
\renewcommand\footrulewidth{0.4pt}

\setlength\parindent{0pt}

%
% Create Problem Sections
%
\newcommand{\be}{\begin{equation}}
\newcommand{\ee}{\end{equation}}
\newcommand{\bes}{\begin{equation*}}
\newcommand{\ees}{\end{equation*}}
\newcommand{\bea}{\begin{flalign*}}
\newcommand{\eea}{\end{flalign*}}


\newcommand{\enterProblemHeader}[1]{
    \nobreak\extramarks{}{Problem \arabic{#1} continued on next page\ldots}\nobreak{}
    \nobreak\extramarks{Problem \arabic{#1} (continued)}{Problem \arabic{#1} continued on next page\ldots}\nobreak{}
}

\newcommand{\exitProblemHeader}[1]{
    \nobreak\extramarks{Problem \arabic{#1} (continued)}{Problem \arabic{#1} continued on next page\ldots}\nobreak{}
    \stepcounter{#1}
    \nobreak\extramarks{Problem \arabic{#1}}{}\nobreak{}
}

\setcounter{secnumdepth}{0}
\newcounter{partCounter}
\newcounter{homeworkProblemCounter}
\setcounter{homeworkProblemCounter}{1}
\nobreak\extramarks{Problem \arabic{homeworkProblemCounter}}{}\nobreak{}

%
% Homework Problem Environment
%
% This environment takes an optional argument. When given, it will adjust the
% problem counter. This is useful for when the problems given for your
% assignment aren't sequential. See the last 3 problems of this template for an
% example.
%
\newenvironment{homeworkProblem}[1][-1]{
    \ifnum#1>0
        \setcounter{homeworkProblemCounter}{#1}
    \fi
    \section{Problem \arabic{homeworkProblemCounter}}
    \setcounter{partCounter}{1}
    \enterProblemHeader{homeworkProblemCounter}
}{
    \exitProblemHeader{homeworkProblemCounter}
}

%
% Homework Details
%   - Title
%   - Due date
%   - Class
%   - Section/Time
%   - Instructor
%   - Author
%

\newcommand{\hmwkTitle}{Assignment\ \#2}
\newcommand{\hmwkDueDate}{Due on 20th September, 2018}
\newcommand{\hmwkClass}{Advanced Quantum Mechanics}
\newcommand{\hmwkClassTime}{}
\newcommand{\hmwkClassInstructor}{}
\newcommand{\hmwkAuthorName}{\textbf{Aditya Vijaykumar}}

%
% Title Page
%

\title{
    %\vspace{2in}
    \textmd{\textbf{\hmwkClass:\ \hmwkTitle}}\\
    \normalsize\vspace{0.1in}\small{\hmwkDueDate\ }\\
%    \vspace{3in}
}

\author{\hmwkAuthorName}
\date{}

\renewcommand{\part}[1]{\textbf{\large Part \Alph{partCounter}}\stepcounter{partCounter}\\}

%
% Various Helper Commands
%

% Useful for algorithms
\newcommand{\alg}[1]{\textsc{\bfseries \footnotesize #1}}

% For derivatives
\newcommand{\deriv}[1]{\frac{\mathrm{d}}{\mathrm{d}x} (#1)}

% For partial derivatives
\newcommand{\pderiv}[2]{\frac{\partial}{\partial #1} (#2)}

% Integral dx
\newcommand{\dx}{\mathrm{d}x}

% Alias for the Solution section header
\newcommand{\solution}{\textbf{\large Solution}}

% Probability commands: Expectation, Variance, Covariance, Bias
\newcommand{\E}{\mathrm{E}}
\newcommand{\Var}{\mathrm{Var}}
\newcommand{\Cov}{\mathrm{Cov}}
\newcommand{\Bias}{\mathrm{Bias}}

\begin{document}

\maketitle

\textit{\textbf{Acknowledgements} - I would like to thank Chandramouli Chowdhury and Junaid Bhat for discussions in this assignment.}
\begin{homeworkProblem}
Before we start looking at the the actual problem itself, let's consider a simpler problem - that of proving $ e^x e^{-x} = 1$. Lets Taylor expand the same and look at terms order by order,
\begin{align*}
e^x e^{-x} &= \qty(1 + x + \dfrac{x^2 }{2!} + \dfrac{x^3}{3!} + \ldots)\qty(1 - x + \dfrac{x^2 }{2!} - \dfrac{x^3}{3!} + \ldots)\\
&= 1 + (x -x) + \qty(\dfrac{x^2 }{2} + \dfrac{x^2 }{2} - x^2) + \qty(\dfrac{x^3 }{6} -  \dfrac{x^3 }{6} + \dfrac{x^3}{2} - \dfrac{x^3}{2} ) + \ldots
\end{align*}
We note that terms cancel out order by order.
\begin{align*}
U &= \sum_n \dfrac{(-i)^n}{n!} \int_{0}^{t} dt_1 \ldots \int_{0}^{t} dt_n \mathcal{T}[H(t_1)\ldots H(t_2)]\\
&= 1 - i \int_0^t dt_1 H(t_1) - \dfrac{1}{2} \int_0^t \int_0^t dt_1 dt_2 \qty[\theta(t_1 - t_2)H(t_1)H(t_2) + \theta(t_2 - t_1)H(t_2)H(t_1)] + \ldots\\
U^\dagger &= 1 + i \int_0^t dt_1 H(t_1) - \dfrac{1}{2} \int_0^t \int_0^t dt_1 dt_2 \qty[\theta(t_2 - t_1)H(t_1)H(t_2) + \theta(t_1 - t_2)H(t_2)H(t_1)] + \ldots\\
\therefore U^\dagger U &= 1 + \qty(- i \int_0^t dt_1 H(t_1) + i \int_0^t dt'_1 H(t'_1) ) + \int_0^t \int_0^t dt_1 dt_2 \qty[H(t_1) H(t_2)]\\
&\qty(- \dfrac{1}{2} \int_0^t \int_0^t dt_1 dt_2 \qty[\theta_{12}H(t_1)H(t_2) + \theta_{21}H(t_2)H(t_1)] - \dfrac{1}{2} \int_0^t \int_0^t dt'_1 dt'_2 \qty[\theta_{2'1'}H(t'_1)H(t'_2) + \theta_{1'2'}H(t'_2)H(t'_1)] )\\&+ \ldots
%&+ \dfrac{1}{4} \int_0^t \int_0^t \int_0^t \int_0^t dt_1 dt_2  dt'_1 dt'_2 \qty[\theta_{12}\theta_{1'2'}H(t_1)H(t_2)H(t_1)H(t_2) + \theta_{21}H(t_2)H(t_1)]
\end{align*}
The terms in the first parenthesis just cancel each other. As the primed variables are just dummies one can ignore the primes and write the terms in the second parenthesis in terms of $ t_1 $ and $ t_2 $. We can rewrite that particular term as follows (using the fact that $ \theta_{12} + \theta_{21} = 1 $),
\begin{align*}
- \dfrac{1}{2} \int_0^t \int_0^t dt_1 dt_2 \qty[\theta_{12}H(t_1)H(t_2) + \theta_{21}H(t_2)H(t_1) + \theta_{21}H(t_1)H(t_2) + \theta_{12}H(t_2)H(t_1)]\\
= - \dfrac{1}{2} \int_0^t \int_0^t dt_1 dt_2 \qty[H(t_1)H(t_2) + H(t_2)H(t_1)] = - \int_0^t \int_0^t dt_1 dt_2 \qty[H(t_1)H(t_2)]
\end{align*}
This shows that the terms cancel. This will hold at higher orders too, albeit with more complicated $ \theta $ functions.
\end{homeworkProblem}

\begin{homeworkProblem}
	We know that the energy eigenstates of the harmonic oscillator form a basis. Hence, the required coherent state $ \ket{z} $ can be written in terms of these eigenstates as,
	\begin{equation*}
	\ket{z} = \sum_{n=0}^{\infty} c_n \ket{n} = \sum_{n=0}^{\infty} c_n \frac{(a^\dagger)^n}{\sqrt{n!}}\ket{0}
	\end{equation*}
	Substituting this in the equation for coherent state $ a\ket{z} = z \ket{z} $,
	\begin{align*}
	 \sum_{n=0}^{\infty} c_n a \ket{n} &=  \sum_{n=0}^{\infty} z c_n \ket{n}\\
	 \sum_{n=1}^{\infty} c_n \sqrt{n} \ket{n-1} &=  \sum_{n=0}^{\infty} z c_n \ket{n}\\
	 \sum_{n=0}^{\infty} c_{n+1} \sqrt{n+1} \ket{n} &=  \sum_{n=0}^{\infty} z c_n \ket{n}\\
	 \therefore c_{n+1} \sqrt{n+1} &= z c_n
	\end{align*}
	
	We have effectively derived a recursion relation for the coefficients $ c_n $. If we start off with $ c_n = \alpha $,
	\begin{equation*}
	c_1 = {z \alpha} \qq{,} c_2 = \frac{z^2 \alpha}{\sqrt{2}} \qq{,} c_3 = \dfrac{z^3 \alpha}{\sqrt{3\vdot 2}} \qq{,} \ldots \qq{,} c_n = \dfrac{z^n \alpha}{\sqrt{n!}}
	\end{equation*}
	So, our coherent state can now be written as,
	\begin{align*}
	\ket{z} &= \alpha \sum_{n=0}^{\infty}  \frac{(za^\dagger)^n}{n!}\ket{0}\\
	&= \alpha e^{a^\dagger z}\ket{0}
	\end{align*}
\end{homeworkProblem}	
\begin{homeworkProblem}
	We know that,
	\begin{equation*}
	{x}(0) = \frac{{a} + {a}^\dagger}{\sqrt{2m\omega}} \qq{,} {p}(0) = \frac{\sqrt{m\omega} ({a} - {a}^\dagger)}{\sqrt{2}i} \qq{,} x(t) = e^{iHt}x(0)e^{-iHt} \qq{,} p(t) = e^{iHt}p(0)e^{-iHt}
	\end{equation*}
	From this, we note the following,
	\begin{align*}
	 x(t)\ket{0} &= e^{iHt}x(0)e^{-iHt}\ket{0}\\
	 &=  e^{-i\omega t/2}e^{iHt}x(0)\ket{0}\\
	 &=  \frac{e^{-i\omega t/2}}{\sqrt{2m\omega}}e^{iHt}\ket{1}\\
	 x(t)\ket{0} &=  \frac{e^{i\omega t}}{\sqrt{2m\omega}}\ket{1} \implies \bra{0}x(t)=\frac{e^{-i\omega t}}{\sqrt{2m\omega}} \bra{1}
	\end{align*}
	\begin{align*}
	\qq{Similarly, }	 p(t)\ket{0} &= e^{iHt}p(0)e^{-iHt}\ket{0}\\
	&=  e^{-i\omega t/2}e^{iHt}p(0)\ket{0}\\
	&=  -\frac{e^{-i\omega t/2}\sqrt{m\omega}}{\sqrt{2}i}e^{iHt}\ket{1}\\
	p(t)\ket{0} &=  -\frac{e^{i\omega t}\sqrt{m\omega}}{\sqrt{2}i}\ket{1} \implies \bra{0}p(t) =  \frac{e^{-i\omega t}\sqrt{m\omega}}{\sqrt{2}i}\bra{1} 
	\end{align*}
	
	Now consider the quantities to be calculated,
	\begin{equation*}
	C_1(t) = \ev{x(t)x(0)}{0} = \frac{e^{-i\omega t}}{\sqrt{2m\omega}} \frac{1}{\sqrt{2m\omega}} = \boxed{\frac{e^{-i\omega t}}{2m\omega}}
	\end{equation*}
	\begin{equation*}
	C_2(t) = \ev{x(t)p(0)}{0} - \ev{p(0)x(t)}{0}= -\dfrac{e^{-i\omega t}}{2i} - \dfrac{e^{i\omega t}}{2i}= \boxed{i \cos \omega t}
	\end{equation*}
	\begin{equation*}
	C_3(t) = \ev{p(t)x(0)}{0} - \ev{x(0)p(t)}{0}= \dfrac{e^{-i\omega t}}{2i} + \dfrac{e^{i\omega t}}{2i}= \boxed{-i \cos \omega t}
	\end{equation*}
\end{homeworkProblem}


\begin{homeworkProblem}
	\textbf{Part (a)}
	\begin{align*}
	Z(\beta) = \Tr e^{-\beta H} &= \sum_{n=0}^{\infty} \expval{e^{-\beta H}}{n}\\
	&= \sum_{n=0}^{\infty} e^{-\beta(n+\frac{1}{2})\omega}\\
	&= \frac{e^{-\beta \omega /2}}{1-e^{-\beta \omega}} = \dfrac{2}{\sinh \dfrac{\beta \omega}{2}}
	\end{align*}
	
	\textbf{Part (b)}\\
	Given that,
	\begin{equation*}
	x(\tau) = \sum_n x_n e^{\frac{2\pi i n \tau}{\beta}}
	\end{equation*}
	As we require $ x(\tau) $ to be real, it follows from above that $ x_n = x_n^* $. Taking $ \omega_n = \frac{2\pi n}{\beta} $
	\begin{equation*}
	S_E = \int_{0}^{\beta} d \tau \qty(\frac{m\dot{x}^2}{2} + \frac{m\omega^2 x^2}{2})
	\end{equation*}
	\begin{align*}
	&=  \sum_{p,q} \qty(-\frac{m x_p x_q \omega_p \omega_q}{2} + \frac{m\omega^2 x_p x_q }{2}) \int_{0}^{\beta} e^{i (\omega_q + \omega_p)\tau} d \tau\\
	&=  \sum_{p,q} \qty(-\frac{m x_p x_q \omega_p \omega_q}{2} + \frac{m\omega^2 x_p x_q }{2}) \beta \delta_{p,-q}\\
	&=  \sum_{q} \qty(-\frac{m x_{-q} x_q \omega_{-q} \omega_q}{2} + \frac{m\omega^2 x_{-q} x_q }{2}) \beta \\
	&=  \frac{m}{2}\sum_{q} \beta\qty(x_{q}^* x_q  \omega_q^2 + \omega^2 x_{q}^* x_q)\\
	-S_E&=  -\frac{m\beta}{2}\sum_{q}x_{q}^* x_q   \qty(\omega_q^2 + \omega^2 )
	\end{align*}
	The required path integral to be done is,
	\begin{align*}
	Z(\beta) &= N \int \mathcal{D}x \exp - \frac{m\beta}{2}\sum_{q}x_{q}^* x_q   \qty(\omega_q^2 + \omega^2 ) \qq{where $ N $ is some normalization}\\
	&= N \int \mathcal{D}x \exp( - m\beta \sum_{q=1}^{\infty} x_{q}^* x_q   \qty(\omega_q^2 + \omega^2 ) - \frac{m\beta x_0^2\omega^2}{2}) \\
	&= N \int dx_0  \exp(-\frac{m\beta x_0^2\omega^2}{2}) \cross \prod_{q=1}^{\infty} \int d x_q d x_q^* \exp -  m\beta x_{q}^* x_q   \qty(\omega_q^2 + \omega^2 ) \\
	&= N \sqrt{\frac{2\pi}{m \beta \omega^2}} \cross \prod_{q=1}^\infty \frac{2\pi}{m \beta (\omega_q^2 + \omega^2 )}\\
	&= N \sqrt{\frac{2\pi}{m \beta \omega^2}} \cross \prod_{q=1}^\infty \frac{2\pi}{m \beta \dfrac{4 q^2 \pi^2}{\beta^2}\qty(1 + \qty(\frac{\beta \omega}{2q\pi})^2)}\\
	&= N \sqrt{\frac{2\pi}{m \beta \omega^2}} \cross \dfrac{\beta \omega}{2 \pi \sinh \dfrac{\beta \omega}{2}} \cross \prod_{q=1}^\infty \frac{\beta}{2 \pi m q^2} \\
	&= N \sqrt{\frac{\beta}{2 \pi m}}  \cross \prod_{q=1}^\infty \frac{\beta}{2 \pi m q^2} \cross \dfrac{1}{\sinh \dfrac{\beta \omega}{2}} \\
	&= N'  \cross \dfrac{1}{\sinh \dfrac{\beta \omega}{2}}
	\end{align*}
	As $ N' $ does not depend on $ \omega $, we can drop it altogether for computation purposes. This gives us the required result.
	
\end{homeworkProblem}
%\pagebreak 	

\end{document}
