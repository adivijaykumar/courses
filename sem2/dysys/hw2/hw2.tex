\documentclass{article}

\usepackage{fancyhdr}
\usepackage{extramarks}
\usepackage{amsmath}
\usepackage{amsthm}
\usepackage{amssymb}
\usepackage{amsfonts}
\usepackage{tikz}
\usepackage{physics}
\usepackage[plain]{algorithm}
\usepackage{algpseudocode}
\usepackage{hyperref}

\usetikzlibrary{automata,positioning}

%
% Basic Document Settings
%

\topmargin=-0.45in
\evensidemargin=0in
\oddsidemargin=0in
\textwidth=6.5in
\textheight=9.0in
\headsep=0.25in

\linespread{1.1}

\pagestyle{fancy}
\lhead{\hmwkAuthorName}
\chead{\hmwkClass\ : \hmwkTitle}
\rhead{\firstxmark}
\lfoot{\lastxmark}
\cfoot{\thepage}

\renewcommand\headrulewidth{0.4pt}
\renewcommand\footrulewidth{0.4pt}

\setlength\parindent{0pt}

%
% Create Problem Sections
%
\newcommand{\be}{\begin{equation}}
\newcommand{\ee}{\end{equation}}
\newcommand{\bes}{\begin{equation*}}
\newcommand{\ees}{\end{equation*}}
\newcommand{\bea}{\begin{flalign*}}
\newcommand{\eea}{\end{flalign*}}


\newcommand{\enterProblemHeader}[1]{
    \nobreak\extramarks{}{Problem \arabic{#1} continued on next page\ldots}\nobreak{}
    \nobreak\extramarks{Problem \arabic{#1} (continued)}{Problem \arabic{#1} continued on next page\ldots}\nobreak{}
}

\newcommand{\exitProblemHeader}[1]{
    \nobreak\extramarks{Problem \arabic{#1} (continued)}{Problem \arabic{#1} continued on next page\ldots}\nobreak{}
    \stepcounter{#1}
    \nobreak\extramarks{Problem \arabic{#1}}{}\nobreak{}
}

\setcounter{secnumdepth}{0}
\newcounter{partCounter}
\newcounter{homeworkProblemCounter}
\setcounter{homeworkProblemCounter}{1}
\nobreak\extramarks{Problem \arabic{homeworkProblemCounter}}{}\nobreak{}

%
% Homework Problem Environment
%
% This environment takes an optional argument. When given, it will adjust the
% problem counter. This is useful for when the problems given for your
% assignment aren't sequential. See the last 3 problems of this template for an
% example.
%
\newenvironment{homeworkProblem}[1][-1]{
    \ifnum#1>0
        \setcounter{homeworkProblemCounter}{#1}
    \fi
    \section{Problem \arabic{homeworkProblemCounter}}
    \setcounter{partCounter}{1}
    \enterProblemHeader{homeworkProblemCounter}
}{
    \exitProblemHeader{homeworkProblemCounter}
}

%
% Homework Details
%   - Title
%   - Due date
%   - Class
%   - Section/Time
%   - Instructor
%   - Author
%

\newcommand{\hmwkTitle}{Homework\ \#1}
\newcommand{\hmwkDueDate}{Due on 14th January, 2019}
\newcommand{\hmwkClass}{Dynamical Systems}
\newcommand{\hmwkClassTime}{}
\newcommand{\hmwkClassInstructor}{}
\newcommand{\hmwkAuthorName}{\textbf{Aditya Vijaykumar}}

%
% Title Page
%

\title{
    %\vspace{2in}
    \textmd{\textbf{\hmwkClass:\ \hmwkTitle}}\\
    \normalsize\vspace{0.1in}\small{\hmwkDueDate\ }\\
%    \vspace{3in}
}

\author{\hmwkAuthorName}
\date{}

\renewcommand{\part}[1]{\textbf{\large Part \Alph{partCounter}}\stepcounter{partCounter}\\}

%
% Various Helper Commands
%

% Useful for algorithms
\newcommand{\alg}[1]{\textsc{\bfseries \footnotesize #1}}

% For derivatives
\newcommand{\deriv}[1]{\frac{\mathrm{d}}{\mathrm{d}x} (#1)}

% For partial derivatives
\newcommand{\pderiv}[2]{\frac{\partial}{\partial #1} (#2)}

% Integral dx
\newcommand{\dx}{\mathrm{d}x}

% Alias for the Solution section header
\newcommand{\solution}{\textbf{\large Solution}}

% Probability commands: Expectation, Variance, Covariance, Bias
\newcommand{\E}{\mathrm{E}}
\newcommand{\Var}{\mathrm{Var}}
\newcommand{\Cov}{\mathrm{Cov}}
\newcommand{\Bias}{\mathrm{Bias}}

\begin{document}

\maketitle
(\textbf{Acknowledgements} - I would like to thank Divya Jagannathan for discussions.)
\\

\begin{homeworkProblem}
	\textbf{Part (a)}\\
	Consider first the set $ E_u $. Any vector in this said would be given by $ V = \sum_i C_j w_j $, where $ w_j $ is a generalized eigenvector of this set. We know,
	\begin{equation*}
	(A - \lambda I)^m w_j = 0 
	\end{equation*}  
	for some $ m $. This also means that,
	\begin{equation*}
	(A - \lambda I) w_j = W_j \implies A w_j = \lambda w_j + W_j
	\end{equation*}
	where $ W_j \in \ker((A - \lambda I)^{m-1})$. Hence $ A w_j  \in E_u $. This also means that $ A^k w_j \in E_u $ for any whole number $ k $, which in turn means that, in general, $\sum_{k} c_k A^k w_j \in E_u$.
	
	Consider,
	\begin{align*}
	e^{At} V = \sum_{i,k} C_j \dfrac{t^k A^k}{k!} w_j \in E_u
	\end{align*}
	Hence Proved that $ E_u $ is an invariant subspace. The argument follows similarly for $ E_s, E_c $.	
	
	If $ \alpha_j, \beta_j, \gamma_j $ are all eigenvectors as defined below, the most general solution of the system in given by,
	\begin{align*}
	\sum_{j} C_j w_j  &= \sum_{j, \alpha_j \in E_u} M_j \alpha_j + \sum_{j, \beta_j \in E_s} N_j \beta_j + \sum_{j, \gamma_j \in E_c} K_j \gamma_j\\
	\implies R^d &= E_u \oplus E_s \oplus E_c
	\end{align*}
	
	\textbf{Part (b)}\\
	For $x_0= \sum_{j} C_j w_j, w_j \in E_s $, $ e^{At}x_0 = \sum_{j}  C_j w_j e^{a_j t} e^{ i b_j t}, a_j < 0$. Hence, we can say,
	\begin{align*}
	\lim\limits_{t \rightarrow \infty } e^{At}x_0 &= \sum_{j}   C_j w_j \lim\limits_{t \rightarrow \infty } e^{a_j t} e^{ i b_j t} = 0 \\
	\lim\limits_{t \rightarrow - \infty } \abs{ e^{At}x_0 } &= \lim\limits_{t \rightarrow - \infty }  \abs{\sum_{j}   C_j w_j e^{a_j t} e^{ i b_j t}} = \infty
	\end{align*}
	
	\textbf{Part (c)}\\
	For $x_0= \sum_{j} C_j w_j, w_j \in E_u $, $ e^{At}x_0 = \sum_{j}  C_j w_j e^{a_j t} e^{ i b_j t}, a_j > 0$. Hence, we can say,
	\begin{align*}
	\lim\limits_{t \rightarrow -\infty } e^{At}x_0 &= \sum_{j}   C_j w_j \lim\limits_{t \rightarrow - \infty } e^{a_j t} e^{ i b_j t} = 0 \\
	\lim\limits_{t \rightarrow  \infty } \abs{ e^{At}x_0 } &= \lim\limits_{t \rightarrow  \infty }  \abs{\sum_{j}   C_j w_j e^{a_j t} e^{ i b_j t}} = \infty
	\end{align*}
	
	\textbf{Part (d)}\\
	For $x_0= \sum_{j} C_j w_j, w_j \in E_s $, $ e^{At}x_0 = \sum_{j}  C_j w_j e^{a_j t} e^{ i b_j t}, a_j < 0$. Hence, we can say,
	
\end{homeworkProblem}



\begin{homeworkProblem}
	\textbf{Part (a)}\\
	
	\textbf{Part (b)}\\
	As we are working with linear systems, any linear combination of the solutions will be linear too. We know that the columns of the fundamental matrix $ X $ are solutions to the ODE, and that they span the solution space. Right multiplying $ X $ with a constant, non-singular matrix $ C $ will give us a matrix which will have linear combination of the columns of $ X $ (ie the solutions to the ODE) according to the entries in $ C $. Due to the condition of non-singularity, this new matrix will also have linearly independent columns. Hence, the new matrix $ Y(t) = X(t) C $ will have columns which are solutions to the ODE and also span the solution space. Hence Proved.
	
	 Left multiplying $ X $ with a constant, non-singular matrix $ B $ will give us a matrix which will have linear combination of the \textit{rows} of $ X $ (which are not the solutions to the ODE) according to the entries in $ B $. This means that $ B X $ will not, in general, be a fundamental matrix for the system. Of course, $ BX $ can be a fundamental matrix if the rows of $ X $ are indeed solutions to the system, ie. if $ X^T = X $.
	\\
	
	\textbf{Part (c)}\\
	From the fact that $ X_1, X_2 $ are fundamental matrices and from the previous part, we can make the following statement,
	\begin{equation*}
	X_2(t) = X_1(t) C 
	\end{equation*}
	where $ C $ is a non-singular matrix.
	
	Consider,
	\begin{align*}
	X_2(t + \omega) &= X_2(t) B_2\\
	X_1(t + \omega) C &= X_2(t) B_2\\
	X_1(t) B_1 C &= X_1(t)  C B_2\\	
	\implies B_1 &= C B_2 C^{-1}
	\end{align*} 
	where inverses could be taken in the last step only because the matrices are known to be nonsingular. Hence proved that $ B_1, B_2 $ are similar.
\end{homeworkProblem}
\end{document}
