\documentclass{article}

\usepackage{fancyhdr}
\usepackage{extramarks}
\usepackage{amsmath}
\usepackage{amsthm}
\usepackage{amssymb}
\usepackage{amsfonts}
\usepackage{tikz}
\usepackage{enumitem}
\usepackage{physics}
\usepackage[plain]{algorithm}
\usepackage{algpseudocode}
\usepackage{hyperref}

\usetikzlibrary{automata,positioning}

%
% Basic Document Settings
%

\topmargin=-0.45in
\evensidemargin=0in
\oddsidemargin=0in
\textwidth=6.5in
\textheight=9.0in
\headsep=0.25in

\linespread{1.1}

\pagestyle{fancy}
\lhead{\hmwkAuthorName}
\chead{\hmwkClass\ : \hmwkTitle}
\rhead{\firstxmark}
\lfoot{\lastxmark}
\cfoot{\thepage}

\renewcommand\headrulewidth{0.4pt}
\renewcommand\footrulewidth{0.4pt}

\setlength\parindent{0pt}

%
% Create Problem Sections
%
\newcommand{\be}{\begin{equation}}
\newcommand{\ee}{\end{equation}}
\newcommand{\bes}{\begin{equation*}}
\newcommand{\ees}{\end{equation*}}
\newcommand{\bea}{\begin{flalign*}}
\newcommand{\eea}{\end{flalign*}}


\newcommand{\enterProblemHeader}[1]{
    \nobreak\extramarks{}{Problem \arabic{#1} continued on next page\ldots}\nobreak{}
    \nobreak\extramarks{Problem \arabic{#1} (continued)}{Problem \arabic{#1} continued on next page\ldots}\nobreak{}
}

\newcommand{\exitProblemHeader}[1]{
    \nobreak\extramarks{Problem \arabic{#1} (continued)}{Problem \arabic{#1} continued on next page\ldots}\nobreak{}
    \stepcounter{#1}
    \nobreak\extramarks{Problem \arabic{#1}}{}\nobreak{}
}

\setcounter{secnumdepth}{0}
\newcounter{partCounter}
\newcounter{homeworkProblemCounter}
\setcounter{homeworkProblemCounter}{1}
\nobreak\extramarks{Problem \arabic{homeworkProblemCounter}}{}\nobreak{}

%
% Homework Problem Environment
%
% This environment takes an optional argument. When given, it will adjust the
% problem counter. This is useful for when the problems given for your
% assignment aren't sequential. See the last 3 problems of this template for an
% example.
%
\newenvironment{homeworkProblem}[1][-1]{
    \ifnum#1>0
        \setcounter{homeworkProblemCounter}{#1}
    \fi
    \section{Problem \arabic{homeworkProblemCounter}}
    \setcounter{partCounter}{1}
    \enterProblemHeader{homeworkProblemCounter}
}{
    \exitProblemHeader{homeworkProblemCounter}
}

%
% Homework Details
%   - Title
%   - Due date
%   - Class
%   - Section/Time
%   - Instructor
%   - Author
%

\newcommand{\hmwkTitle}{Homework\ \#3}
\newcommand{\hmwkDueDate}{Due on 25th January, 2019}
\newcommand{\hmwkClass}{Dynamical Systems}
\newcommand{\hmwkClassTime}{}
\newcommand{\hmwkClassInstructor}{}
\newcommand{\hmwkAuthorName}{\textbf{Aditya Vijaykumar}}

%
% Title Page
%

\title{
    %\vspace{2in}
    \textmd{\textbf{\hmwkClass:\ \hmwkTitle}}\\
    \normalsize\vspace{0.1in}\small{\hmwkDueDate\ }\\
%    \vspace{3in}
}

\author{\hmwkAuthorName}
\date{}

\renewcommand{\part}[1]{\textbf{\large Part \Alph{partCounter}}\stepcounter{partCounter}\\}

%
% Various Helper Commands
%

% Useful for algorithms
\newcommand{\alg}[1]{\textsc{\bfseries \footnotesize #1}}

% For derivatives
\newcommand{\deriv}[1]{\frac{\mathrm{d}}{\mathrm{d}x} (#1)}

% For partial derivatives
\newcommand{\pderiv}[2]{\frac{\partial}{\partial #1} (#2)}

% Integral dx
\newcommand{\dx}{\mathrm{d}x}

% Alias for the Solution section header
\newcommand{\solution}{\textbf{\large Solution}}

% Probability commands: Expectation, Variance, Covariance, Bias
\newcommand{\E}{\mathrm{E}}
\newcommand{\Var}{\mathrm{Var}}
\newcommand{\Cov}{\mathrm{Cov}}
\newcommand{\Bias}{\mathrm{Bias}}

\begin{document}

\maketitle
(\textbf{Acknowledgements} - I would like to thank Divya Jagannathan for discussions.)
\\
\begin{homeworkProblem}
	Given $ \overline{\lim\limits_{t \rightarrow \infty}} f(t) = a \ne \pm \infty$, we can think of two broad cases :-
	\\
	
	{\textbf{Case I}} - $ \lim\limits_{t \rightarrow \infty} f(t) $ exists  $ \implies \lim\limits_{t \rightarrow \infty} f(t) = a $.
	\begin{enumerate}[label=(\alph*)]
		\item $\lim\limits_{t \rightarrow \infty} [f(t) - a - \epsilon] = - \epsilon$
		\item $\overline{\lim\limits_{t \rightarrow \infty}} [f(t) - a - \epsilon] = - \epsilon$
		\item  $\lim\limits_{t \rightarrow \infty} [f(t) - a + \epsilon] =  \epsilon$
		\item  $\overline{\lim\limits_{t \rightarrow \infty}} [f(t) - a + \epsilon] =  \epsilon$
		\end{enumerate}
	
	{\textbf{Case II}} - $ \lim\limits_{t \rightarrow \infty} f(t) $ does not exist  $ \implies f(t) \le a $ as $ t \rightarrow \infty  $.
	\begin{enumerate}[label=(\alph*)]
		\item $\lim\limits_{t \rightarrow \infty} [f(t) - a - \epsilon] $ does not exist.
		\item $\overline{\lim\limits_{t \rightarrow \infty}} [f(t) - a - \epsilon] = - \epsilon$
		\item  $\lim\limits_{t \rightarrow \infty} [f(t) - a + \epsilon] $ does not exist
		\item  $\overline{\lim\limits_{t \rightarrow \infty}} [f(t) - a + \epsilon] =  \epsilon$
	\end{enumerate}
\end{homeworkProblem}




\begin{homeworkProblem}
	content...
\end{homeworkProblem}









\begin{homeworkProblem}
	The most general quadratic time-independent Hamiltonian for a system with $ n $ degrees of freedom is,
	\begin{equation*}
	H = \dfrac{1}{2}\sum_{i=1,j=1}^{n}  A_{ij} q_i q_j + B_{ij} p_i p_j + C_{ij} q_i p_j
	\end{equation*}
	where $ A_{ij} = A_{ji}, B_{ij} = B_{ji}, C_{ij} = C_{ji}, $ and $ q_i $ and $ p_i $ are the generalized positions and momenta respectively. Using Hamilton's equations of motion, one can write,
	\begin{equation*}
	\dot{q}_i = \sum_{j=1}^n B_{ij} p_j + C_{ij} q_j (2 - \delta_{ij}) \qq{and} \dot{p}_i = - \sum_{j=1}^n A_{ij} q_j - C_{ij} p_j (2 - \delta_{ij})
 	\end{equation*}
 	
\end{homeworkProblem}
\end{document}
