\documentclass{article}

\usepackage{fancyhdr}
\usepackage{extramarks}
\usepackage{amsmath}
\usepackage{amsthm}
\usepackage{amssymb}
\usepackage{amsfonts}
\usepackage{tikz}
%\usepackage{physics}
\usepackage[plain]{algorithm}
\usepackage{algpseudocode}
\usepackage{hyperref}
\usepackage[arrowdel]{physics}

\usetikzlibrary{automata,positioning}

%
% Basic Document Settings
%

\topmargin=-0.45in
\evensidemargin=0in
\oddsidemargin=0in
\textwidth=6.5in
\textheight=9.0in
\headsep=0.25in

\linespread{1.1}

\pagestyle{fancy}
\lhead{\hmwkAuthorName}
\chead{\hmwkClass\ : \hmwkTitle}
\rhead{\firstxmark}
\lfoot{\lastxmark}
\cfoot{\thepage}

\renewcommand\headrulewidth{0.4pt}
\renewcommand\footrulewidth{0.4pt}

\setlength\parindent{0pt}

%
% Create Problem Sections
%
\newcommand{\be}{\begin{equation}}
\newcommand{\ee}{\end{equation}}
\newcommand{\bes}{\begin{equation*}}
\newcommand{\ees}{\end{equation*}}
\newcommand{\bea}{\begin{flalign*}}
\newcommand{\eea}{\end{flalign*}}


\newcommand{\enterProblemHeader}[1]{
    \nobreak\extramarks{}{Problem \arabic{#1} continued on next page\ldots}\nobreak{}
    \nobreak\extramarks{Problem \arabic{#1} (continued)}{Problem \arabic{#1} continued on next page\ldots}\nobreak{}
}

\newcommand{\exitProblemHeader}[1]{
    \nobreak\extramarks{Problem \arabic{#1} (continued)}{Problem \arabic{#1} continued on next page\ldots}\nobreak{}
    \stepcounter{#1}
    \nobreak\extramarks{Problem \arabic{#1}}{}\nobreak{}
}

\setcounter{secnumdepth}{0}
\newcounter{partCounter}
\newcounter{homeworkProblemCounter}
\setcounter{homeworkProblemCounter}{1}
\nobreak\extramarks{Problem \arabic{homeworkProblemCounter}}{}\nobreak{}

%
% Homework Problem Environment
%
% This environment takes an optional argument. When given, it will adjust the
% problem counter. This is useful for when the problems given for your
% assignment aren't sequential. See the last 3 problems of this template for an
% example.
%
\newenvironment{homeworkProblem}[1][-1]{
    \ifnum#1>0
        \setcounter{homeworkProblemCounter}{#1}
    \fi
    \section{Problem \arabic{homeworkProblemCounter}}
    \setcounter{partCounter}{1}
    \enterProblemHeader{homeworkProblemCounter}
}{
    \exitProblemHeader{homeworkProblemCounter}
}

%
% Homework Details
%   - Title
%   - Due date
%   - Class
%   - Section/Time
%   - Instructor
%   - Author
%

\newcommand{\hmwkTitle}{Pset\ \#2}
\newcommand{\hmwkDueDate}{Due on 6th February, 2019}
\newcommand{\hmwkClass}{Electromagnetism}
\newcommand{\hmwkClassTime}{}
\newcommand{\hmwkClassInstructor}{}
\newcommand{\hmwkAuthorName}{\textbf{Aditya Vijaykumar}}

%
% Title Page
%

\title{
    %\vspace{2in}
    \textmd{\textbf{\hmwkClass:\ \hmwkTitle}}\\
    \normalsize\vspace{0.1in}\small{\hmwkDueDate\ }\\
%    \vspace{3in}
}

\author{\hmwkAuthorName}
\date{}

\renewcommand{\part}[1]{\textbf{\large Part \Alph{partCounter}}\stepcounter{partCounter}\\}

%
% Various Helper Commands
%

% Useful for algorithms
\newcommand{\alg}[1]{\textsc{\bfseries \footnotesize #1}}

% For derivatives
\newcommand{\deriv}[1]{\frac{\mathrm{d}}{\mathrm{d}x} (#1)}

% For partial derivatives
\newcommand{\pderiv}[2]{\frac{\partial}{\partial #1} (#2)}

% Integral dx
\newcommand{\dx}{\mathrm{d}x}

% Alias for the Solution section header
\newcommand{\solution}{\textbf{\large Solution}}

% Probability commands: Expectation, Variance, Covariance, Bias
\newcommand{\E}{\mathrm{E}}
\newcommand{\Var}{\mathrm{Var}}
\newcommand{\Cov}{\mathrm{Cov}}
\newcommand{\Bias}{\mathrm{Bias}}

\begin{document}

\maketitle
(\textbf{Acknowledgements} - I would like to thank Junaid Majeed for discussions.)
\\

\begin{homeworkProblem}
	\textbf{Zangwill - Problem} 24.1\\
	\textbf{Part (a)}
	Given, 
	\begin{align*}
	\pdv{F_i}{\dot{r}_j} &= - \pdv{F_j}{\dot{r}_i} \\
	\implies \pdv[2]{F_i}{\dot{r}_j}{\dot{r}_k} &= - \pdv{F_j}{\dot{r}_i}{\dot{r}_k}\\
	\pdv[2]{F_i}{\dot{r}_k}{\dot{r}_j}&= - \pdv{F_j}{\dot{r}_k}{\dot{r}_i}\\
	-\pdv[2]{F_k}{\dot{r}_i}{\dot{r}_j} &=  \pdv{F_k}{\dot{r}_j}{\dot{r}_i}\\
	\implies  \pdv{F_k}{\dot{r}_j}{\dot{r}_i} &=0
	\end{align*}
	From Hemholtz first relation, we know that integrating the above equation should give us an object which is antisymmetric under the exchange of indices. We can take this into account by introducing our friendly neighbourhood antisymmetric object, namely the \textit{Levi-Civita symbol}.
	\begin{equation*}
	\implies \pdv{F_i}{\dot{r}_j} =  \epsilon_{ijk} Q_k (\vb{r},t) \implies F_i = \epsilon_{ijk} Q_k (\vb{r},t) \dot{r}_j + P(\vb{r}, t)
	\end{equation*}
	Hence proved.
	\\
	
	\textbf{Part (b)}
	The second Helmholtz relation says,
	\begin{align*}
	\pdv{F_i}{r_j} - \pdv{F_j}{r_i} &= \dfrac{1}{2} \dv{t} \qty(\pdv{F_i}{\dot{r}_j} - \pdv{F_j}{\dot{r}_i} )\\
	\pdv{P_i}{r_j} - \pdv{P_j}{r_i} + \epsilon_{ijk} \qty(\pdv{Q_k}{r_j} + \pdv{Q_k}{r_i})  &=\dv{t} (\epsilon_{ijk} Q_k(\vb{r},t) )\\
	\pdv{P_i}{r_j} - \pdv{P_j}{r_i} + \epsilon_{ijk} \qty(\pdv{Q_k}{r_j} + \pdv{Q_k}{r_i})  &=\epsilon_{ijk} \qty(\pdv{Q_k}{t} + \pdv{Q_k}{r_a}\dot{r}_a)\\
	\end{align*}
	As we can see, there are no terms involving $ \dot{r}_a $ on the LHS. Hence, the term involving $ \dot{r}_a $ on the RHS should be zero!
	\begin{equation*}
	\implies \pdv{Q_k}{r_a}\dot{r}_a = \grad{Q} = 0
	\end{equation*}
	Next, we multiply both sides of the equation with $ \epsilon_{l j i} $, and use the fact that $ \epsilon_{lmk} \epsilon_{ijk} = (\delta_{li} \delta_{mj} - \delta_{lj} \delta_{mi}) $,
	\begin{align*}
	\epsilon_{l j i} \qty(\pdv{P_i}{r_j} - \pdv{P_j}{r_i}) + \epsilon_{l j i}\epsilon_{ijk}  \qty(\pdv{Q_k}{r_j} + \pdv{Q_k}{r_i})  &= \epsilon_{l j i} \epsilon_{ijk}  \qty(\pdv{Q_k}{t} )\\
	\epsilon_{lmk} \qty(\pdv{P_i}{r_j} - \pdv{P_j}{r_i}) +  \qty(\pdv{Q_k}{r_m} + \pdv{Q_k}{r_l} - \pdv{Q_k}{r_l} - \pdv{Q_k}{r_m})  &=  \qty(\pdv{Q_k}{t} )\\
	\therefore \epsilon_{lmk} \qty(\pdv{P_i}{r_j} - \pdv{P_j}{r_i})  &=  \qty(\pdv{Q_k}{t} )\\
	\therefore \curl{P} &= \pdv{Q}{t}
	\end{align*}
	
	\textbf{Zangwill - Problem} 24.2\\
	\textbf{Part (a)}\\
	\begin{align*}
	H(r,p) &= \dfrac{p^2}{2m} + g(r) + p^2 f(r) = \qty(\dfrac{1 + 2 fm}{2m}) p^2 + g(r)\\
	\implies \dot{r} &= \pdv{H}{p} = \dfrac{p}{m} + 2p f(r) \\
	\implies p &= \dfrac{m \dot{r}}{1 + 2 fm}
	\end{align*}
	\begin{equation*}
	\therefore L(r, \dot r) = \dfrac{m \dot{r}^2}{1 + 2 fm} - \dfrac{m \dot{r}^2}{2(1 + 2 fm)} - g(r) = \dfrac{m \dot{r}^2}{2(1 + 2 fm)} - g(r)
	\end{equation*}
	\textbf{Part (b)}\\
	We write the Euler Lagrange equations for this Lagrangian,
	\begin{align*}
	\dv{t} \qty(\dfrac{m \dot{r}}{1 + 2 fm}) + \pdv{g}{r} &= 0  \\
	\dfrac{m \ddot{r}}{1 + 2 fm} - \dfrac{m \dot{r}}{(1 + 2 fm)^2} \dot{r} \dv{f}{r} + \dv{g}{r}&= 0 \\
	{m \ddot{r}} - \dfrac{m \dot{r}^2}{(1 + 2 fm)} \dv{f}{r} + (1 + 2 fm)\dv{g}{r}&= 0
	\end{align*}
	As one can see from the above equation, the force depends on $ \ddot{r}, \dot{r}, f , g $.
	\\
	
	\textbf{Zangwill - Problem} 24.4\\
	Let's first work with $ c=1 $. We shall restore the factors of $ c $ at the end.
	\begin{align*}
	S &= - m\int \dd{s} - g \int \dd{s} \phi (\va{r}(s))\\
	&= - m\int \dd{t} \dv{s}{t} - g \int \dd{t} \dv{s}{t} \phi (\va{r}(s))\\
	S &= \int - \dfrac{m}{\gamma} \dd{t} +  \int \dfrac{- g \phi}{\gamma } \dd{t} 
	\end{align*}
	From the above form of action $ S $, the Lagrangian is evidently,
	\begin{equation*}
	L = - \dfrac{m + g \phi}{\gamma} \qq{where} \gamma = \dfrac{1}{\sqrt{1 - \va{v} \vdot \va{v}}} =  \dfrac{1}{\sqrt{1 - \dot{\va{r}} \vdot\dot{\va{r}} }}
	\end{equation*}
	\begin{align*}
	\therefore \dv{t} \qty( \pdv{L}{\dot{\va{r}}}) &= \pdv{L}{\va{r}} \\
	\dv{(m + g \phi) \gamma \dot{\va{r}}}{t} &= - \dfrac{g \grad{\phi}}{\gamma}
	\end{align*}
	Restoring the factors of $ c $,
	\begin{equation*}
	\dv{(m + g \phi/c) \gamma \dot{\va{r}}}{t} = - \dfrac{g c \grad{\phi}}{\gamma}
	\end{equation*}
	This differs from the electric field force equation, and has an extra term on the left (second term).
	\\
	
	\textbf{Zangwill - Problem} 24.11\\
	\begin{equation*}
	L_{CS} = \int \dd^3 r [\rho \phi - \va{j} \vdot \va{A} + 1/2 \{\epsilon_0 (\va{E}^2 - c^2 \va{B}^2) - \phi  (\va{d} \vdot \va{B}/c) + \va{d} \vdot (\va{A} \cross \va{E}/c) + d_0 \va{A} \vdot \va{B}  \}]
	\end{equation*}
	It is given that the Lagrangian remains invariant under usual gauge tranformations $ \phi \rightarrow \phi + \partial_t {\lambda} $ and $ \va{A} \rightarrow \va{A} - \grad{\lambda} $. Hence,
	\begin{align*}
	L'_{CS} - L_{CS} = \int \dd^3 r [ \rho \partial_t \lambda + \va{j} \vdot \grad{\lambda} + 1/2(- \va{d} \vdot (\grad{\lambda} \cross \va{E}/c) - d_0 \grad{\lambda} \vdot \va{B} - \partial_t \lambda (\va{d} \vdot \va{B}/c)  )] \\
	= \int \dd^3 r [ \partial_t(\rho \lambda) - \lambda \partial_t \rho + \div{(\lambda \va{j})} - \lambda \div{\va{j}} + 1/2(- \va{d} \vdot (\grad{\lambda} \cross \va{E}/c) - d_0 \grad{\lambda} \vdot \va{B} - \partial_t \lambda (\va{d} \vdot \va{B}/c)  )]
	\end{align*}
	The first four terms above vanish in the integral - the first and the third because they are boundary terms and the other two because of conservation.
	\begin{align*}
	L'_{CS} - L_{CS} &= \int \dd^3 r [ 1/2(- \va{d} \vdot (\grad{\lambda} \cross \va{E}/c) - d_0 \grad{\lambda} \vdot \va{B} - \partial_t \lambda (\va{d} \vdot \va{B}/c)  )] \\
	L'_{CS} - L_{CS} &= \int \dd^3 r [ 1/2(\va{d} \vdot \lambda/c \curl{\va{E}}  - d_0 \div{\lambda \va{B}} + d_0 \lambda \div{\va{B}}  - \partial_t [\lambda (\va{d} \vdot \va{B}/c) )] + \lambda/c(\partial_t\va{d} \vdot \va{B} + \partial_t\va{B} \vdot \va{d})] \\
	&= \int \dd^3 r [ 1/2(- \va{d} \vdot\curl{(\lambda \va{E}/c)} + \lambda\va{d} \vdot\curl{( \va{E}/c)}  + d_0 \lambda \div{\va{B}} - d_0 \div - \partial_t \lambda (\va{d} \vdot \va{B}/c)  )]
	\end{align*}
	
	\textbf{Zangwill - Problem} 24.12\\
	\begin{equation*}
	L = \va{j} \vdot \va{A} - \rho \phi - \dfrac{1}{2} \epsilon_0 (\va{E}^2 - c^2 \va{B}^2 ) - \epsilon_0 \va{E} \vdot (
	\grad{\phi} + \dot{\va{A}} ) - \epsilon_0 c^2 \va{B} \vdot (\curl{\va{A}})
	\end{equation*}
	\begin{align*}
	\pdv{L}{\va{E}} = - \epsilon_0 \va{E} - \epsilon_0  (\grad{\phi} + \dot{\va{A}} ) &\qq{,} \pdv{L}{\dot{\va{E}}} = 0 \implies \va{E} = -\grad{\phi} - \dot{\va{A}} \\
	\pdv{L}{\va{B}} = \epsilon_0 c^2 \va{B} - \epsilon_0 c^2 \curl{\va{A}} &\qq{,}  \pdv{L}{\dot{\va{B}}} = 0 \implies \va{B} = \curl{\va{A}}\\
	\pdv{L}{\phi} = - \rho - \epsilon_0 \pdv{\va{E} \vdot \grad{\phi}}{\phi} =  - \rho + \epsilon_0 \div{\va{E}} &\qq{,} \pdv{L}{\dot{\phi}} = 0 \implies \div{\va{E}} = \dfrac{\rho}{\epsilon_0} \\
	\pdv{L}{\va{A}} = \va{j}  -\epsilon_0 c^2  \pdv{\va{B} \vdot (\curl{\va{A}})}{\va{A}} =  \va{j}  -\epsilon_0 c^2 & \pdv{(\va{A} \vdot (\curl{\va{B}}) + \div{\va{A} \cross \va{B}} )}{\va{A} } = \va{j} -\epsilon_0 c^2 \curl{\va{B}}\\
	\pdv{L}{\dot{\va{A}}} = - \epsilon_0 \va{E} &\implies \curl{B} = \dfrac{\va{j}}{\epsilon_0 c^2} + \dfrac{1}{c^2} \dot{\va{E}}
	\end{align*}
	The first two equations can be rewritten as $ \div{B} = 0 $ and $ \curl{E} = - \pdv{\va{B}}{t} $. Hence we have all the Maxwell equations.
	
	There are 7 primary constraints (whose momenta vanish).
\end{homeworkProblem}


\begin{homeworkProblem}[2]
	\textbf{Part (a)}\\
	Given that,
	\begin{align*}
	L_D &= \sum_a \qty(\dfrac{1}{2} m_a \va{u}_a^2 + \dfrac{1}{8c^2} m_a \va{u}_a^4 ) + \sum_a \sum_{b\ne a} \qty[ - \dfrac{q_a q_b}{8 \pi \epsilon_0 r_{ab}} + \dfrac{q_a q_b}{16 \pi \epsilon_0 c^2 r_{ab}} \qty( \va{u}_a \vdot \va{u}_b + (\va{u}_a \vdot \vu{r}_{ab}) (\va{u}_b \vdot \vu{r}_{ab}) )]   \\
	\implies \pdv{L_D}{\va{u}_a}  &= \qty( m_a \va{u}_a+ \dfrac{1}{2c^2} m_a \va{u}_a^2 \va{u}_a ) + \sum_{b\ne a} \qty[\dfrac{q_a q_b}{16 \pi \epsilon_0 c^2 r_{ab}} \qty(  2 \va{u}_b + 2 \vu{r}_{ab} (\va{u}_b \vdot \vu{r}_{ab}) )]  \\
	  \pdv{L_D}{\va{u}_a}  &= \underbrace{ \sum_{b\ne a} \qty[\dfrac{q_b}{8 \pi \epsilon_0 c^2 r_{ab}} \qty(  \va{u}_b + \vu{r}_{ab} (\va{u}_b \vdot \vu{r}_{ab}) )]}_{\va{p}_a^{kin}}  +q_a   \underbrace{ \sum_{b\ne a} \qty[\dfrac{q_b}{8 \pi \epsilon_0 c^2 r_{ab}} \qty(  \va{u}_b + \vu{r}_{ab} (\va{u}_b \vdot \vu{r}_{ab}) )]}_{\va{A}_a}
	\end{align*}
	which is the required form.
	
	\textbf{Part (b)}\\
	
	\textbf{Part (c)}\\
	Consider,
	\begin{align*}
	\pdv{r_{ab}}{\va{r}_a} &= \pdv{\sqrt{(\va{r}_a - \va{r}_b)\vdot (\va{r}_a - \va{r}_b)}}{\va{r}_a} = \dfrac{2 }{2 r_{ab}} (\va{r}_a - \va{r}_b) = \vu{r}_{ab} \qq{and} \\
	\pdv{\vu{r}_{ab}}{\va{r}_a} &= \pdv{(\va{r}_{ab}/r_{ab})}{\va{r}_a} = \dfrac{1}{r_{ab}}  - \dfrac{\va{r}_{ab} \vdot \vu{r}_{ab}}{r_{ab}^2} = 0
	\end{align*}
	\begin{align*}
	\implies \pdv{L_D}{\va{r}_a} &=  2\sum_{b\ne a} \qty[ - \dfrac{q_a q_b}{8 \pi \epsilon_0 r_{ab}^2} \vu{r}_{ab} + \dfrac{q_a q_b \vu{r}_{ab}}{16 \pi \epsilon_0 c^2 r_{ab}^2} \qty( \va{u}_a \vdot \va{u}_b + (\va{u}_a \vdot \vu{r}_{ab}) (\va{u}_b \vdot \vu{r}_{ab}) )]    \\
	-q_a \pdv{\va{r}_a} (\phi_a - \va{u}_a \vdot \va{A}_a ) &=  \sum_{b\ne a} \qty[ - \dfrac{q_a q_b}{4 \pi \epsilon_0 r_{ab}^2} \vu{r}_{ab} + \dfrac{q_a q_b \vu{r}_{ab}}{8 \pi \epsilon_0 c^2 r_{ab}^2} \qty( \va{u}_a \vdot \va{u}_b + (\va{u}_a \vdot \vu{r}_{ab}) (\va{u}_b \vdot \vu{r}_{ab}) )]   \\
	\implies \pdv{L_D}{\va{r}_a} &= -q_a \pdv{\va{r}_a} (\phi_a - \va{u}_a \vdot \va{A}_a ) = -q_a \grad_a{\phi_a}  + q_a \grad_a{(\va{u}_a \vdot \va{A}_a)}
	\end{align*}
	Note that, in the first expression, there is an extra factor of $ 2 $ due to the summation over $ a $.
	
	\textbf{Part (d)}\\
	We write our equations of motion,
	\begin{align*}
	\dv{t} \qty(\pdv{L_D}{\va{u}_a} ) &= \pdv{L_D}{\va{r}_a} \\
	\dv{\va{p}_a^{kin}}{t} + q_a \dot{\va{A}}_a &= -q_a \grad_a{\phi_a}  + q_a \grad_a{(\va{u}_a \vdot \va{A}_a)}
	\end{align*}
	Using $ \grad{(\va{a} \vdot \va{b}) } = (\va{a} \vdot \grad) \va{b} + (\va{b} \vdot \grad) \va{a} + \va{a} \cross (\curl{\va{b}}) +  \va{b} \cross (\curl{\va{a}} )  $,
	\begin{align*}
	\dv{\va{p}_a^{kin}}{t} + q_a \partial_t\va{A}_a + q_a ( \va{u}_a \vdot \grad_a )\va{A}_a &= -q_a \grad_a{\phi_a}  + q_a ((\va{u}_a \vdot \grad_a) \va{A}_a + (\va{A}_a \vdot \grad_a) \va{u}_a + \va{u}_a \cross (\curl{\va{A}_a}) +  \va{A}_a \cross (\curl{\va{u}_a }) ) \\
	\dv{\va{p}_a^{kin}}{t} + q_a \partial_t\va{A}_a &= -q_a \grad_a{\phi_a}  + q_a ((\va{u}_a \vdot \grad_a) \va{A}_a + (\va{A}_a \vdot \grad_a) \va{u}_a + \va{u}_a \cross (\curl{\va{A}_a}) +  \va{A}_a \cross (\curl{\va{u}_a }) )
	\end{align*}
\end{homeworkProblem}

\begin{homeworkProblem}
	\textbf{Part (a)}\\
	\begin{equation*}
	L_{BI} = \dfrac{B_0^2}{\mu_0} - \dfrac{B_0}{\mu_0} \sqrt{B_0^2 + \va{B}^2 - \dfrac{1}{c^2} \va{E}^2  - \dfrac{(\va{E} \vdot \va{B})^2}{(cB_0)^2}}
	\end{equation*}
	We know that,
	\begin{align*}
	\va{D} = \pdv{L_{BI}}{\va{E}} &= - \dfrac{B_0}{2\mu_0} \dfrac{-\dfrac{2 \va{E}}{c^2} - \dfrac{2(\va{E}\vdot \va{B}) \va{B}}{c^2 B_0^2}}{\sqrt{B_0^2 + \va{B}^2 - \dfrac{1}{c^2} \va{E}^2  - \dfrac{(\va{E} \vdot \va{B})^2}{(cB_0)^2}}} \\
	\implies c\va{D} &=  \dfrac{B_0}{\mu_0 c} \dfrac{{ \va{E}} + \dfrac{(\va{E}\vdot \va{B}) \va{B}}{ B_0^2}}{\sqrt{B_0^2 + \va{B}^2 - \dfrac{1}{c^2} \va{E}^2  - \dfrac{(\va{E} \vdot \va{B})^2}{(cB_0)^2}}} \\
	\implies c\va{D}&= \eta_E (\va{E} + \alpha c \va{B})
	\end{align*}
	\begin{align*}
	\va{H} = -\pdv{L_{BI}}{\va{B}} &=  \dfrac{B_0}{2 \mu_0} \dfrac{{2 \va{B} - \dfrac{2  (\va{E} \vdot \va{B}) \va{E}}{(cB_0)^2}}}{\sqrt{B_0^2 + \va{B}^2 - \dfrac{1}{c^2} \va{E}^2  - \dfrac{(\va{E} \vdot \va{B})^2}{(cB_0)^2}}} \\
	\va{H} &= \dfrac{B_0}{ \mu_0 c} \dfrac{{ c\va{B} - \dfrac{  (\va{E} \vdot \va{B}) \va{E}}{cB_0^2}}}{\sqrt{B_0^2 + \va{B}^2 - \dfrac{1}{c^2} \va{E}^2  - \dfrac{(\va{E} \vdot \va{B})^2}{(cB_0)^2}}}\\
	\implies \va{H} &= \eta_E(c \va{B} - \alpha \va{E})
 	\end{align*}
 	From the above expressions, one can write,
 	\begin{align*}
 	c \va{D} \vdot \va{H} &= \eta_E^2 [- \alpha\va{E}^2 + \alpha c^2 \va{B}^2 + (1 - \alpha^2 )c \va{E} \vdot \va{B}] \\
 	&= \eta_E^2 [- \alpha\va{E}^2 + \alpha c^2 \va{B}^2 + (1 - \alpha^2 )c^2 \alpha B_0^2 ]\\
 	&= \eta_E^2 (c^2 \alpha B_0^2 )\qty[- \dfrac{\va{E}^2 }{c^2 B_0^2}+  \dfrac{\va{B}^2 }{B_0^2}+ (1 - \alpha^2 )]\\
 	c \va{D} \vdot \va{H} &= \eta_E^2 \va{E} \vdot \va{B} c \dfrac{1}{\eta_E^2 (\mu_0 c)^2}\\
 	\implies  (\mu_0 c)^2 \va{D} \vdot \va{H} &=\va{E} \vdot \va{B} \\
 	\implies \alpha &= \dfrac{(\mu_0 c)^2 \va{D} \vdot \va{H}}{c B_0^2}
 	\end{align*} 	
 	\textbf{Part (b)}\\
 	\begin{align*}
 	\va{D}^2 - \dfrac{1}{c^2} \va{H}^2 &= \dfrac{\eta_E^2}{c^2} (\va{E}^2 + \alpha^2 c^2 \va{B}^2 + 2 \alpha c \va{E} \vdot \va{B} - c^2 \va{B}^2 - \alpha^2 \va{E}^2 + 2 \alpha c \va{E} \vdot \va{B}) \\
 	&= \dfrac{\eta_E^2}{c^2} ((1- \alpha^2)(\va{E}^2 - c^2 \va{B}^2) + 4 \alpha^2 c^2 B_0^2 )
 	\end{align*}
 	But we know,
 	\begin{align*}
 	\eta_E = \dfrac{1}{ \mu_0 c} \dfrac{1}{\sqrt{1 - \alpha^2 + \dfrac{1}{B_0^2} \qty(\va{B}^2 - \dfrac{1}{c^2} \va{E}^2)}} \implies B_0^2 \qty(\dfrac{1}{(\mu_0 c)^2 \eta_E^2} + \alpha^2 -1) = \qty(\va{B}^2 - \dfrac{1}{c^2} \va{E}^2) \\
 	\eta_H = \dfrac{\mu_0 c}{\sqrt{1 - \alpha^2 + \qty(\dfrac{\mu_0 c}{B_0})^2 \qty(\va{D}^2 - \dfrac{1}{c^2} \va{H}^2)}} \implies \qty[ \qty(\dfrac{\mu_0c}{\eta_H})^2 + \alpha^2 -1 ] \qty(\dfrac{B_0}{\mu_0 c})^2 = \va{D}^2 - \dfrac{1}{c^2} \va{H}^2 
 	\end{align*}
 	Hence, we can rewrite the expression as,
 	\begin{align*}
	 \qty[ \qty(\dfrac{\mu_0c}{\eta_H})^2 + \alpha^2 -1 ] \qty(\dfrac{B_0}{\mu_0 c})^2  &= - {\eta_E^2} \qty((1- \alpha^2) \qty[ B_0^2 \qty(\dfrac{1}{(\mu_0 c)^2 \eta_E^2} + \alpha^2 -1)] + 4 \alpha^2 c^2 B_0^2 ) \\
	 \qty(\dfrac{\mu_0c}{\eta_H})^2 + \alpha^2 -1   &=  (\alpha^2 - 1) \qty[ {1} + {\eta_E^2}(\mu_0 c)^2 (\alpha^2 -1)] - 4 \alpha^2 c^2 {\eta_E^2} (\mu_0 c)^2 \\
	  \qty(\dfrac{\mu_0c}{\eta_H})^2   &= (\mu_0 c)^2 (\alpha^2 - 1)  {\eta_E^2} (\alpha^2 -1) - 4 \alpha^2 c^2 {\eta_E^2} \\
	   \qty(\dfrac{1}{\eta_H})^2   &=  {\eta_E^2} \qty[(\alpha^2 - 1)^2  - {4 \alpha^2 c^2 }] = \eta_E^2 (\alpha^2 + 1)^2
 	\end{align*}
 	\begin{equation*}
 	\qq{Hence,} \eta_E \eta_H (1 + \alpha^2) = 1
 	\end{equation*}
 	We know, 
 	\begin{equation*}
 	\va{H} = \eta_E(c \va{B} - \alpha \va{E}) \qq{and} c\va{D} = \eta_E (\va{E} + \alpha c \va{B})
 	\end{equation*}
 	Hence,
 	\begin{align*}
 	\alpha c \va{D} + \va{H} = \eta_E c ( 1 + \alpha^2 ) \va{B} &\implies \va{B} = {\eta_H} \qty({\alpha  \va{D} + \dfrac{\va{H}}{c}}) \\
 	c \va{D} - \alpha \va{H} = \eta_E (1 + \alpha^2 ) \va{E} &\implies \va{E} = \eta_H (c \va{D} - \alpha \va{H})
 	\end{align*}
 	Applying $ \dfrac{\va{E}}{\mu_0 c} \leftrightarrow \va{H}$ and $ \dfrac{\va{B}}{\mu_0 c} \leftrightarrow \va{D} $, we have,
 	\begin{align*}
 	\eta_H = \dfrac{\mu_0 c}{\sqrt{1 - \alpha^2 + \qty(\dfrac{1}{B_0})^2 \qty(\va{B}^2 - \dfrac{1}{c^2} \va{E}^2)}} &= (\mu_0 c)^2 \eta_E\\
 	\dfrac{\va{E}}{\mu_0 c} = \eta_E(\mu_0 c^2 \va{D} - \alpha \mu_0 c \va{H}) &\implies \va{E} = (\mu_0 c)^2 \eta_E(c \va{D} - \alpha \va{H}) = \eta_H (c \va{D} - \alpha \va{H}) \\
 	c\dfrac{\va{B}}{\mu_0 c} = \eta_E (\mu_0 c) (\va{H} + \alpha c \va{D}) &\implies \va{B} = {\eta_H} \qty({\alpha  \va{D} + \dfrac{\va{H}}{c}}) \\
 	 \mu_0 c \va{D} = {\eta_H} \dfrac{1}{\mu_0 c} \qty({\alpha  \va{B} + \dfrac{\va{E}}{c}}) &\implies c\va{D} = \eta_E (\va{E} + \alpha c \va{B})\\
 	 \mu_0 c \va{H} = \dfrac{\eta_H}{\mu_0 c} (c \va{B} - \alpha \va{E}) &\implies \va{H} = \eta_E(c \va{B} - \alpha \va{E})
 	\end{align*}
 	Thus, we have shown that BI theory has a duality transformation.
 	
 	\textbf{Part (c)}\\
	\begin{align*}
	\delta  L_{BI} =& - \dfrac{B_0}{2\mu_0} \dfrac{1}{\sqrt{B_0^2 + \va{B}^2 - \dfrac{1}{c^2} \va{E}^2  - \dfrac{(\va{E} \vdot \va{B})^2}{(cB_0)^2}}} \qty(2 \va{B} \vdot \delta \va{B} - \dfrac{2 \va{E} \vdot \delta \va{E}}{c^2} - \dfrac{2 \va{E} \vdot \va{B}}{(cB_0)^2} (\va{B} \vdot \delta \va{E} + \va{E} \vdot \delta \va{B})) \\
	&= c \eta_E (\va{E} \vdot \delta \va{E} ) = - \div{\qty(\eta_E \dfrac{\va{E}}{c} \delta \phi)} - \partial_t \qty(\eta_E \dfrac{\va{E}}{c} \vdot \delta \va{A}) + \div{\qty(\eta_E \dfrac{\va{E}}{c})} + \partial_t \qty{\eta_E \dfrac{\va{E}}{c}} \vdot \delta \va{A}
	\end{align*}
	\begin{align*}
	c \eta_E (\va{E} \vdot \delta \va{E} ) &= - \div{\qty(\eta_E \dfrac{\va{E}}{c} \delta \phi)} - \partial_t \qty(\eta_E \dfrac{\va{E}}{c} \vdot \delta \va{A}) + \div{\qty(\eta_E \dfrac{\va{E}}{c})} + \partial_t \qty{\eta_E \dfrac{\va{E}}{c}} \vdot \delta \va{A} \\
	-c \eta_E (\va{B} \vdot \delta \va{B} ) &= \div{(c \eta_E \va{B} \cross \delta \va{A})}  = \delta \va{A} \vdot \curl{c \eta_E \va{B}}\\
	\eta_E\alpha \va{E} \vdot \delta \va{B} &= \eta_E \alpha \va{E} \vdot \curl{\delta \va{A}} - \delta \va{A} \vdot \curl{(c \eta_E \va{B})}\\
	\eta_E\alpha \va{B} \vdot \delta \va{E} &= - \div{(\eta_E \alpha \va{E} \cross \delta \va{A})} - \partial_t (\eta_E \alpha \va{B} \vdot \delta \va{A})
	\end{align*}
	
 	\textbf{Part (d)}\\
 	Including the point charge,
 	\begin{equation*}
 	L = \dfrac{B_0^2}{\mu_0} - \dfrac{B_0}{2\mu_0} \sqrt{B_0^2 + \va{B}^2 - \dfrac{1}{c^2} \va{E}^2  - \dfrac{(\va{E} \vdot \va{B})^2}{(cB_0)^2}} + \va{J} \vdot \va{A} - \rho \phi
 	\end{equation*}
\end{homeworkProblem}
\end{document}
