\documentclass{article}

\usepackage{fancyhdr}
\usepackage{extramarks}
\usepackage{amsmath}
\usepackage{amsthm}
\usepackage{amssymb}
\usepackage{amsfonts}
\usepackage{tikz}
%\usepackage{physics}
\usepackage[plain]{algorithm}
\usepackage{algpseudocode}
\usepackage{hyperref}
\usepackage[arrowdel]{physics}

\usetikzlibrary{automata,positioning}

%
% Basic Document Settings
%

\topmargin=-0.45in
\evensidemargin=0in
\oddsidemargin=0in
\textwidth=6.5in
\textheight=9.0in
\headsep=0.25in

\linespread{1.1}

\pagestyle{fancy}
\lhead{\hmwkAuthorName}
\chead{\hmwkClass\ : \hmwkTitle}
\rhead{\firstxmark}
\lfoot{\lastxmark}
\cfoot{\thepage}

\renewcommand\headrulewidth{0.4pt}
\renewcommand\footrulewidth{0.4pt}

\setlength\parindent{0pt}

%
% Create Problem Sections
%
\newcommand{\be}{\begin{equation}}
\newcommand{\ee}{\end{equation}}
\newcommand{\bes}{\begin{equation*}}
\newcommand{\ees}{\end{equation*}}
\newcommand{\bea}{\begin{flalign*}}
\newcommand{\eea}{\end{flalign*}}


\newcommand{\enterProblemHeader}[1]{
    \nobreak\extramarks{}{Problem \arabic{#1} continued on next page\ldots}\nobreak{}
    \nobreak\extramarks{Problem \arabic{#1} (continued)}{Problem \arabic{#1} continued on next page\ldots}\nobreak{}
}

\newcommand{\exitProblemHeader}[1]{
    \nobreak\extramarks{Problem \arabic{#1} (continued)}{Problem \arabic{#1} continued on next page\ldots}\nobreak{}
    \stepcounter{#1}
    \nobreak\extramarks{Problem \arabic{#1}}{}\nobreak{}
}

\setcounter{secnumdepth}{0}
\newcounter{partCounter}
\newcounter{homeworkProblemCounter}
\setcounter{homeworkProblemCounter}{1}
\nobreak\extramarks{Problem \arabic{homeworkProblemCounter}}{}\nobreak{}

%
% Homework Problem Environment
%
% This environment takes an optional argument. When given, it will adjust the
% problem counter. This is useful for when the problems given for your
% assignment aren't sequential. See the last 3 problems of this template for an
% example.
%
\newenvironment{homeworkProblem}[1][-1]{
    \ifnum#1>0
        \setcounter{homeworkProblemCounter}{#1}
    \fi
    \section{Problem \arabic{homeworkProblemCounter}}
    \setcounter{partCounter}{1}
    \enterProblemHeader{homeworkProblemCounter}
}{
    \exitProblemHeader{homeworkProblemCounter}
}

%
% Homework Details
%   - Title
%   - Due date
%   - Class
%   - Section/Time
%   - Instructor
%   - Author
%

\newcommand{\hmwkTitle}{Pset\ \#3}
\newcommand{\hmwkDueDate}{Due on 25th February, 2019}
\newcommand{\hmwkClass}{Electromagnetism}
\newcommand{\hmwkClassTime}{}
\newcommand{\hmwkClassInstructor}{}
\newcommand{\hmwkAuthorName}{\textbf{Aditya Vijaykumar}}

%
% Title Page
%

\title{
    %\vspace{2in}
    \textmd{\textbf{\hmwkClass:\ \hmwkTitle}}\\
    \normalsize\vspace{0.1in}\small{\hmwkDueDate\ }\\
%    \vspace{3in}
}

\author{\hmwkAuthorName}
\date{}

\renewcommand{\part}[1]{\textbf{\large Part \Alph{partCounter}}\stepcounter{partCounter}\\}

%
% Various Helper Commands
%

% Useful for algorithms
\newcommand{\alg}[1]{\textsc{\bfseries \footnotesize #1}}

% For derivatives
\newcommand{\deriv}[1]{\frac{\mathrm{d}}{\mathrm{d}x} (#1)}

% For partial derivatives
\newcommand{\pderiv}[2]{\frac{\partial}{\partial #1} (#2)}

% Integral dx
\newcommand{\dx}{\mathrm{d}x}

% Alias for the Solution section header
\newcommand{\solution}{\textbf{\large Solution}}

% Probability commands: Expectation, Variance, Covariance, Bias
\newcommand{\E}{\mathrm{E}}
\newcommand{\Var}{\mathrm{Var}}
\newcommand{\Cov}{\mathrm{Cov}}
\newcommand{\Bias}{\mathrm{Bias}}

\begin{document}

\maketitle
(\textbf{Acknowledgements} - I would like to thank Junaid Majeed for discussions.)
\\

\begin{homeworkProblem}[1]
	Consider, with $ \psi^R_G (r) = \dfrac{1}{(2\pi)^{d/2} R^d} \exp(- \dfrac{r^2}{2 R^2}) $,
	\begin{align*}
	\dfrac{1}{R} \pdv{\psi^R_G}{R} &= \dfrac{1}{R}  \dfrac{1}{(2\pi)^{d/2} R^d} \exp(- \dfrac{r^2}{2 R^2}) \qty[\dfrac{-d}{R} - \dfrac{-2 r^2}{2 R^3} ] \\
	\dfrac{1}{R} \pdv{\psi^R_G}{R} &= \dfrac{\psi^R_G(r)}{R^2} \qty[-d + \dfrac{ r^2}{ R^2} ]
	\end{align*}
	Now consider,
	\begin{align*}
	\laplacian{\psi^R_G} &= \dfrac{1}{r^{d-1}} \pdv{r}(r^{d-1} \pdv{\psi^R_G}{r}) \\
	&= \pdv[2]{\psi^R_G}{r} + \dfrac{d-1}{r} \pdv{\psi^R_G}{r} \\
	&= -\dfrac{\psi_G^R}{R^2} + \dfrac{r^2}{R^4} \psi_G^R + \dfrac{d-1}{r} \qty(\dfrac{-r}{R^2}) \psi_G^R\\
	\laplacian{\psi^R_G} &= \dfrac{\psi^R_G(r)}{R^2} \qty[-d + \dfrac{ r^2}{ R^2} ] = \dfrac{1}{R} \pdv{\psi^R_G}{R}
	\end{align*}
	
	Consider now Green's vector field,
	\begin{align*}
	\va{G}[\psi_G^R, \phi_\lambda] &= \psi_G^R \grad \phi_\lambda - \phi_\lambda \grad{\psi_G^R} \\
	\implies \div{\va{G}} &= \psi_G^R \laplacian{\phi_\lambda} - \phi_\lambda \laplacian{\psi_G^R} \\
	\int \dd^d \va{r} (\div{\va{G}}) &=\int \dd^d{\va{r} } ( \psi_G^R \laplacian{\phi_\lambda} - \phi_\lambda \laplacian{\psi_G^R}) \\
	\int_{S_{d-1}} \va{G} \cdot \dd{\va{a}} &=\int \dd^d{\va{r} } ( \psi_G^R \laplacian{\phi_\lambda} - \phi_\lambda \laplacian{\psi_G^R}) \\
	\implies 0 &= \int \dd^d{\va{r} } ( \psi_G^R \laplacian{\phi_\lambda} - \phi_\lambda \laplacian{\psi_G^R}) 
	\end{align*}
	From the above manipulations, we can see that,
	\begin{align*}
	\int \dd^d{\va{r} } (\psi_G^R \laplacian{\phi_\lambda} - \phi_\lambda \laplacian{\psi_G^R} ) &= 0 \\
	\int \dd^d{\va{r} } \qty(\psi_G^R \lambda {\phi_\lambda} - \phi_\lambda \dfrac{1}{R} \pdv{\psi^R_G}{R} ) &= 0\\
	\implies \lambda \int \dd^d{\va{r} } \qty(\psi_G^R  {\phi_\lambda}) &= \dfrac{1}{R} \pdv{R} \int \dd^d{\va{r} } \phi_\lambda  {\psi^R_G} \\
	\implies  \int \dd^d{\va{r} } \phi_\lambda  {\psi^R_G} = C \exp(\dfrac{\lambda R^2}{2})
	\end{align*}
	Good, so now we have got an expression for the gaussian average. All that is left is to figure out the parameter $ C $. For this we note that the Gaussian distribution approaches a Dirac delta function as $ R \rightarrow 0 $, and then write,
	\begin{align*}
	\int \dd^d{\va{r} } \phi_\lambda \lim\limits_{R \rightarrow 0} {\psi^R_G (\va{r} - \va{r}_0)} &= C \lim\limits_{R \rightarrow 0}  \exp(\dfrac{\lambda R^2}{2}) \\
	\int \dd^d{\va{r} } \phi_\lambda {\delta (\va{r} - \va{r}_0)} &= C \\
	\implies C &= \phi_\lambda(\va{r}_0)
	\end{align*} 
	\begin{equation*}
	\implies \boxed{\int \dd^d{\va{r} } \ \phi_\lambda(\va{r})  {\psi^R_G} (\va{r} - \va{r}_0) = \phi_\lambda(\va{r}_0) \exp(\dfrac{\lambda R^2}{2})}
	\end{equation*}
	
	\textbf{Part (b)}\\
	\begin{align*}
	\implies \int \dd^d{\va{r} } \ \phi_\lambda(\va{r})  \dfrac{1}{(2\pi)^{d/2} R^d} \exp(- \dfrac{\abs{\va{r} - \va{r}_0}^2}{2 R^2})   &= \phi_\lambda(\va{r}_0) \exp(\dfrac{\lambda R^2}{2}) \\
	\implies \int \dd^d{\va{r} } \ \phi_\lambda(\va{r})  \dfrac{1}{(2\pi)^{d/2} R^d} \exp(- \dfrac{r^2}{2 R^2}) \exp(- \dfrac{r_0^2}{2 R^2}) \exp(\dfrac{\va{r}_0 \vdot \va{r}}{R^2})   &= \phi_\lambda(\va{r}_0) \exp(\dfrac{\lambda R^2}{2}) \\
	\implies \int \dd^d{\va{r} } \ \phi_\lambda(\va{r}) \psi_G^R(\va{r}) \exp(- \dfrac{r_0^2}{2 R^2}) \exp(\dfrac{\va{r}_0 \vdot \va{r}}{R^2})   &= \phi_\lambda(\va{r}_0) \exp(\dfrac{\lambda R^2}{2})
	\end{align*}
	Taking averages over $ \va{r}_0 $ and noting that $ \kappa = \dfrac{r}{R^2} $,
	\begin{equation*}
	 \int \dd^d{\va{r} } \ \phi_\lambda(\va{r}) \psi_G^R(\va{r}) \exp(- \dfrac{r_0^2}{2 R^2}) I_0 \qty(d;  \dfrac{r}{R^2} r_0)  = \ev{\phi_\lambda(\va{r}_0)} \exp(\dfrac{\lambda R^2}{2})
	\end{equation*}
	Now, let's make the substitution, $ r_0 = \kappa R^2 $ and $ \phi_\lambda (r) = J_0(d; k r) \implies \lambda = - k^2$
	\begin{align*}
	\int \dd^d{\va{r} } \ J_0(d; k r) \psi_G^R(\va{r}) \exp(- \dfrac{\kappa^2 R^2}{2}) I_0 \qty(d;  \kappa r)  &= \ev{J_0(d; k r_0)} \exp(-\dfrac{k^2 R^2}{2}) \\
	\int \dd^d{\va{r} } \ J_0(d; k r) I_0 \qty(d;  \kappa r)  \psi_G^R(\va{r})  &= J_0(d; k \kappa R^2)\exp(\dfrac{(\kappa^2 - k^2)R^2}{2})
	\end{align*}
	Similarly, taking $ \phi_\lambda (r) = I_0(d; \kappa' r) \implies \lambda = \kappa'^2$,
	\begin{align*}
	\int \dd^d{\va{r} } \ I_0(d; \kappa' r) \psi_G^R(\va{r}) \exp(- \dfrac{\kappa^2 R^2}{2}) I_0 \qty(d;  \kappa r)  &= \ev{I_0(d; \kappa' r_0)} \exp(\dfrac{\kappa'^2 R^2}{2}) \\
	\int \dd^d{\va{r} } \ I_0(d; \kappa' r) I_0 \qty(d;  \kappa r)  \psi_G^R(\va{r})  &= I_0(d; \kappa' \kappa R^2) \exp(\dfrac{(\kappa^2 + \kappa'^2)R^2}{2})
	\end{align*}
	Now, let's make the substitution, $ r_0 = i k_2 R^2 $ and $ \phi_\lambda (r) = J_0(d; k_1 r) \implies \lambda = - k_1^2$,
	\begin{align*}
	\int \dd^d{\va{r} } \ J_0(d; k_1 r) \psi_G^R(\va{r}) \exp( \dfrac{k_2^2 R^2}{2}) I_0 \qty(d; i k_2 r)  &= \ev{J_0(d; i k_1 k_2 R^2)} \exp(-\dfrac{k_1^2 R^2}{2}) \\
	\int \dd^d{\va{r} } \ J_0(d; k_1 r)  J_0 \qty(d;  k_2 r) \psi_G^R(\va{r})  &= I_0(d; k_1 k_2 R^2)\exp(-\dfrac{(k_1^2 + k_2^2) R^2}{2})
	\end{align*}
	
	\textbf{Part (c)}\\
	\textbf{Part (d)}\\
	Consider, 
	\begin{equation*}
	I_1 = \int_0^\infty \dfrac{\dd{R}}{R^{2 \Delta +1 }}  \exp(-\dfrac{r^2}{2 R^2})
	\end{equation*}
	Substitute $ R^2 = \dfrac{r^2}{2 u} \implies \dd{u} = - \dfrac{r^2}{R^3} \dd{R} \implies \dd{R} = - \dfrac{r\dd{u}}{ (2u)^{3/2}}$ ,
	\begin{align*}
	I_1 &= \int_0^\infty  \dfrac{r \dd{u}}{ (2u)^{3/2}} \qty(\dfrac{2u}{r^2})^{\Delta + 1/2}  \exp(-u) \\
	&= \dfrac{2^{\Delta - 1}}{r^{2\Delta}} \int_0^\infty  \dd{u} {u}^{\Delta -1}  \exp(-u) \\
	I_1 &= \dfrac{2^{\Delta - 1}}{r^{2\Delta}} \Gamma(\Delta) \qq{;} \qq{for} \Delta > 0
	\end{align*}
	Now consider,
	\begin{equation*}
	I_2 = \int_{0}^{\infty} \dfrac{\dd{R}}{R^{2 \Delta + 1}} (2 \pi)^{d/2} R^d \exp(-\dfrac{k^2 R^2}{2})
	\end{equation*}
	Substitute $ \dfrac{k^2 R^2}{2} = u \implies \dd{u} = k^2 R \dd{R} , R = \qty(\dfrac{2u}{k^2})^{1/2} $,
	\begin{align*}
	I_2 &= \int_{0}^{\infty}\dfrac{\dd{u}}{k^2 } \dfrac{1}{R^{2 \Delta + 2}} (2 \pi)^{d/2} \qty(\dfrac{2u}{k^2})^{d/2} \exp(-u) \\
	&= \int_{0}^{\infty}\dfrac{\dd{u}}{k^2 } \qty(\dfrac{k^2}{2u})^{\Delta + 1} (2 \pi)^{d/2} \qty(\dfrac{2u}{k^2})^{d/2} \exp(-u) \\
	&=  \dfrac{2^{d - \Delta - 1} \pi^{d/2}}{k^{2 \Delta - d}} \int_{0}^{\infty} \dd{u} u^{d/2 - \Delta - 1 } \exp(-u) \\
	I_2 &= \dfrac{2^{d - \Delta - 1} \pi^{d/2}}{k^{2 \Delta - d}} \Gamma(d/2 - \Delta) \qq{;} \qq{for} \dfrac{d}{2} > \Delta
	\end{align*}
	From Gaussian averaging, we know that,
	\begin{align*}
	\int \dd^d{\va{r} } \ \phi_{-k^2}(\va{r})  {\psi^R_G} (\va{r} - \va{r}_0) &= \phi_{-k^2}(\va{r}_0) \exp(-\dfrac{k^2 R^2}{2}) \\
	\int \dd^d{\va{r} } \ \phi_{-k^2}(\va{r}) \int_{0}^{\infty} \dfrac{\dd{R}}{R^{2 \Delta + 1}} (2 \pi)^{d/2} R^d {\psi^R_G} (\va{r} - \va{r}_0) &= \phi_{-k^2}(\va{r}_0) \int_{0}^{\infty} \dfrac{\dd{R}}{R^{2 \Delta + 1}} (2 \pi)^{d/2} R^d \exp(-\dfrac{k^2 R^2}{2}) \\
	\int \dd^d{\va{r} } \ \phi_{-k^2}(\va{r}) \dfrac{2^{\Delta - 1}}{\abs{\va{r} - \va{r}_0}^{2\Delta}} \Gamma(\Delta)  &= \phi_{-k^2}(\va{r}_0) \dfrac{2^{d - \Delta - 1} \pi^{d/2}}{k^{2 \Delta - d}} \Gamma(d/2 - \Delta) \\
	\implies \int \dd^d{\va{r} } \ \phi_{-k^2}(\va{r}) \dfrac{\Gamma(\Delta)}{\abs{\va{r} - \va{r}_0}^{2\Delta}}   &= \phi_{-k^2}(\va{r}_0) \dfrac{2^{d - 2\Delta } \pi^{d/2}}{k^{2 \Delta - d}} \Gamma(d/2 - \Delta)
	\end{align*}
\end{homeworkProblem}

\begin{homeworkProblem}[2]
	\textbf{Part (a)}\\
	We know,
	\begin{align*}
	I_0(d;x) &= \sum_{m=0}^{\infty} \dfrac{\Gamma(d/2)}{m! \ \Gamma(d/2 + m)} \dfrac{x^{2m}}{2^{2m}} \\
	\dfrac{x^2}{d^2} I_0(d + 2; x) &= \sum_{m=0}^{\infty} \dfrac{\Gamma(d/2 + 1)}{m! \ \Gamma(d/2 + m + 1)} \dfrac{x^{2m + 2}}{2^{2m}} \\
	&= \sum_{m=0}^{\infty} \dfrac{\Gamma(d/2) \vdot d/2}{m! \ \Gamma(d/2 + m + 1)} \dfrac{x^{2m + 2}}{2^{2m + 2}} \dfrac{2^2}{d^2} \\
	\dfrac{x^2}{d^2} I_0(d + 2;x) &= \dfrac{2}{d} \sum_{m=0}^{\infty} \dfrac{\Gamma(d/2) }{m! \ \Gamma(d/2 + m + 1)} \dfrac{x^{2m + 2}}{2^{2m + 2}}
	\end{align*}
	Let's proceed and take the derivative,
	\begin{align*}
	\dfrac{x}{d} \dv{x} I_0(d;x) &= \dfrac{x}{d}  \sum_{m=0}^{\infty} \dfrac{\Gamma(d/2)}{m! \ \Gamma(d/2 + m)} \dv{x} \qty(\dfrac{x^{2m}}{2^{2m}}) \\
	&= \dfrac{1}{d}  \sum_{m=1}^{\infty} \dfrac{\Gamma(d/2)}{m! \ \Gamma(d/2 + m)}  \dfrac{(2m) x^{2m}}{2^{2m}} \\
	&=  \dfrac{1}{d}  \sum_{m=0}^{\infty} \dfrac{\Gamma(d/2)}{(m+1)! \ \Gamma(d/2 + m + 1)}  \dfrac{2(m+1) x^{2m + 2}}{2^{2m + 2}} \impliedby (m \rightarrow m+	1) \\
	\dfrac{x}{d} \dv{x} I_0(d;x) &=  \dfrac{2}{d}  \sum_{m=0}^{\infty} \dfrac{\Gamma(d/2)}{m! \ \Gamma(d/2 + m + 1)}  \dfrac{ x^{2m + 2}}{2^{2m + 2}} 
	\end{align*}
	Let's now consider the third part,
	\begin{align*}
	I_0(d-2;x) - I_0(d,x) &= \sum_{m=0}^{\infty} \qty[ \dfrac{\Gamma(d/2 - 1)}{ \ \Gamma(d/2 + m -1 )} - \dfrac{\Gamma(d/2)}{\ \Gamma(d/2 + m)} ] \dfrac{x^{2m}}{m! \ 2^{2m}} \\
	&= \sum_{m=1}^{\infty} \qty[ \dfrac{\Gamma(d/2 - 1)}{ \ \Gamma(d/2 + m -1 )} - \dfrac{\Gamma(d/2)}{\ \Gamma(d/2 + m)} ] \dfrac{x^{2m}}{m! \ 2^{2m}} \\
	&= \sum_{m=0}^{\infty} \qty[ \dfrac{\Gamma(d/2 - 1)}{ \ \Gamma(d/2 + m )} - \dfrac{\Gamma(d/2)}{\ \Gamma(d/2 + m + 1)} ] \dfrac{x^{2m+2}}{(m+1)! \ 2^{2m}} \\
	&= \sum_{m=0}^{\infty} \qty[\dfrac{\Gamma(d/2)}{\ \Gamma(d/2 + m + 1)}  \dfrac{ (d/2 + m )}{(d/2 - 1)} - \dfrac{\Gamma(d/2)}{\ \Gamma(d/2 + m + 1)} ] \dfrac{x^{2m+2}}{(m+1)! \ 2^{2m}} \\
	&= \sum_{m=0}^{\infty} \qty[ \dfrac{2(m+1)}{d-2} ] \dfrac{\Gamma(d/2)}{\ \Gamma(d/2 + m + 1)} \dfrac{x^{2m+2}}{(m+1)! \ 2^{2m}} \\
	&= \dfrac{2}{d-2} \sum_{m=0}^{\infty}  \dfrac{\Gamma(d/2)}{\ \Gamma(d/2 + m + 1)} \dfrac{x^{2m+2}}{m! \ 2^{2m}} \\
	\dfrac{d - 2}{d}(I_0(d-2;x) - I_0(d,x)) &=  \dfrac{2}{d}  \sum_{m=0}^{\infty} \dfrac{\Gamma(d/2)}{m! \ \Gamma(d/2 + m + 1)}  \dfrac{ x^{2m + 2}}{2^{2m + 2}}
	\end{align*}
	\begin{equation*}
	\implies \boxed{ \dfrac{x}{d} \dv{x} I_0(d;x) = \dfrac{x^2}{d^2} I_0(d + 2; x) = \dfrac{d - 2}{d}\qty[ I_0(d-2;x) - I_0(d,x)] }
	\end{equation*}
	
	\textbf{Part (b)}\\
	Schafli's contour integral is given by,
	\begin{equation*}
	I_0(d;x) = \oint_\mathcal{C} \dfrac{\dd{z}}{2 \pi i} \dfrac{\Gamma(d/2)}{z^{d/2}} \exp(z + \dfrac{x^2}{4z})
	\end{equation*}
	where the contour $ \mathcal{C} = C[-\infty - i0, 0+, -\infty + i0] $.
	\begin{align*}
	\dfrac{x}{d} \dv{x} I_0(d;x) &= \oint_\mathcal{C} \dfrac{x}{d} \dfrac{\dd{z}}{2 \pi i} \dfrac{\Gamma(d/2)}{z^{d/2}} \dfrac{x}{2z} \exp(z + \dfrac{x^2}{4z}) \\
	&= \dfrac{x^2}{d^2}\oint_\mathcal{C} \dfrac{\dd{z}}{2 \pi i} \dfrac{(d/2) \cdot \Gamma(d/2)}{z^{(d+2)/2}}  \exp(z + \dfrac{x^2}{4z}) \\
	&= \dfrac{x^2}{d^2}\oint_\mathcal{C} \dfrac{\dd{z}}{2 \pi i} \dfrac{ \Gamma((d+2)/2)}{z^{(d+2)/2}}  \exp(z + \dfrac{x^2}{4z})\\
	\implies \dfrac{x}{d} \dv{x} I_0(d;x) &= \dfrac{x^2}{d^2} I_0(d + 2; x)
	\end{align*}	
	\begin{align*}
	 I_0(d-2;x) - I_0(d,x) &= \oint_\mathcal{C} \dfrac{\dd{z}}{2 \pi i} \qty[\dfrac{\Gamma(d/2 - 1)}{z^{d/2 -1}} - \dfrac{\Gamma(d/2)}{z^{d/2}}] \exp(z + \dfrac{x^2}{4z})\\
	 &= 
	\end{align*}
	
	\textbf{Part (c)}\\
	We are given,
	\begin{align*}
	I_0(d;x) &\approx \dfrac{e^x}{\abs{S^{d-1}}} \qty(\dfrac{2\pi}{x})^{\frac{d-1}{2}} \sum_{n=0}^{\infty} \dfrac{(-1)^n \Gamma(\frac{d-1}{2} + n)}{(2x)^n n! \ \Gamma(\frac{d-1}{2} -n )} \\
	\implies \dfrac{x}{d} \dv{x} I_0(d;x) &\approx \dfrac{x}{d} \qty[\dfrac{e^x}{\abs{S^{d-1}}}  \qty(\dfrac{2\pi}{x})^{\frac{d-1}{2}} \sum_{n=0}^{\infty} \dfrac{(-1)^n \Gamma(\frac{d-1}{2} + n)}{(2x)^n n! \ \Gamma(\frac{d-1}{2} -n )} ]
	\end{align*}
\end{homeworkProblem}

\begin{homeworkProblem}
	\textbf{Part (a)}\\
	Given,
	\begin{align*}
	V(z, \rho) &= \int_{0}^{2 \pi} \dfrac{\dd{\alpha}}{2\pi} g(z - i \rho \cos \alpha) \\
	\laplacian{V(z, \rho)} &= \int_{0}^{2 \pi} \dfrac{\dd{\alpha}}{2\pi} \laplacian g(z - i \rho \cos \alpha) \\
	&= \int_{0}^{2 \pi} \dfrac{\dd{\alpha}}{2\pi} \qty[
	\dfrac{1}{\rho} \pdv{\rho}(\rho \pdv{(g(z - i \rho \cos \alpha))}{\rho}) + \pdv[2]{g(z - i \rho \cos \alpha)}{z} ] \\
	&= \int_{0}^{2 \pi} \dfrac{\dd{\alpha}}{2\pi} \qty[
	\dfrac{1}{\rho} \pdv{\{\rho g'(t) (-i \cos \alpha)\}}{\rho}  + g''(t)  ] \qq{;} t = z - i \rho \cos \alpha \\
	&= \int_{0}^{2 \pi} \dfrac{\dd{\alpha}}{2\pi} \qty[
	- \dfrac{i g'(t ) \cos \alpha}{\rho} - g''(t) \cos^2 \alpha  + g''(t)  ] \\
	&= \int_{0}^{2 \pi} \dfrac{\dd{\alpha}}{2\pi} \qty[
	- \dfrac{i g'(t ) \cos \alpha}{\rho} + g''(t) \sin^2 \alpha   ]
	\end{align*}
	Now we proceed to integrate the first term by parts,
	\begin{align*}
	\laplacian{V(z, \rho)} &=- \eval{\dfrac{i g'(t ) \sin \alpha}{\rho}}_{0}^{2 \pi} + \int_{0}^{2 \pi} \dfrac{\dd{\alpha}}{2\pi} \qty[	 {i g''(t ) i \sin \alpha \sin \alpha} + g''(t) \sin^2 \alpha   ]\\
	\implies \laplacian{V(z, \rho)} &= 0 
	\end{align*}
	Hence, we have proved that $ V $ solves Laplace equation.
	\\
	
	\textbf{Part (b)}\\
	Consider,
	\begin{align*}
	g(z - i \rho \cos \alpha ) &= \sum_{n=0}^{\infty} g^{(n)}(z) (-i)^n \cos^n \alpha  \dfrac{\rho^n}{n!} \\
	\int_{0}^{2\pi} \dfrac{\dd{\alpha}}{2 \pi} g(z - i \rho \cos \alpha ) &= \sum_{n=0}^{\infty} g^{(n)}(z) (-i)^n \dfrac{\rho^n}{n!} \int_{0}^{2\pi} \dfrac{\dd{\alpha}}{2 \pi} \cos^n \alpha 
	\end{align*}
	\begin{equation*}
	\int_{0}^{2\pi} \frac{\dd{\alpha}}{2 \pi} \cos^n \alpha  = 0 \qq{for} n = 1,3, 5 \ldots \qq{;} \int_{0}^{2\pi} \frac{\dd{\alpha}}{2 \pi} \cos^n \alpha  = \dfrac{1}{2^n} {}^nC_{n/2} \qq{for} n = 0, 2, 4, \ldots
	\end{equation*}
	\begin{equation*}
	\implies V(g,z) = \sum_{m=0}^{\infty} g^{(2m)}(z) (-1)^{m} \dfrac{\rho^{2m} }{(2m)!} \dfrac{1}{2^{2m}} {}^{2m}C_{m}
	\end{equation*}
	But, $ g(z) = V(z,0) \implies g^{(n)}(z)  = V^{n}(z,0)$. Hence, we write,
	\begin{align*}
	\boxed{V(g,z) = V(z,0) - \dfrac{\rho^2}{4} V^{(2)}(z,0) + \dfrac{\rho^4}{64} V^{(4)}(z,0) + \ldots}
	\end{align*}
	\textbf{Part (c)}\\
	\begin{align*}
	V(g,z) &= \sum_{m=0}^{\infty} (-1)^{m} \dfrac{\rho^{2m} }{(2m)!} \dfrac{1}{2^{2m}} {}^{2m}C_{m}  \pdv[2m]{z} V(z,0) \\
	&= \sum_{m=0}^{\infty} (-1)^{m} \dfrac{1 }{(m!)^2} \qty(\dfrac{1}{2} \rho  \pdv{z})^{2m} V(z,0) \\
	V(g,z) &= J_0\qty(d=2;\rho  \pdv{z} ) V(z,0)
	\end{align*}
	Consider,
	\begin{align*}
	E_z(z,\rho) = - \pdv{V(z,\rho)}{z} = - \sum_{m=0}^{\infty} (-1)^{m} \dfrac{1 }{(m!)^2} \qty(\dfrac{1}{2} \rho  \pdv{z})^{2m} \pdv{V(z,0)}{z}  = J_0\qty(d=2;\rho  \pdv{z} ) E_z(z,0) \\
	E_\rho(z,\rho) = - \pdv{V(z,\rho)}{\rho} = - \rho \dfrac{1}{\rho} \pdv{\rho} J_0\qty(d=2;\rho  \pdv{z} ) V(z,0) = - \dfrac{\rho}{2} J_0\qty(d=4;\rho  \pdv{z} ) \pdv[2]{V}{z} = \dfrac{\rho}{2} J_0\qty(d=4;\rho  \pdv{z} ) \pdv{E_z}{z}
	\end{align*}
	\textbf{Part (d)}\\
	\begin{align*}
	V_l (z,\rho) &= J_0 \qty( \rho \pdv{z})  \\
	&= \sum_{m=0}^{\infty} (-1)^{m} \dfrac{1 }{(m!)^2} \qty(\dfrac{1}{2} \rho  \pdv{z})^{2m} \dfrac{z^l}{l!}  \\
	&= \sum_{m=0}^{\infty} (-1)^{m} \dfrac{1 }{(m!)^2} \dfrac{\rho^{2m}}{2^{2m}}  \dfrac{l!}{(l-2m)!} \dfrac{z^{l-2m}}{l!} \qq{if} l\ge 2m \qq{else} =0\\
	V_l (z,\rho) &= \sum_{m=0}^{\infty} (-1)^{m} \dfrac{1 }{(m!)^2} \dfrac{\rho^{2m}}{2^{2m}}  \dfrac{z^{l-2m}}{(l-2m)!}
	\end{align*}
	\begin{equation*}
	\boxed{V_1(z,\rho) = z  \qq{and} V_2 (\rho,z) = \dfrac{z^2}{2} - \dfrac{\rho^2}{4}}
	\end{equation*}
\end{homeworkProblem}


\end{document}
