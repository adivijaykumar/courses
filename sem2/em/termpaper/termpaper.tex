\documentclass[a4paper,11pt]{article}

\usepackage{physics}
\usepackage{amsmath}
\usepackage{amssymb}
\usepackage{amsmath}
\usepackage{amsthm, mathtools}
%\usepackage{hyperref}
\usepackage{color}
\usepackage{jheppub}
\usepackage[T1]{fontenc} % if needed

\newcommand{\be}{\begin{equation}}
\newcommand{\ee}{\end{equation}}
\newcommand{\bes}{\begin{equation*}}
\newcommand{\ees}{\end{equation*}}
\newcommand{\bea}{\begin{flalign*}}
\newcommand{\eea}{\end{flalign*}}

%\linespread{1.0}
%\setlength{\parindent}{0em}
%\setlength{\parskip}{0.8em}

\title{\textbf{Multipole Expansion of Gravitational Waves}}
\author{Aditya Vijaykumar}
\affiliation{International Centre for Theoretical Sciences, Bengaluru, India.}
\emailAdd{aditya.vijaykumar@icts.res.in}
\abstract{Term Paper for the Electromagnetism Course, January-April 2019}

\begin{document}
\maketitle


\section{Gravitational Waves}
\textit{Main Reference : Gravitation - Foundations and Frontiers by Thanu Padmanabhan}

\subsection{Facts about the Field Equations}
The field equations are given by,
\begin{equation*}
G_{ab}  = R_{ab} - \dfrac{1}{2} g_{ab} R = 8 \pi T_{ab}
\end{equation*}
where,
\begin{align*}
R_{ijkl} = \dfrac{1}{2} (\partial_k \partial_l g_{im}  + \partial_i \partial_m g_{kl} - \partial_k \partial_m g_{il} &- \partial_i \partial_l g_{km}) + g_{np} (\Gamma^n_{kl} \Gamma^p_{im} - \Gamma^p_{il} \Gamma^n_{km}) \\
R_{ab} &= R_{iajb} g^{ij} \\
R &= R_{ab} g^{ab}
\end{align*}
The equations are symmetric under the exchange of indices, and hence there are really \textbf{ten} equations from the face of it. But we also note that the Bianchi identity tells us that $ \nabla_a G^a_{b} = 0 $, \textcolor{red}{Where does this come from? Read.} giving us \textbf{four} constraint equations. So really, we have \textbf{six} independent functions of spacetime coordinates to solve for. This suggests naively that though we have ten independent components in $ g_{ab} $ only six of them play a part in dynamics, and the other four are just fixed by the evolution of these six components. Let's now observe a few other things :-

\begin{itemize}
	\item Second derivatives of time are contained only in $ R_{0 \alpha 0 \beta} $, where $ \alpha, \beta $ run only over the space coordinates. \textit{The only terms which have double time derivatives}, hence, are $ \ddot{g}_{\alpha \beta} $.
	\item $ \nabla_a G^a_{b} = 0 \implies \nabla_0 G^0_{b} =-  \nabla_\alpha G^\alpha_{b}  $. The RHS has terms with double time derivatives, which means that $ G^0_{b} $ should have \textit{only one time derivative}.
	\item Further, the space-time and the time-time components have first time derivatives which are propotional to $ \dot{g}_{\alpha \beta} $. The time derivatives of the type $ \dot{g}_{0 a} $ \textit{do not appear in Einstein's equations}.
\end{itemize}
This now gives us an alternate way to understand the evolution equations. To evolve the equations, we need to provide the initial values of $ g_{\alpha \beta} $ and $ \dot{g}_{\alpha \beta} $ on a time slice. The space-time and the time-time Einstein's equations fix the value of the other four components. \textcolor{red}{Add Harmonic Gauge details here.}

\subsection{Weak field limit of gravity} 
Consider $ g_{ab} = \eta_{ab} + \epsilon h_{ab} $, with $  \eta = \text{diag} (-1,1,1,1) $, and $ \epsilon $ being a small number. We now write down the field equations with terms only up to $ \order{\epsilon} $. After the dust settles, we get,
\begin{equation*}
\partial_n \partial_m h + \Box h_{mn} - \partial_n \partial_r h^r_m - \partial_m \partial_r h^r_n - \eta_{mn} \qty(\Box h - \partial_r \partial_s h^{sr}) = - 16 \pi \kappa T_{mn}
\end{equation*}
We effect a change of variables as $ \bar{h}_{mn} = h_{mn} - \dfrac{\eta_{mn}}{2} h $. The equation now becomes,
\begin{equation*}
\Box \bar{h}_{mn} + \eta_{mn} \partial_r \partial_s \bar{h}^{rs} - \partial_n \partial_r \bar{h}^r_m - \partial_m \partial_r \bar{h}^r_n = - 16 \pi \kappa T_{mn}
\end{equation*}
Noticing that the above equation is invariant under gauge transformations of the form \begin{equation*}
\bar{h}'^{mr} = \bar{h}^{mr} - \partial^m \xi^r - \partial^r \xi^m + \eta^{mr} \partial_s \xi^s
\end{equation*}
and imposing the harmonic gauge constraint $ \Box \xi^m =0 $, \textcolor{red}{review this calculation} we get,
\begin{equation*}
\Box \bar{h}^{mn} = - 16 \pi \kappa T_{mn}
\end{equation*}
This shows us that some sort of gravitational waves exist, with or without a source.
\subsection{Gravitational waves in a flat background}
By symmetry considerations of the Riemann tensor and upto first order in perturbation, we can write,
\begin{equation*}
\Box R_{bcmn } = 8 \pi [\partial_b(\partial_m \bar{T}_{nc} - \partial_n \bar{T}_{mc} ) - \partial_c(\partial_m \bar{T}_{nb} - \partial_n \bar{T}_{mb} )] = 8 \pi \bar{T}_{bcmn}
\end{equation*}
where $ \bar{T}_{ij} = T_{ij} -\dfrac{g_{ij} T}{2} $. The equation in invariant under infinitesimal coordinate transformations. Let's first consider the vacuum case $ \Box R_{bcmn } =0 \implies R_{abmn} = C_{abmn} e^{i k_a x^a}, k^a k_a = 0$. The Bianchi identity gives, \textcolor{red}{Read Up}
\begin{equation*}
C_{bcmn} k_a + C_{camn}k_b + C_{abmn}k_c =0 
\end{equation*}
We now choose the wavevector to be oriented aong the $ z $-axis \textit{ie} $ k_a = (-\omega, 0 , 0 , \omega) $. Setting index $ c=0 $, we get
\begin{align*}
C_{abmn} = \dfrac{1}{\omega}(C_{b0mn} k_a + C_{0amn}k_b ) &= \dfrac{1}{\omega}(C_{b0mn} k_a - C_{a0mn}k_b ) \\
n \rightarrow 0 \implies C_{a0bm} &= \dfrac{1}{\omega}(C_{b0m0} k_a - C_{a0m0}k_b )
\end{align*}
Substituting the second equation into the first, we can figure that $ C_{abmn} $ can be specified completely in terms of the form $ C_{i0j0} $. Furthermore, substituting $a=0$ in the second equation gives,
\begin{align*}
 C_{00bm} &= \dfrac{1}{\omega}(C_{b0m0} (-\omega) - C_{00m0}k_b ) \implies C_{00m0}=0
\end{align*}
This means that $ i,j $ cannot be zero, and hence only the terms of the form $ C_{\alpha 0 \beta 0} $ survive this ordeal.

\section{Symmetric Trace-free Tensors}

\end{document}
