\documentclass{article}

\usepackage{fancyhdr}
\usepackage{extramarks}
\usepackage{amsmath}
\usepackage{amsthm}
\usepackage{amssymb}
\usepackage{amsfonts}
\usepackage{tikz}
\usepackage{physics}
\usepackage[plain]{algorithm}
\usepackage{algpseudocode}
\usepackage{hyperref}

\usetikzlibrary{automata,positioning}

%
% Basic Document Settings
%

\topmargin=-0.45in
\evensidemargin=0in
\oddsidemargin=0in
\textwidth=6.5in
\textheight=9.0in
\headsep=0.25in

\linespread{1.1}

\pagestyle{fancy}
%\lhead{\hmwkAuthorName}
\chead{\hmwkClass\ : \hmwkTitle}
\rhead{\firstxmark}
\lfoot{\lastxmark}
\cfoot{\thepage}

\renewcommand\headrulewidth{0.4pt}
\renewcommand\footrulewidth{0.4pt}

\setlength\parindent{0pt}

%
% Create Problem Sections
%
\newcommand{\be}{\begin{equation}}
\newcommand{\ee}{\end{equation}}
\newcommand{\bes}{\begin{equation*}}
\newcommand{\ees}{\end{equation*}}
\newcommand{\bea}{\begin{flalign*}}
\newcommand{\eea}{\end{flalign*}}


\newcommand{\enterProblemHeader}[1]{
    \nobreak\extramarks{}{Problem \arabic{#1} continued on next page\ldots}\nobreak{}
    \nobreak\extramarks{Problem \arabic{#1} (continued)}{Problem \arabic{#1} continued on next page\ldots}\nobreak{}
}

\newcommand{\exitProblemHeader}[1]{
    \nobreak\extramarks{Problem \arabic{#1} (continued)}{Problem \arabic{#1} continued on next page\ldots}\nobreak{}
    \stepcounter{#1}
    \nobreak\extramarks{Problem \arabic{#1}}{}\nobreak{}
}

\setcounter{secnumdepth}{0}
\newcounter{partCounter}
\newcounter{homeworkProblemCounter}
\setcounter{homeworkProblemCounter}{1}
\nobreak\extramarks{Problem \arabic{homeworkProblemCounter}}{}\nobreak{}

%
% Homework Problem Environment
%
% This environment takes an optional argument. When given, it will adjust the
% problem counter. This is useful for when the problems given for your
% assignment aren't sequential. See the last 3 problems of this template for an
% example.
%
\newenvironment{homeworkProblem}[1][-1]{
    \ifnum#1>0
        \setcounter{homeworkProblemCounter}{#1}
    \fi
    \section{Problem \arabic{homeworkProblemCounter}}
    \setcounter{partCounter}{1}
    \enterProblemHeader{homeworkProblemCounter}
}{
    \exitProblemHeader{homeworkProblemCounter}
}

%
% Homework Details
%   - Title
%   - Due date
%   - Class
%   - Section/Time
%   - Instructor
%   - Author
%

\newcommand{\hmwkTitle}{Tutorial\ \#3	}
\newcommand{\hmwkDueDate}{ICTS Summer School on Gravitational Wave Astronomy, 2019. }
\newcommand{\hmwkClass}{Advanced General Relativity}
\newcommand{\hmwkClassTime}{}
\newcommand{\hmwkClassInstructor}{Sudipta Sarkar}
\newcommand{\hmwkAuthorName}{\textbf{Tutorial Instructors : Aditya Vijaykumar, Md Arif Shaikh}}

%
% Title Page
%

\title{
    %\vspace{2in}
    \textmd{\textbf{\hmwkClass:\ \hmwkTitle}}\\
    \normalsize\vspace{0.1in}\small{\hmwkDueDate\ }\\
%    \vspace{3in}
}

\author{\hmwkAuthorName}
\date{}

\renewcommand{\part}[1]{\textbf{\large Part \Alph{partCounter}}\stepcounter{partCounter}\\}

%
% Various Helper Commands
%

% Useful for algorithms
\newcommand{\alg}[1]{\textsc{\bfseries \footnotesize #1}}

% For derivatives
\newcommand{\deriv}[1]{\frac{\mathrm{d}}{\mathrm{d}x} (#1)}

% For partial derivatives
\newcommand{\pderiv}[2]{\frac{\partial}{\partial #1} (#2)}

% Integral dx
\newcommand{\dx}{\mathrm{d}x}

% Alias for the Solution section header
\newcommand{\solution}{\textbf{\large Solution}}

% Probability commands: Expectation, Variance, Covariance, Bias
\newcommand{\E}{\mathrm{E}}
\newcommand{\Var}{\mathrm{Var}}
\newcommand{\Cov}{\mathrm{Cov}}
\newcommand{\Bias}{\mathrm{Bias}}

\begin{document}

\maketitle

\begin{homeworkProblem}
	\textbf{Part (a)}\\
	Given,
	\begin{equation}\label{key}
	\dd{s}^2 = - \qty(1- \dfrac{2M}{r}) \dd{t}^2 + \qty(1 - \dfrac{2M}{r})^{-1} \dd{r}^2 + r^2 \qty(\dd{\theta}^2 + \sin^2 \theta \dd{\phi}^2)
	\end{equation}
	Using an affine parameter $ \lambda $, we can rewrite these equations as,
	\begin{equation}\label{e2}
	\kappa = \qty(\dv{s}{\lambda})^2 = - \qty(1- \dfrac{2M}{r}) \dot{t}^2 + \qty(1 - \dfrac{2M}{r})^{-1} \dot{r}^2 + r^2 \qty(\dot{\theta}^2 + \sin^2 \theta \dot{\phi}^2)
	\end{equation}
	where $ \kappa = -1,0,1 $ for timelike, null and spacelike geodesics respectively. Corresponding to the Killing vectors $ \partial_t $ and $ \partial_\phi $, we would have the following conserved quantities,
	\begin{equation}\label{e3}
	\tilde{E} = 2\qty(1 - \dfrac{2M}{r}) \dot{t} \qq{and} \tilde{L} = 2 r^2 \sin^2 \theta \dot{\phi}
	\end{equation}
	Substituting $\dot{t} $ and $ \dot{\phi} $ from (\ref{e3}) to (\ref{e2}) and assuming planar orbit ($ \theta = \pi/2 $), we get,
	\begin{align}\label{key}
	\kappa &= - \qty(1- \dfrac{2M}{r}) \dfrac{\tilde{E}^2}{4 \qty(1 - \dfrac{2M}{r})^2}+ \qty(1 - \dfrac{2M}{r})^{-1} \dot{r}^2 + r^2 \dfrac{\tilde{L}^2}{4 r^4 \sin^4 \theta} \\
	&= - \qty(1- \dfrac{2M}{r})^{-1} \dfrac{\tilde{E}^2}{4}+ \qty(1 - \dfrac{2M}{r})^{-1} \dot{r}^2 + \dfrac{\tilde{L}^2}{4 r^2}\\
	\kappa - \dfrac{2M\kappa}{r} &= -  \dfrac{\tilde{E}^2}{4}+  \dot{r}^2 + \dfrac{\tilde{L}^2}{4 r^2} \qty(1 - \dfrac{2M}{r})
	\end{align}
	\begin{equation}\label{key}
	\therefore \dfrac{\dot{r}^2}{2} - \dfrac{1}{2}\qty({\kappa} +\dfrac{\tilde{E}^2}{4} ) + \dfrac{M\kappa}{r} + \dfrac{\tilde{L}^2}{8r^2} - \dfrac{\tilde{L}^2 M}{4 r^3} = 0
	\end{equation}
	Defining $ E = \dfrac{\tilde{E}}{2} $ and $ L = \dfrac{\tilde{L}}{2} $, we have,
	\begin{equation}\label{e8}
	\dfrac{\dot{r}^2}{2} + V_{eff}(r) = 0
	\end{equation}
	where $ V_{eff}(r) = - \dfrac{E^2 + \kappa}{2} + \dfrac{M\kappa}{r} + \dfrac{{L}^2}{2r^2} - \dfrac{{L}^2 M}{ r^3}  $
	\\
	
	\textbf{Part (b)}\\
	Substituting $ \kappa = 0$ and $ \dot{r} = 0 $ in (\ref{e8}), and further using (\ref{e3}) we have,
	\begin{equation}\label{e9}
	E^2 = \dfrac{L^2}{ r^2} \qty(1 - \dfrac{2M}{r}) \implies \qty(\dv{t}{\phi})^2  = r^2 \qty(1 - \dfrac{2M}{r})^{-1}
	\end{equation} 
	Now, writing down the radial geodesic for the Schwarzschild metric, we have,
	\begin{equation}\label{key}
	\dv{\tau} \qty[2\qty(1 - \dfrac{2M}{r})^{-1} \dot{r} ] = - \dfrac{2M}{r^2} \dot{t}^2  - \qty(1 - \dfrac{2M}{r})^{-2} \qty(\dfrac{2M}{r^2}) \dot{r}^2 + 2 r (\dot{\theta}^2 + \sin^2 \theta \dot{\phi}^2)
	\end{equation}
	Using $ \dot{r}=0 $, $ \theta = \pi/2 $, we get,
	\begin{equation}\label{key}
	\dfrac{2M}{r^2} \dot{t}^2 = 2 r \dot{\phi}^2 \implies \qty(\dv{t}{\phi})^2 = \dfrac{r^3}{M}
	\end{equation}
	Comparing with (\ref{e9}), we have,
	\begin{equation}\label{key}
	r^2 \qty(1 - \dfrac{2M}{r})^{-1} = \dfrac{r^3}{M} \implies 1 - \dfrac{2M}{r} = \dfrac{M}{r} \implies r = 3M
	\end{equation}
	Hence, there is a null circular orbit at $ r = 3M $. To decide stability, one needs to check the value of $ \pdv[2]{V_{eff}}{r} $ at $ r=r_p=3M $,
	\begin{equation}\label{key}
	\eval{\pdv[2]{V_{eff}}{r}}_{r=r_p} =  \dfrac{3L^2}{r_p^4} - 12\dfrac{L^2 M}{r_p^5} = - \dfrac{L^2}{(3M)^4} < 0
	\end{equation}
	Hence, this orbit is unstable.
	
	
	
	\textbf{Part (c)}\\
	Radial null geodesics will have $ \kappa =0  $ and $ L=0 $. Hence,
	\begin{align}\label{key}
	\dot{r}^2 = E^2 &= \qty(1 - \dfrac{2M}{r})^2 \dot{t}^2 \\
	\implies \dv{r}{t} &= \pm \qty(1 - \dfrac{2M}{r}) \\
	a
	\end{align}
	
	\textbf{Part (d)}
\end{homeworkProblem}

\begin{homeworkProblem}
	The area of a black hole $ A \propto M^2 $. The area theorem hence implies,
	\begin{equation}\label{ehawk}
	M_3^2 \ge M_1^2 + M_2^2 
	\end{equation}
	Consider the two numbers $ M_1^2 $ and $ M_2^2 $. Let's write down the $ AM \ge GM $ inequality \textit{ie.},
	\begin{align}\label{e10}
	\dfrac{M_1^2 + M_2^2}{2} &\ge M_1 M_2\\
	\implies M_1^2 + M_2^2 &\ge \dfrac{M_1^2 + M_2^2}{2} + M_1 M_2 \\
	\implies M_1^2 + M_2^2 &\ge \dfrac{(M_1 + M_2)^2}{2}
	\end{align}
	
	Combining (\ref{ehawk}) and the last inequality, we have,
	\begin{equation}\label{key}
	M_3^2 \ge \dfrac{(M_1 + M_2)^2}{2} \implies \dfrac{M_3}{M_1 + M_2} \ge \dfrac{1}{\sqrt{2}}
	\end{equation}
	Hence, 
	\begin{equation}\label{key}
 \eta \le 1 - \dfrac{1}{\sqrt{2}}
	\end{equation}
\end{homeworkProblem}
\end{document}
