\documentclass{article}

\usepackage{fancyhdr}
\usepackage{extramarks}
\usepackage{amsmath}
\usepackage{amsthm}
\usepackage{amssymb}
\usepackage{amsfonts}
\usepackage{tikz}
\usepackage{physics}
\usepackage[plain]{algorithm}
\usepackage{algpseudocode}
\usepackage{hyperref}

\usetikzlibrary{automata,positioning}

%
% Basic Document Settings
%

\topmargin=-0.45in
\evensidemargin=0in
\oddsidemargin=0in
\textwidth=6.5in
\textheight=9.0in
\headsep=0.25in

\linespread{1.1}

\pagestyle{fancy}
\lhead{\hmwkAuthorName}
\chead{\hmwkClass\ : \hmwkTitle}
\rhead{\firstxmark}
\lfoot{\lastxmark}
\cfoot{\thepage}

\renewcommand\headrulewidth{0.4pt}
\renewcommand\footrulewidth{0.4pt}

\setlength\parindent{0pt}

%
% Create Problem Sections
%
\newcommand{\be}{\begin{equation}}
\newcommand{\ee}{\end{equation}}
\newcommand{\bes}{\begin{equation*}}
\newcommand{\ees}{\end{equation*}}
\newcommand{\bea}{\begin{flalign*}}
\newcommand{\eea}{\end{flalign*}}


\newcommand{\enterProblemHeader}[1]{
    \nobreak\extramarks{}{Problem \arabic{#1} continued on next page\ldots}\nobreak{}
    \nobreak\extramarks{Problem \arabic{#1} (continued)}{Problem \arabic{#1} continued on next page\ldots}\nobreak{}
}

\newcommand{\exitProblemHeader}[1]{
    \nobreak\extramarks{Problem \arabic{#1} (continued)}{Problem \arabic{#1} continued on next page\ldots}\nobreak{}
    \stepcounter{#1}
    \nobreak\extramarks{Problem \arabic{#1}}{}\nobreak{}
}

\setcounter{secnumdepth}{0}
\newcounter{partCounter}
\newcounter{homeworkProblemCounter}
\setcounter{homeworkProblemCounter}{1}
\nobreak\extramarks{Problem \arabic{homeworkProblemCounter}}{}\nobreak{}

%
% Homework Problem Environment
%
% This environment takes an optional argument. When given, it will adjust the
% problem counter. This is useful for when the problems given for your
% assignment aren't sequential. See the last 3 problems of this template for an
% example.
%
\newenvironment{homeworkProblem}[1][-1]{
    \ifnum#1>0
        \setcounter{homeworkProblemCounter}{#1}
    \fi
    \section{Problem \arabic{homeworkProblemCounter}}
    \setcounter{partCounter}{1}
    \enterProblemHeader{homeworkProblemCounter}
}{
    \exitProblemHeader{homeworkProblemCounter}
}

%
% Homework Details
%   - Title
%   - Due date
%   - Class
%   - Section/Time
%   - Instructor
%   - Author
%

\newcommand{\hmwkTitle}{Assignment\ \#2}
\newcommand{\hmwkDueDate}{Due on 4th February, 2019}
\newcommand{\hmwkClass}{Advanced Statistical Mechanics}
\newcommand{\hmwkClassTime}{}
\newcommand{\hmwkClassInstructor}{}
\newcommand{\hmwkAuthorName}{\textbf{Aditya Vijaykumar}}

%
% Title Page
%

\title{
    %\vspace{2in}
    \textmd{\textbf{\hmwkClass:\ \hmwkTitle}}\\
    \normalsize\vspace{0.1in}\small{\hmwkDueDate\ }\\
%    \vspace{3in}
}

\author{\hmwkAuthorName}
\date{}

\renewcommand{\part}[1]{\textbf{\large Part \Alph{partCounter}}\stepcounter{partCounter}\\}

%
% Various Helper Commands
%

% Useful for algorithms
\newcommand{\alg}[1]{\textsc{\bfseries \footnotesize #1}}

% For derivatives
\newcommand{\deriv}[1]{\frac{\mathrm{d}}{\mathrm{d}x} (#1)}

% For partial derivatives
\newcommand{\pderiv}[2]{\frac{\partial}{\partial #1} (#2)}

% Integral dx
\newcommand{\dx}{\mathrm{d}x}

% Alias for the Solution section header
\newcommand{\solution}{\textbf{\large Solution}}

% Probability commands: Expectation, Variance, Covariance, Bias
\newcommand{\E}{\mathrm{E}}
\newcommand{\Var}{\mathrm{Var}}
\newcommand{\Cov}{\mathrm{Cov}}
\newcommand{\Bias}{\mathrm{Bias}}

\begin{document}

\maketitle
(\textbf{Acknowledgements} - I would like to thank Aditya Sharma and Junaid Majeed for discussions.)

\begin{homeworkProblem}[1]
	The two-particle Virial Coefficient $ b_2 $ is given by,
	\begin{align*}
	b_2 &= \int \dd^d \va{q}_1 \dd^d \va{q}_2 U(\va{q}_1  - \va{q}_2) \\
	&= A S_{d-1}^2 \int \dd q_1  \dd{q_2}  q_1^{d-1} q_2^{d-1}  \dfrac{1}{\abs{q_1 - q_2}^\sigma}
	\end{align*}
\end{homeworkProblem}



\begin{homeworkProblem}[2]
	\textcolor{red}{Do Part (a)}\\
	\textbf{Part (b)}\\
	Let's denote the Vandermonde determinant by $ D_n $,
	\begin{align*}
	D_n = \mdet{1  & x_1 & x_1^2 & \ldots & x_1^{n-1}\\ 1  & x_2 & x_2^2 & \ldots & x_2^{n-1} \\ \vdots & \vdots & \vdots & \vdots & \vdots \\ 1  & x_n & x_n^2 & \ldots & x_n^{n-1}}
	\end{align*}
	We prove the required statement by using row and column operations. We first use $ R_n \rightarrow R_n - R_{n-1}, n=2,3 \ldots, n $. We then have,
	\begin{align*}
	D_n = \mdet{1  & x_1 & x_1^2 & \ldots & x_1^{n-1}\\ 0  & x_2 - x_1 & x_2^2 - x_1^2 & \ldots & x_2^{n-1} - x_1^{n-1} \\ \vdots & \vdots & \vdots & \vdots & \vdots \\ 0  & x_n - x_{1} & x_n^2 - x_{1}^2 & \ldots & x_n^{n-1} - x_{1}^{n-1}}
	\end{align*}
	We now proceed to make the topmost row elements $ 0 $, save for the first element.
	We use $ C_n \rightarrow C_n - x_1 C_{n-1}, n=2,3 \ldots, n $,
	
	\begin{align*}
	D_n &= \mdet{1  & 0 & 0& \ldots & 0\\ 0  & x_2 - x_1 & x_2(x_2 - x_1) & \ldots & x_2^{n-2}(x_2 - x_1) \\ \vdots & \vdots & \vdots & \vdots & \vdots \\ 0  & x_n - x_{1} & x_n(x_n - x_{1}) & \ldots & x_n^{n-2}(x_n - x_{1}) } \\
	&=\mdet{ x_2 - x_1 & x_2(x_2 - x_1) & \ldots & x_2^{n-2}(x_2 - x_1) \\ \vdots & \vdots & \vdots & \vdots & \vdots \\  x_n - x_{1} & x_n(x_n - x_{1}) & \ldots & x_n^{n-2}(x_n - x_{1}) } \\
	&=\qty(\prod_{i=2}^{n} x_i - x_{1}) \mdet{1  & x_2 & x_2^2 & \ldots & x_2^{n-2}\\ 1  & x_3 & x_3^2 & \ldots & x_3^{n-2} \\ \vdots & \vdots & \vdots & \vdots & \vdots \\ 1  & x_n & x_n^2 & \ldots & x_n^{n-2}} \\
	\end{align*}
	
	\begin{align*}
	\implies D_n &= \qty(\prod_{i=2}^{n} x_i - x_{1}) D_{n-1} \\
	D_n &= \qty(\prod_{i=2}^{n} x_i - x_{1}) \qty(\prod_{i=3}^{n-1} x_i - x_{2}) D_{n-2}\\
	&=  \qty(\prod_{i=2}^{n} x_i - x_{1}) \qty(\prod_{i=3}^{n-1} x_i - x_{2}) \ldots D_2\\
	D_n &= \prod_{1\le j<i \le n} x_i - x_j
	\end{align*}
	Hence Proved.
	
\end{homeworkProblem}

\begin{homeworkProblem}[4]
	For $ N  $ particles in a harmonic trap,
	\begin{align*}
	\psi(x_1, x_2 , \ldots x_N) = \dfrac{1}{\sqrt{N!}} \det\phi_j (x_j) = \dfrac{1}{N!} \det A_{ij}  \\
	\qq{where} \phi_i(x) = \qty(\dfrac{a^2}{\pi})^{1/4} \dfrac{1}{\sqrt{2^i i!}} H_i(a x) e^{-a^2 x^2/2} \qq{,} a^2 = \dfrac{m \omega}{\hbar}
	\end{align*}
	
	The probability density is,
	\begin{align*}
	P(\{ x_i\}) &= \dfrac{1}{N!} \det A^T A  = \dfrac{1}{N!} \det K
	\end{align*}
	We define the average number density as,
	\begin{align*}
	\ev{\rho(x)} &= \sum_i \ev{\dfrac{1}{N} \delta (x- x_i)}\\
	&= \dfrac{1}{N}  \sum_i \int \prod_j \dd{x_j} \delta (x- x_i) P(\{x_k\})
	\end{align*}
	Let's have a closer look at the integral above. The integral over the delta function with replace the $ x_i $ in $ P(\{x_k\}) $ with $ x $. Expanding the sum over $ i $ will give us $ N $ terms, each having one argument replaced by $ x $. But we know that $ P(x_1, x_2) = P(x_2, x_1)$, and hence we can relabel the terms in the summation, and get the following expression,
	\begin{align*}
	\ev{\rho(x)} &= \int \prod_{j=1}^{N-1} \dd{x_j}  P(x, x_1, x_2 , \ldots, x_{N-1})\\
	&= \int \prod_{j=1}^{N-1} \dd{x_j} \dfrac{1}{N!} \det K
	\end{align*}
\end{homeworkProblem}


\end{document}

