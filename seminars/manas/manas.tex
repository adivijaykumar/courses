
\documentclass[a4paper,11pt]{article}

\usepackage{physics}
\usepackage{amsmath}
\usepackage{amssymb}
\usepackage{amsmath}
\usepackage{amsthm, mathtools}
%\usepackage{hyperref}
\usepackage{color}
\usepackage{jheppub}
\usepackage[T1]{fontenc} % if needed

% My Documents
\newcommand{\be}{\begin{equation}}
\newcommand{\ee}{\end{equation}}
\newcommand{\bes}{\begin{equation*}}
\newcommand{\ees}{\end{equation*}}
\newcommand{\bea}{\begin{flalign*}}
\newcommand{\eea}{\end{flalign*}}

\def\a{\alpha}
\def\b{\beta}
\def\g{\gamma}
\def\s{\sigma}
%\linespread{1.0}
%\setlength{\parindent}{0em}
%\setlength{\parskip}{0.8em}

\title{\textbf{Aspects of non-equilibrium Physics}}
\author{Aditya Vijaykumar}
\affiliation{International Centre for Theoretical Sciences, Bengaluru, India.}
\emailAdd{aditya.vijaykumar@icts.res.in}

\begin{document}
\maketitle

\section{Introduction and Motivation}
Non-equilibrium systems can be - open, closed and quenched (changing parameters of the Hamiltonian suddenly so that the system as a whole is out of equilibrium). They arise in classical (integrable models) as well as quantum systems.

\textbf{(question by Junaid on nonlinear Schrodinger equation, and a long detour about that. Manas writes the Schrodinger equation with a $ \abs{\psi}^2 $ potential, and discusses a few methods of discretization of this equation. The essence of the discussion was that the nonlinear schrodinger equation can be reduced (in something called the reductive perturbation expansion) to a minimal model of the KdV equation. $ u_t = -\partial_x (\alpha u^2 + \beta u_{xx}) $). The solution to $ \beta=0 $ case is $ u(x,t) = U_0(x - u(x,t) t) $ So if we start out with lets say a Lorentzian profile, the $ u^2  $ term will cause the profile to curve to the right and steepen. But the derivative term will now convert the steepening to oscillations.) } 

Classical non-equilibrium systems can be classifield as :-
\begin{itemize}
	\item Integrable Models
	\item Non-integrable Models
	\item Classical Field Theoretic systems
\end{itemize}

Quantum non-equilibrium systems :-
\begin{itemize}
	\item \textbf{Quantum Lattice Models (Incommensurate Models)}- $ H = \sum_i (a_i^\dagger a_{i+1} + \text{h.c}) \sum_i (a_i^\dagger a_{i}) $. Lets say one starts with $ \omega_i = \lambda \cos(2 \pi b i), b = 4/3 $. The model basically repeats itself after every three steps in $ i $, ie.$ i \rightarrow i + 3 $. Under this conditions this system is called a ballistic system. But if $ b = \dfrac{\sqrt{5 -1}}{2} $, this model never repeats itself. Such models are called incommensurate models. For $ \lambda < 1 $, these systems behave ballistic, but when $ \lambda > 1 $, the system is localized. The current goes down. When $ \lambda = 1 $, it's called the critical condition. Just by a model where there are no real interactions, we see a lot of different phases with respect to the parameter $ \lambda $.
	\item \textbf{Hybrid Quantum Systems} - Systems made of fermionic and bosonic degrees of freedom. Let's say we have a mesoscopic systems (with only a source and sink) and two quantum dots in between. Lets say one couples this fermionic system to a bosonic system. How do the bosonic degrees of freedom affect the fermionic current and vice versa.
	\item \textbf{Designing Quantum Hamiltonian Systems} - Open Quantum Phase Transitions, Open Quantum Spin Chains, Property of non-Hermitian systems, emergent phenomena, Quantum Devices, understanding dark states
\end{itemize}

Seni-classical non-equilibrium systems :-
\begin{itemize}
	\item PDEs due to cold atoms, multi component gases
\end{itemize}



















\end{document}