
\documentclass[a4paper,11pt]{article}

\usepackage{physics}
\usepackage{amsmath}
\usepackage{amssymb}
\usepackage{amsmath}
\usepackage{amsthm, mathtools}
%\usepackage{hyperref}
\usepackage{color}
\usepackage{jheppub}
\usepackage[T1]{fontenc} % if needed

% My Documents
\newcommand{\be}{\begin{equation}}
\newcommand{\ee}{\end{equation}}
\newcommand{\bes}{\begin{equation*}}
\newcommand{\ees}{\end{equation*}}
\newcommand{\bea}{\begin{flalign*}}
\newcommand{\eea}{\end{flalign*}}

\def\a{\alpha}
\def\b{\beta}
\def\g{\gamma}
\def\s{\sigma}
%\linespread{1.0}
%\setlength{\parindent}{0em}
%\setlength{\parskip}{0.8em}

\title{\textbf{Aspects of non-equilibrium Physics}}
\author{Aditya Vijaykumar}
\affiliation{International Centre for Theoretical Sciences, Bengaluru, India.}
\emailAdd{aditya.vijaykumar@icts.res.in}

\begin{document}
\maketitle

\section{Introduction and Motivation}
Non-equilibrium systems can be :-
\begin{itemize}
	\item \textbf{Open} - Interactions with the \textit{environment} are considered.
	\item \textbf{Closed} - the system is considered to be isolated
	\item \textbf{Quenched} - parameters of the Hamiltonian suddenly so that the system as a whole is out of equilibrium
\end{itemize}
Let's look at examples of classical, semi-classical and quantum non-equilibrium systems.
\section{Classical non-equilibrium Systems} 

Classical non-equilibrium systems can be classifield as follows :-
\begin{itemize}
	\item \textbf{Integrable Models} - The most famous example is the \textit{Calogero Family of Models}, with a Hamiltonian given by,
	\begin{equation*}
	H(p,q) = \dfrac{1}{2}\sum_{n} (p_n^2 + \omega^2 q_n^2) + g^2 \sum_{m,n ; m \ne n }\dfrac{1}{(q_n - q_m)^2}
	\end{equation*}
	The \textit{Discrete non-linear Schrodinger Equation} is another example of an classical integrable system. The non-linear Schrodinger Equation is given by,
	\begin{equation*}
	i \pdv{\psi}{t} = - \pdv[2]{\psi}{x} + g \abs{\psi }^2 \psi
	\end{equation*}
	which can be thought of being derived from the Hamiltonian,
	\begin{equation*}
	H = \int \qty[\dfrac{1}{2} \qty(\pdv{\psi}{x})^2 + \dfrac{g}{2} \abs{\psi}^4] dx
	\end{equation*}
	One encounters these equations in the models for cold atomic gases or non-linear optics.
	\item \textbf{Non-integrable Models} - One can discretize the above non-linear Schrodinger equation, and get an expression for the Hamiltonian as follows,
	\begin{equation*}
	H = \sum_{j=0}^{N-1} \qty[\dfrac{1}{2m} \abs{\psi_{j+1} - \psi_j}^2 + \dfrac{1}{2} \abs{\psi_{j}}^4]
	\end{equation*}
	giving the equations of motion,
	\begin{align*}
	i \pdv{\psi_j}{t} = -\dfrac{1}{2m } \Delta \psi_j + g \abs{\psi_j}^2
	\end{align*}
	which is a non-integrable system.
	
	\item \textbf{Classical Field Theoretic Models} - The nonlinear schrodinger equation can be reduced (in something called the \textit{reductive perturbation expansion}) to a minimal model of the KdV equation. 
	\begin{equation*}
	u_t = -\partial_x (\alpha u^2 + \beta u_{xx}) 
	\end{equation*} The solution to $ \beta=0 $ case is,
	\begin{equation*}
	u(x,t) = U_0(x - u(x,t) t)
	\end{equation*}
	 So if we start out with lets say a Lorentzian profile, the $ u^2  $ term will cause the profile to curve to the right and steepen. But the derivative term will now convert the steepening to oscillations.
\end{itemize}

\section{Quantum non-equilibrium Systems}
\begin{itemize}
	\item \textbf{Quantum Lattice Models (Incommensurate Models)} - Consider the following Hamiltonian,
	\begin{equation*}
	H = \sum_i (a_i^\dagger a_{i+1} + \text{h.c}) +  \sum_i \omega_i (a_i^\dagger a_{i})
	\end{equation*}
	Lets say one starts with $ \omega_i = \lambda \cos(2 \pi b i), b = 4/3 $. The model basically repeats itself after every three steps in $ i $, ie.$ i \rightarrow i + 3 $. Under this conditions this system is called a ballistic system. But if $ b = \dfrac{\sqrt{5} -1}{2} $ (golden mean), this model never repeats itself. Such models are called incommensurate models. For $ \lambda < 1 $, these systems behave ballistic, but when $ \lambda > 1 $, the system is localized. The current goes down. When $ \lambda = 1 $, it's called the critical condition. Just by a model where there are no real interactions, we see a lot of different phases with respect to the parameter $ \lambda $.
	\item \textbf{Hybrid Quantum Systems} - Systems made of fermionic and bosonic degrees of freedom. Let's say we have a mesoscopic systems (with only a source and sink) and two quantum dots in between. Lets say one couples this fermionic system to a bosonic system. We ask -\textit{ How do the bosonic degrees of freedom affect the fermionic current and vice versa?}
	\item \textbf{Designing Quantum Hamiltonian Systems} - Open Quantum Phase Transitions, Open Quantum Spin Chains, Property of non-Hermitian systems, emergent phenomena, Quantum Devices, understanding dark states
\end{itemize}

\section{Semi-classical non-equilibrium Systems}
PDEs due to cold atoms, multi component gases etc. fall under this category. We can also have systems with are inherently quantum in their large $ N $ limit.


\end{document}